\documentclass[output=paper
,modfonts
,nonflat]{langsci/langscibook} 
\bibliography{/Users/tgrano/Dropbox/work/ling.bib} 
% add all extra packages you need to load to this file  
\usepackage{tabularx} 

%%%%%%%%%%%%%%%%%%%%%%%%%%%%%%%%%%%%%%%%%%%%%%%%%%%%
%%%                                              %%%
%%%           Examples                           %%%
%%%                                              %%%
%%%%%%%%%%%%%%%%%%%%%%%%%%%%%%%%%%%%%%%%%%%%%%%%%%%% 
%% to add additional information to the right of examples, uncomment the following line
% \usepackage{jambox}
%% if you want the source line of examples to be in italics, uncomment the following line
% \renewcommand{\exfont}{\itshape}
% \usepackage{lipsum}
% \usepackage[normalem]{ulem}
\usepackage{tikz}
\usepackage{tikz-qtree}
\usepackage{tikz-qtree-compat}
\usepackage{tabularx}
\usepackage{verbatim}
\usepackage{pifont}    %for pointing hand, checkmarks, crosses
\usepackage{tipa}
\usepackage{amssymb}
\usepackage{amsmath}
\usepackage{subcaption}
\usepackage{csquotes}
\usepackage{multirow}
\usepackage{multicol}
\usepackage{outlines} 
%\usepackage{wrapfig}
\usepackage{enumitem}
\usepackage{rotating}  
%\usepackage{ling-macros-custom}
\usepackage{stmaryrd}
%\usepackage{qtree}
\usepackage{./langsci/styles/langsci-optional}
\usepackage{./langsci/styles/langsci-lgr}
\usepackage{hhline}
\usepackage{langsci-gb4e}
\usepackage[linguistics]{forest}
 



%% hyphenation points for line breaks
%% Normally, automatic hyphenation in LaTeX is very good
%% If a word is mis-hyphenated, add it to this file
%%
%% add information to TeX file before \begin{document} with:
%% %% hyphenation points for line breaks
%% Normally, automatic hyphenation in LaTeX is very good
%% If a word is mis-hyphenated, add it to this file
%%
%% add information to TeX file before \begin{document} with:
%% %% hyphenation points for line breaks
%% Normally, automatic hyphenation in LaTeX is very good
%% If a word is mis-hyphenated, add it to this file
%%
%% add information to TeX file before \begin{document} with:
%% \include{localhyphenation}
\hyphenation{
affri-ca-te
affri-ca-tes
com-ple-ments
}
\hyphenation{
affri-ca-te
affri-ca-tes
com-ple-ments
}
\hyphenation{
affri-ca-te
affri-ca-tes
com-ple-ments
}
 
\newcommand{\mc}[1]{\textsc{#1}}	% morpheme glossing as small caps: does not clash with fontspec
\def\Id#1{\vskip-.cm\hskip.01cm\vtop{#1}} % for controlling where the example breaks onto the next line


\setlength{\emergencystretch}{2pt}
\newcolumntype{s}{>{\hsize=.5\hsize}X}
\newcolumntype{m}{>{\hsize=.25\hsize}X}


 

%makes subscripts
\newcommand{\sub}[1]{\textsubscript{#1}}
\newcommand{\tikzmark}[1]{\tikz[overlay,remember picture]\node(#1){};}

%draws normal arrow
\newcommand*{\DrawArrow}[3][]{%
	\begin{tikzpicture}[overlay,remember picture, >=latex']
	\draw[rounded corners, -latex,<-,semithick,#1]  (#2) -- ++(0,-1.2em)coordinate (a) -- 
	($(a-|#3)$) -- (#3);
	\end{tikzpicture}%
}

%draws dashed arrow
\newcommand*{\DrawXArrow}[3][]{%
    % #1 = draw options
    % #2 = left point
    % #3 = right point
    \begin{tikzpicture}[overlay,remember picture,>=latex']
       % \draw [-latex, #1] ($(#2)+(0.1em,0.5ex)$) to ($(#3)+(0,0.5ex)$);
		\draw[rounded corners, -latex,<-,semithick,#1]  (#2) -- ++(0,-1.2em)coordinate (a) -- 
		node%[font=\footnotesize]
		{\ding{56}}
		($(a-|#3)$) -- (#3);
    \end{tikzpicture}%
}% 

\newcommand*{\DrawArrowok}[4][]{%
	% #1 = draw options
	% #2 = left point
	% #3 = right point
	\begin{tikzpicture}[overlay,remember picture, >=latex']
	% \draw [-latex, #1] ($(#2)+(0.1em,0.5ex)$) to ($(#3)+(0,0.5ex)$);
	\draw[rounded corners, -latex,<-,semithick,#1]  (#2) -- ++(0,-1.2em)coordinate (a) -- 
	%node[below,font=\footnotesize]{#4} 
	node[midway,fill=white,font=\normalsize]{#4}
	($(a-|#3)$) -- (#3);
	\end{tikzpicture}%
}% 

\newcommand*{\DrawLXArrow}[3][]{%
	% #1 = draw options
	% #2 = left point
	% #3 = right point
	\begin{tikzpicture}[overlay,remember picture,>=latex']
	% \draw [-latex, #1] ($(#2)+(0.1em,0.5ex)$) to ($(#3)+(0,0.5ex)$);
	\draw[rounded corners, -latex,<-,semithick,#1]  (#2) -- ++(0,-1.8em)coordinate (a) -- 
	node%[font=\footnotesize]
	{\ding{56}}
	($(a-|#3)$) -- (#3);
	\end{tikzpicture}%
}% 
\usetikzlibrary{arrows,shapes,positioning,shadows,trees,calc}
\usetikzlibrary{decorations.text}

 
\newcommand{\all}[1]{\ensuremath{\forall #1}}

\newcommand{\bari}{\ipabar{\i}{.5ex}{1.1}{}{}} 

\let\oldemptyset\emptyset
\let\emptyset\varnothing
 


\usetikzlibrary{calc,positioning,tikzmark}

%hyman
\newcommand{\higr}[1]{{\color{red}#1}}
 
 
\definecolor{lsDOIGray}{cmyk}{0,0,0,0.45}

%newkirk
\newcommand{\ix}[1]{{\color{red}\textsubscript{#1}}} %probably i.nde.x
\newcommand{\ol}[1]{{\color{red}\textit{#1}}} %probably o.bject l.anguage
\newcommand{\alert}[1]{{\color{red}\textbf{#1}}}
\newenvironment{context}{\color{red}
\smallskip 

\scriptsize 
}{\color{black}\normalsize }
\newcommand{\m}[1]{{\color{red}\textsc{#1}}}
\newcommand{\den}[1]{{\color{red}#1}}
\newcommand{\type}[1]{{\color{red}#1}}
\newcommand{\denol}[1]{{\color{red}#1}}
\newcommand{\bex}{\ea}
\newcommand{\fex}{\z}
\newcommand{\set}{{\color{red} SET}}
 
\def\denotes#1{$\lbrack\!\lbrack${#1\/}$\rbrack\!\rbrack$}      % denotes
\newcommand{\citeNP}{\citealt}


\makeatletter
\let\thetitle\@title
\let\theauthor\@author 
\makeatother


\newcommand{\togglepaper}[1][0]{ 
  \bibliography{../localbibliography}
  \papernote{\scriptsize\normalfont
    \theauthor.
    \thetitle. 
    To appear in: 
    Samson Lotven,   Silvina Bongiovanni,   Phillip Weirich,   Robert Botne \&  Samuel Gyasi Obeng (eds.),  
    African linguistics across the disciplines: Selected papers from the 48th Annual Conference on African Linguistics 
    Berlin: Language Science Press. [preliminary page numbering]
  }
  \pagenumbering{roman}
  \setcounter{chapter}{#1}
  \addtocounter{chapter}{-1}
}

 

\def\denotes#1{$\lbrack\!\lbrack${#1\/}$\rbrack\!\rbrack$}      % denotes

\title{Control of logophoric pronouns in Gengbe} 
\author{%
 Thomas Grano\affiliation{Indiana University}\lastand 
 Samson Lotven\affiliation{Indiana University} 
}

% \chapterDOI{} %will be filled in at production

% \epigram{}

\abstract{Control is a phenomenon in which the subject of an embedded clause (the `controlled argument') is obligatorily bound by an argument of the immediately embedding predicate (the `controller'). Cross-linguistic research has revealed variation in how control sentences are syntactically instantiated, though no studies to date have documented cases where the controlled argument is realized by an overt logophoric pronoun. Based on novel field data, we argue that precisely this happens in Gengbe (Gbe, Niger-Congo), though only with some embedding verbs (such as {\em d{\Z}\'i} `want') and only when the embedded clause has potential (as opposed to jussive) mood marking. We propose an account of the facts whereby control complements are property-denoting and whereby logophoricity and jussive mood are two independent routes for creating property-denoting clauses. The upshot is a view of control as an emergent phenomenon; there is no `control construction' or `control pronoun' (PRO) but rather several independent components of the grammar that interact to give rise to control under certain conditions for principled type-theoretic reasons.}

\newcommand{\Z}{ʒ}
\newcommand{\D}{ɖ}
\newcommand{\ny}{ɲ}
\newcommand{\R}{ɾ}

\newcommand{\á}{\'{ã}}
\newcommand{\É}{\'{\~{ε}}}
\newcommand{\È}{\`{\~{ε}}}
\newcommand{\í}{\'{\~{i}}}
\newcommand{\ì}{\`{\~{i}}}
\renewcommand{\O}{ɔ}
\newcommand{\Ó}{\'{\~{ɔ}}}
\newcommand{\Ò}{\`{\~{ɔ}}}
\newcommand{\ú}{\'{ũ}}
\newcommand{\ù}{\`{ũ}}

%\let\bf\textbf
%\let\sc\textsc



\begin{document}

\maketitle
\section{Introduction} 






Control is a phenomenon in which the subject of an embedded clause (the `controlled argument') is obligatorily bound by an argument of the immediately higher embedding predicate (the `controller'). In (\ref{1}), for example, the unexpressed subject of {\em leave} (represented here as PRO) can only be understood as coreferential with {\em Bill}; it cannot be coreferential with {\em John} nor can it take an antecedent outside the sentence. (See e.g. \citealt{landau13} for a recent survey of the vast theoretical literature on control.)

\ea (John$_{1}$ said that) {\bf Bill$_{2}$} wants [{\bf PRO}$_{*1/2/*3}$ to leave]. \label{1}
\z

Cross-linguistic research has revealed a fair amount of variation in the syntax of control. Diverging from the pattern instantiated by English, some languages like Tsez (a Nakh-Daghestanian language spoken in northeast Caucasus) evidence {\sc backward control}, wherein the controlled argument is overt and the controller is covert, as in (\ref{2}).

\ea
\gll {\bf $\emptyset$} [{\bf kid-b\={a}} ziya b-i\v{s}r-a] y-oq-si.\\
{\sc ii}.{\sc abs} girl.{\sc ii}-{\sc erg} cow.{\sc abs} {\sc iii}-feed-{\sc inf} {\sc ii}-begin-{\sc pst}.{\sc evid}\\
\glt `The girl began to feed the cow.' \hfill ({\sc tsez}, \citealt{pp02}:248) \label{2}
\z

Still other languages exhibit {\sc copy control}, wherein both the controller and the controlled argument are overtly represented, as in San Luis Quiavin\'i Zapotec, illustrated in (\ref{3}).

\ea
\gll R-c\`a\`a'z {\bf Gye'eihlly} [g-auh {\bf Gye'eihlly} bxaady].\\
{\sc hab}-want Mike {\sc irr}-eat Mike grasshopper\\
\glt `Mike wants to eat grasshopper.' \hfill({\sc san luis quiavin\'i zapotec},\\ \vspace{-.5cm} \null\hspace{9.36cm} Lee 2003:102) \label{3}
\z

 \nocite{lee03}

Another pattern involves control of an overt anaphor, as found for example in Korean and illustrated in (\ref{4}).


\ea
\gll {\bf Inho-ka} Jwuhi-eykey [{\bf caki-ka} cip-ey ka-keyss-ko] yaksok-ha-yess-ta.\\
Inho-{\sc nom} Jwuhi-{\sc dat} self-{\sc nom} home-{\sc loc} go-{\sc vol}-{\sc comp} promise-do-{\sc pst}-{\sc comp}\\
\glt `Inho promised Jwuhi to go home.' \hfill({\sc korean}, \citealt{madigan08a}:237) \label{4}
\z

Finally, yet another pattern involves control of an overt nominative expression, which \cite{szabolcsi09} has shown obtains in Hungarian under some conditions, as in (\ref{5}).

\ea
\gll Szeretn\'ek [csak {\bf \'en}  magas lenni].\\
would.like.1{\sc sg} only I  tall be.{\sc inf}\\
\glt `I want it to be the case that only I am tall.' \hfill({\sc hungarian}, \citealt{szabolcsi09}:8) \label{5}
\z

The focus of this paper is yet another pattern which is to our knowledge unattested in the control literature: control of an overt logophoric pronoun.\footnote{\cite{landau15}, for example, citing \cite{culy94}, says that overt logophoric pronouns are never found in control complements.}
 We argue that this happens in Gengbe, in sentences like (\ref{6}).

\ea
\gll (K\`of\'i$_{1}$ b\'e) {\bf \'Am\'{\~a}$_{2}$}  d{\Z}\'i [b\'e {\bf j\`e$_{*1/2/*3}$}  l\'a {\D}\`u n{\ú}].\\
%\gll (K\`of\'i$_{1}$ b\'e) {\bf \'Am\'{\~a}$_{2}$}  d{\IPA Z}\'i [b\'e {\bf j\`e$_{*1/2/*3}$}  l\'a {\IPA \:d}\`u n{\IPA \'{\~u}}].\\
Kofi say Ama want {\sc comp} {\sc log} {\sc pot} eat thing\\
\glt `(Kofi said that) Ama wants to eat.' \label{wll} \hfill ({\sc gengbe}) \label{6}
\z

%\exg. (K\`of\'i$_{1}$ b\'e) {\bf \'Am\'{\~a}$_{2}$}  d{\Z}\'i [b\'e {\bf j\`e$_{*1/2/*3}$}  l\'a {\D}\`u n{\ú}].\\
%\exg. (K\`of\'i$_{1}$ b\'e) {\bf \'Am\'{\~a}$_{2}$}  d{\IPA Z}\'i [b\'e {\bf j\`e$_{*1/2/*3}$}  l\'a {\IPA \:d}\`u n{\IPA \'{\~u}}].\\
%Kofi say Ama want {\sc comp} {\sc log} {\sc pot} eat thing\\
%`(Kofi said that) Ama wants to eat.' \label{wll} \hfill ({\sc gengbe}, this talk)

In what follows, after providing more background on Gengbe and our data collection (section 2), we will show that control of  logophors in Gengbe obtains only with some embedding predicates, and only with potential (as opposed to jussive) mood marking (section 3). Then, in section 4, drawing on  relevant theoretical literature, we will sketch a formal semantic account of the observed facts wherein control complements denote properties (\citealt{chierchia84, dowty85}) and wherein there are two routes to propertyhood: (i) a logophoric subject (\citealt{pearson15}) or (ii) jussive mood marking (\citealt{zpp12}). The upshot is a view of control as an emergent phenomenon; there is no `control construction' or `control pronoun' but rather several independent components of the grammar that interact to produce control under certain conditions for principled type-theoretic reasons. We conclude in section 5.\footnote{The core data and analysis presented in this paper are reported also in \cite{gl17}, where we focus on the implications of the Gengbe data for theories of mood.}



\section{Background on Gengbe and our data collection}


Gengbe (also known as {\em Gen} or {\em Mina}) is a Niger-Congo (Kwa) language closely related to Ewe and spoken in southern Togo and Benin. According to Ethnologue, it has 278,900 speakers worldwide.  All Gengbe data reported in this paper were collected via elicitation sessions at Indiana University during 2014--2016 with Gabriel Mawusi, a native Gengbe speaker from Batonou, Togo. These sessions were conducted by Samson Lotven and supported by Professor Samuel Obeng.



\section{Core data and puzzles}

We begin with the observation that Gengbe has mood markers that are used to express future possibility and deontic necessity, respectively, as illustrated in (\ref{pot}) and (\ref{juss}). Following \cite{essegbey08} on the  cognate  Ewe particle {\em a}, we label the former {\sc pot}({\sc ential}), and following \cite{ameka08} on  cognate Ewe particle {\em ne}, we label the latter {\sc juss}({\sc ive}).


\ea 
\gll Ak\'u {\bf l\'a} {\D}\`u n\'{\~u}.\\
%\gll Ak\'u {\bf l\'a} {\IPA \:d}\`u n{\IPA \'{\~u}}.\\
Aku {\sc pot} eat thing\\
\glt `Aku will/might eat.' \label{pot} %\hfill {\sc pot}({\sc ential} {\sc mood})\\  \hspace*{\fill} (following \ct{essegbey08} on Ewe cognate {\em a})
\z

\ea
\gll Ak\'u {\bf n{\É}} {\D}\`u n\'{\~u}.\\
%\gll Ak\'u {\bf n{\IPA \'{\~E}}} {\IPA \:d}\`u n{\IPA \'{\~u}}.\\
Aku {\sc juss} eat thing\\
\glt`Aku should eat.' / `I want Aku to eat.' \label{juss} %\hfill {\sc juss}({\sc ive mood})\\  \hspace*{\fill}  (following \ct{ameka08} on Ewe cognate {\em ne})
\z

Against this backdrop, the first puzzle we want to consider is that when embedded under {\em d{\Z}\'i} `want', (\ref{pot}) (with potential marking) results in an unacceptable sentence whereas (\ref{juss}) (with jussive marking) results in an acceptable sentence:

\ea
\gll *\'Am\'{\~a} {\bf d{\Z}\'i} [b\'e {\bf \`Ak\'u} {\bf l\'a} {\D}\`u n\'{\~u}].\\
%\gll *\'Am\'{\~a} {\bf d{\IPA Z}\'i} [b\'e {\bf \`Ak\'u} {\bf l\'a} {\IPA \:d}\`u n{\IPA \'{\~u}}].\\
Ama want {\sc comp} Aku {\sc pot} eat thing\\
\glt Intended: `Ama wants Aku to eat.' \label{wal} % \hfill want+full-NP+POT $\rightarrow$ *
\z

\ea
\gll \'Am\'{\~a} {\bf d{\Z}\'i} [b\'e {\bf \`Ak\'u} {\bf n{\É}} {\D}\`u n\'{\~u}].\\
%\gll \'Am\'{\~a} {\bf d{\IPA Z}\'i} [b\'e {\bf \`Ak\'u} {\bf n{\IPA \'{\~E}}} {\IPA \:d}\`u n{\IPA \'{\~u}}].\\
Ama want {\sc comp} Aku {\sc juss} eat thing\\
\glt `Ama wants Aku to eat.' \label{wan}  %\hfill want+full-NP+JUSS $\rightarrow$ OK
\z

At first glance, the puzzle in (\ref{wal})--(\ref{wan}) seems familiar enough: in many languages, verbs that embed clausal complements can only combine with clauses that bear a particular mood. In Romance languages, for example, `want' requires the subjunctive mood. (See e.g. \citealt{palmer01} for an overview.) Hence we might hypothesize that in Gengbe, {\em d{\Z}\'i} `want' requires jussive mood. 

But additional data reveal that this cannot be the whole story. Gengbe has a logophoric pronoun {\em  j\`e} which in sentences like (\ref{bl})--(\ref{bll}) behaves like other logophoric pronouns reported in the literature: it must be anteceded by an attitude holder, and multiple embedding gives rise to an ambiguity in antecedent choice whereby any c-commanding attitude holder can serve as the antecedent (see e.g.~\citealt{clements75, pearson15}).\footnote{We use underscores in {\em k\'{\~a}\_{\D}\'o\_\'e\_d{\Z}\'i} `be certain' to signal that it is morphologically complex, consisting of a verb {\em k\'{\~a}} `cut' and a third-person singular pronoun  {\em \'e} flanked by two adpositions {\em {\D}\'o} `at' and {\em d{\Z}\'i} `top'. An English gloss more faithful to this underlying structure would be `count on it that
\ldots'.  We nonetheless gloss  {\em k\'{\~a}\_{\D}\'o\_\'e\_d{\Z}\'i}  as `be.certain' in the interest of perspicuity. }

\ea
\gll  {\bf \'Am\'{\~a}$_{1}$} k\'{\~a}\_{\D}\'o\_\'e\_d{\Z}\'i [b\'e {\bf j\`e$_{1/*2}$}  {\D}\`u n\'{\~u}].\\
%\gll  {\bf \'Am\'{\~a}$_{1}$} k{\IPA \'{\~a}}{\IPA \:d}\'o\'ed{\IPA Z}\'i [b\'e {\bf j\`e$_{1/*2}$}  {\IPA \:d}\`u n{\IPA \'{\~u}}].\\
Ama be.certain {\sc comp} {\sc log}  eat thing\\
\glt `Ama is certain that she (= Ama)  ate.' \label{bl}
\z

\ea 
\gll {\bf K\`of\'i$_{1}$} b\'e {\bf \'Am\'{\~a}$_{2}$} k\'{\~a}\_{\D}\'o\_\'e\_d{\Z}\'i [b\'e {\bf j\`e$_{1/2/*3}$}  {\D}\`u n\'{\~u}].\\
Kofi say Ama be.certain {\sc comp} {\sc log}  eat thing\\
\glt `Kofi said Ama is certain that he/she (= Kofi/Ama)  ate.' \label{bll}
\z

This leads us to our second puzzle. When {\em  j\`e} is embedded under  {\em d{\Z}\'i} `want', both potential and jussive mood become acceptable in the embedded clause, but with consequences for antecedent choice. When the potential marker is used, the logophor can only be anteceded by the immediately higher subject, instantiating a control relation as illustrated in (\ref{wll}), whereas when the jussive marker is used, the logophor is obligatorily obviative with respect to the immediately higher subject and instead must be bound remotely, as illustrated in (\ref{wln}).

\ea
\gll K\`of\'i$_{1}$ b\'e {\bf \'Am\'{\~a}$_{2}$} {\bf d{\Z}\'i} [b\'e {\bf j\`e$_{*1/2/*3}$} {\bf l\'a} {\D}\`u n\'{\~u}].\\
Kofi say Ama want {\sc comp} {\sc log} {\sc pot} eat thing\\
\glt `Kofi said Ama wants to eat.' \label{wll}  \hfill {\sc control}%\hfill want+LOG+POT $\rightarrow$ {\sc control}%\hfill {\em l\'a} $\rightarrow$ {\sc control}
\z

\ea
\gll {\bf K\`of\'i$_{1}$} b\'e  \'Am\'{\~a}$_{2}$ {\bf d{\Z}\'i} [b\'e {\bf j\`e$_{1/*2/*3}$} {\bf n{\É}} {\D}\`u n\'{\~u}].\\
Kofi say Ama want {\sc comp} {\sc log} {\sc juss} eat thing\\
\glt `Kofi said Ama wants him ( = Kofi) to eat.' \label{wln}  \hfill {\sc obviation}
\z


As expected given this second puzzle, if the desire reports in (\ref{wll})--(\ref{wln}) appear unembedded, the former is grammatical and has an obligatory control interpretation (\ref{wll'}), whereas the latter is simply ungrammatical (\ref{wln'}): the logophor demands an antecedent, but the jussive marker forces obviation with respect to the only potential antecedent, leading to an irreconcilable conflict.

\ea
\gll  \'Am\'{\~a}$_{1}$  d{\Z}\'i [b\'e  j\`e$_{1/*2}$ {\bf l\'a} {\D}\`u n\'{\~u}].\\
Ama want {\sc comp} {\sc log} {\sc pot} eat thing\\
\glt `Ama wants to eat.' \label{wll'}  %\hfill {\sc control}%\hfill want+LOG+POT $\rightarrow$ {\sc control}%\hfill {\em l\'a} $\rightarrow$ {\sc control}
\z

\ea
\gll *\'Am\'{\~a} d{\Z}\'i [b\'e  j\`e {\bf n{\É}} {\D}\`u n\'{\~u}].\\
Ama want {\sc comp} {\sc log} {\sc juss} eat thing\\
\glt Intended: `Ama wants to eat.' \label{wln'}  %\hfill {\sc obviation}
\z





 Other attitude predicates that pattern like {\em d{\Z}\'i} `want' with respect to this puzzle include {\em w{\`{ɔ}}\_s\'us\'u} `intend' ({\em lit.}: `do thought'), {\em d{\Z}\`e\_\`agb\`agb\'a} `try' ({\em lit.:} `do ability'), {\em l{\Ò}} `agree', and {\em fj{\È}\_d{\Z}{\`{ɔ}}gb\`e} `pledge'. 

In contrast with `want' and the other attitude predicates just mentioned, the majority of attitude predicates do not behave likewise with respect to these puzzles; these predicates include {\em k\'{\~a}\_{\D}\'o\_\'e\_d{\Z}\'i} `be certain', {\em {\ny}{\á}} `know', {\em gbl{\Ò}} `say', and {\em k\'u\`u\_d{\R}\`{\~i}\'{\~i}} `dream' ({\em lit.:} `die dream'). With these attitude predicates, complement clauses do not have to contain an overt mood marker (as already illustrated for `be certain' in (\ref{bl})--(\ref{bll}) above). And when they do contain an overt mood marker, both potential and jussive mood are compatible with a full-NP subject, as seen in (\ref{bal})--(\ref{ban}). Furthermore, with a logophoric subject, potential marking gives rise to ambiguity in antecedent choice (\ref{bal2}) whereas jussive marking patterns like it does for `want' in forcing obviation (\ref{ban2}).

\ea
\gll K\`of\'i b\'e \'Am\'{\~a} {\bf k\'{\~a}\_{\D}\'o\_\'e\_d{\Z}\'i} [b\'e {\bf \`Ak\'u} {\bf l\'a} {\D}\`u n\'{\~u}].\\
Kofi say Ama be.certain {\sc comp} Aku {\sc pot} eat thing\\
\glt`Kofi said Ama is certain that Aku will eat.' \label{bal}  
\z

\ea
\gll K\`of\'i b\'e \'Am\'{\~a}  {\bf k\'{\~a}\_{\D}\'o\_\'e\_d{\Z}\'i} [b\'e {\bf \`Ak\'u} {\bf n{\É}} {\D}\`u n\'{\~u}].\\
Kofi say Ama be.certain {\sc comp} Aku {\sc juss} eat thing\\
\glt `Kofi said Ama is certain that Aku should eat.' \label{ban}  
\z

\ea 
\gll {\bf K\`of\'i$_{1}$} b\'e {\bf \'Am\'{\~a}$_{2}$} {\bf k\'{\~a}\_{\D}\'o\_\'e\_d{\Z}\'i} [b\'e {\bf j\`e$_{1/2}$} {\bf l\'a} {\D}\`u n\'{\~u}].\\
Kofi say Ama be.certain {\sc comp} {\sc log} {\sc pot} eat thing\\
\glt `Kofi said Ama is certain that he/she (= Kofi/Ama) will eat.'  \label{bal2}
\z

\ea
\gll {\bf K\`of\'i$_{1}$} b\'e  \'Am\'{\~a}$_{2}$ {\bf k\'{\~a}\_{\D}\'o\_\'e\_d{\Z}\'i} [b\'e {\bf j\`e$_{1/*2}$} {\bf n{\É}} {\D}\`u n\'{\~u}].\\
Kofi say Ama be.certain {\sc comp} {\sc log} {\sc juss} eat thing\\
\glt `Kofi said Ama is certain that he ( = Kofi) should eat.'  \label{ban2}
\z

\noindent See \cite{gl17} for further discussion and analysis of this class of predicates. In what follows, we focus on the class exemplified by `want'.



\section{Toward an account}

In  (\ref{ws}), we illustrate a run-of-the-mill modal semantics for {\em want}-sentences modeled after Hintikka's (\citeyear{hintikka69}) influential approach to attitude reports, achieved compositionally via the denotation for {\em want} supplied in (\ref{w}). Here, WANT($x$,$w$) denotes the set of worlds compatible with $x$'s desires in $w$. This is no doubt an oversimplified view of the semantics of desire reports for reasons discussed in such works as \cite{heim92}, but it is sufficient for our purposes, where all that is crucial is that {\em want} denotes some kind of relation between individuals and propositions.


\ea 
\denotes{John wants Bill to eat}$^{w}$\\ = $\forall w^{\prime}$[$w^{\prime}$ $\in$ WANT($j$,$w$) $\rightarrow$ EAT($b$) in $w^{\prime}$]\\
$\approx$ `All those worlds compatible with what John wants in $w$ are worlds in which Bill eats.' \label{ws}
\z

\ea 
\denotes{want}$^{w}$ = $\lambda p_{\langle st\rangle}\lambda x.\forall w^{\prime}$[$w^{\prime}$ $\in$ WANT($x$,$w$) $\rightarrow$ $p$($w^{\prime}$)] \label{w}
\z

When we turn to control sentences like (\ref{wcs2}), on the other hand, we observe that the matrix subject {\em John} appears to play two roles semantically, naming both the attitude holder and the individual who eats in those worlds compatible with the attitude holder's desires. In other words, it has a denotation like (\ref{wcs}).\footnote{Something not captured by the denotation in (\ref{wcs}) and that we abstract away from since it is orthogonal to our purposes is that attitude reports expressed by control sentences have an obligatory {\em de se} semantics. See Stephenson 2010; Pearson 2015, 2016 for recent approaches.\nocite{stephenson10, pearson15, pearson16}}

\ea 
John wants to eat. \label{wcs2}
\z

\ea 
\denotes{John wants to eat}$^{w}$\\  = $\forall w^{\prime}$[$w^{\prime}$ $\in$ WANT($j$,$w$) $\rightarrow$ EAT($j$) in $w^{\prime}$]\\
$\approx$ `All those worlds compatible with what John wants in $w$ are worlds in which John eats.' \label{wcs}
\z

Borrowing an insight from \cite{chierchia84, dowty85}, we can achieve this compositionally with a revised semantics for {\em want} as in (\ref{w2}). Here, {\em want} denotes a relation between individuals and properties. When the individual $x$ is plugged in, it values both the attitude holder and the unsaturated argument associated with the property.

\ea
\denotes{want$^{\prime}$}$^{w}$ = $\lambda P_{\langle  e, st\rangle}\lambda x.\forall w^{\prime}$[$w^{\prime}$ $\in$ WANT($x$,$w$) $\rightarrow$ $P$($x$)($w^{\prime}$)] \label{w2}
\z

Turning our attention back to Gengbe, we propose that {\em n{\É}} `{\sc juss}' contributes an individual argument whereas {\em l\'a} `{\sc pot}' does not. As schematized in (\ref{mm}), this means that if the semantic type of an unmarked clause in Genge is proposition-denoting (type $\langle st\rangle$), a jussive-marked clause is property-denoting (type $\langle e,st\rangle$). Potential marking, by contrast, has no type-theoretic effect; a potential-marked clause is proposition-denoting. 


\ea
a. [Kofi eat]$_{\langle st\rangle}$ $\rightarrow$ [$\lambda x$ . Kofi {\sc juss} eat]$_{\langle e,st\rangle}$\\
b. [Kofi eat]$_{\langle st\rangle}$ $\rightarrow$ [Kofi {\sc pot} eat]$_{\langle st\rangle}$ \label{mm}
\z

This is a natural extension of ideas developed by \cite{portner04, portner07, zpp12} that imperative clauses (and jussive clauses more generally) are property-denoting rather than proposition-denoting. Intuitively, the individual argument introduced by the jussive marker can be thought of as the individual who bears the responsibility for bringing about the action named by the clause.  See \citealt{gl17} for further discussion.


%\vspace{-.25cm}

With these proposals in place, the first part of the puzzle is now solved, provided Gengbe {\em d{\Z}\'i} `want' has the property-theoretic denotation in (\ref{w2}). As schematized in (\ref{p1s}), when `want' combines with a potential-marked clause, the result is a type mismatch, because `want' needs to combine with a property-denoting complement and yet its complement is propositional. When `want' combines with a jussive-marked clause, on the other hand, there is no problem, since jussive clauses are property-denoting.


\ea
a. Ama want$_{\langle\langle e,st\rangle,\langle e,st\rangle\rangle}$ [Aku POT eat]$_{\langle st\rangle}$  \hfill $\leftarrow$ *!\\
b. Ama want$_{\langle\langle e,st\rangle,\langle e,st\rangle\rangle}$ [$\lambda x$.Aku JUSS eat]$_{\langle e,st\rangle}$   \hfill $\leftarrow$ {\em ok!} \label{p1s}
\z



In order to solve the second part of the puzzle, we need to say something about the semantic analysis of logophors. Following \cite{heim02, vs02, vs03, pearson15}, we adopt the proposal that what distinguishes logophoric pronouns from ordinary pronouns is that logophors are obligatorily bound by an attitude predicate, thereby creating a derived property for the binding attitude predicate to combine with. As schematized in (\ref{log}), this means that  if a logophor is embedded under two attitude predicates, it can in principle be bound either by the immediately embedding attitude predicate or by the more distant one, but if it is not bound by either, the result is ungrammatical.\footnote{According to an anonymous reviewer, a logophor can be embedded under a non-attitude verb as long as it appears in a clause introduced by the complementizer {\em be}, as in the following example supplied by the reviewer (using Ewe orthography but holding for Gengbe as well, according to the reviewer).


\ea
\gll Kofi yi Lome be ye-a-ƒle av{\O}.\\
Kofi go Lome {\sc comp} {\sc log}-{\sc subj}-buy cloth\\
\glt `Kofi went to Lome to buy cloth.'
\z

We note, however, the following example due to \cite{pearson15} showing that logophors in Ewe cannot be embedded under a causative predicate even in the presence of the complementizer {\em be}. 

\ea
\gll Kofi w{\O} be e/*ye dzo.\\
Kofi do {\sc comp} 3{\sc sg}/{\sc log} leave\\
\glt `Kofi caused himself to leave.' (\citealt{pearson15}:96)
\z

In light of the asymmetry between (i) and (ii), we hypothesize that (i) may contain a silent attitude predicate that licenses the logophor; plausibly it has a meaning like `intend' given that the rationale clause in (i) is paraphrasable as `with the intention of buying cloth'.

}


\ea
a. Kofi say [ Ama {\bf be.certain} [ $\lambda x$. [ {\bf LOG}$_{x}$ eat ] ] ] \hfill $\leftarrow$ {\em ok!}\\
b. Kofi {\bf say} [ $\lambda x$. [ Ama be.certain  [ {\bf LOG}$_{x}$ eat ] ] ] \hfill $\leftarrow$ {\em ok!} \\
c. Kofi say  [ Ama be.certain  [ LOG$_{x}$ eat ] ] \hfill $\leftarrow$ {\em *!} \label{log}
\z

This approach to logophors, together with the other proposals already introduced, are sufficient for solving the second part of the puzzle. To see this, consider (\ref{local})--(\ref{remote}). In (\ref{local}), where the complement to `want' has potential marking, local binding of the logophor is licit since this will yield the needed property-theoretic denotation for the complement to `want', but remote binding is ruled out since it results in a proposition-denoting complement to `want'. In (\ref{remote}) with jussive marking, by contrast, the opposite obtains: local binding of the logophor results in a type $\langle e,\langle e,st\rangle\rangle$ denotation for the complement to `want', yielding a type mismatch, whereas remote binding of the logophor preserves the property-denoting status of the complement clause and is hence licit. 


\ea 
a. Kofi  say Ama [want$_{\langle\langle e,st\rangle,\langle e,st\rangle\rangle}$ [$\lambda x$. LOG$_{x}$ POT eat]$_{\langle e,st\rangle}$ ] \hfill $\leftarrow$ {\em ok!}\\
b. Kofi  say [$\lambda x$.Ama want$_{\langle\langle e,st\rangle,\langle e,st\rangle\rangle}$ [LOG$_{x}$ POT eat]$_{\langle st\rangle}$ ] \hfill $\leftarrow$ {\em *!} \label{local}
\z

\ea 
a. Kofi  say Ama [want$_{\langle\langle e,st\rangle,\langle e,st\rangle\rangle}$ [$\lambda x\lambda y$. LOG$_{x}$ JUSS eat]$_{\langle e,\langle e,st\rangle\rangle}$ ] \hfill $\leftarrow$ {\em *!}\\
b. Kofi  say [$\lambda x$.Ama want$_{\langle\langle e,st\rangle,\langle e,st\rangle\rangle}$ [$\lambda y$. LOG$_{x}$ JUSS eat]$_{\langle e,st\rangle}$ ] \hfill $\leftarrow$ {\em ok!} \label{remote}
\z


Finally, consider what happens when the desire reports in (\ref{local})--(\ref{remote}) are unembedded. When potential marking is used as in (\ref{local'}), local binding of the logophor leads to type-theoretic well-formedness. The only other option is to leave the logophor unbound, which leads both to a type mismatch as well as to a violation of the constraint that logophors must be bound. When jussive marking is used as in (\ref{remote'}), on the other hand, local binding of the logophor leads to a type mismatch and non-binding of the logophor violates the constraint that logophors must be bound. Thus we accurately predict ungrammaticality for such cases.\footnote{An anonymous reviewer asks what happens when a non-logophoric pronoun is used in place of a logophoric pronoun in configurations in cases like (\ref{local'})--(\ref{remote'}). Unfortunately, we do not have data on this, but we can say a few words about what our theory predicts. Insofar as non-logophoric pronouns are distinct from logophoric pronouns in that they are {\em optionally} (as opposed to obligatorily) bound by attitude predicates, the prediction of our theory is that they should be grammatical both with potential marking (since they can be bound) and with jussive marking (since they need not be bound). That being said, Pearson (\citeyear{pearson15}:97) reports variation among Ewe speakers in whether they accept bound construals of non-logophoric pronouns. Possibly, rejection of a bound construal could be due to a kind of pragmatic blocking effect whereby the failure to use a logophor biases an interpreter toward a non-bound interpretation. If logophors force a {\em de se} construal then such a blocking effect would be nullified by setting up a non-{\em de se} context so that the logophor would not be a viable alternative; however, a central claim of \cite{pearson15} is that logophors (at least in Ewe) need not be construed {\em de se}, {\em contra} the received wisdom.}



\ea 
a. Ama [want$_{\langle\langle e,st\rangle,\langle e,st\rangle\rangle}$ [$\lambda x$. LOG$_{x}$ POT eat]$_{\langle e,st\rangle}$ ] \hfill $\leftarrow$ {\em ok!}\\
b. Ama want$_{\langle\langle e,st\rangle,\langle e,st\rangle\rangle}$ [LOG$_{x}$ POT eat]$_{\langle st\rangle}$  \hfill $\leftarrow$ {\em *!} \label{local'}
\z

\ea 
a. Ama [want$_{\langle\langle e,st\rangle,\langle e,st\rangle\rangle}$ [$\lambda x\lambda y$. LOG$_{x}$ JUSS eat]$_{\langle e,\langle e,st\rangle\rangle}$ ] \hfill $\leftarrow$ {\em *!}\\
b. Ama want$_{\langle\langle e,st\rangle,\langle e,st\rangle\rangle}$ [$\lambda y$. LOG$_{x}$ JUSS eat]$_{\langle e,st\rangle}$  \hfill $\leftarrow$ {\em *!} \label{remote'}
\z




In a nutshell, the observed interaction between mood choice and logophoric antecedent choice falls out automatically from familiar, previously proposed ideas about the type-theoretic effects of mood markers, logophors, and embedding verbs.\footnote{Another consequence of this approach is that `be certain' and other predicates that pattern like it  need to be type-theoretically flexible in being able to take either a property or a proposition as their first argument. See \citealt{gl17} for illustration and further discussion.}




\section{Conclusions}

Our central conclusions are twofold. First, on the empirical end, we have argued that Gengbe exhibits  control of logophors under particular syntactic conditions in a way that depends on the choice of the embedding verb and on the mood marking in the clause containing the logophor. On the theoretical end, we have shown how to account for the relevant facts in a system whereby independently acting parts (verbs, logophors, and mood markers)  interact with each other to give rise to control in certain combinations for principled type-theoretic reasons. As already alluded to in the introduction, one consequence of this system is a view of control as an `emergent' phenomenon: no single element in the structure of the sentence is responsible for control, it is only their interaction that ends up mattering.

If the type-theoretic principles that form the backbone of our proposal have wide cross-linguistic currency, why is control of logophors seemingly so rare?  We suggest that this is because it depends on the convergence of two features that vary independently. First, not all languages have logophors to begin with, so for obvious reasons, only those languages that have logophors have the potential for control of logophors. Second, in many languages, verbs like `want' that typically take control complements also typically tend to take structurally impoverished complements that preclude an overt subject. Gengbe, on the other hand, has both logophors and the syntax to support a logophor under a verb like `want', thereby giving rise to the right conditions for the phenomenon to emerge.


\section*{Abbreviations}


{\sc abs} = absolutive, {\sc comp} = complementizer, {\sc dat} = dative, {\sc erg} = ergative, {\sc evid}  = evidential, {\sc hab} = habitual, {\sc inf} = infinitive, {\sc irr} = irrealis,  {\sc juss} = jussive, {\sc loc} = locative, {\sc log} = logophor, {\sc nom} = nominative, {\sc pot} = potential, {\sc pst} = past, {\sc vol} = volitional, 1/3{\sc sg} = 1st/3rd-person singular
 
\section*{Acknowledgements}

Thanks first and foremost go to our Gengbe linguistic consultant Gabriel Mawusi. We also thank Samuel Obeng for supporting this research, and we thank audiences at the Indiana University Semantics Reading Group (August 2016), {\em Sinn und Bedeutung} 21 in Edinburgh (September 2016), and ACAL 48 at Indiana University (March 2017) for their feedback on the work presented here. Finally, thanks are due to two anonymous reviewers for helpful suggestions on an earlier version of this paper.

\printbibliography[heading=subbibliography,notkeyword=this]

\end{document}


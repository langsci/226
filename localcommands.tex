% Font size in Paper Footer
\renewcommand{\lsChapterFooterSize}{\scriptsize}



\newcommand{\mc}[1]{\textsc{#1}}	% morpheme glossing as small caps: does not clash with fontspec
\def\Id#1{\vskip-.cm\hskip.01cm\vtop{#1}} % for controlling where the example breaks onto the next line


\setlength{\emergencystretch}{2pt}
\newcolumntype{s}{>{\hsize=.5\hsize}X}
\newcolumntype{m}{>{\hsize=.25\hsize}X}

\newcommand*{\glossify}[1]{\textsc{\MakeLowercase{#1}}}

%makes subscripts
\newcommand{\sub}[1]{\textsubscript{#1}}
\newcommand{\tikzmark}[1]{\tikz[overlay,remember picture]\node(#1){};}

%draws normal arrow
\newcommand*{\DrawArrow}[3][]{%
	\begin{tikzpicture}[overlay,remember picture, >=latex']
	\draw[rounded corners, -latex,<-,semithick,#1]  (#2) -- ++(0,-1.2em) coordinate (a) -- 
	($(a-|#3)$) -- (#3);
	\end{tikzpicture}%
}

%draws dashed arrow
\newcommand*{\DrawXArrow}[3][]{%
    % #1 = draw options
    % #2 = left point
    % #3 = right point
    \begin{tikzpicture}[overlay,remember picture,>=latex']%
       % \draw [-latex, #1] ($(#2)+(0.1em,0.5ex)$) to ($(#3)+(0,0.5ex)$);
		\draw[rounded corners, -latex,<-,semithick,#1]  (#2) -- ++(0,-1.2em)coordinate (a) -- 
		node[fill=white,inner sep=0pt]%[font=\footnotesize]
		{\ding{56}}
		($(a-|#3)$) -- (#3);
    \end{tikzpicture}%
}% 

\newcommand*{\DrawArrowok}[4][]{%
	% #1 = draw options
	% #2 = left point
	% #3 = right point
	\begin{tikzpicture}[overlay,remember picture, >=latex']
	% \draw [-latex, #1] ($(#2)+(0.1em,0.5ex)$) to ($(#3)+(0,0.5ex)$);
	\draw[rounded corners, -latex,<-,semithick,#1]  (#2) -- ++(0,-1.2em)coordinate (a) -- 
	%node[below,font=\footnotesize]{#4} 
	node[midway,fill=white,inner sep=0pt,font=\normalsize]{#4}
	($(a-|#3)$) -- (#3);
	\end{tikzpicture}%
}% 

\newcommand*{\DrawLXArrow}[3][]{%
	% #1 = draw options
	% #2 = left point
	% #3 = right point
	\begin{tikzpicture}[overlay,remember picture,>=latex']
	% \draw [-latex, #1] ($(#2)+(0.1em,0.5ex)$) to ($(#3)+(0,0.5ex)$);
	\draw[rounded corners, -latex,<-,semithick,#1]  (#2) -- ++(0,-1.8em)coordinate (a) -- 
	node[fill=white,inner sep=0pt]%[font=\footnotesize]
	{\ding{56}}
	($(a-|#3)$) -- (#3);
	\end{tikzpicture}%
}% 
\usetikzlibrary{arrows,shapes,positioning,shadows,trees,calc}
\usetikzlibrary{decorations.text}

 
\newcommand{\all}[1]{\ensuremath{\forall #1}}

\newcommand{\bari}{\sout{\i}} 

\let\oldemptyset\emptyset
\let\emptyset\varnothing
 


\usetikzlibrary{calc,positioning,tikzmark}

%hyman
\newcommand{\higr}[1]{{\color{red}#1}}
 
 
\definecolor{lsDOIGray}{cmyk}{0,0,0,0.45}

%newkirk
\newcommand{\ix}[1]{\ensuremath{_{#1}}}   % ix =index; in text-semantics mode, allows upright subscript indices with non-math mode text 
\newenvironment{context}{\medskip\par\noindent%
			{\bfseries\sffamily Context: }% 
			}%open environment
			{\par}%close environment
\newcommand{\den}[1]{\ensuremath{\llbracket}\,{#1}\,\ensuremath{\rrbracket}}
\newcommand{\type}[1]{\ensuremath{\langle{#1}\rangle}} % Type brackets for type-theory 	 ... \type{e,\type{s,t}}
\newcommand{\denol}[1]{\den{\textit{#1}}} % Denotation with object language
\newcommand{\set}[1]{\ensuremath{\{~ #1 ~\}}} % Encloses a set (easier to see in source)

\def\denotes#1{$\lbrack\!\lbrack${#1\/}$\rbrack\!\rbrack$}      % denotes
\newcommand{\citeNP}{\citealt}


\makeatletter
\let\thetitle\@title
\let\theauthor\@author 
\makeatother


\newcommand{\togglepaper}[1][0]{ 
  \bibliography{../localbibliography}
  \papernote{\scriptsize\normalfont
    \theauthor.
    \thetitle. 
    To appear in: 
    Samson Lotven,   Silvina Bongiovanni,   Phillip Weirich,   Robert Botne \&  Samuel Gyasi Obeng (eds.),  
    African linguistics across the disciplines: Selected papers from the 48th Annual Conference on African Linguistics 
    Berlin: Language Science Press. [preliminary page numbering]
  }
  \pagenumbering{roman}
  \setcounter{chapter}{#1}
  \addtocounter{chapter}{-1}
}
\newcommand{\Z}{ʒ}
\newcommand{\D}{ɖ}
\newcommand{\ny}{ɲ}
\newcommand{\R}{ɾ}

\newcommand{\á}{\'{ã}}
\newcommand{\É}{\'{\~{ε}}}
\newcommand{\È}{\`{\~{ε}}}
\newcommand{\í}{\'{\~{i}}}
\newcommand{\ì}{\`{\~{i}}}
\renewcommand{\O}{ɔ}
\newcommand{\Ó}{\'{\~{ɔ}}}
\newcommand{\Ò}{\`{\~{ɔ}}}
\newcommand{\ú}{\'{ũ}}
\newcommand{\ù}{\`{ũ}}
% % % \newcommand{\nolistbreak}{%
% % %   \let\oldpar\par\def\par{\oldpar\nobreak}% Any \par issues a \nobreak
% % %   \@nobreaktrue% Don't break with first \item
% % % }
\newcommand{\NUM}{{\textsc{num}}}
\makeatletter
\makeatletter

\documentclass[output=paper,modfonts,nonflat]{langsci/langscibook} 
\title{Universal quantification in the nominal domain in Kihehe}
\author{Kelly Kasper-Cushman\affiliation{Indiana University}}
\abstract{This study provides a description of the universal quantifiers \textit{mbe-\textsc{agr2}-li} `all' and \textit{kila} `every' in Kihehe, a Bantu language spoken in south-central Tanzania. Following a description of general properties of these quantifiers, this study analyzes how the Kihehe data bear on the phenomena of collectivity and distributivity, and situates the Kihehe data with regard to recent crosslinguistic work on the typology of quantifiers \cite{matthewson13}.}

\IfFileExists{../localcommands.tex}{%hack to check whether this is being compiled as part of a collection or standalone
  % add all extra packages you need to load to this file  
\usepackage{tabularx} 

%%%%%%%%%%%%%%%%%%%%%%%%%%%%%%%%%%%%%%%%%%%%%%%%%%%%
%%%                                              %%%
%%%           Examples                           %%%
%%%                                              %%%
%%%%%%%%%%%%%%%%%%%%%%%%%%%%%%%%%%%%%%%%%%%%%%%%%%%% 
%% to add additional information to the right of examples, uncomment the following line
% \usepackage{jambox}
%% if you want the source line of examples to be in italics, uncomment the following line
% \renewcommand{\exfont}{\itshape}
% \usepackage{lipsum}
% \usepackage[normalem]{ulem}
\usepackage{tikz}
\usepackage{tikz-qtree}
\usepackage{tikz-qtree-compat}
\usepackage{tabularx}
\usepackage{verbatim}
\usepackage{pifont}    %for pointing hand, checkmarks, crosses
\usepackage{tipa}
\usepackage{amssymb}
\usepackage{amsmath}
\usepackage{subcaption}
\usepackage{csquotes}
\usepackage{multirow}
\usepackage{multicol}
\usepackage{outlines} 
%\usepackage{wrapfig}
\usepackage{enumitem}
\usepackage{rotating}  
%\usepackage{ling-macros-custom}
\usepackage{stmaryrd}
%\usepackage{qtree}
\usepackage{./langsci/styles/langsci-optional}
\usepackage{./langsci/styles/langsci-lgr}
\usepackage{hhline}
\usepackage{langsci-gb4e}
\usepackage[linguistics]{forest}
 



   
\newcommand{\mc}[1]{\textsc{#1}}	% morpheme glossing as small caps: does not clash with fontspec
\def\Id#1{\vskip-.cm\hskip.01cm\vtop{#1}} % for controlling where the example breaks onto the next line


\setlength{\emergencystretch}{2pt}
\newcolumntype{s}{>{\hsize=.5\hsize}X}
\newcolumntype{m}{>{\hsize=.25\hsize}X}


 

%makes subscripts
\newcommand{\sub}[1]{\textsubscript{#1}}
\newcommand{\tikzmark}[1]{\tikz[overlay,remember picture]\node(#1){};}

%draws normal arrow
\newcommand*{\DrawArrow}[3][]{%
	\begin{tikzpicture}[overlay,remember picture, >=latex']
	\draw[rounded corners, -latex,<-,semithick,#1]  (#2) -- ++(0,-1.2em)coordinate (a) -- 
	($(a-|#3)$) -- (#3);
	\end{tikzpicture}%
}

%draws dashed arrow
\newcommand*{\DrawXArrow}[3][]{%
    % #1 = draw options
    % #2 = left point
    % #3 = right point
    \begin{tikzpicture}[overlay,remember picture,>=latex']
       % \draw [-latex, #1] ($(#2)+(0.1em,0.5ex)$) to ($(#3)+(0,0.5ex)$);
		\draw[rounded corners, -latex,<-,semithick,#1]  (#2) -- ++(0,-1.2em)coordinate (a) -- 
		node%[font=\footnotesize]
		{\ding{56}}
		($(a-|#3)$) -- (#3);
    \end{tikzpicture}%
}% 

\newcommand*{\DrawArrowok}[4][]{%
	% #1 = draw options
	% #2 = left point
	% #3 = right point
	\begin{tikzpicture}[overlay,remember picture, >=latex']
	% \draw [-latex, #1] ($(#2)+(0.1em,0.5ex)$) to ($(#3)+(0,0.5ex)$);
	\draw[rounded corners, -latex,<-,semithick,#1]  (#2) -- ++(0,-1.2em)coordinate (a) -- 
	%node[below,font=\footnotesize]{#4} 
	node[midway,fill=white,font=\normalsize]{#4}
	($(a-|#3)$) -- (#3);
	\end{tikzpicture}%
}% 

\newcommand*{\DrawLXArrow}[3][]{%
	% #1 = draw options
	% #2 = left point
	% #3 = right point
	\begin{tikzpicture}[overlay,remember picture,>=latex']
	% \draw [-latex, #1] ($(#2)+(0.1em,0.5ex)$) to ($(#3)+(0,0.5ex)$);
	\draw[rounded corners, -latex,<-,semithick,#1]  (#2) -- ++(0,-1.8em)coordinate (a) -- 
	node%[font=\footnotesize]
	{\ding{56}}
	($(a-|#3)$) -- (#3);
	\end{tikzpicture}%
}% 
\usetikzlibrary{arrows,shapes,positioning,shadows,trees,calc}
\usetikzlibrary{decorations.text}

 
\newcommand{\all}[1]{\ensuremath{\forall #1}}

\newcommand{\bari}{\ipabar{\i}{.5ex}{1.1}{}{}} 

\let\oldemptyset\emptyset
\let\emptyset\varnothing
 


\usetikzlibrary{calc,positioning,tikzmark}

%hyman
\newcommand{\higr}[1]{{\color{red}#1}}
 
 
\definecolor{lsDOIGray}{cmyk}{0,0,0,0.45}

%newkirk
\newcommand{\ix}[1]{{\color{red}\textsubscript{#1}}} %probably i.nde.x
\newcommand{\ol}[1]{{\color{red}\textit{#1}}} %probably o.bject l.anguage
\newcommand{\alert}[1]{{\color{red}\textbf{#1}}}
\newenvironment{context}{\color{red}
\smallskip 

\scriptsize 
}{\color{black}\normalsize }
\newcommand{\m}[1]{{\color{red}\textsc{#1}}}
\newcommand{\den}[1]{{\color{red}#1}}
\newcommand{\type}[1]{{\color{red}#1}}
\newcommand{\denol}[1]{{\color{red}#1}}
\newcommand{\bex}{\ea}
\newcommand{\fex}{\z}
\newcommand{\set}{{\color{red} SET}}
 
\def\denotes#1{$\lbrack\!\lbrack${#1\/}$\rbrack\!\rbrack$}      % denotes
\newcommand{\citeNP}{\citealt}


\makeatletter
\let\thetitle\@title
\let\theauthor\@author 
\makeatother


\newcommand{\togglepaper}[1][0]{ 
  \bibliography{../localbibliography}
  \papernote{\scriptsize\normalfont
    \theauthor.
    \thetitle. 
    To appear in: 
    Samson Lotven,   Silvina Bongiovanni,   Phillip Weirich,   Robert Botne \&  Samuel Gyasi Obeng (eds.),  
    African linguistics across the disciplines: Selected papers from the 48th Annual Conference on African Linguistics 
    Berlin: Language Science Press. [preliminary page numbering]
  }
  \pagenumbering{roman}
  \setcounter{chapter}{#1}
  \addtocounter{chapter}{-1}
}

 
  \bibliography{../localbibliography}
  \togglepaper[13]
}{}




\begin{document}

\maketitle

\section{Introduction}

This paper presents a description of the universal quantifiers \textit{mbe-\textsc{agr2}-li} `all' and \textit{kila} `every' in Kihehe (G.62; \citealt{maho09}), a Bantu language spoken in south-central Tanzania around the town of Iringa.
Regarding quantification in Bantu, \citealt[383]{zerbian08} note that ``few studies exist which touch upon quantification [in (whatever) Bantu languages]'' (but see \citealt{landman16, landmanip} for two recent works). Additionally, this documentation contributes to an understanding of quantifiers crosslinguistically, and to the development of a typology of quantifiers, as called for by \cite{matthewson13} (for recent studies on quantification in natural languages, see the works in \citealt{gil13, keenan12, matthewson08} and \citealt{paperno17}).

Sections~\ref{sec:kasper:2} and~\ref{sec:kasper:3} describe general properties of \textit{mbe-\textsc{agr2}-li} `all' and \textit{kila} `every,' respectively. Section 4 considers how the data in Kihehe bear on the properties of collectivity and distributivity associated with the universal quantifiers (see, e.g., \citet{szabolcsi10}). Finally, Section 5 situates the Kihehe data with regard to three crosslinguistic typological generalizations of quantifiers put forth in \citealt{matthewson13}. 


\section{\emph{Mbe-\textsc{agr2}-li} `all'}\label{sec:kasper:2}

This section provides an overview of the universal quantifier \emph{mbe-\textsc{agr2}-li} `all.'\footnote{The meaning of the individual components \textit{mbé-} and \textit{-li} is not clear; these components always appear together with an infixed agreement marker in this construction.} It starts wtih a discussion of general agreement patterns before providing a description of coordination and the partitive construction. 

\subsection{General properties}

 The quantifier \emph{mbe-\textsc{agr2}-li} follows the noun it modifies and agrees with it in noun class, as shown in (\ref{ex:kaspercushman:allpeople}) in which it modifies the plural Class 2 noun \textit{vanu} `people.' The agreement marker, labeled \textsc{agr2}, is an infix.  

\begin{exe}
\ex \label{ex:kaspercushman:allpeople}
\gll vá-nu mbé-va-li \\
2-person all-2-all \\
\glt `all people'/`all of the people'\footnotemark \\

\end{exe}
\footnotetext{Tone is marked on examples where transcription is reliable. Data are transcribed using orthographic conventions based on Kiswahili. Vowel length is not typically marked. Syllabic nasals are marked.}

The morphological form of the agreement marker for this quantifier patterns with that for demonstratives, numbers, and possessives, as shown in (\ref{ex:kaspercushman:alltrees}).  This is unlike the morphological form of the agreement marking on adjectives, which is identical in form to the noun class marker, as in (\ref{ex:kaspercushman:manytrees}) (this latter pattern includes the quantificational adjectives \textit{-olofu} `many/much,' \textit{-ongefu} `many/much,' and \textit{-kefu} `few'; however, \textit{-ngi} `other' patterns with demonstratives). When other modifiers are present in the NP, as in (\ref{ex:kaspercushman:alltrees}),   \textit{mbe-\textsc{agr2}-li} appears last.


\begin{exe}

\ex\begin{xlist}

\ex 
\gll i-mi-biki gi-tayi mbe-ge-li\\ 
\textsc{ppf}-4-tree 4-four all-4-all\\
\glt `all four trees' \\ \label{ex:kaspercushman:alltrees}

\ex 
\gll i-mi-biki my-ongefu\\
\textsc{ppf}-4-tree 4-many \\
\glt `many trees' \label{ex:kaspercushman:manytrees}

\end{xlist}

\end{exe}



A singular noun can be modified by \textit{mbe-\textsc{agr2}-li}. Example (\ref{wholebody}) shows a noun from Class 3; here, \textit{all + N} is translated as `the whole N.'

\begin{exe} 

\ex 
\gll \s{m}-víli mbé-gu-li  \label{wholebody} \\   
3-body all-3-all \\
\glt `a/the whole body' \\  

\end{exe}


\textit{Mbe-\textsc{agr2}-li} can take locative noun class agreement. With the Class 16 agreement (\textit{pa-}), for example, the quantifier translates as `all of the time.'

\begin{exe}
\ex 
\gll mbé-pa-li \\
all-16-all \\
\glt `all of the time' \\
\end{exe}


The quantifier can agree with a plural personal subject (see also \citealt{jerro13}).

\begin{exe}   

\ex 
\gll tu-bít-e mbé-tu-li \label{subjectagreement} \\
\textsc{1pl}-go-\textsc{imp} all-\textsc{1pl}-all \\  
\glt `Let's all go!'

\end{exe}


There is no separate lexical item to express \textit{ex:kaspercushman:both}; \textit{mbe-\textsc{agr2}-li} is used in this context.  If disambiguation is necessary, the number \textit{-vili} `two' is inserted, as in (\ref{ex:kaspercushman:both}).

\begin{exe} 

\ex 
\gll a-v-ana va-víl\^i mbe-va-li va-támw-a \\ 
\textsc{ppf}-2-child 2-two all-2-all 2-be\_sick-\textsc{fv} \\ \label{ex:kaspercushman:both}
\glt `Both of the children are sick.' \\

\end{exe}


In citation form, the head noun modified by \textit{mbe-\textsc{agr2}-li} can optionally occur with the pre-prefix.  However, when the head noun is an argument, the head noun must appear with the pre-prefix; compare example (\ref{ex:kaspercushman:allsick}) with example (\ref{ex:kaspercushman:allpeople}); see also the discussion in Section \ref{ex:kaspercushman:type} and \citealt{gambarage16}. 
 
\begin{exe}
\ex 
\gll a-v-ána mbé-va-li va-támw-a. \\
\textsc{ppf}-2-child all-2-all 2-be\_sick-\textsc{fv} \\
\glt `All of the children are sick.' \\ \label{ex:kaspercushman:allsick}
\end{exe}

Note that the pre-prefix is required even if a demonstrative is present as in (\ref{ex:kaspercushman:demonstrative}) (see also the discussion in \citealt[32--33]{matthewson13}). 
\begin{exe}

\ex \label{ex:kaspercushman:demonstrative}
\gll A-va-na i-v-o mbe-va-li va-tamw-a \\
\textsc{ppf}-2-child \textsc{prox.dem}-2-\textsc{add} all-2-all 2-be\_sick-\textsc{fv} \\
\glt `All of those children are sick.' \\
\end{exe}


\textit{Mbe-\textsc{agr2}-li} can stand alone as a pronoun, as shown in (\ref{ex:kaspercushman:allofthemdied}--\ref{ex:kaspercushman:allarechildren}).  Example (\ref{ex:kaspercushman:allarechildren}) shows a copular structure with a predicate noun. 

\begin{exe}
\ex \label{ex:kaspercushman:allofthemdied} 
\gll mbé-se-li sí-fw-e \\ 
all-10-all 10-die-\textsc{pst}.\textsc{fv} \\
\glt `All (of them) died.' 

\end{exe}

\begin{exe}
\ex \label{ex:kaspercushman:allarechildren} 
\gll mbé-va-li v-ana \\
all-2-all 2-child \\
\glt `All are children.'
\end{exe}


Finally, an example of the universal quantifier in a text is given in (\ref{ex:kaspercushman:luwuko}). The example is from \citet{ex:kaspercushman:luwuko}\todo{Please check this citation} (9:25), the Kihehe translation of the Book of Exodus.

\begin{exe} 
\ex\label{ex:kaspercushman:luwuko}
\gll na i-ndonya i-nya-ma-ganga y-a-denyanz-ile \\ 
and \textsc{ppf}-9.rain \textsc{ppf}-of-6-stone 9-\textsc{p3}-strike\_down-\textsc{pst}\\


\gll \textbf{i-fi-melo} \textbf{mbe-fi-li} \textbf{i-fy-a} \textbf{mu}=\textbf{mi-gunda}, \\
\textsc{ppf}-8-seedling all-8-all \textsc{ppf}-8-\textsc{assoc} \textsc{loc}=4-field,\\


\gll y-a-nanz-ile \textbf{i-mi-biki} \textbf{mbe-gi-li} \textbf{i-j-a}  \\
9-\textsc{p3}-destroy-\textsc{pst} \textsc{ppf}-4-tree all-4-all \textsc{ppf}-4-\textsc{assoc}  \\


\gll \textbf{mu}=\textbf{mi-gunda} \\ 
\textsc{loc}=4-field \\

\glt `Then the hail struck down \textbf{all of the seedlings in the fields}, and destroyed \textbf{all of the crops in the fields}.' \hfill{(Luwuko 9:25)} \\
\end{exe}

\subsection{Coordination with \emph{mbe-\textsc{agr2}-li}}

In a coordinate structure such as \textit{all of the N$_{1}$ and N$_{2}$}, \emph{mbe-\textsc{agr2}-li}  can modify either $N_{1}$ or both $N_{1}$ and $N_{2}$.  \emph{Mbe-\textsc{agr2}-li} can be placed after either the first noun or the second noun in the coordinate structure and have the interpretation `all $N_{1}$ and all $N_{2}$,' although the most natural reading will be that of \emph{mbe-\textsc{agr2}-li} modifying the closest N (compare (\ref{ex:kaspercushman:allboysgirls}) with (\ref{ex:kaspercushman:allgirls})). To disambiguate, \emph{mbe-\textsc{ag2}-li} can be placed after both nouns, as in (\ref{ex:kaspercushman:allboysallgirls}).

\begin{exe} 

\ex \begin{xlist} \label{ex:kaspercushman:coordination}

\ex Structure: \emph{all N$_{1}$ and N$_{2}$}\\
\gll a-va-kwámisi mbé-va-li n(a)	a-vá-hí\`inza \label{ex:kaspercushman:allboysgirls}\\  
\textsc{ppf}-2-boy all-2-all and \textsc{ppf}-2-girl \\
\glt Most natural interpretation: `all of the boys and some (not all) of the girls' \\
    Possible interpretation: `all of the boys and all of the girls' \\
\ex  Structure: \emph{N$_{1}$ and all N$_{2}$}\\
\gll a-va-kwamisi n(a) a-va-hí\`inza mbe-va-li \label{ex:kaspercushman:allgirls}  \\  
\textsc{ppf}-2-boy and \textsc{ppf}-2-girl all-2-all\\
\glt Most natural interpretation: `Some (not all) of the boys and all of the girls' \\
Possible interpretation: `all of the boys and all of the girls' \\
\ex  Structure: \emph{all N$_{1}$ and all N$_{2}$} \\
\gll a-va-kwamisi mbe-va-li n(a) a-va-hí\`inza mbe-va-li  \label{ex:kaspercushman:allboysallgirls} \\  
\textsc{ppf}-2-boy all-2-all  and \textsc{ppf}-2-girl all-2-all\\
\glt `all of the boys and all of the girls' \\
\end{xlist}
\end{exe} 

In the examples in (\ref{ex:kaspercushman:coordination}), both nouns in the coordinate structure belong to the same noun class. Nouns from different noun classes can be coordinated using the structure \textit{all of the $N_{1}$ and $N_{2}$}, but the most natural interpretation of that structure is `all of the $N_{1}$ and some of the $N_{2}$.'  When the intended meaning is `all of the $N_{1}$ and all of the $N_{2}$,' it is more felicitous to modify both nouns with \textit{all}.  

\begin{exe}
\ex \begin{xlist}
\ex 
\gll a-va-nu mbe-va-li n(a) i-senga \\
\textsc{ppf}-2-person all-2-all and \textsc{ppf}-10.cow \\
\glt Most natural interpretation: `all of the people and some (not all) of the cows' \\
Possible interpretation: `all of the people and all of the cows' \\
\ex 
\gll a-va-nu mbe-va-li n(a) i-senga mbe-se-li \\
\textsc{ppf}-2-person all-2-all and \textsc{ppf}-10.cow all-10-all \\
\glt `all of the people and all of the cows' 
\end{xlist}
\end{exe}

\subsection{Partitive constructions with \emph{mbe-\textsc{agr2}-li}}

In English, there is a surface difference between quantified nouns in partitive constructions (e.g., \textit{all of the N}) and non-partitive constructions (\textit{all N}); for example, \textit{I went to a party last night and talked to *all linguists/all of the linguists} (example adapted from an example in Matthewson, 2001, p. 170; see Matthewson, 2001, and Matthewson, 2013, for a discussion of the debate regarding the partitive construction and the semantic denotation of quantifiers). In Kihehe, there is no overt morphological element corresponding to \textit{of} in English partitive constructions. Additionally, all nouns modified by \emph{mbe-\textsc{agr2}-li} must first combine with the pre-prefix.  Thus, we find a lack of surface contrast between contexts in which the quantified noun is contextually restricted versus non-contextually restricted, for both count nouns, as in (\ref{ex:kaspercushman:eatgrass}), and mass nouns, as in (\ref{ex:kaspercushman:drinktea}).

\begin{exe} 

\ex \label{ex:kaspercushman:partitive} \begin{xlist} \label{ex:kaspercushman:eatgrass}

\ex Generic\\
\gll \textbf{i-senga} \textbf{mbe-se-li} s-ay-i-ly-(a) a-ma-soli \\
\textsc{ppf}-10.cow all-10-all 10-\textsc{hab}-\textsc{prs}-eat-\textsc{fv} \textsc{ppf}-6-grass \\
\glt `All cows do eat grass.'

\ex  Contextually restricted\\
\gll ke=nyele ku=\s{m}-nada n-gus-is(e) \textbf{i-senga} \textbf{mbe-se-li} \\
\textsc{aux}=\textsc{1sg}.go.\textsc{pst} \textsc{loc}=3-auction \textsc{1sg}-sell-\textsc{pst} \textsc{ppf}-10.cow all-10-all \\
\glt `I went to the auction and sold all of the cows.'
\end{xlist}
\end{exe}

\begin{exe}

\ex \begin{xlist} \label{ex:kaspercushman:drinktea}
\ex Generic\\
\gll nd-i-wend-(a) \textbf{i-fy-ayi} \textbf{mbe-fe-li} \\
\textsc{1sg}-\textsc{prs}-like-\textsc{fv} \textsc{ppf}-8-tea all-8-all \\
\glt `I like all tea.'

\ex Contextually restricted (Context: I made two pots of tea a while ago.)\\
\gll \textbf{i-fy-ayi} \textbf{mbe-fe-li} fy-e=nelike fi-$\emptyset$-pos-ile \\
\textsc{ppf}-8-tea all-8-all 8-\textsc{rel}=\textsc{1sg}.cook.\textsc{pst} 8-\textsc{p1}-be\_cool-\textsc{pst} \\
\glt `All of the tea which I made is cold.' 

\end{xlist}
\end{exe}



\section{\emph{Kila} `every'} \label{sec:kaspercushman:every}\label{sec:kasper:3}
This section presents a general overveiw of the universal distributive quantifier, \textit{kila}, `every.' \textit{Kila} is a borrowing from Kiswahili (which borrowed it from Arabic; \citealt{zerbian08}). It is unknown how recently \textit{kila} was borrowed into Kihehe, but evidence suggests that \textit{kila} is a recent borrowing. It does not appear in \citegen{spiss1900} Kihehe-German dictionary, and it is not used in the \citeauthor{ex:kaspercushman:luwuko} text. However, \textit{kila} is used commonly by Kihehe speakers, except for the oldest generation.

\subsection{General properties}

Similar to what \citealt{zerbian08} note for Kiswahili, in Kihehe \textit{kila} must appear before the noun it modifies. It does not show agreement in noun class, and the noun modified by \textit{kila} may not occur with the pre-prefix.\footnote{Steve
    Franks and an anonymous reviewer both point out that \textit{kila} and the pre-prefix are in complementary distribution.  This suggests that a quantified noun with \textit{kila} is a Q + NP structure, while quantification with \textit{mbe-\textsc{agr2}-li} is a Q + DP structure. This is discussed with regard to crosslinguistic generalizations in \sectref{sec:kaspercushman:5}. \label{foot:kaspercushman:1}
}

\begin{exe} 
\ex \begin{xlist}
\ex[]{
\gll kila mw-ana \\ 
every 1-child \\
\glt `every child'\label{ex:kaspercushman:everychild} }

\ex[*]{ 
\gll mw-ana kila \\
  1-child every \\
\glt `every child' }

\ex[*]{ 
\gll  kila u-mw-ana \label{ex:kaspercushman:everyppfn}\\
every \textsc{ppf}-1-child \\
\glt `every child'}
\end{xlist}
\end{exe} 

\textit{Kila} can modify both singular nouns, as in (\ref{ex:kaspercushman:everychild}), and plural nouns (\ref{ex:kaspercushman:everygroup}). In (\ref{ex:kaspercushman:everygroup}), \textit{kila N} is interpreted as `every group of N' (\ref{ex:kaspercushman:everygroup}).   


\begin{exe}

\ex 
\gll \textbf{kila} \textbf{va-nu} v-ay-i-pig-ag-a i-kasi kwa u-\s{m}-twa \\
every 2-person 2-\textsc{hab}-\textsc{prs}-work-\textsc{dur}-\textsc{fv} \textsc{ppf}-9.work for \textsc{ppf}-1-chief \\
\glt `Every group of people does work for the chief.' \label{ex:kaspercushman:everygroup}

\end{exe}

Finally, note that \textit{kila} cannot stand alone as a pronoun.

\subsection{\textit{Kila} in coordinate structures}

In general, \textit{kila} seems to resist coordination. Structures such as \textit{every N$_{1}$ and N$_{2}$} without the repetition of \textit{kila} were judged to be unnatural with the intended meaning was \textit{every N$_{1}$ and every N$_{2}$}.  Additionally, for several contexts created to elicit coordination with \textit{kila}, coordination with \textit{mbe-\textsc{agr2}-li} was judged to be more natural (or the only coordination structure possible). However, I was able to elicit a coordinated structure in the form of \textit{every N$_{1}$ and every N$_{2}$} which my informant judged to be natural. This context would also permit the quantifier \textit{mbe-\textsc{agr2}-li}.  Note that in the example (\ref{ex:kaspercushman:everyphotos}) with \textit{kila}, the two coordinated NPs contain relative clauses, while in example (\ref{ex:kaspercushman:allphotos}) with \textit{mbe-\textsc{agr2}-li}, the coordinated DPs do not.

\begin{exe}

\ex Context: I went to Tanzania. While I was there, I went on a safari in a \textit{mbugayawayama} (a Kiswahili word for `reserve' or `park'). In the park... \label{ex:kaspercushman:photos}\\

\begin{xlist}

\ex 
\gll nd-a-tov-ige i-picha y-a \textbf{kila} \textbf{\s{m}-biki} \\
\textsc{1sg}-\textsc{p3}-beat-\textsc{dur.pst} \textsc{ppf}-9.picture 9-\textsc{assoc} every 3-tree \\ \label{ex:kaspercushman:everyphotos}


\gll gw-e-nd-a-gu-won-ige \textbf{na} \textbf{kila} \textbf{ki-koko} \\
3-\textsc{rel}-\textsc{1sg}-\textsc{p3}-3-see-\textsc{dur.pst} and every 7-animal \\


\gll ch-e-nd-a-ki-won-ige. \\
  7-\textsc{rel}-\textsc{1sg}-\textsc{p3}-7-see-\textsc{dur.pst}  \\
  
\glt `I took a picture of every plant which I saw and every animal which I saw.' 



\ex 
\gll nd-a-tov-ige i-picha y-a \textbf{i-mi-biki}  \\ 
\textsc{1s}-\textsc{p3}-beat-\textsc{dur.pst} \textsc{ppf}-9.picture 9-\textsc{assoc} \textsc{ppf}-4-tree \\ \


\gll \textbf{mbe-ge-li} \textbf{n(a)} \textbf{i-fi-koko} \textbf{mbe-fe-li} \label{ex:kaspercushman:allphotos} \\
 all-4-all and \textsc{ppf}-8-animal all-8-all  \\
 
 
\gll fy-e-nd-a-fi-won-ige. \\
 8-\textsc{rel}-\textsc{1s}-\textsc{p3}-8-see-\textsc{dur.pst} \\
 
\glt `I took a picture of all the plants and animals that I saw.'

\end{xlist}
\end{exe}


\subsection{\emph{Kila}: Like \emph{every} or like \emph{each}?}  

There is no separate lexical item in Kihehe expressing \textit{each}. For English, linguists have investigated several differences between \textit{each} and \textit{every}; three of these differences are summarized in \tabref{tab:kaspercushman:everyeach}. Considering these differences, one can ask if the uses of \textit{kila} in Kihehe correspond more to the syntax and semantics of \textit{each}, or more to those of \textit{every}. 

\begin{table}
\small
\caption{Some properties of \emph{each} and \emph{every} \citep{BeghelliStowell1997}}
\label{tab:kaspercushman:everyeach}
\begin{tabularx}{\textwidth}{l@{~~}p{42mm}@{~~}Q@{}}
\lsptoprule
Property                 & \textit{Each}                                                                                            & \textit{Every}                                                                                                 \\ 
\midrule
Ability to float         & can float                                                 & cannot float                                                                                          \\ 
Modification by `almost'  & cannot be modified by \textit{almost}                     & can be modified by \textit{almost}                                                                           \\ 
Generics                 & \textit{each N} cannot be understood as a generic          & \textit{every N} can be understood as a   generic                  \\ 
\lspbottomrule
\end{tabularx}
\todo[inline]{maybe use + and -- in table cells}
\end{table}

For the properties in \tabref{tab:kaspercushman:everyeach}, \textit{kila} patterns with \textit{every}, rather than \textit{each}.  First, \textit{kila} cannot float. It can appear in only one position, preceding the noun it modifies. Second, as shown in (\ref{ex:kaspercushman:almostevery}), \textit{kila} can be modified by \textit{kalibiya} `almost,' which is a borrowing from Kiswahili. 

\begin{exe}
\ex 
\gll Kalibiya kila ki-nu ki-sis-ile     \\
almost every 7-thing 7-be\_used\_up-\textsc{pst} \\
\glt `almost every thing is gone' \\ \label{ex:kaspercushman:almostevery}
\end{exe}

Third, \emph{kila} can be used to make generic statements, as shown in example (\ref{ex:kaspercushman:everycow}).

\begin{exe}


\ex 
\gll kila senga y-ay-i-ly-(a) a-ma-soli \label{ex:kaspercushman:everycow}\\
every 9.cow 9-\textsc{hab}-\textsc{prs}-eat-\textsc{fv} \textsc{ppf}-6-grass \\
\glt `Every cow eats grass.' 

\end{exe}


In addition to these properties, \citealt[150]{vendler62}, states that ```each'...directs one's attention to the individuals as they appear, in some succession or other, one by one.''  Consider example (\ref{ex:kaspercushman:onebyone}), based on a similar example in \citealt[150]{vendler62}. The example suggests that in Kihehe, when the context establishes a one-by-one interpretation, \textit{kila} becomes infelicitous.\footnote{An anonymous reviewer notes that this example ``suggests that Kihehe employs the typologically widespread reduplication strategy for expressing `strong distributivity' (in the sense of Beghelli and Stowell, 1997)'' and wonders if the reduplication method for expressing strong distributivity blocks \textit{kila} in appearing in such contexts. The use of reduplication to indicate distributivity is also supported by the Kihehe data in (\ref{mangos}), in which reduplication gives rise to an unambiguous distributive meaning of ``2 mangos per child'' (on reduplication and distributivity, see also \citealt{balusu13}). 
\begin{exe}
\ex \label{mangos} 
\gll \s{M}-pel-e kila mw-an(a) a-ma-yembe ga-vili ga-vili \\
1-give-\textsc{imp} every 1-child \textsc{ppf}-6-mango 6-two 6-two \\
\glt `Give every child 2 mangos each.' \\
\end{exe}

A final point of comparison between English \textit{every} and \textit{kila}. Beghelli and Stowell (1997, p. 98) note that \textit{every} allows collective readings, as in \textit{It took every boy to lift the piano} (their example 33a). \textit{Kila} does not allow collective readings. This is discussed in the following section in conjunction with example (\ref{ex:kaspercushman:competition}) and footnote \ref{foot:kaspercushman:competition}.}

\begin{exe}

 \ex Context: After I purchased a basket of mangos, I noticed that some of them were starting to rot. I want you to look at each one carefully so that you can sort the mangos into those that are still good to eat, and those that are spoiled. I tell you: \label{ex:kaspercushman:onebyone} \\

\begin{xlist}
\judgewidth{\textsc{ok}}
\ex[\#]{ 
\gll  li-lav-e kila li-yembe \\
 5-see-\textsc{imp} every 5-mango \\
\glt `Examine every mango.'}

\ex[\textsc{ok}]{ 
\gll  li-lav-e lí-mwí-lí-mw\^i \\
5-see-\textsc{imp} 5-one-5-one \\ \label{reduplication}
\glt `Examine (them) one after another.'}

\end{xlist}
\end{exe}


\section{Collectivity and distributivity}

This section discusses the properties of collective and distributive interpretations with the universal quantifiers in Kihehe. 
Regarding \textit{-ote} `all' and \textit{kila} in Kiswahili, a reviewer of \citealt[396]{zerbian08} ``suggests \textit{kila} is inherently distributive whereas \textit{-ote} is underspecified concerning the distributive/collective distinction.''  This section presents data which suggest that the behavior of quantifiers in Kihehe are compatible with the hypothesis that \textit{mbe-\textsc{agr2}-li} is underspecified for collectivity vs. distributivity, while \textit{kila} is distributive.

\subsection{Distributive contexts}

In the contexts that are distributive, \textit{mbe-\textsc{agr2}-li} is acceptable. For the context presented in (\ref{ex:kaspercushman:didgreet}), \textit{kila} is also acceptable (\ref{ex:kaspercushman:everygreeted}).\footnote{An anyonymous reviewer asks why \textit{kila} is felicitous here (\ref{ex:kaspercushman:everygreeted}), while it is not felicitous in either (\ref{reduplication}) or (\ref{ex:kaspercushman:reduplicationstone}). One possibility is that the context in (\ref{ex:kaspercushman:didgreet}) does not impose that the acts of greeting were conducted in strict succession, one after another.  More research is needed to clearly define the properties of the contexts which permit or require one strategy (\textit{kila} or reduplication) versus another.} 


\begin{exe}

\ex Context: The chief was sitting in the front of the hall.  Before the meeting started, the villagers went up individually to greet the chief. All of the villagers present had to greet the chief before the meeting started. Then at last, the meeting began. This evening... \label{ex:kaspercushman:didgreet}

\begin{xlist}

\ex 
\gll \textbf{a-va-nu} \textbf{mbe-va-li} va-mu-$\emptyset$-hunje u-mu-twa \\
\textsc{ppf}-2-person all-2-all 2-1-\textsc{p1}-greet.\textsc{pst} \textsc{ppf}-1-chief \\
\glt `All of the people greeted the chief.' \\



\ex 
\gll \textbf{Kila} \textbf{mu-nu} a-\s{m}-$\emptyset$-hunje u-mu-twa \\
every 1-person 1-1-\textsc{p1}-greet.\textsc{pst} \textsc{ppf}-1-chief  \\ \label{ex:kaspercushman:everygreeted}
\glt `Every person greeted the chief.' 

\end{xlist}
\end{exe}


However, \textit{kila} is not acceptable in all distributive contexts. In example (\ref{ex:kaspercushman:stone}), \textit{mbe-\textsc{agr2}-li} is possible, but \textit{kila} is not felicitous. As with example (\ref{ex:kaspercushman:onebyone}), the context creates an interpretation of `individually, one after another.'  


\begin{exe}
\ex Context: The boys in the village were arguing about who was the strongest. Then one boy saw a big rock and said ``Whoever throws the rock the farthest is the strongest.'' So one at a time, the boys threw the rock. What happened?   \label{ex:kaspercushman:stone}

\begin{xlist}

\ex 
\gll \textbf{a-va-kwamisi} \textbf{mbe-va-li} v-a-hom-it(e) i-li-ganga \\
\textsc{ppf}-2-boy all-2-all 2-\textsc{p3}-throw-\textsc{pst} \textsc{ppf}-5-stone \\
\glt `All of the boys threw the rock.'




\ex 
\gll \# \textbf{kila} \textbf{\s{m}-kwamisi} (a)-a-hom-it(e) i-li-ganga \\
{} every 1-boy 1-\textsc{p3}-throw-\textsc{pst} \textsc{ppf}-5-stone \\ \label{ex:kaspercushman:reduplicationstone}
\glt `Every boy threw the rock.'




\ex 
\gll \textsc{ok} \textbf{u-mu-nu} \textbf{yu-mwi-yu-mwi} (a)-a-hom-it(e) i-li-ganga \\  
{} \textsc{ppf}-1-person 1-one-1-one 1-\textsc{p3}-throw-\textsc{pst} \textsc{ppf}-5-stone \\
\glt `The people threw the stone one after another.'

\end{xlist}
\end{exe}


\subsection{Collective contexts}

In contexts that are necessarily collective, only \textit{mbe-\textsc{agr2}-li} is felicitous.  The use of \textit{kila} creates distributive interpretations that are not compatible with the contexts. 

\begin{exe}

 \ex Context: A fire destroyed someone's house in the village. The person was elderly and could not rebuild her house, so the other villagers decided to help her. This morning... 

\begin{xlist}

\ex[]{
\gll \textbf{a-va-baba} \textbf{mbe-va-li} v-a-senz-ile kangi i-nyumba \\
\textsc{ppf}-2-man all-2-all 2-\textsc{p3}-build-\textsc{pst} again \textsc{ppf}-9.house\\
\gll ye=ke=yi-ka-pye  \\ 
 9.\textsc{rel}=\textsc{aux}=9-\textsc{p2}-burn.\textsc{pst} \\
\glt `All the men rebuilt the house which had burned.' }

\ex[\#]{ 
\gll  \textbf{kila} \textbf{\s{m}-baba} (a)-a-senz-ile kangi i-nyumba \\
 every 1-man 1-\textsc{p3}-build-\textsc{pst} again \textsc{ppf}-9.house \\
\gll ye=ke=yi-ka-pye   \\ 
9.\textsc{rel}=\textsc{aux}=9-\textsc{p2}-burn.\textsc{pst}\\
\glt `Every man rebuilt the house which had burned.' \\ 
\textit{Comment from informant: This would have to mean that there were many houses being built.}}

\end{xlist}
\end{exe}


\citealt[121]{szabolcsi10} states that ``To test collective readings it is advisable to employ punctual accomplishment verbs such as \textit{lift up}, as opposed to \emph{lift}, which has an activity reading.'' Based on this statement, the context in (\ref{ex:kaspercushman:wagon}) was designed to test collective readings with the verb \textit{kunyanyula}, `lift up.' The informant's judgments indicate that  \textit{mbe-\textsc{agr2}-li} is acceptable with a collective reading, while \textit{kila} requires a distributive reading. 

\begin{exe}

 \ex Context: There was a child trapped under a wagon that had fallen over in the road. The child screamed, and people rushed to the wagon. In order to rescue the child... \label{ex:kaspercushman:wagon}

\begin{xlist}
\ex[]{
\gll \textbf{a-va-nu} \textbf{mbe-va-li} va-$\emptyset$-nyanyuw(e) i-li-tololi \\
\textsc{ppf}-2-person all-2-all 2-\textsc{p1}-lift\_up.\textsc{pst} \textsc{ppf}-5-wagon  \\
\glt `All of the people lifted up the wagon.'}
\ex[\#]{ 
\gll \textbf{kila} \textbf{mu-nu} a-$\emptyset$-nyanyuw(e) i-li-tololi \\ 
 every 1-person 1-\textsc{p1}-lift\_up.\textsc{pst} \textsc{ppf}-5-wagon  \\
\glt `Every person lifted up the wagon.' \\ \label{ex:kaspercushman:everywagon}
\textit{Comment from informant: This would mean that everyone has his/her own wagon.}}
\end{xlist}
\end{exe}


However, as Robert Botne (pers. comm.) points out, we cannot be sure that \textit{kunyanyula} is necessarily a punctual accomplishment verb (i.e., is it interpreted as `lift up' or `lift'?). Additional evidence that \textit{kila} requires a distributive interpretation while \textit{mbe-\textsc{agr2}-li} is acceptable with a collective reading comes from the pair in (\ref{ex:kaspercushman:competition}), modeled after the data in \citealt[396--397]{zerbian08}. These data also show that similar to what Zerbian and Krifka (2008, p. 397) describe for Kiswahili, in Kihehe \textit{kila} is not grammatical with \textit{pamw\^i} `together' (\ref{ex:kaspercushman:everytogether}).\footnote{Note also that it is ungrammatical to add \textit{pamw\^i} `together' to the utterance in example (\ref{ex:kaspercushman:everywagon}).\label{foot:kaspercushman:competition}}


\begin{exe}
\ex  \label{ex:kaspercushman:competition} \begin{xlist}
\ex[]{
\gll \textbf{a-va-kwamisi} \textbf{mbe-va-li} v-a-lavis-ig\^e a-ma-shindano pamw\^i \\
\textsc{ppf}-2-boy all-2-all 2-\textsc{p3}-watch-\textsc{dur.pst} \textsc{ppf}-6-competition together\\
\glt `All of the boys watched the competition together.'}

\ex[*]{ \label{ex:kaspercushman:everytogether} 
\gll \textbf{kila} \textbf{\s{m}-kwamisi} (a)-a-lavis-ig\^e a-ma-shindano pamw\^i \\
 every 1-boy 1-\textsc{p3}-watch-\textsc{dur.pst} \textsc{ppf}-6-competition together \\
\glt `Every boy watched the competition together.'}
\end{xlist}
\end{exe}

 

\section{Kihehe universal quantifiers and cross-linguistic generalizations}\label{sec:kaspercushman:5}  \label{ex:kaspercushman:type}
\citet{matthewson13} calls for work towards a typology of quantifiers, and puts forth eight crosslinguistic generalizations based on a survey of 37 different languages. In this section I situate the Kihehe data with respect to three of the crosslinguistics generalizations that bear on syntactic differences among the universal quantifiers.\footnote{Landman has a similar goal for her (2016) study of Logoori; she finds that the generalizations from Matthewson (2013) under consideration for her study apply to the distributive quantifier \textit{vuri} `every,' but not to the quantifier \textit{-oosi} `all,' ultimately analyzing \textit{-oosi} as a DP-internal modifier. Whether this analysis could extend to Kihehe \textit{mbe-\textsc{agr2}-li} is left an open question for future research.}  



First, consider Matthewson's Generalization 6:

\begin{exe}
\ex Gen6: It is common for a word translated as `all' to look as if it attaches to a full DP, even when other quantifiers do not. \citep[35]{matthewson13}
\end{exe}

As described above, the Kihehe data show that \textit{mbe-\textsc{agr2}-li} must combine with a head noun + pre-prefix when the head noun is an argument. The data are consistent with the hypothesis that the pre-prefix on nouns is a D, and one of its functions is to combine with NPs to form DP arguments. I come to this hypothesis applying the argumentation laid out in \citealt{matthewson01}. \citealt{gambarage16} arrives at a similar analysis of the pre-prefix in Nata, a Bantu language also spoken in Tanzania; see his work for a much more exhaustive investigation of this claim as well as for references of number of authors who argue that the pre-prefix is a D; see \citealt{longobardi94} regarding Ds forming DP arguments.

In support of this hypothesis, note that in Kihehe, the head noun of a DP argument containing the quantifier \textit{mbe-\textsc{agr2}-li} must appear with the pre-prefix. Main clause quantified DPs are ungrammatical without the pre-prefix (\ref{ex:kaspercushman:alldied}).

\begin{exe}
\ex[*]{ \label{ex:kaspercushman:alldied} 
\gll \textbf{Sénga} \textbf{mbé-se-li} sí-fwe \label{ex:kaspercushman:allcowsnoppf2}\\ 
10.cow all-10-all 10-die.\textsc{pst} \\
\glt `All the cows died.' }
\end{exe}

While DP arguments must contain a pre-prefix on the head noun, the pre-prefix cannot occur on a predicate noun or a predicate adjective. When the pre-prefix does occur in these contexts, it creates a relative clause (\ref{ex:kaspercushman:chief}--\ref{ex:kaspercushman:tall}). 

\begin{exe}
\ex \label{ex:kaspercushman:chief} \begin{xlist}

\ex 
\gll u-mu-nu yu-la \textbf{mu-twa} \\
\textsc{ppf}-1-person 1-\textsc{dem.dist} 1-chief \\
\glt `That person is a/the chief.'


\ex 
\gll u-mu-nu yu-la \textbf{u-mu-twa} \\
\textsc{ppf}-1-person 1-\textsc{dem.dist} \textsc{ppf}-1-chief \\
\glt `that person who is a/the chief'

\end{xlist}
\end{exe}

\begin{exe}

\ex \label{ex:kaspercushman:tall} \begin{xlist}

\ex 
\gll a-v-ana \textbf{va-tali} \\
\textsc{ppf}-2-child 2-tall  \\
\glt `The children are tall'


\ex 
\gll a-v-ana \textbf{a-va-tali} \\
\textsc{ppf}-2-child \textsc{ppf}-2-tall \\
\glt `the children who are tall'

\end{xlist}
\end{exe}

Pre-prefixes are not needed on mass nouns when stating general information (\ref{ex:kaspercushman:wulasi}). 


\begin{exe}

\ex \label{ex:kaspercushman:wulasi} 
\gll \textbf{wu-lasi} wu-nono \\
14-wulasi 14-sweet \\
\glt `\textit{ex:kaspercushman:wulasi} is sweet'

\end{exe}

The hypothesis that a DP noun argument must be marked with a pre-prefix when combing with the universal quantifier is strongly supported by the Kihehe data.  However, one possible counter example to this hypothesis is given in example (\ref{ex:kaspercushman:mango}). This minimal pair demonstrates the finding that the requirement of the pre-prefix on the head noun seems to be stronger for subjects and direct objects than for indirect objects.  My informant judged (\ref{ex:kaspercushman:noppfio}) only marginally worse than (\ref{yesppfio}). 

\begin{exe} 
\ex  \label{ex:kaspercushman:mango} \begin{xlist}
\ex[]{ Pre-prefix on indirect object head noun\\
\gll Va-pel-(e) a-ma-yembe ga-vili \textbf{a-v-ana} \textbf{mbe-va-li} \\
2-give-\textsc{imp} \textsc{ppf}-6-mango 6-two \textsc{ppf}-2-child all-2-all\\ 
\glt `Give two mangos to all of the children.' \label{yesppfio} 
}
\ex[?]{ No pre-prefix on indirect object head noun\\
\gll Va-pel-(e) a-ma-yembe ga-vili \textbf{v-ana} \textbf{mbe-va-li} \\
2-give-\textsc{imp} \textsc{ppf}-6-mango 6-two 2-child all-2-all\\ \label{ex:kaspercushman:noppfio}\
\glt `Give two mangos to all of the children.'}

\end{xlist}
\end{exe}

Thus, the Kihehe evidence is consistent with Generalization 6 in that `all' appears to combine with a full DP.  The second half of Generalization 6, ``even when other quantifiers do not [attach to a full DP]'' will be discussed in tandem with Matthewson's Generalization 7. 


\begin{exe}
\ex Gen7: In some languages, distributive universals appear to combine directly with NP, while other quantifiers do not. \cite[36]{matthewson13} 
\end{exe}

As described above, unlike \textit{mbe-\textsc{agr2}-li}, the distributive universal quantifier \textit{kila} may not combine with a noun + pre-prefix, and in fact, \textit{kila} and the pre-prefix are in complementary distribution (see footnote \ref{foot:kaspercushman:1}). If the pre-prefix is a D which attaches to NPs to form DP arguments, and \textit{kila} may never attach to a noun with a pre-prefix, then this suggests that \textit{kila} combines with NPs. Thus, this generalization applies to Kihehe. 

Finally, consider Matthewson's Generalization 8: 

\begin{exe}
\ex Gen8: A secondary pattern is to distinguish distributive universal quantifiers from other universal quantifiers, but for the former to use some other strategy such as reduplication, affixation, or adverbial quantification \cite[37]{matthewson13}
\end{exe}

As discussed above, the data suggest that Kihehe uses reduplication to express distributivity.  While the strategy of reduplication is in need of more exploration, a preliminary hypothesis is that Kihehe has one strategy of distributive quantification in which the quantifier \textit{kila}, borrowed from Kiswahili, combines with an NP, and one strategy of distributive quantification which uses reduplication. 

\section{Conclusion}

This paper describes general properties of the universal quantifiers \textit{mbe-\textsc{agr2}-li} `all' and \textit{kila} `every' in Kihehe. It then discusses the quantifiers in light of the properties of collectivity and distributivity, and provides data that suggest that Kihehe has an additional strategy of reduplication for expressing distributive quantification. Ultimately, this paper argues that Kihehe attests patterns similar to what has been observed for other Bantu languages: \textit{mbe-\textsc{agr2}-li} is compatible with both collective and distributive contexts, while \textit{kila} is distributive. Unlike English \textit{every}, \textit{kila} is ungrammatical with a collective reading. However, while distributive, \textit{kila} is infelicitous in contexts in which a `one by one' or `one after another' reading is required, though the exact constraints on \textit{kila}, and a more thorough investigation of the strategy of reduplication, are left for future research. Finally, this paper considers the Kihehe data in light of the typology of quantifiers crosslinguistically, specifically arguing that the Kihehe data are congruent with three of Matthewson's (2013) crosslinguistic generalizations. 

\section*{Acknowledgements}
I would like to thank Richard Nyamahanga for providing the Kihehe language data and judgements presented in this paper. Data were collected by the author through elicitation sessions at Indiana University.  I would like to thank Robert Botne and my fellow students in L654: Field Methods, Indiana University, Spring 2016, for their suggestions and discussion of this data. I would also like to thank Tom Grano and Steve Franks for formative comments on earlier versions of this work. Finally, I would like to thank the audience members at ACAL 48, and an anonymous reviewer for helpful feedback and suggestions.

\section*{Abbreviations}
Bare numerals (e.g., 1, 2, 3) refer to noun class markers or agreement markers for that noun class.\\

\noindent\begin{tabularx}{.5\textwidth}{@{}lQ@{}}
\textsc{1sg} & first person singular\\
\textsc{1pl} & first person plural\\
\textsc{add} & proximal to addressee\\
\textsc{adj} & adjective\\
\textsc{agr2} & agreement marker 2\\
\textsc{assoc} & associative marker\\
\textsc{aux} & auxiliary\\
\textsc{dem} & demonstrative\\
\textsc{dur} & durative\\
\textsc{fv} & final vowel\\
\textsc{hab} & habitual\\
\textsc{imp} & imperative\\
\textsc{impf} & imperfective\\
\end{tabularx}%
\begin{tabularx}{.5\textwidth}{@{}lQ@{}}
\textsc{loc} & locative\\
N & noun\\
\textsc{p1} & recent past\\
\textsc{p2} & hodiernal past\\
\textsc{p3} & remote past\\
\textsc{ppf} & pre-prefix\\
\textsc{poss} & possessive\\
\textsc{prox} & proximal\\
\textsc{prs} & present\\
\textsc{pst} & past\\
\textsc{rel} & relative\\
\# & infelicitous\\
\\
\end{tabularx}

\sloppy
\printbibliography[heading=subbibliography,notkeyword=this]
\end{document}

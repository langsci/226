\documentclass[output=paper,
modfonts
]{langscibook} 

\title{Complement clause C-agreement beyond subject phi-agreement in Ikalanga}

\author{%
 Rose Letsholo\affiliation{University of Botswana}\lastand
 Ken Safir\affiliation{Rutgers University}
}

% \chapterDOI{} %will be filled in at production
% \epigram{} 

 \abstract{ 
% Abstract goes here
 }

\begin{document}
\maketitle

\section{Introduction} 

Complementizer agreement with matrix verb subjects in languages like Lubukusu \citep{Diercks2013} raises many issues concerning the syntax of agreement and the status of agreeing complementizers, but rich descriptions beyond Lubukusu are rare. In addition to Lubukusu, C-agreement with the matrix subject has been attested for Kinande and Ibibio \citep{Baker2008}, Mande languages \citep{Idiatov2010}, Luvale, Luchazi, Chokwe and Lunda \citep{Kawasha2007}, Limbum \citep{Nformi2017a} and Kipsigis \citep{Diercks2017}. Only the recent account of Kipsigis, however, comes close to the thorough description Diercks provides for Lubukusu. In this paper we  provide a rich description and analysis of C-agreement and complement clause complementizer (CCC) distributions in Ikalanga, which uncovers a much more articulated set of relationships between matrix clause properties and the morphology of an agreeing complementizer. In particular, we show that the agreeing CCC is sensitive not only to matrix subject phi-features, but matrix voice and tense as well. Although we suggest an analysis, our main goal in this essay is to set boundary conditions on what any truly explanatory treatment must account for.

\section{The Ikalanga Pattern}

Ikalanga CCCs for non-infinitival clauses include invariant \textit{kuti} and \textit{kuyi}, and two CCCs that agree with matrix subjects, AGR-\textit{ti} and AGR-\textit{yi}. The morpheme \textit{–ti} is identical to the verb ‘say’ \textit{-ti} and neither AGR-\textit{ti}, nor any other CCC, is possible with main verb \textit{–ti}. However, AGR-\textit{ti}, when it occurs with other verbs, not only agrees with matrix subjects, but also partially agrees with matrix tense. We will argue further that both \textit{kuti} and \textit{kuyi} are also based on the \textit{–ti} ‘say’ root which has the suppletive form \textit{-yi} when it is passive. \textit{Kuyi}, in at least some of its distribution, agrees for passive voice with the matrix verb. The \textit{ku-} of \textit{kuti} and \textit{kuyi} is a default form where the c17 prefix \textit{ku-} affixes to \textit{–ti} (and suppletive \textit{-yi}) in the absence of agreement.  We argue that all the \textit{–ti}-based complementizers are essentially the same form and that they are not verbs – grammaticalization is incomplete in an interesting way. This pattern has consequences for what the assumed locality relations are between matrix T(ense)  and elements inside vP. Although sensitivity to voice and subject phi-features is feasible within the vP-phase in minimalist reasoning, interaction of a CCC with matrix T raises interesting questions that require further assumptions which we will explore.

The following examples show that what we are calling the AGR-\textit{ti} complementizer agrees with the matrix subject for person and noun class. Examples (\ref{1}a,b) show agreement for noun classes c1 and c2 and examples (\ref{1}c,d) show agreement for noun class and person. The vowel /e/ in \textit{eti} (\ref{a}) and \textit{beti} in (\ref{b}) is underlyingly /a/ before regressive assimilation. The rest of the examples in (\ref{1}) illustrate agreement for c3 through c10. The presence of -\textit{ka-} as in \textit{-kati} will be addressed later.\footnote{Our glosses diverge slightly from Leipzig Glossing Conventions in that we follow Afranaph Project glossing (Afranaph Database, ongoing) \nocite{AfranaphDatabaseOngoing} and in particular subject agreement is treated as `subject marker' (\mc{sm}) to avoid prejudging the agreement or pronominal status of subjects in Bantu languages and more generally (also \mc{om} for `object marker'). Noun class numbers are preceded by ‘c’ to facilitate Afranaph searches, as are other departures from Leipzig conventions. Person is marked with 1\textsuperscript{st} and 2\textsuperscript{nd} and noun class glosses determine plurality, as in \mc{sm}.c1.1\textsuperscript{st} which is first person singular where c2 would be plural. Third person is treated as default. Other glosses include \mc{aux} (auxiliary), \mc{caus} (causative), fv (final vowel), \mc{fut} (future), \mc{loc} (locative), \mc{neg} (negation), \mc{pass} (passive),  \mc{prn} (pronoun), \mc{prs} (present), \mc{pst}1 (recent past), \mc{pst}2 (remote past), \mc{rcm} (reciprocal verbal affix), \mc{rfm} (reflexive verbal affix), and \mc{sbjv} (subjunctive). Associative plural, \textit{bo}- in Ikalanga, is informally glossed as \&. \mc{inf} (infinitive) is used instead of c17 in Ikalanga, which it may be identical to. For Kinande, \mc{iv} is `initial vowel'.}  


\ea \label{1}
\ea \label{a} \gll ú-nó-dw-à Néó è-tí á-téng-é	lórì\\
\mc{sm}.c1-\mc{prs}-tell-fv Neo \mc{sm}.c1-that \mc{sm}.c1-buy-\mc{sbjv}	c9.car\\
\glt He/she is telling Neo that she should buy a car/to buy a car 

\ex \label{b} \gll íbò b-á-dw-á Nchídzì	bè-tí á-tèng-è lórì \\
\mc{prn}.c2	\mc{sm}.c2-\mc{pst1}-tell-fv	Nchidzi	\mc{sm}.c2-that \mc{sm}.c1-buy-\mc{sbjv} c9.car \\
\glt They (pl) told Nchidzi that he should buy a car/ to buy a car.

\ex \label{c} \gll nd-ó-dw-à		 Néó	ndì-tí 		 á-tèng-è		lórì \\
\mc{sm}.c1.1\textsuperscript{st}-\mc{prs}-tell-fv  Neo  \mc{sm}.c1.1\textsuperscript{st}-that	 \mc{sm}.c1-buy-\mc{sbjv}	c9.car \\
\glt I am telling Neo to buy a car.
  
\ex \label{d} \gll ìswì   t-à-dw-á   Nchídzì  tì-tí  á-tèng-è	 lórì \\
\mc{prn}.c2.1\textsuperscript{st}  \mc{sm}.c2.1\textsuperscript{st}-\mc{pst1}-tell-fv   Nchidzi  \mc{sm}.c2.1\textsuperscript{st}-that \mc{sm}.c1-buy-\mc{sbjv} c9.car\\
\glt We told Nchidzi that he should buy a car/ to buy a car.

\ex \label{e} \gll mpání	w-áká-dw-à	nsú	ú-kàtì	ú-ízèl-é \\
mophane.c3	\mc{sm}.c3-\mc{pst2}-tell-fv acacia-c3	\mc{sm}.c3-that	\mc{sm}.c3-sleep-\mc{sbjv}\\
\glt The mophane tree told the acacia tree to sleep.

\ex \label{f} \gll	mipání		y-áká-dw-à			nsú		í-kàtì		ú-ízèl-é\\
mophane.c4	\mc{sm}.c4-\mc{pst2}-tell-fv	acacia-c3	\mc{sm}.c4-that	\mc{sm}.c3-sleep-\mc{sbjv}\\
\glt The mophane trees told the acacia tree to sleep.  (lit. The mophane trees told the acacia  tree that it (the acacia tree) should sleep.)
\z
\z 

Additional examples illustrating agreement for c5-c10 are to be found in the Afranaph Database (ongoing), see example IDs \#15250-\#15255.

\subsection{The Lubukusu Pattern}

Our expectations of what C-agreement with matrix subject phenomena look like have been largely set by \citeauthor{Diercks2013}' (2013) analysis of Lubukusu, and so we outline some key features of the Lubukusu pattern as a background comparison for our account of Ikalanga. Lubukusu has three sorts of complementizers that are typically used for indicative complement clauses: invariant \textit{bali}, \textit{mbo}, and AGR-\textit{li}, the last of which agrees with the matrix subject. 
\begin{enumerate}
\item[A.] The root \textit{–li} of AGR-\textit{li} is related either to a copula or a c5 agreement marker. Indicative clausal subjects show c5 agreement with matrix subject agreement markers (SMs). 
\item[B.] A fairly wide class of verbs (at least 19), including verbs of speaking, desire, perception and epistemic verbs, take indicative or subjunctive complement clauses that can have AGR-\textit{li}. 
\item[C.] Lubukusu AGR-\textit{li} agrees with the matrix subject of the verb selecting the complement clause, whether the matrix subject is an active agent or experiencer, or if it is the passivized object.
\item[D.] Direct objects of double complement verbs do not block C-agreement with the matrix subject and the agreement is not keyed to perspectives. \citet{Diercks2013} shows C-agreement with the matrix subject, not the information source in prepositional object position, as in (\ref{2}) (glossing from source).
\end{enumerate}

\ea \label{2}
\gll Khw-a-ulila  khukhwama khu Sammy  \textbf{khu-li}/*ali ba-limi   ba-a-funa ka-ma-indi. \\
1pl\mc{s-pst}-hear from  \mc{loc} 1Sammy 1pl-that  2-farmers 2\mc{s-pst}-harvest 6-6-maize \\
\glt We heard from Sammy that the farmers harvested the maize.
\z

\subsection{The role of the `say' root in Ikalanga \textit{-ti} based complementizers}

One difference from Lubukusu is that the lexical root of the Ikalanga agreeing CCC is morphologically identical to the root of a verb meaning ‘say’. When \textit{–ti} is used as a matrix verb, no \textit{-ti}  complementizer is possible, be it \textit{kuti}, \textit{kuyi} or AGR-\textit{ti}.

\ea \label{3}
\ea\label{3a} \gll nd-à-tí Néó w-á-téng-à	lórì \\
	\mc{sm}.c1.1\textsuperscript{st}-\mc{pst}1-say	Neo	\mc{sm}.c1-\mc{pst}1-buy-fv		c9.car.\\ 
	\glt I said that Neo has bought a car.

\ex\label{3b}  \gll	ù-kà-tì			Nchídzì	w-á-énd-à \\
     	\mc{sm}.c1.2nd-\mc{prs}-say	Nchidzi	\mc{sm}.c1-\mc{pst}1-go-fv \\
	\glt You say/are saying  Nchidzi is gone/ has left.

\ex\label{3c}    \gll à-kà-tì			Nchídzì	w-á-énd-à \\
      	\mc{sm}.c1-\mc{prs}-say	Nchidzi	\mc{sm}.c1-\mc{pst}-go-fv \\
     	\glt He says/is saying Nchidzi is gone/ has left. 

\ex\label{3d}  \gll Néó	ú-nòò-tì		b-àná		bá-zh-è.\\
	Neo	\mc{sm}.c1-\mc{fut}-say	c2-child	\mc{sm}.c2-come-\mc{sbjv}\\
	\glt Neo will say (that) the children should come.

\z
\z 

Since many languages have grammaticalized `say' complementizers (see, e.g., \citealt{Heine2002}), a natural hypothesis might be that AGR-\textit{ti} not only looks like a verb, it is one, and that is why it is missing when the main verb means `say'. There are a number of reasons that we do not analyze AGR-\textit{ti} as a synchronic verb. 

One is that the morphology borne by AGR-\textit{ti} is impoverished by comparison with main verb \textit{–ti}. For example,  recent past (\mc{pst}1) of main verb \textit{–ti} (\ref{3a}), the present (\ref{3b}), and the future (\ref{3d}),  are never used on the ‘AGR-\textit{ti}’ that introduces a complement clause (whenever the main verb is not \textit{–ti}). The suppletive \textit{–kà-} which is \mc{prs} for only the verb \textit{–ti} in (\ref{3b})   can occur on AGR-\textit{ti}, but only in concordance with remote past.\footnote{Remote past (\mc{pst2}) might be morphologically composed of recent past (\mc{pst1}) \textit{–a-} and remote past (\mc{pst2}) \textit{–ka-} but we will represent it as undecomposed \textit{–aka-}. This morphological analysis is not crucial in our discussion of C distributions.}  

Additional evidence that AGR-\textit{ti} is not acting as a verb comes from contexts where the verb \textit{–ti} is embedded after AGR-\textit{ti}. [based on Afranaph ID559]

\ea \gll Edgar w-ákà-dw-à  Bill à-kà-tì  á-tí  Mary 	ú-noò-n-dá\\
        Edgar \mc{sm}.c1-\mc{pst2}-tell-fv Bill \mc{sm}.c1-\mc{pst}2-that \mc{sm}.c1-say Mary  \mc{sm}.c1-\mc{prs-om}.c1-love\\
       \glt Edgar told Bill to say that Mary loved him.
       \z 
       
If \textit{àkàtì} were sufficient to embed the verb `say', then we would not expect \textit{átí} to follow it. In fact, there is another verb meaning `say' or `speak' in Ikalanga that requires AGR-\textit{ti} as its complementizer. If AGR-\textit{ti} were a main verb it would be redundant in (\ref{5}).

\ea\label{5} \gll Néó w-ákà-leb-a    (à)-kà-ti  Nchidzi w-ákà-téng-á  lórì\\
         Neo \mc{sm}.c1-\mc{pst2}-say-fv  \mc{sm}.c1-\mc{pst2}-that Nchidzi \mc{sm}.c1-\mc{pst2}-buy-fv car\\
     \glt  Neo said it that Nchidzi had bought a car.
            \z
            
Moreover, all matrix verbs that take subjunctive or indicative complements require that the CCC be either AGR-\textit{ti}, \textit{kuti}, or \textit{kuyi}, an observation most naturally stated as (\ref{6}) rather than a semantically inappropriate claim that every finite complement clause is introduced by the verb meaning `say'.\footnote{There are some verbs that take a slightly different \textit{–ti}-based `as that' complementizer (similar to one found in Kinande) which we do not discuss.
\ea \gll bo-Nchidzi  ba-no-zwi-tham-a  se-u-nga-ti	  ba-no-lil-a\\
      \&-Nchidzi   \mc{sm}.c2-\mc{prs-rfm}-make-fv  as-\mc{sm}.c1-as-that  \mc{sm}.c2-\mc{prs}-cry-fv\\
     \glt Nchidzi and others are pretending to be crying. \\
     Or: Nchidzi and others are pretending like they are crying.\z}  

\ea \label{6}
\textup{In Ikalanga, C is obligatory for any non-infinitival complement clause, unless the 	main verb is} \textit{-ti}.
\z

We conclude that \textit{–ti}-based C in Ikalanga is not a verb, however closely it is historically related to the verb \textit{–ti}, and \textit{–ti}-based complementizers are fully distinguishable from the verb \textit{-ti}. We return to the relationship between –ti-based C and the verb –ti in \sectref{sec:letsholo:3.4}, however. 

\subsection{Sensitivity of AGR-\textit{ti} to Tense}

Ikalanga departs from what we know of the other C-agreement languages in that AGR-\textit{ti} is sensitive to the tense of the matrix verb. However, as just remarked, AGR-\textit{ti} morphology is often impoverished; Tense agreement on AGR-\textit{ti} only matches the matrix verb for remote past \textit{-aká-}. For any other matrix tense there is no tense affix on AGR-\textit{ti}. Where \textit{–ti} cannot bear \mc{pst2}, the non-agreeing \textit{kuti} is used.

\ea 
\ea[]{ \label{7a} \gll Maria w-ákà-dw-a Nchídzì  (á)-kà-tì    á-bík-è\\
	Maria \mc{sm}.c1-\mc{pst}2-tell-fv 	Nchidzi   \mc{sm}.c1-\mc{pst}2-that  \mc{sm}.c1-cook-\mc{sbjv}\\
	\glt Maria told Nchidzi that he should cook.}
    
\ex[]{ \label{7b} \gll	bo-Maria  b-ákà-dw-à   Nchídzì   bá-kà-tì   á-bík-è\\
	\&-Maria  \mc{sm}.c2-\mc{pst}2-tell-fv Nchidzi \mc{sm}.c2-\mc{pst}2-that \mc{sm}.c1-cook-\mc{sbjv}\\
  	\glt Maria and others told Nchidzi that he should cook.} \z \z 


\ea
\ea[]{\label{8a} \gll Néó  w-á-zwí-bùzw-à (á)-kà-tì  à  Nchídzì w-ákà-téng-á  lórì\\
           Neo  \mc{sm}.c1-\mc{pst}1-\mc{rfm}-ask-fv \mc{sm}.c1-\mc{pst}2-that 	\mc{q}-part Nchidzi \mc{sm}.c1-\mc{pst}2-buy-fv car\\
	\glt Neo wondered whether Nchidzi had bought a car.} 
    
 \ex[*]{\gll Néó  ú-nò-zwí-bùzw-à (á)-kà-tì à  Nchídzì w-á-kà-téng-á lórì\\
	Neo	\mc{sm}-c1-\mc{prs}-\mc{rfm}-ask-fv  \mc{sm}.c1-\mc{pst}2-that \mc{q}-part Nchidzi \mc{sm}.c1-\mc{pst}2-buy car\\
	\glt Neo wonders whether Nchidzi had bought a car.}
    
  \ex[]{\gll Néó  ú-nòò-zwí-bùzw-à  kuti à  Nchídzì w-ákà-téng-á lórì  			tshwá\\
		Neo  \mc{sm}.c1-\mc{fut}-\mc{rfm}-ask-fv that  \mc{q}-part  Nchidzi \mc{sm}.c1-\mc{pst}2-buy-fv car  new\\
		\glt Neo will wonder whether Nchidzi bought a new car.}
\z  \z 

Thus in addition to the phi-features of the matrix subject, the shape of AGR-\textit{ti} is also sensitive to tense.

\subsection{Sensitivity to Voice}

When a verb taking a clausal complement is passivized, the form of the CCC is typically \textit{kuyi}, which does not permit any subject or tense agreement ((\ref{9b}) is after ID654 in Afranaph), but some speakers (probably older ones) also accept agreeing AGR-\textit{yi} (which we have verified allows agreement for all persons and noun classes, not illustrated here).

\ea  
\ea \label{9a} \gll	(ìmì) nd-àká-dw-à Nchídzì ndí-kà-tì á-téng-é lórì.\\
	I \mc{sm}.c1.1\textsuperscript{st}-\mc{pst}2-tell-fv Nchidzi \mc{sm}.c1.1\textsuperscript{st}-\mc{pst}2-that \mc{sm}.c1-buy-\mc{sbjv}	car\\
	\glt I told Nchidzi that he should buy a car.

\ex \label{9b} \gll (ìmì) nd-àká-dw-íw-à   kù-yí     Mary à-á-tó-ndí-d-á\\
     	 I  \mc{sm}.c1.1\textsuperscript{st}-\mc{pst}2-tell-\mc{pass}-fv \mc{sm}.c1.1\textsuperscript{st}-that Mary
         \mc{neg}-\mc{sm}.c1-\mc{pst}1-\mc{om}.c1.1\textsuperscript{st}-like-fv\\
     \glt I was told that Mary did not like me.

\ex \label{9c} \gll ìmì nd-àká-dw-íw-à  ndì-yí Mary à-á-tó-ndí-d-á\\
         I \mc{sm}.c1.1\textsuperscript{st}-\mc{pst}2-tell-\mc{pass}-fv \mc{sm}.c1.1\textsuperscript{st}-that Mary
         \mc{neg}-\mc{sm}.c1-\mc{pst}1-\mc{om}.c1.1\textsuperscript{st}-like-fv\\
      	\glt I was told that Mary did not like me.	
\z \z

Even verbs like \textit{-budz}- that allow both \textit{kuti} and AGR-\textit{ti} in the active favor the \textit{-yi} form when the immediately superordinate verb is passivized, though \textit{kuti} is always possible, as in (\ref{10a}).  Verbs that do not allow AGR-\textit{ti} like \textit{-dum}- also can take \textit{kuyi} (but not \mc{agr}-\textit{yi})  when they are passivized - compare (\ref{10c},d).

\ea
\ea \label{10a} \gll Néó ú-nò-èmùl-à kù-búdz-íw-á kùyí/kùtí	á-tèng-é lórì\\
	Neo	\mc{sm}.c1-\mc{prs}-wish-fv	\mc{inf}-tell-\mc{pass}-fv	that	\mc{sm}.c1-buy-\mc{sbjv}	car\\
	\glt Neo wishes to be told to buy a car. 

\ex \label{10b} \gll Néó  ú-nò-èmùl-à kù-búdz-íw-á è-yí	 á-tèng-é lórì\\
	Neo	\mc{sm}.c1-\mc{prs}-wish-fv	\mc{inf}-tell-\mc{pass}-fv  \mc{sm}.c1-that \mc{sm}.c1-buy-\mc{sbjv}	car\\
	\glt Neo wishes to be told to buy a car.

\ex \label{10c} \gll Néó ú-nò-dùm-à kùyí/kùtì Nchídzì  w-ákà-bál-á búkà\\
	Neo	 \mc{sm}.c1-\mc{prs}-believe-fv that Nchidzi   \mc{sm}.c1-\mc{pst}2-read-fv book.c9\\
	\glt Neo believes that Nchidzi read a book.

\ex \label{10d} \gll ku-ó-dùm-ìw-à kùyí/kùtì  Nchídzì	w-ákà-bál-á búkà\\
	\mc{sm}.c17-\mc{prs}-believe-\mc{pass}-fv that Nchidzi	\mc{sm}.c1-\mc{pst}2-read-fv book.c9\\
	\glt It is believed that Nchidzi read a book.\footnotemark
    \z\z
\footnotetext{The \mc{sm}.c17 appears to be \textit{ku}- where /\textit{u}/ becomes a glide before /\textit{a}/ and deletes before /\textit{o}/. Nothing turns on this.} 

A number of points emerge in this data. First, both \textit{kuti} and \textit{kuyi} Cs are possible for passivized verbs and for some speakers AGR-\textit{yi} is also possible. However, AGR-\textit{yi} is not normally possible when the matrix verb is not passivized. Also, notice that the forms like (\ref{10b}) suggest that agreement for voice and agreement with the `subject' are both distinct from tense agreement. (\ref{10b}) shows c1 agreement on AGR-\textit{yi} even though the infinitive has no subject agreement – it is marked only with the invariant c17 marker for infinitives. Thus the C must be agreeing with the PRO subject of the `to be told' infinitive, determined to be c1 because it is controlled by \textit{Neo}. This fact generalizes to AGR-\textit{ti}, which also can agree for noun class with a controlled infinitival subject. 

\ea \label{11} \gll	Néó	wá-kà-bé-è-shák-à kù-dw-á bàìsáná  è-tí bá-tèng-é 		lórì\\
	Neo	\mc{sm}.c1-\mc{pst}2-\mc{aux}-\mc{sm}.c1-want-fv \mc{inf}-tell-fv c2.boys \mc{sm}.c1-that
    \mc{sm}.c2-buy-\mc{sbjv}	car\\
    \glt Neo wanted to tell the boys that they should buy a car.\z

Subject clauses always take \textit{kuti}. When an active verb like –\textit{sup}-, `prove',  that takes only \textit{kuti} in its complement clause is passivized, it can take either \textit{kuti} or \textit{kuyi} in its postverbal clausal complement. However, when the passivized clause is in sentential subject position (as marked by subject agreement on the verb), the C must be \textit{kuti}, as in (\ref{12c}), and it cannot be \textit{kuyi}.

\ea
\ea \label{12a}\gll nsèkísì w-ákà-súp-à kùtì Nchídzì w-ákà-kwíb-á márí\\
	prosecutor.1 \mc{sm}.c1-\mc{pst}2-prove-fv that Nchidzi
    \mc{sm}.c1-\mc{pst}2-steal-fv c6.money\\
	\glt The prosecutor proved that Nchidzi stole the money.

\ex \label{12b} \gll ku-ákà-súp-íw-à kùtì/kùyí Nchídzì w-ákà-kwíb-á márí\\
	c17-\mc{pst}2-prove-\mc{pass}-fv that Nchidzi \mc{sm}.c1-\mc{pst}2-steal-fv 			c6.money\\
	\glt It was proved that Nchidzi stole money.

\ex \label{12c} \gll kùtì Nchídzì w-ákà-kwíb-á márí kw-ákà-súp-íw-à né nsèkísì \\
 That Nchidzi \mc{sm}.c1-\mc{pst}2-steal-fv c6.money c17-\mc{pst}2-prove-\mc{pass}-fv 	by	prosecutor\\
 \glt That Nchidzi stole the money was proved by the prosecutor. \z \z

This shows that the distribution of \textit{kuyi} and AGR-\textit{ti} is sensitive to voice. It suggests that \textit{kuti} is necessary in subject position because passive does not c-command it. Notice further, that –\textit{sup}- is a verb that does not take AGR-\textit{ti}. Thus, even verbs that only co-occur with \textit{kuti} in their active form also license \textit{kuyi}. This shows that the \textit{kuti/kuyi} alternation is independent of the \textit{kuti}/AGR-\textit{ti} alternation. The agreement and distributions of \textit{–ti} and \textit{–yi} are summarized in Table \ref{tab1} and Table \ref{tab2}, respectively =

\begin{table}
\begin{tabular}{lcc} 
 \lsptoprule
 Agreement features & \textit{-ti} & \textit{-yi}  \\ \midrule
 Subject Agr & \ding{52} & \ding{52}  \\
 Tense Agr & \ding{52} & \ding{52}  \\ 
 \lspbottomrule
\end{tabular}
\caption{Summary of the C-agreement morphemes}
\label{tab1}
\end{table}

\begin{table}
\begin{tabular}{llcc} 
 \lsptoprule
 \multicolumn{2}{l}{Distribution} & \textit{-ti} & \textit{-yi}  \\ \midrule
 Voice: & Active  & \ding{52} & \ding{56} \\
    	  & Passive & \ding{52} & \ding{52}  \\\tablevspace
Mood: & Subjunctive & \ding{52} & \ding{52}  \\  
		& Interrogative & \ding{52} & \ding{56} \\\tablevspace
\multicolumn{2}{l}{Introducing clausal complement} & \ding{52} & \ding{56} \\ 
\lspbottomrule
\end{tabular}
\caption{Complementizer distribution of \textit{–ti} in comparison with \textit{–yi}}
\label{tab2}
\end{table}

\subsubsection{Suppletion of \textit{-ti}}

What makes it clear that there is an agreement relation between passive voice and \textit{kuyi} or AGR-\textit{yi}, however is the –\textit{yi} shape itself. When main verb –\textit{ti} is passivized, the root is suppletive, taking the form -\textit{yi}, as illustrated in (\ref{13b}), and unlike the AGR-\textit{yi} C, it can be inflected for tense.

\ea
\ea \gll ku-ákà-yì Nchídzì à-á-pò\\
	\mc{sm}.c17-\mc{pst}2-say.\mc{pass} Nchidzi	Neg-\mc{sm}.c1-there\\
	\glt It was said that Nchidzi was not there.

\ex \label{13b} \gll kó-ò-yì Nchídzì	á-tèng-è lórì tshwá\\
	\mc{sm}.c17-\mc{fut}-say.\mc{pass}	Nchidzi	\mc{sm}.c1-buy-\mc{sbjv}	car	new\\
     \glt It will be said that Nchdizi should buy a new car. \z \z

Thus it seems that AGR-\textit{yi} could be a form of suppletion parallel to the \textit{–ti/-yi} alternation of the main verb –\textit{ti}. It is plausible to see this as a form of morphological concord or agreement, especially since phi-agreement on AGR-\textit{yi} is also possible when it is a complementizer (as in (\ref{14})). Moreover, AGR-\textit{yi}, just like AGR-\textit{ti}, also permits tense agreement for \mc{pst2}.  

\ea\label{14} \gll Néó	w-ákà-dw-iw–á á-ká-yì á-tèng-é lórì\\
	Neo	\mc{sm}.c1-\mc{pst}2-tell-\mc{pass}-fv  \mc{sm}.c1-\mc{pst2}-that \mc{sm}.c1-buy-\mc{sbjv}	car\\
	\glt Neo was told to buy a car/ Neo was told that she should buy a car. \z

It appears that verbs that most favor AGR-\textit{ti} are those that permit AGR-\textit{yi}, but we have not checked every case.

\subsubsection{\textit{Kuyi} without passive concord}

\textit{Kuyi} can be used for emotive complement clauses in the absence of passive morphology, but in these cases, it appears to simply be an infinitival passive complement of the matrix verb. Thus no complementizer preceding infinitival \textit{kuyi} is expected. \textit{Kuyi} is functioning as a main verb, and as such, it is the one verb that does not introduce a C before its complement clause. Consistent with the infinitival passive analysis, AGR-\textit{yi} is not possible for any of (\ref{15}a-d) as illustrated by (\ref{15}e). %last line does not make sense

\ea \label{15}
\ea[]{\gll Néó  ú-nòò-chénám-à kù-yí á-tèng-é lórì\\
	Neo  \mc{sm}.c1-\mc{fut}-surprised-fv	\mc{inf}-say.\mc{pass} \mc{sm}.c1-buy-\mc{sbjv} car\\
	\glt Neo will be surprised to be told that she should buy a car.}

\ex[]{ \gll Néó ú-nòò-gwádzík-à kù-yí á-tèng-é lórì\\
	Neo \mc{sm}.c1-\mc{fut}-hurt-fv \mc{inf}-say.\mc{pass} \mc{sm}.c1-buy-\mc{sbjv}	car\\
	\glt Neo will be hurt to be told that she should buy a car.}

\ex[]{\gll Néó  ú-nòò-sháth-á kù-yí á-tèng-é lórì\\
	Neo  \mc{sm}.c1-\mc{fut}-happy-fv	\mc{inf}-say.\mc{pass} \mc{sm}.c1-buy-\mc{sbjv} car\\
	\glt Neo will be happy to be told that she should buy a car.}

\ex[]{ \gll Néó  ú-nòò-d-á kù-yí á-tèng-é lórì\\
	Neo  \mc{sm}.c1-\mc{fut}-happy-fv	\mc{inf}-say.\mc{pass} \mc{sm}.c1-buy-\mc{sbjv} car\\
	\glt Neo will be happy to be told that she should buy a car.}
    
\ex[*]{\gll Néó ú-nòò-chénám-à	è-yí  á-tèng-é lórì\\
         Neo  \mc{sm}.c1-\mc{fut}-surprise-fv  \mc{sm}.c1-say.\mc{pass} 		\mc{sm}.c1-buy-\mc{sbjv} car\\
		\glt Neo will be surprised to be told that she should buy a car.}
\z \z 

Other cases that may be instances where \textit{kuyi} acts like a passivized infinitival complement may include instances where it functions as an evidential. 

\ea \label{16}  \gll Néó  w-ákà-wh-á kùtì/kùyí Nchídzì w-ákà-téng-á lórì tshwá\\
	Neo	\mc{sm}.c1-\mc{pst}2-hear-fv  that  Nchidzi \mc{sm}.c1-\mc{pst}2-buy-fv car	new\\
	\glt Neo heard that Nchidzi had bought a new car.
\z

In these contexts, it appears that the truth of the proposition introduced by \textit{kuyi} may have some sort of evidential import. Verbs like \textit{wh}- `hear' can also take \textit{kuti}, as in (\ref{16}), but when \textit{kuyi} is used, the source of the information is evaluated differently. While (\ref{16}) with \textit{kuti} does not indicate how the information came to the matrix subject hearer (someone could have made this claim directly to the matrix subject), with \textit{kuyi} it indicates that that the matrix subject heard that it was said that Nchidzi had bought a car. This indicates further distance between the matrix subject and the evidence.\footnote{Moreover, it is possible for a source phrase to be licensed for (\ref{16}), though it is not clear whether it is \textit{kuyi} or `hear' itself that makes the source phrase possible, since \textit{kuti} is also possible here. 

\ea \gll Néó w-ákà-wh-á kùyí Nchídzì  w-ákà–téng-á lórì ndí  John\\
Neo \mc{sm}.c1-\mc{pst}2-hear-fv	that Nchidzi \mc{sm}.c1-\mc{pst}2-buy-fv car by John\\
\glt Neo heard from John that Nchidzi had bought a car. \z 
} 

At present, it is not clear that the infinitival passive complement analysis is appropriate for (\ref{n17}a,b), that is, whether the agreement is about what is said or about what is believed. These examples require further study. 

\ea\label{n17} \ea \gll Néó w-ákà-dúm-à kùyí Nchídzì w-ákà-téng-á lórì tshwá\\
	Neo	\mc{sm}.c1-\mc{pst}2-agree-fv that  Nchidzi  \mc{sm}.c1-\mc{pst}2-buy-fv car new\\
	\glt Neo agreed that Nchidzi had bought a new car.

\ex  \gll ìngwì 	m-ó-lándùl-à kùyí Nchídzì  w-ákà-téng-á lórì tshwá\\
	You.pl \mc{sm}.c2-\mc{prs}-refute-fv that Nchidzi  \mc{sm}.c1-\mc{pst}2-buy-fv	car new\\
	\glt You (PL) refute that Nchidzi has bought a new car.  \z \z  


Apart from instances where \textit{kuyi} is a passivized infinitive (and cases like (\ref{n17}a,b)), however, the concord approach to the appearance of \textit{kuyi} in place of AGR-\textit{ti} still seems like the best generalization, especially given the sentential subject facts. The generalization in (\ref{18}) seems to capture the central pattern.

\ea \label{18} \textup{The –\textit{ti}-based complementizers are allomorphs}. \z

In addition to the fact that all of these \textit{–ti/-yi}-based forms are almost in complementary distribution (if the optionality of (\ref{10a}),  (\ref{10d}) and (\ref{12b}) arise from competing analyses), they also all occur in the same high clausal position, above the question particle that can introduce a matrix yes-no question as well as an indirect yes-no question. The verb -\textit{buzw}- can take either AGR-\textit{ti} (as in (\ref{8a})) or invariant \textit{kuti}. The presence of the Q-particle, also used in matrix questions, is obligatory.

\ea
\ea[]{\label{19a}\gll Néó w-ákà-zwí-bùzw-à (à)-kà-tì à Nchídzì  w-ákà-téng-á lórì  		tshwa\\
	Neo \mc{sm}.c1-\mc{pst}2-\mc{rfm}-ask-fv \mc{sm}.c1-\mc{pst}2-that \mc{q}-part Nchidzi \mc{sm}.c1-\mc{pst}2-buy-fv car new\\
	\glt Neo wondered whether Nchidzi had bought a new car.}

\ex[]{\label{19b}\gll Néó w-ákà-zwí-bùzw-à kùtì à Nchídzì w-ákà-téng-á lórì tshwá\\
	Neo \mc{sm}.c1-\mc{pst}2-\mc{rfm}-ask-fv that \mc{q}-part Nchidzi \mc{sm}.c1-\mc{pst}2-buy-fv 	  car new\\
	\glt Neo wondered whether Nchidzi had bought a new car.}

\ex[*]{\gll	Néó w-ákà-zwí-bùzw-à kùtì Nchídzì w-ákà-téng-á lórì tshwá\\
      Neo \mc{sm}.c1-\mc{pst}2-\mc{rfm}-ask-fv that	Nchidzi c1-\mc{pst}2-buy-fv car new\\
	\glt Neo wondered whether Nchidzi had bought a car.}

\ex[]{\gll Néó  w-ákà-búzw-ìw-á kùyí 	à	Nchídzì  w-ákà-téng-á lórì\\
	Neo \mc{sm}.c1-\mc{pst}2-ask-\mc{pass}-fv that \mc{q}-part	Nchidzi \mc{sm}.c1-\mc{pst}2-buy-fv	car\\
	\glt Neo was asked whether Nchidzi had bought a car.}\z\z

\subsection{More on the morphology of AGR-\textit{ti}}

There are some deformations in the shape of AGR-\textit{ti} that do not simply result from morphologically composing the SM, the (remote past) tense, and \textit{–ti}. Although we alert the reader to them here, they do not change the basic generalization, namely, that the possible forms of AGR-\textit{ti} (and AGR-\textit{yi}) are entirely predictable from the SM and tense of the matrix verb. 

We have shown that remote past –\textit{(a)ká}- shows up on AGR-\textit{ti} when it also appears on the matrix verb. However, if the matrix verb is negated then the shape of matrix remote past tense is affected, surfacing as \textit{zo}- instead of –\textit{ká}. In this case, \textit{ká}- can still appear on AGR-\textit{ti}, as in (\ref{20b}) .

\ea \label{20}
\ea \label{20b} \gll Néó à-á-zò-dw-à Nchídzì á-kà-tì á-tèng-è lórì\\
	Neo	\mc{neg}-\mc{sm}.c1-\mc{neg.pst}-tell-fv Nchidzi	\mc{sm}.c1-\mc{pst}2-that \mc{sm}.cl-buy-\mc{sbjv}	car\\
	\glt Neo did not tell Nchidzi that he should buy a car.

\ex\label{20a} \gll  Néó à-á-zò-dw-à Nchídzì	è-tí á-tèng-è lórì\\
    Neo	\mc{neg}-\mc{sm}.c1-\mc{neg.pst}-tell-fv Nchidzi	\mc{sm}.c1-that	\mc{sm}.cl-buy-\mc{sbjv}	car\\
	\glt Neo did not tell Nchidzi that he should buy a car.

\ex \label{20c} \gll (ìwè) à-ú-zò-dw-à Nchídzì  ú-tí á-tèng-è lórì\\
	You \mc{neg}-\mc{sm}.c1.2\textsuperscript{nd}-\mc{neg.pst}-tell-fv Nchidzi \mc{sm}.c1.2\textsuperscript{nd}-that \mc{sm}.c1-buy-\mc{sbjv} car\\
    \glt You did not tell Nchidzi that he should buy a car. \z \z  

In (\ref{20a}) where \textit{ka}- optionally does not appear, SM \textit{a}- will regressively assimilate raising to front \textit{e}- (we do not know why the /\textit{a}/ of \textit{ka}- does not undergo regressive assimilation before the /\textit{i}/ of \textit{kati}). The high \textit{u}- of SM.c1.2\textsuperscript{nd}  would not be affected, as in (\ref{20c}).

The paradigm for the SM followed by the complementizer –\textit{ti} is presented in Table \ref{tab3} (the tone of –\textit{ti} is influenced by its context).\footnote{See also \citet[73]{Chebanne2010a}. For more complete accounts of subject agreement and their relations to tense, see \citet{Mathangwane1999} and \citet{Letsholo2002}.} 

\begin{table}
\begin{tabular}{llllll} 
 \lsptoprule
 \mc{sm}.c1.3\textsuperscript{rd} & \mc{sm}.c1.1\textsuperscript{st} & \mc{sm}.c1.2\textsuperscript{nd} & \mc{sm}.c2.3\textsuperscript{rd} & \mc{sm}.c2.1\textsuperscript{st} & \mc{sm}.c2.2\textsuperscript{nd}    \\\midrule
 ú/wá/á  & ndì &  ù & bá  & tì  & mù  \\ 
 \lspbottomrule
\end{tabular}
\caption{SM paradigm}
\label{tab3}
\end{table}

It is not clear whether or not SM.c1, which is 3\textsuperscript{rd} if unmarked, is \textit{wá}- when it precedes PST configuration or if it is just /\textit{ú}/ fused with the PST1 \textit{-a}- that follows it (as suggested in \citealt{Letsholo2002}), raising the tone on /\textit{a}/. For subjunctive, interrogative and negated sentences, c1 C-agreement appears to have the form \textit{a}-.

The paradigm for SM-\textit{ti} combinations when there is no PST2 matching with the AGR-\textit{ti} is as in Table \ref{tab4}. When the matrix has SM.c2, the /\textit{a}/ of \textit{ba}- regressively assimilates to /\textit{e}/ before the /\textit{i}/ of -\textit{ti}, as does SM.c1 when it is \textit{a}-.

\begin{table}
\begin{tabular}{llllll} 
 \lsptoprule
 \mc{sm}.c1-\textit{ti} & \mc{sm}.c1.1\textsuperscript{st}-\textit{ti} & \mc{sm}.c1.2\textsuperscript{nd}-\textit{ti} & \mc{sm}.c2-\textit{ti} & \mc{sm}.c2.1\textsuperscript{st}-\textit{ti} & \mc{sm}.c2.2\textsuperscript{nd}-\textit{ti}\\ \midrule
 ú/è-tí  & ndì-tí &  ù-tí & bè-tí  & tì-tí  & mù-tí  \\ \lspbottomrule
\end{tabular}
\caption{SM-\textit{ti} paradigm without PST2}
\label{tab4}
\end{table}

If \textit{ká} occurs on AGR-\textit{ti} matching matrix PST2, the outputs in Table \ref{tab5} are possible (note that the SM we are treating as \textit{a}- is the only one that is optional before \textit{-ka}).

\begin{table}
\begin{tabular}{llllll} 
 \lsptoprule
 \mc{sm}.c1.3\textsuperscript{rd} & \mc{sm}.c1.1\textsuperscript{st} & \mc{sm}.c1.2\textsuperscript{nd} & \mc{sm}.c2.3\textsuperscript{rd} & \mc{sm}.c2.1\textsuperscript{st} & \mc{sm}.c2.2\textsuperscript{nd}    \\ 
 -\mc{pst}2-\textit{ti}  &  -\mc{pst}2-\textit{ti}&  -\mc{pst}2-\textit{ti}&  -\mc{pst}2-\textit{ti}&  -\mc{pst}2-\textit{ti}&  -\mc{pst}2-\textit{ti} \\ \midrule
(á)-kà-tì & hàtì/ndì-kà-tì & ù-kà-tì & bá-kà-tì & tí-kà-tì & mú-kà-tì   \\ \lspbottomrule
\end{tabular}
\caption{SM-PST2-\textit{ti} paradigm}
\label{tab5}
\end{table}

Where \textit{ka} cannot appear, the corresponding forms in Table \ref{tab2} are possible. We assume that \textit{hàtì} is a suppletive option.


We stress, however, that the morphological details of the forms of AGR-\textit{ti} do not obscure the main point that interests us in this essay, namely, that the possible forms of AGR-\textit{ti} are entirely determined by its relation to matrix tense, phi-features and voice.\footnote{There are one or two deviations from the facts as described that we do not understand, such as the following example, where \textit{ka} appears in AGR-\textit{ti} when we do not expect it to.

\ea\label{24} \gll (ìwè)	ù-nó-léb-á ù-kà-tì Nchídzì w-ákà-téng-á lórì\\				
you \mc{sm}.c1.2\textsuperscript{nd}-\mc{prs}-say-fv  \mc{sm}.c1.2\textsuperscript{nd}-\mc{prs}-that  Nchidzi \mc{sm}.c1-\mc{pst}2-buy-fv	car\\ 
	\glt You are saying that Nchidzi bought a car. \z

It is possible that this is an instance where the suppletive \textit{kàtì} form for main verb PRS-\textit{ti} is echoed on AGR-\textit{ti}, in which case we also see tense concord for PRS, at least in these cases. We have no more to say about such examples, though they deserve more research. It has been suggested to us that \textit{-ka}- could be understood as a consecutive marker of some kind, both in its position on the verb stem and its position on the CCC. It is also the case in Ikalanga that \textit{ka}- can be a consecutive marker as in other Bantu languages, as illustrated in the examples below (tones omitted):

\ea\label{FN7.1} \gll Neo w-aka-tem-a miti, ka	kubunganya	maswazwi, ka a-pisa\\
      Neo \mc{sm}.c1-\mc{pst}2-cut-fv c4.tree \mc{cons} gather c6.branches \mc{cons} \mc{om}.c6-burn\\
      \glt Neo cut the trees, gathered together the branches and burned them.
      
 \ex\label{FN7.2}  \gll Ingwi	m-aka-tem-a	miti, mu-ka kubunganya maswazwi, mu-ka-a-pis-a\\
       \mc{prn}c2.2\textsuperscript{nd}	\mc{sm}.c2.2\textsuperscript{nd}-\mc{pst}2-cut-fv	c4.tree	\mc{sm}.c2-\mc{cons} gather	 c6.branches 			\mc{sm}.c2.2\textsuperscript{nd}-\mc{cons}-\mc{om}.c6-burn-fv\\
	\glt The men cut the trees, gathered together the branches and burned them.  \z

We are not sure if the events in (\ref{24}) can be said to be consecutive in nature in the way that examples (\ref{FN7.1}) and (\ref{FN7.2}) are. We do not see a way to relate these observations to \textit{ka} on C.}


\subsection{The paucity of AGR-\textit{ti}-taking verbs vs. invariant \textit{kuti}}

It would seem that selection for AGR-\textit{ti} is a largely lexical affair limited to only a few verbs, including the roots -\textit{buzw}- `ask', and -\textit{leb}- `speak, say', -\textit{budz}- `tell' (\ref{25c})  and \textit{-dw}- `tell' (an instruction), which have all been exemplified, in addition to -\textit{dum}- `concede/agree/believe', -\textit{landul}- `disagree', and  \textit{-alakan}- `think' (not illustrated here). By comparison, recall that Diercks cites at least 19 Lubukusu verbs which permit AGR-\textit{li}.

\ea
\ea \gll Néó  w-ákà-dúm-à (à)-kà-tì Nchídzì  w-ákà-téng-á lórì\\
	Neo \mc{sm}.c1-\mc{pst}2-agree/concede-fv \mc{sm}.c1-\mc{pst}2-that Nchidzi
    \mc{sm}.c1-\mc{pst}2-buy-fv car\\
	\glt Neo conceded that Nchidzi had bought a car.

\ex \gll Néó w-ákà-lándúl-à (à)-kà-tì Nchídzì à-á-zò-téng-á lórì \\
	Neo \mc{sm}.c1-\mc{pst}2-disagree-fv \mc{sm}.c1-\mc{pst}2-that Nchidzi \mc{neg}-\mc{sm}.c1-\mc{prs}-buy-fv car \\
	\glt Neo disagreed (saying?) that Nchidzi has not bought a new car.

\ex \label{25c} \gll nd-àká-kú-búdz-à ndì-tí ú-tèng-è lórì\\
	\mc{sm}.c1.1\textsuperscript{st}-\mc{pst}2-\mc{om}.c1.2\textsuperscript{nd}-tell-fv \mc{sm}.c1.1\textsuperscript{st}-that
    \mc{sm}.c1.2\textsuperscript{nd}-buy-\mc{sbjv} car\\
	\glt I told you to buy a car. \z\z

Consultants differ as to whether or not –\textit{dw}- can also take \textit{kuti} complements, but they agree that \textit{-leb}-, \textit{-budz}-,and \textit{-buzw}- can (see (\ref{19a},b) for \textit{buzw}-).

\ea
\ea \gll Néó w-ákà-léb-á kùtì  Nchídzì w-ákà-téng-á lórì tshwá\\
	Neo	\mc{sm}.c1-\mc{pst}2-tell-fv that Nchidzi \mc{sm}.c1-\mc{pst}2-buy-fv car new\\
	\glt Neo said that Nchidzi bought a car.

\ex \gll (ìmì) nd-àká-kú-búdz-à kùti/hà-tì ú-tèng-è lórì\\
      I \mc{sm}.c1.1\textsuperscript{st}-\mc{pst}2-\mc{om}.c1.2\textsuperscript{nd}-tell-fv that/\mc{sm}.c1.1\textsuperscript{st}-that \mc{sm}.c1.2\textsuperscript{nd}-buy-\mc{sbjv} car\\
	\glt I told you to buy a car. \z\z

However, as we saw in the case of (\ref{12a},b), whatever determines the distribution of \textit{kuyi} concord applies more generally than whatever determines which verbs allow AGR-\textit{ti}.

So far, we have no indication that there is a successful generalization about verbs that take \textit{kuti}, other than that \textit{kuti} may be a default for verbs that take indicative or subjunctive complements when AGR-\textit{ti} is not available. In some cases it is still available even when AGR-\textit{ti} is a possible choice. Evidence that \textit{kuti} is, at least in some contexts, a default form, is that it is always used for sentential subjects, where the clause determines \mc{sm}.c17, even for a verb like \textit{-dum}- that we know to be an AGR-\textit{ti}-taking verb (in contrast to \textit{-sup}- in (\ref{12b}).

\ea\label{n23} \gll kùtì bàthù bà-njínjí à-bá-tó-thòph-à kw-áká-dúmí-gw-àn-à\\
      that c2.people c2.Agr-many \mc{neg}-\mc{sm}.c2-\mc{neg.prs}-vote-fv \mc{sm}.c17-\mc{pst}2-agree-\mc{pass}-\mc{rcm}-fv\\
	\glt That many people don't vote was agreed on. \z

As noted earlier, this suggests that in clausal subject position, \textit{kuti} is in a position where there is no c-commanding tense, voice, or subject to agree with.


\subsection{The status of the allomorphy hypothesis}

We suggested in (\ref{18}) that all \textit{-ti}-based Cs are allomorphs of each other, and at this point, there is reason to believe that the distribution of AGR-\textit{ti}, AGR-\textit{yi}, and \textit{kuti} is predictable from the voice, tense and subject phi-features of the matrix verb, once we determine which predicates permit AGR-\textit{ti}. We have treated \textit{kuyi} as potentially in concord with passivized verbs in some contexts, but in others where it appears to be evidential, we treat it as the infinitival passive form of AGR-\textit{ti}.

\section{Theoretical questions about Locality}

If syntactic relations like agreement are always phase-internal, a central tenet in minimalist theorizing, then we must determine whether all the agreement relations we posit are phase-local. In typical phase-based accounts, C and v or Voice are the phase heads. There is discussion in the literature (\citealt{Kratzer1996,Harley2013,Legate2012,Safir2017})  concerning whether the phase edge is the functional head v, which determines that a root is verbal or a higher Voice head (which takes vP as a complement). 
Following Kratzer and others, we assume that Voice selects vP and can introduce the external argument (EA). Alternatively, it might be assumed that the EA is introduced in Spec vP and can raise to Spec VoiceP (in parentheses in (ref{28})), thus inhabiting the phase edge.\footnote{\label{fn8} It is possible that VoiceP is always the phase edge, but only occasionally introduces the EA. We will assume that only some light verbs are banned from introducing their EA in Spec V, such as \mc{caus}, while most other verbs assign their EA in Spec vP. Arguments against assigning EA to Spec vP in \citet{Pylkkaenen2008} are primarily based on evidence about \mc{caus}. Any argument exiting VoiceP will have to pass through its edge, however. See also \citet{Safir2017}  for discussion of the left periphery of the verbal domain.}  Thus the maximal span of locality extends from the edge of the VoiceP downward to the edge of the CP (C, Spec CP, and adjunctions to CP, if these are different from Spec CP) The Voice phase does not include the higher (bolded) T or anything above it, nor anything below AGR-\textit{ti} (C) such as the lower (bolded) TP. 


\ea \label{28} \textup{[\textbf{T}  { }  [\textsubscript{VoiceP} (EA) [\textsubscript{Voice} Voice [\textsubscript{vP} EA v …[\textsubscript{CP} …AGR-\textit{ti} \textbf{[}\textsubscript{\textbf{TP}} \textbf{…]}] ]] ]]} \z


In 3.1 we examine the phase-internal relations between the EA and Voice with respect to AGR-\textit{ti} and in 3.2-3 we suggest an analysis based on Voice agreement that could instantiate the allomorphy analysis proposed in (\ref{18}). 

	Notice now that matrix T and AGR-\textit{ti} do not share a phase in (\ref{28}). Thus we would not ordinarily expect any agreement relation to hold between those two heads, unless AGR-\textit{ti} or a phrase below Voice that contains AGR-\textit{ti} somehow escapes to the edge of VoiceP. We consider `escape' strategies in sections 3.3-4 and 3.5.


\subsection{Agreement internal to the VoiceP}\label{s3.1}

On the view that a passivized object passes through Spec VoiceP (and perhaps through Spec vP, where v selected by passive Voice does not assign an EA), we may expect that agreeing C in some languages would be sensitive to whatever inhabits the argument position that is most local to the Voice head. In Lubukusu (\ref{29}a,b) (from \citealt[368]{Diercks2013}, glosses ours), the verb agrees with the passivized subject.

\ea\label{29}
\ea \gll (Ese) n-a-bol-el-a Nelsoni \textbf{n-di} ba-keni ba-a-ch-a\\
          I \mc{sm}.c1.1\textsuperscript{st}-\mc{pst}1-say-\mc{appl}-fv c1.Nelson \mc{sm}.c1.1st-that c2-guests
          \mc{sm}.c2-\mc{pst}1-go-fv\\
		\glt I told Nelson that the guests left.

\ex \gll Sammy ka-bol-el-w-a \textbf{a-li} ba-keni b-ol-a\\
		c1.Sammy \mc{sm}.c1-say-\mc{appl}-\mc{pass}-fv c1-that 2-guests \mc{sm}.c2-arrive-fv\\
	\glt  Sammy was told that the guests arrived. \z\z

That such a relationship between Spec VoiceP is possible in both active and passive is what would be expected, but this is not what we find in Kinande, nor is it what we always find in Ikalanga. In Kinande, another C-agreement language, agreeing C (agreeing with matrix subject only) in the active reverts to default non-agreeing form in the passive, which suggests that it is primarily sensitive to passive voice, not agreement with a DP. Thanks to Prof. Philip Ngessimo Mutaka for these Kinande examples.

\ea \label{30}
\ea	Yoháni mwásirisyákumbusy’ 	abakolhw’ 	ati bálwé b’erisom’  echapítre 2
	\gll Yohani mo-a-sirisya-buki-a  a-ba-kolho a-ti ba-lue ba
    e-ri-som-a {e-chapitre 2}\\
	John     mo-\mc{sm}.c1-\mc{fut}-remind-fv \mc{iv}-c2-student \mc{sm}.c1-\mc{comp}    \mc{sm}.c2-\mc{aux} c2.\mc{lk} 		\mc{iv}-\mc{inf}-read-fv {\mc{iv}-chapter 2} \\
	\glt John reminded the students that they should read chapter 2

\ex abakolhó móbásirisyabukibw’ ambu bálwé b’erisom echapítre 2
	\gll a-ba-kolho mo-ba-sirisya-buk-i-bu-a ambu   ba-lue ba
    e-ri-som-a {e-chapitre 2}\\
	\mc{iv}-c2-student mo-\mc{sm}.c2-\mc{fut}-remember-\mc{caus}-\mc{pass}-fv \mc{comp} \mc{sm}.c2-\mc{aux} c2.\mc{lk} 	 \mc{iv}-\mc{inf}-read-fv  {\mc{iv}-chapter 2}\\
    \glt The students were reminded that they should read chapter 2.
    \z \z

However, the shift to \textit{ambu} has a semantic consequence, namely, it removes the speaker’s responsibility for the following proposition, and which echoes a similar distinction related to non-agreeing voice in Ikalanga perception complements. Nonetheless, the shift from the agreeing C to the non-agreeing C suggests that it is passive Voice in Kinande that fails to facilitate C-agreement with the matrix subject.

\subsection{What we know so far}

The strongest theory of the distribution of \textit{kuti}, \textit{kuyi}, AGR-\textit{ti} and AGR-\textit{yi} is that they are allomorphs (apart from the infinitival passive \textit{kuyi}), as this imposes certain analytic requirements that stipulated distributions do not. By definition, the form of the C should be predicted by its syntactic and/or morphological context, so it is a rule that determines the allomorph. After all, the AGR-\textit{yi}/AGR-\textit{ti} alternation is not a predicate specific relation even if the availability of AGR-\textit{ti} is predicate specific and must be stipulated somehow. These, then, are the alternations predicted.

\ea \label{Allo} \textup{The allomorphy theory as driven by Voice}
\ea \textup{AGR-ti alternates with AGR-\textit{yi} when the superordinate verb is active or passive, respectively.}
\ex Kuti \textup{is almost always available as a default where the superordinate verb is active or passive.}
\ex Kuyi \textup{is possible when the superordinate verb is passive} \z \z

We have distinguished two kinds of verbs that take finite CP complements in terms of the C alternation that they permit. There is a small class of verbs that select AGR-\textit{ti} and those verbs participate in the AGR-\textit{ti}/AGR-\textit{yi} alternation.  Just about all verbs can take kuti complements. Passivized, these verbs allow for a \textit{kuyi} C when the CP is postverbal. Some verbs take \textit{kuyi} complements that are understood as impersonal infinitival passives, but \textit{kuyi} is acting as a verb in these cases. 

We have also established certain relations that have to be captured in any analysis of the Iklalanga phenomena. 

\ea\label{n28} 
\ea	\textup{C-Agreement for voice is independent of agreement for tense (see \ref{10b}).}
\ex	\textup{C-Agreement for phi-features with active or passive voice is independent of tense (see \ref{10b} and (\ref{11})).}
\ex \textup{C-Agreement for phi-features is independent of matrix agreement on T (as in (\ref{10b}) and (\ref{11})).}
\ex \textup{C-Agreement for tense is only possible if there is agreement for phi-features.} \z\z 


In what follows, we argue that (\ref{n28}a--c) are captured by tying C-agreement relations to Voice, and we postpone discussion of (\ref{n28}d), which is licensed by a different locality relation, until 3.5. 


\subsection{Modeling Voice agreement with C}
Up to this point, we have been making the case that the relation between Voice and –\textit{ti} complementizers is one of agreement in a descriptive sense, that is, the \textit{–ti/yi} alternation covaries with passive marking on the verb. Although \citet[353-370]{Diercks2010a} discusses and rejects modeling Lubukusu C-agreement as agreement with Voice, we believe modeling C-agreement as a Voice-C relation is the best account of Ikalanga and perhaps correct for Lubukusu as well. As Diercks points out, a Voice-C relation suggests why direct objects or other intervening nominals do not shift agreement to non-subjects, since Voice only agrees with the nominal closest to it, namely, the EA or a passivized nominal that passes through Spec-VoiceP (or Spec-vP). Diercks rejects the Voice-C model, however because he assumes that the causative affix (CAUS) takes VoiceP complements, and if so, the affected subject, rather than the subject of CAUS, should control agreement. Under different assumptions about the position of the EA, Voice-C relation may actually produce the right result. If most predicates assign EAs to Spec vP, but only certain light verbs like CAUS (a species of v) assign their EA in Spec VoiceP (see fn. \ref{fn8}), then CAUS can take a vP complement that has an EA, but would not embed a Voice projection below it. The structure is illustrated in (\ref{n29}).

\ea\label{n29} \textup{[\textsubscript{VoiceP} EA  [ Voice [\textsubscript{vP} [\textsubscript{v}  CAUS [\textsubscript{vP}  EA [\textsubscript{v} V...  [\textsubscript{CP} [C... ]] ]] ]] ]]} \z 

If this approach is viable, then only the causer argument can antecede Voice, so only the causer argument can control C-agreement. This is the right prediction for Lubukusu, but in Ikalanga (\ref{n30}) the introduction of a causative affix blocks C-agreement altogether, perhaps as a form of defective intervention that neither the movement theory nor our agreement theory fully predicts.  

\ea\label{n30} \gll ba-isana	b-aka-buzw-is-a			Neo	mme-abe	kuti	kene b-aka-teng-a		ma-bisi.\\
       c2-boys	\mc{sm}.c2-\mc{pst}2-ask-\mc{caus}-fv	Neo	mother-hers	that whether \mc{sm}.c2a-\mc{pst}2-buy-fv	c6-melons\\
    \glt The boys made Neo ask her mother whether they had bought watermelons. \z 

Thus treating the Voice-C relation as the core of C-agreement has advantages,\footnote{\citet[367-369]{Diercks2010a} rejects the Voice-C relation as a model for C-agreement in Lubukusu in favor of a control analysis, in part because of examples where C-agreement holds for clausal complements to direct object nouns. These cases, also found in Ikalanga, are indeed puzzling, but they are so for all accounts, including the proposal of \citet{Diercks2017a}, discussed in 3.5.}  especially in Ikalanga, where we have morphological evidence that the voice of C covaries with the voice of the immediately superordinate verb

At this point, more technical issues arise as to how `agreement' is to be modeled within a theoretical approach. In minimalist theories, such as \citet{Chomsky2001}, 
agreeing heads with unvalued features are `probes' which search for `goals' (typically nominals) in order to value their (probing) features. This probing operation is called `Agree'. Theorists divide over whether all Agree is downward-looking, upward looking, or both.\footnote{As Diercks \textit{et. al.} (2017) point out, the ``delayed valuation'' of \citet{Carstens2016}, which allows a probe to search `up' if there is no agreement goal locally c-commanded by the probe, appears to predict that nominals intervening between C and the superordinate subject should result in agreement with non-subjects. Carstens argues that the intervention does not occur due to independent factors, but those factors deserve further scrutiny. Both Diercks \textit{et. al.} and the proposal made here do not require additional mechanisms to make the right prediction with respect to this intervention.} They also disagree about whether or not Agree is the operation that ensures that anaphors agree with their antecedents. Following \citet{Rooryck2011}, \citet{Diercks2017a} assume that anaphoric features of AGR-\textit{ti} raise to adjoin to the vP (or VoiceP) phase head where they can probe down to get the features of the EA. Then these agreement features are morphologically realized only in the C position, not the higher vP adjoined position. The inheritance of morphologically realized features downward to a copy does not follow from Agree or copy theory, however. The only overt evidence they offer for C-movement to vP is the behavior of `say' verbs in languages like Ikalanga that don’t allow C to follow them. Although we adopt a version of their analysis for the Lubukusu `say' verb below, the motivation we provide does not justify generalized C-to-vP (or VoiceP) adjunction.

We agree with Diercks \textit{et. al.} that Voice and C are in an antecedent-anaphor relation, but we do not treat the antecedent-anaphor relation as an Agree relation. Rather it is a morphological relation that results from anaphoric elements that find their antecedent features within the same phase, as proposed in \citet{Safir2014}. The relation between Voice and the EA may also be seen as anaphoric. Thus Voice gets phi-features from the EA or whatever passes through its Spec VoiceP, and then an anaphoric voice feature on C is anteceded by Voice along with its agreement features that were anaphorically valued by Spec VoiceP. On this model, phi-agreement on AGR-\textit{ti} and AGR-\textit{yi} only proceeds by virtue of agreement with Voice. When Voice is active, the agreeing C is AGR-\textit{ti} and when Voice is passive, we get AGR-\textit{yi}. Since we could not find a semantic generalization that characterizes the small class of verbs that license agreeing C, selection for agreeing C must be treated as a lexical matter. There are various ways to stipulate this, but how that is done need not require anything special, so we do not address it here.

Our account predicts that agreement on Voice is independent of agreement introduced by finite T. As (\ref{11}) illustrates, this is the correct prediction, since AGR-\textit{ti} agrees with the subject of an infinitive (presumably PRO) in the absence of finite agreement introduced by T.

	There are few verbs, if any, that take finite complements and are completely incompatible with \textit{kuti}. We propose that the \textit{kuti/kuyi} alternation arises when the anaphoric feature on –\textit{ti} has no phi-feature, so the anaphoric feature on –\textit{ti} is only sensitive to passive/active. The absence of an anaphoric feature on -\textit{ti} also yields \textit{kuti}, as in cases where there is a finite sentential subject. The C of the CP subject is not anteceded by Voice and so cannot successfully agree.
    
Voice sensitivity ascribed to AGR-C may provide insight into what is going on in Kinande and Lubukusu. In Lubukusu, C-agreement with the subject is also possible with matrix active verb subjects as well as passivized subjects. As in Diercks’ (2010) hypothetical account of Voice agreement, we assume that phi-agreement is parasitic on Voice in Lubukusu, but unlike Ikalanga, voice has no morphological exponent on C. Kinande, like Lubukusu, has no C-morphology for voice. In Kinande, however, when the matrix verb is passivized, C-agreement is blocked. This would follow if passive Voice in Kinande is not anaphoric for phi-features, thus has none to transmit.\footnote{A further challenge is raised by C-agreement in Limbum. \citet{Nformi2017a} reports that direct objects block c-agreement in Limbum, which otherwise looks just like Lubukusu in the relevant respects. Accounts designed for the Lubukusu and Ikalanga patterns so far do not account for Limbum. Nformi proposes that the direct object blocks Agree between upward-probing C and the EA (see also Carstens, 2016, on Lubukusu).}


\subsection{Raising of \textit{-ti} to v}\label{sec:letsholo:3.4}

Diercks \textit{et. al.} (2017) suggest that the reason `say' verbs in some languages resist taking any complementizer is that the `say' C in those languages raises into the matrix clause. In their account, it is raising to an adjoined position on vP or VoiceP and it is independent of whether or not there is a lexical matrix verb. By contrast, we propose to limit the raising of `say' C 
by permitting it only when it fills an empty v position in the matrix clause. Both our account and Diercks \textit{et. al.}, however, assume that the -\textit{ti} root begins as a complementizer, not as a verb, as `say' complementizers are more commonly speculated to be. In Ikalanga, this would be manifested by the –\textit{ti} root taking on verbal status and being licensed to bear verbal morphology, which it cannot do when it remains the head of C. 

\ea 
\ea\label{n31a} \textup{[TP T  [\textsubscript{VoiceP} EA [Voice [\textsubscript{vP} [\textsubscript{v} v [\textsubscript{CP} -\textit{ti} [\textsubscript{TP} …]] ]] ]] ]}
\ex\label{n31b}	\textup{[TP T  [\textsubscript{VoiceP} EA [Voice-\textbf{v-\textit{ti}} [\textsubscript{vP} [\textsubscript{v} v-\textbf{\textit{ti}}… [\textsubscript{CP} -\textbf{\textit{ti}} [\textsubscript{TP} …]] ]] ]] ]}
\z \z 

Raising of C to v is illustrated in (\ref{n31b}) (copies bolded), presumably with subsequent raising of v to Voice, part of the phase edge, where agreement with T is possible. This limited account of `say' C-raising is thus morphologically motivated, but there is an independent reason to suppose that the v corresponding to `say' might often be morphologically null.

\citet{Grimshaw2015} proposes that the semantics of  `say' verbs follows a strikingly regular semantic decomposition based on what she calls the `say' schema. She notes that English \textit{say} is a kind of vanilla report of a particular form of speech act that has at least two arguments, a source and a `linguistic material' argument that amounts to content of what was said (setting aside the addressee or goal argument). One can talk nonsense or speak words, but neither of these two speech acts involves a linguistic material argument, as the direct objects have no propositional content. She then notes that three classes of verbs are based on the `say' schema, namely, `say by means' (\textit{mutter, shriek, mumble}), `say with attitude' (\textit{bitch, gripe}) and `discourse role' verbs (\textit{ask, announce, comment, remark, tell}). She points out that the three classes do not overlap, in that there are no monomorphemic verbs that mean `announce by shrieking', `ask in a bitching way' or `mutter by bitching', etc. She proposes that all three sets of verbs modify the same single mutable feature of the `say' schema to create other verbs. The `say' schema itself is just a template of argument places waiting for suitable morphology. Our suggestion is that in languages like Ikalanga, the predicate is filled by a complementizer from the argument that distinguishes the `say' schema by virtue of the presence of a linguistic material argument.\footnote{This proposal grew out of a discussion with Jane Grimshaw, personal communication.}

If this proposal is on the right track, then the Diercks \textit{et. al.} suggestion that C raises to become main verb `say' is essentially correct, but it does not support their proposal that movement of C to matrix vP or VoiceP is in any way general.

 

\subsection{Locality between matrix T and AGR-\textit{ti}}

The more challenging locality relation for phase theory is the relationship between T and the complement clause C, insofar as they are separated by a phase boundary. At least two different approaches come to mind that might address this puzzle.

One way is to motivate (covert) movement of AGR-C into the periphery of the VoiceP phase in order to be high enough to agree with T. In the past, anaphors that can only agree with subjects and never with objects have been posited to move to the matrix T where only the Spec-TP locally c-commands them (on ‘hoisting’ analyses, see \citealt{Safir2013}). The motivation for moving nominal anaphors in this way was always thin, driven by the needs of the locality of binding, rather than any morphological property of tense apparent on the element that moves.\footnote{Diercks et. al. attempt to motivate movement of anaphoric features to adjoined vP (or VoiceP) position on the basis of creating `referential' vPs, which, to avoid crashing, must have no meaningful, but unvalued features in them. It is unclear how they can account for cases like \textit{The men considered themselves to have praised themselves}, where the EA of \textit{praise} has no valued features to contribute, so movement of anaphoric features of the direct object to vP is not motivated or helpful. Such derivations should crash in their account.}

This proposal, essentially that of Diercks \textit{et. al.} (2017), is anaphor movement, except what moves has no nominal character. The vestige of T-agreement on C is a posited residue of covert head-movement of AGR-\textit{ti} to the edge of VoiceP, as in (\ref{34}) (bolding=copies).

\ea \label{34}
\textup{[\textsubscript{TP} T  [\textsubscript{VoiceP}} \textbf{AGR-\textit{ti}} \textup{[\textsubscript{VoiceP} EA [\textsubscript{Voice} …[\textsubscript{CP}} \textbf{AGR-\textit{ti}} \textup{[\textsubscript{TP} …]] ]] ]]}
 \z

The AGR-\textit{ti} adjoined to VoiceP is an unpronounced higher copy resulting from movement. Adjoined to the VoiceP, abstract AGR-\textit{ti} is then in the VoiceP phase edge and local to T. A serious problem for such an account is that Spell-out should apply to the lower copy of AGR-\textit{ti} (the one that is overt) as soon as the VoiceP phase is complete. T is outside the VoiceP phase, however, so the lowest C copy of the complement clause could only get its tense agreement by a counter-cyclic operation of inheritance down into the closed VoiceP phase. This counter-cyclic Spell-out breaches the Phase Impenetrability Condition \citep{Chomsky2001}.


An alternative account would be to assume that the complement clause can extrapose to a position at the edge of VoiceP where the edge of the complement clause CP phase would be visible to T in the matrix clause CP phase. Notice that it is possible for a complement clause to be separated from adjacency to the verb by an adverb.

\ea \label{35}
	\gll Néó w-ákà-léb-á chósèlélè (à)-kà-tì Nchídzì w-ákà-téng-á lórì  \\
	Neo \mc{sm}.c1-\mc{pst}2-say-fv indeed \mc{sm}.c1-\mc{pst}2-that Nchidzi \mc{sm}.c1-\mc{pst}2-buy-fv 	car	\\
	\glt Neo said it definitely that Nchidzi had bought a new car. \z

If the complement clause can extrapose by adjunction to VoiceP, where C is on the edge of CP, and CP is on the edge of VoiceP, then C is in the same phase as T.

\ea \label{36} \textup{[\textbf{T } [\textsubscript{VoiceP} [\textsubscript{VoiceP} …(EA) [\textsubscript{Voice} v [\textsubscript{VP} …}[\textsubscript{CP} AGR-C [TP]] { }\textup{]] ]]} [\textsubscript{CP} \underline{AGR-C} [TP]] { } \textup{]} \z


The (underlined) overt copy of AGR-C in the extraposed CP at the edge of the complement clause domain is no longer separated from T by a phase boundary in (\ref{36}). The underlined AGR-C can thus agree with Voice in the higher phase if Voice (and v) is incorporated into T, as is usually assumed in Bantu (but clausal subjects would still be outside the c-command domain of T, so would not agree in voice, as shown in (\ref{n23})). Thus T-agreement (from anaphoric tense features parasitic on agreeing C) would be direct, not indirect via copy inheritance down, as in (\ref{34}). 

\citet{Rackowski2005} have proposed extraposition of exactly the sort we propose to make the CP phase edge susceptible to wh-extraction, and in particular, with accompanying agreement relations established with the extraposed clause. We set aside investigation of this parallel (pointed out to us by Michael Diercks, personal communication) for future investigation.


\section{Conclusion}

Although our approach holds out the promise of applying more generally to other  C-agreement systems, empirical studies of matrix C-agreement are still sparse and our proposals will have to be tested against the additional patterns that may be discovered (including \citealt{Diercks2017},  not addressed here). Nonetheless, the relations between voice and tense and subordinate C-agreement uncovered in Ikalanga will have to be accounted for in any future approach.



%\section*{Abbreviations}
\section*{Acknowledgements}

The authors acknowledge the support of NSF BCS \#1324404 which was crucial for this research. We would also like to thank Michael Diercks whose commentary considerably influenced our revisions and Samson Lotven for his good advice and forbearance. 

\printbibliography[heading=subbibliography,notkeyword=this]

\end{document}

\documentclass[output=paper]{langscibook} 
\title{Tone, orthographies, and phonological depth in African languages}
\author{Michael Cahill}

\abstract{Marking of tone in African orthographies has historically been a challenge, not only for linguistic and analytical reasons, but also because most designers of these orthographies have been educated in non-tonal languages. After a review of lexical vs. grammatical tone, this paper examines various strategies that have been used for marking both lexical and grammatical tone in several East and West African languages, as well as cases in which tone is not marked. The question of the desired phonological depth of an orthography is discussed, especially when applied to tonal processes. Many phonologists do not apply theory more recent than Chomsky and \citet{Halle1968} to orthographies. However, the more recent bifurcation of rules into lexical and postlexical provides a psycholinguistically supported phonological level at which tone marking can be based: the output of the lexical level. Experimental evidence supports this \textsc{lexical} level as more readable than either a \textsc{phonemic} or a \textsc{deep} level.  A tonal typology of languages also guides what types of languages more predictably would need lexical tone marking. Recommendations for orthographical implementation are given in the conclusion.}

\begin{document}
\maketitle

\section{Introduction}\label{sec:Introduction:1}

Marking of tone in African orthographies was considered problematic even before the 1928 Rejaf Language Conference, where permission was rather grudgingly given to mark tones in Sudanese languages when absolutely necessary: “For tonal representations, the consensus was that only high tones should be marked, with an acute accent, and only if necessary for a particular language” \citep{Miner2003}.

One reason for this rather tepid approval was that most developers of orthographies either were Europeans or were educated in European languages, which of course are not tonal. The result was that many writing systems for African languages avoided tone marking, and tone was often not studied in any depth. Matters improved only somewhat two years after Rejaf with a cross-continental proclamation: 

\begin{quote}
In books for Africans, tones, generally speaking, need only be marked when they have a grammatical function, or when they serve to distinguish words alike in every other respect; and even then they may be sometimes omitted when the context makes it quite clear which word is intended. As a rule, it will suffice to mark the high or the low tone only. (\citealt[14, referring to Rejaf and 12 other documents]{IIALC1930})
\end{quote}

This guidance sounds strikingly modern, both in what it says and does not say. Note that this statement specifies books “for Africans,” not for foreigners, so it primarily has local literacies in mind. It laudably distinguishes grammatical from lexical tone, and for the latter, advocates what is called “selective tone marking” today -- marking tone only on minimal pairs, and even then, only when they are words likely to be confused in context. Tone marking is still considered a challenge today. It is not uncommon for orthography developers to not mark tone at all, either for principled reasons, or because they cannot deal with it, or because they do not consider it important (see \citealt{Cahill2000} for a critique of omitting all tone markings).

This paper begins (\sectref{sec:LexicalVgrammaticalTone:2}) with a review of the distinction between lexical and grammatical tone. \sectref{sec:HowToneMarked:3} examines methods that have been used to represent both lexical and grammatical tone (or not) in various African orthographies. In \sectref{sec:PhonTheoryOrtho:4}, I examine two major topics for assisting decisions in tone marking: the appropriate phonological level for orthographies, and a two-fold typological division of African languages. I close in \sectref{sec:Conclusion:5} with some recommendations for representing tone in African orthographies, and a brief re-examination of the selective tone marking issue.

\section{Lexical vs. grammatical tone: Review}
\label{sec:LexicalVgrammaticalTone:2}
Lexical tone is a difference in pitch that distinguishes one \textit{lexeme} from another. Samples of this are given in \REF{ex:LexicalToneDifferencesNouns:1}.\footnote{I follow a common notation for tone transcriptions that indicates tone levels with various diacritics: á = high, à = low, ā = mid, â = falling, ǎ = rising, and \textsuperscript{!}á = downstepped high. An unmarked tone is generally mid in a 3-level system. Unless indicated by other labeling, phonetic transcriptions are enclosed in square brackets [a], while orthographic representations are in angle brackets 〈a〉. ISO codes for languages are noted in the usual square brackets, e.g., [kma] for Kɔnni in \REF{ex:LexicalToneDifferencesNouns:1}.}

\ea Lexical tone differences in nouns\\
\label{ex:LexicalToneDifferencesNouns:1}
\ea
\langinfo{Kɔnni}{}{\citealt[306]{Cahill2007}}\smallskip\\
%Kɔnni [kma] (Ghana) \citep[306]{Cahill2007}\\
\begin{tabularx}{\linewidth}{@{}XXX@{}}
\textit{kpááŋ} & \textit{kpá\textsuperscript{!}}\textit{áŋ} &  \textit{kpàáŋ}\\
‘oil’          & ‘guinea~fowl’                              & ‘back~of~head’ \\
\end{tabularx}
% % % \gll \textit{kpááŋ} \textit{kpá\textsuperscript{!}}\textit{áŋ} \textit{kpàáŋ}\\
% % % ‘oil’ ‘guinea~fowl’ ‘back~of~head’ \\
\ex
Mono [mnh] (D. R. Congo) \citep[198]{Olson2005}\smallskip\\
% \gll {\textit{áwá}} {\textit{\=aw\=a}}  {\textit{àwà}}\\
% ‘diarrhea’ ‘road’ ‘fear’ \\
\begin{tabularx}{\linewidth}{@{}XXX@{}}
\textit{áwá} & {\textit{\=aw\=a}} &  {\textit{àwà}}\\
‘diarrhea’   &  ‘road’            &  ‘fear’ \\
\end{tabularx}
\z\z
Grammatical tone, on the other hand, distinguishes one \textit{grammatical category} from another. There are many grammatical categories which can be thus distinguished. Some of the more common ones are given in \xxref{tab:PersonByGrammaticalTone:2}{tab:OtherRelations:7} and \tabref{ex:SingPluralGrammaticalTone:3}. Not every person distinction is differentiated by tone in these or other languages; it is typically only two pronouns of the set that are so distinguished.

\ea Person distinguished by grammatical tone\label{tab:PersonByGrammaticalTone:2}
\ea Jur Modo  [bex]  (Sudan) \citep[80]{Persson2004}\smallskip\\
    \begin{tabularx}{\linewidth}{@{}XXX@{}}
    \textit{nì}   & \textit{ní} & \\
    ‘her’         & ‘their’     & \\
    \end{tabularx}
\ex Lyele [lee] (Burkina Faso) \citep[57]{Kutsch2014}\smallskip\\ 
    \begin{tabularx}{\linewidth}{@{}XXX@{}}
    \textit{ń}    & \textit{ǹ}   & \\
    \textsc{2sg}  & \textsc{3sg} & \\
    \end{tabularx}
\z
\z

In some languages (e.g., Tarok in \tabref{ex:SingPluralGrammaticalTone:3}), tone distinguishes singulars from plurals in only a subset of nouns, while in others (e.g., Koro Waci, Ndrulo), tone change is the normal method of making plurals from singular nouns. It appears that in the majority of languages which exhibit tone change to mark plural nouns, the plural nouns are in some way higher toned than the singular. However, this is not universal, as will be seen in Karaboro in section \sectref{sec:HowToneMarked:MarkGrammatical:3}.


\begin{table}
\begin{tabularx}{\textwidth}{lXXX} 
\lsptoprule
& Singular & Plural & Gloss \\\midrule
\multicolumn{4}{l}{a. Ndrulo [led] (Uganda) \citep[60]{Kutsch2014}}\\
& \textit{vìnì} & \textit{víní} & ‘his sister/s’  \\
& \textit{djānì} & \textit{djání} & ‘his father/s’  \\
\multicolumn{4}{l}{b. Koro Waci [bqv] (Nigeria) (Rachelle Wenger, p.c.)\ia{Wenger, Rachelle@Wenger, Rachelle}}\\
& \textit{ɪsʊr} & \textit{ɪsʊr} & ‘he-goat/s’  \\
& \textit{ɪtɔmɪ} & \textit{ɪtɔmɪ} & {‘work/s’}  \\
& \textit{ìbǔr} & \textit{íbûr} & ‘slime/s’  \\
\multicolumn{4}{l}{c. Tarok  [yer] (Nigeria) \citep[90–91]{Longtau2008}}\\
& \textit{ìfàng} & \textit{īfáng} & ‘fingers/s’  \\
& \textit{ìnà} & \textit{īnà} & ‘cow/s’  \\
& \textit{ǹtúng} & \textit{\={n}túng}  & ‘hyena/s’ \\
\lspbottomrule
\end{tabularx}
\caption{Singular/plural nouns distinguished by grammatical tone\label{ex:SingPluralGrammaticalTone:3}}
\end{table}

Though verbal aspect may be the most common grammatical category distinguished by tone, as in \REF{tab:VerbAspectByGrammaticalTone:4}, other categories are not rare. \REF{tab:LocativeByGrammaticalTone:5} shows an example of tone distinguishing a locative from the bare noun, \REF{tab:SubjObjRelationsByTone:6} exemplifies the syntactic subject/object feature distinguished solely by tone, and \REF{tab:OtherRelations:7} exhibits a miscellany of language-specific grammatical relations distinguished by tone. 

\ea Verbal aspect distinguished by grammatical tone\label{tab:VerbAspectByGrammaticalTone:4}\\Mbembe [mfn] (Nigeria) \citep{Barnwell1969}\smallskip\\
\begin{tabularx}{\linewidth}{@{}XXX@{}}
 \textit{ɔkɔn}  ‘you sang’         & \textit{ɔkɔn}  {‘you should sing’}                          & \textit{móchí}  ‘he will eat’    \\    
 \textit{ɔkɔn} {‘you have sung’}   & \textit{ɔk\textsuperscript{!}}\textit{ɔn}  ‘if you sing’    & \textit{mòchí}  ‘he will not eat’\\
\end{tabularx}
\z

\ea Locative distinguished by grammatical tone\label{tab:LocativeByGrammaticalTone:5}\\Fur [fvr] (Sudan) \citep[61]{Kutsch2014}\smallskip\\
\begin{tabularx}{\linewidth}{@{}XXX@{}}
 \textit{bàrù}  ‘country’ & \textit{bàrú}  ‘in the country’ & \\
 \textit{dɔŋá}  {‘hand’} & \textit{dɔŋà}  ‘in the hand’ & \\
 \textit{ʊtʊ}  ‘fire’ & ʊtʊ  ‘in the fire’ & \\
\end{tabularx}
\z

\ea Subject/object relations distinguished by grammatical tone\label{tab:SubjObjRelationsByTone:6}\\
Sabaot [spy]  (Uganda) \citep[66]{Kutsch2014}\smallskip\\
\textit{kɪbakaac kwààn}  ‘his father left him’\\
\textit{kɪbakaac kwáán}  ‘he left his father’\\
\z

\ea Other relations\label{tab:OtherRelations:7}\\
Lugungu [rub] (Uganda) \citep[10]{Moe1999}\smallskip\\
\textit{mulogo muhandú} ‘an old witch’ \\
\textit{múlógó muhandú} ‘the witch is old’ \\
\textit{múlógô muhandú} ‘the witch, (she) is old’ \\
\z

\section{How tone is marked}
\label{sec:HowToneMarked:3}

  Local orthography developers and outside linguists have developed astonishingly varied and sometimes creative ways of marking tone in languages. In contrast, some languages do not mark tone at all, even if they are distinctly tonal, and I start with these.

  \subsection{No tone marking}
  \label{sec:HowToneMarked:NoToneMark:1}
    Here I look at a few languages with no orthographic tone marking at all. Interestingly, sometimes tone marking appears to be crucial to reading, and in other cases less so.

  The consensus among linguists I have spoken to is that the common way of writing Hausa in \REF{tab:cahill:VariousVerbalAspects:8} (there are other systems) is quite difficult to read. This is especially due to the fact that the \textit{grammatical} tone, as in the example, is not marked, and there are many situations where this ambiguity is impossible to resolve by the context.

\ea Various verbal aspects\label{tab:cahill:VariousVerbalAspects:8}\\
Hausa [hau] (Nigeria)  \citep{Harley2012}\smallskip\\
\begin{tabularx}{\linewidth}{@{}XXX@{}}
[jáá tàfí] & [jáà tàfí] & [jà tàfí]\\
$\langle$ya tafi$\rangle$ & $\langle$ya tafi$\rangle$ & $\langle$ya tafi$\rangle$\\
‘he went’  &  ‘he may go’   & ‘he should go’\\
\end{tabularx}
\z
    
    Kumam [kdi] (Uganda) has both lexical and grammatical tone: \textit{abe} can mean either ‘an egg’ or ‘a lie,’ while \textit{ebedo} can mean either ‘he lives’ or ‘he lived.’ However, tone is not marked at all in Kumam, and 60\% of people surveyed agreed that it is more difficult to read the Kumam Bible than Bibles in other languages \citep{Edonyu2015}.

     Kɔnni [kma] (Ghana) orthography does not mark tone. However, in contrast to the above languages, this seems not to make a significant difference in readability (my personal observation). In this language minimal pairs are few, so there is a fairly small functional load for lexical tone. Furthermore, there is very little grammatical tone in the language. People are able to read aloud fluently.

  \subsection{Marking lexical tone} 
  \label{sec:HowToneMarked:MarkLexical:2}
    Lexical tone, if it is marked, is marked by diacritics more frequently than not. Rangi [lag] of Tanzania, for example, marks lexical High tone, but only on nouns (e.g., \textit{ikúfa} ‘bone’, \citealt{Stegen2005}). Similarly, Akoose marks High tone (e.g., \textit{edíb} [èdíb] ‘river’) and contours (e.g., \textit{kɔ̂d}  [kɔ̂d] ‘age’), but leaves Low unmarked \citep[13]{Hedinger2011}.

    In a few cases, tone has been marked by punctuation marks before each word, especially in Côte d’Ivoire (e.g., \citealt{Bolli1978}). Examples of the punctuation marks used are displayed in \tabref{tab:cahill:LexicalToneNotationCoteLang:9}.

    
    \begin{table}
      \begin{tabularx}{\textwidth}{XXXXXXXX}
      \lsptoprule
      extra high & high & mid & low & extra low & mid-low falling & low-high rising & high-low falling\\\midrule
      ″CV & ′CV & CV & {}-CV & =CV & CV- & {}-CV’ & ‘CV-\\
      \lspbottomrule
      \end{tabularx}
      \caption{Lexical tone notation for Côte d’Ivoire languages \citep[58]{Kutsch2014}\label{tab:cahill:LexicalToneNotationCoteLang:9}}
    \end{table}

    This system can handle up to five tone levels, necessary in some languages of Côte d’Ivoire. This is exemplified as follows in Attié, which has four contrastive levels of tone (but does not have extra low).

    \ea \label{ex:AttieContrastiveTone:1}
    \langinfo{Attié}{}{Matthew 6:30a}\\
    \glt 'Pɛte {\textquotedbl}yi {\textquotedbl}fa, 'fa {\textquotedbl}kan'a 'lö {\textquotedbl}a -bë ko fon- 'tshɛn'a tɔ, 'eyipian -Zö -wɔ' sɛn  'e hɛn dzhi ko \ldots    
    \z

  

  \subsection{Marking grammatical tone} 
  \label{sec:HowToneMarked:MarkGrammatical:3}
    Different languages have used a wide variety of strategies for indicating grammatical tone. One strategy is using diacritics, and often these mark a phonetic tone which instantiates a particular grammatical category, as in \REF{tab:DiacriticPhoneticsMeaning:10}, with the Daffo variety of Lis Ma Ron.

\ea Diacritic showing both phonetics and meaning\label{tab:DiacriticPhoneticsMeaning:10}\\
Lis Ma Ron [cla] (Nigeria) \citep{Harley2012}\smallskip\\
\begin{tabularx}{\linewidth}{@{}XXX@{}}
\textit{á} &  \textit{à} & \\
‘you (male)’ & `he'      & \\
\end{tabularx}
\z 
    
    Akoose exhibits a somewhat unusual pattern in that the singular and plural nouns for class 9/10 are identical, but the distinction is made by tone on the agreement prefix of the following \textit{verb}: 

    %\begin{stylelsTableHeading}
    %Table : Displaced diacritics in Akoose \citep[13]{Hedinger2011}
    %\end{stylelsTableHeading}
    %\begin{table}
    %\begin{tabularx}{\textwidth}{XXXX} 
    %    \lsptoprule
    %    & phonetic & orthography & gloss\\
    %    Akoose [bss] & [ngù: \textbf{è}délé] & \textit{nguu} \textbf{\textit{e}}\textit{délé} & ‘the pig is heavy’\\
    %    (Cameroon) & [ngù: \textbf{é}délé] & \textit{nguu} \textbf{\textit{é}}\textit{délé} & ‘the pigs are heavy’\\
    %    \lspbottomrule
    %    \end{tabularx}
    %\end{table}
    \protectedex{
    \begin{exe}
        \ex \label{ex:DisplacedDiacriticsAkoose:2}
        \langinfo{Akoose}{}{\citealt[13]{Hedinger2011}}
        \begin{xlist}
            \ex
                \gll [ngù: \textbf{è}délé]\\
                 \textit{nguu} \textbf{\textit{e}}\textit{délé}\\
                \glt ‘the pig is heavy’\\
            \ex 
                \gll [ngù: \textbf{é}délé]\\
                   \textit{nguu} \textbf{\textit{é}}\textit{délé}\\
                \glt ‘the pigs are heavy’\\
        \end{xlist}
    \end{exe}
    }
    
Some languages indicate grammatical tone by letters which are otherwise unused. For example, Gangam [gng] (Togo) marks grammatical tone, not phonetically, but with other symbols to indicate the \textit{meaning}. The imperfective is marked with the letter 〈h〉 and the perfective with an apostrophe 〈’〉 (See \citealt{HigdonEtAl2000}, also \citealt{Roberts2013} for more examples. Phonetic transcription is from Jean Reimer p.c.\ia{Reimer, Jean@Reimer, Jean}).

    \begin{exe}
        \ex \label{ex:ToneByUnusedLetters:3}
        \langinfo{Gangam}{}{\citealt{HigdonEtAl2000}}
        \begin{xlist}
            \ex {\itshape N bɛnge' [bɛ́ŋge] Miganganm ya kaanm.}
               \glt  `I learned to read Gangam.'
            \ex {\itshape N laan bɛngeh [bɛ̄ŋgé] Miganganm ya kaanm nɛ.}
            \glt `I am learning to read Gangam.'
        \end{xlist}
\end{exe}
    
Similarly, Etung \REF{tab:EtungPronouns} uses 〈h〉 to differentiate pronouns which differ only by tone.

\ea Pronouns in Etung [etu] (Nigeria, \citealt{Harley2012})\label{tab:EtungPronouns}\setlength{\multicolsep}{0pt}
\begin{multicols}{2}\ea\gll\relax [á]\\ $\langle$ah$\rangle$ \\\glt `they'
\ex\gll\relax [à]\\ $\langle$a$\rangle$ \\\glt `he'
\z\end{multicols}\z
    
Other languages double some letters to differentiate pronouns which differ only by tone \REF{tab:JurModoLetterDoubling}.

\ea Pronouns in Jur Modo [bex] (Sudan, \citealt{Persson2004})\label{tab:JurModoLetterDoubling}\setlength{\multicolsep}{0pt}
\begin{multicols}{2}\ea \gll\relax [nì] \\ $\langle$nï$\rangle$ \\\glt ‘her’
\ex \gll\relax [ní] \\ $\langle$nnï$\rangle$ \\\glt ‘their’
\z\end{multicols}\z

A number of languages indicate various grammatical tone functions by means of punctuation or other non-alphabetic marks. Karaboro, as displayed in \REF{tab:KaraboroPlurals}, uses a word-final hyphen to indicate plurals (in those cases which are not indicated by a segmental marker), which all happen to end in a low tone.

\ea Plurals in Karaboro [xrb] (Burkina Faso, SIL 2009, as cited in \citealt{Roberts2013})\label{tab:KaraboroPlurals}\setlength{\multicolsep}{0pt}
\begin{multicols}{3}
\ea\gll\relax\ob kāī, kāì\cb \\           $\langle$kai, kai-$\rangle$ \\\glt ‘affair, affairs’
\ex\gll\relax\ob gjɔɔ, gjɔɔ\cb \\         $\langle$jɔɔ, jɔɔ-$\rangle$ \\\glt ‘net, nets’
\ex\gll\relax\ob sààpjé, sàápjè\cb \\     $\langle$saapye, saapye-$\rangle$ \\\glt ‘rabbit, rabbits’
\z\end{multicols}\z
    
The old Ejagham orthography, now changed to a different system, used punctuation extensively to indicate various verbal aspectual forms \REF{tab:cahill:fiaghamOrtho:13}.

% % Table : Old Ejagham orthography [etu] (Nigeria \& Cameroon)\\
% %              (\citealt{Bird1999b}, corrected by John Watters pc)
    \begin{table}
        \begin{tabularx}{\textwidth}{llXl}
            \lsptoprule
            Orthographic Rule & Phonetic & Orthography & Gloss\\\midrule
            colon =  \textsc{perfect} & [émè]       & $\langle$e:me$\rangle$ & ‘we have swallowed’\\
            space =  \textsc{perfective} & [èmê]    & $\langle$e me$\rangle$ & ‘we swallowed’\\
            apostrophe = \textsc{hortative} & [éme] & $\langle$e’me$\rangle$ & ‘let us swallow’\\
            hyphen =   \textsc{conditional} & [émě] & $\langle$e-me$\rangle$ & ‘when we swallow’\\
            no symbol =  \textsc{noun} & [èmè]      & $\langle$eme$\rangle$ & ‘neck’\\
            \lspbottomrule
        \end{tabularx}
        \caption{Old Ejagham orthography [etu] (Nigeria \& Cameroon)\\
             (\citealt{Bird1999b}, corrected by John Watters, p.c.\ia{Watters, John@Watters, John})\label{tab:cahill:fiaghamOrtho:13}}
    \end{table}
    
The Bokyi orthography (\tabref{tab:cahill:BokyiOrtho:14}) uses a system that appears rather unusual to most readers in its employment of a variety of non-alphabetic symbols, but it is currently in use.


    \begin{table}
        \begin{tabularx}{\textwidth}{XXX}
        \lsptoprule
        Phonetic & Orthography & Gloss\\\midrule
        \ob ǹtsè\cb & $\langle$nce$\rangle$ & ‘going’\\
        \ob ǹtsâ\cb & $\langle$n-ca$\rangle$  & ‘I go’\\
        \ob ńtsè\cb & $\langle$n/ce$\rangle$  & ‘I went’  \\
        \ob \={n}ńtsè\cb &  $\langle$nn/ce$\rangle$ & ‘I have gone’\\
        \ob ńtʃì ǹ-tsâ\cb & $\langle$n/chi  n-ca$\rangle$ & ‘I will go’\\
        \ob \={n}ńtséē\cb & $\langle$n*-ce*$\rangle$ & ‘I don't go’\\
        \ob ǹdátsèē\cb & $\langle$n*da/ce*$\rangle$  & ‘I didn't go’\\
        \ob \={m}ḿbátʃì ǹtsáā\cb & $\langle$n*ba/chi  n-ca*$\rangle$ & ‘I will not go’  \\
        \lspbottomrule
        \end{tabularx}
        \caption{Bokyi [bky] (Nigeria) orthography (Harley, p.c.)\ia{Harley, Matthew@Harley, Matthew}\label{tab:cahill:BokyiOrtho:14}}
    \end{table}

The Bungu language is one of the more complex illustrations of grammatical tone marking. It uses both diacritics and punctuation marks to indicate the interaction of person and aspect in the verbal system. Many words are segmentally identical, and vowel length is not contrastive, putting a greater load on tone. At this point, no lexical tone is marked (though the orthography is still being adjusted), and \tabref{tab:BunguGrammaticalToneMarking} does not give the entire picture of grammatical tone. Other complexities exist as well, such as tone marking of objects. 

\begin{table}
        \begin{tabular}{lll}
        \lsptoprule
\multicolumn{2}{l}{Orthography} & Gloss\\\midrule
        $\langle$w\"{a}kala$\rangle$  & [wàkála] & ‘you bought (recent)’            \\
        $\langle$wákala$\rangle$  & [wákála] & ‘he bought (recent)’                 \\
        $\langle$waakala$\rangle$  & [waːkála] & ‘they bought (recent)’             \\
        $\langle$wakala$\rangle$   & [wakála]   & ‘they will buy’           \\
        $\langle$\^{}wakala$\rangle$  & [wa\v{} kala]   & ‘they are buying’ \\
        $\langle$:w\"{a}kala$\rangle$ & [wákala] & {‘you have already bought’}\\
        $\langle$:wákala$\rangle$  & [wakála]  & {‘he has already bought’}\\
        $\langle$:waakala$\rangle$  & [wǎːkala] & {‘they have already bought’}\\
        $\langle$:nakala$\rangle$  & [nákala] & {‘I have already bought’}\\
        $\langle$\^{}nakala$\rangle$ & [nǎkala]  & {‘I am buying’}\\
\lspbottomrule
        \end{tabular}
        \caption{Bungu [wun] (Tanzania, \citealt{Katterhenrich2016}) \textmd{(Low tone is unmarked)}. Key:  Colon: \textsc{completive}; Carat: \textsc{progressive}; Umlaut: \textsc{2sg.subj.past}; Accent: \textsc{3sg.subj.past}; Double vowel: \textsc{3pl.subj.past}.}
        \label{tab:BunguGrammaticalToneMarking}
    \end{table}

\subsection{Marking both lexical and grammatical tone with diacritics}
\label{sec:HowToneMarked:LexicalGrammatical:4}

Zinza [zin] (Echizinza) marks both lexical and grammatical tone, with accent marks for high, rising, and falling (see \ref{tab:ZinzaAccentMarking}).

\ea Marking both lexical and grammatical tone with accent in Zinza \citep{Matthews2010}\label{tab:ZinzaAccentMarking}
\ea lexical tone\smallskip\\
    \begin{tabularx}{\linewidth}{@{}XXX@{}}
    \textit{enzóka} & \textit{omuyǎnda} & \\
    ‘snake’ &   ‘child, youth’& \\
    \end{tabularx}
\ex grammatical tone\smallskip\\
    \begin{tabularx}{\linewidth}{@{}XXX@{}}
    \textit{aleeba} & \textit{aléeba} & \\
    ‘he looked’ & ‘he (habitually) looks’ & \\
    \end{tabularx}
\z
\z

\section{Phonological theory and orthography}
\label{sec:PhonTheoryOrtho:4}

The above discussion has assumed that tones are completely stable, i.e., that underlying tones and surface tones are the same. The question of when or if to mark the results of \textit{tone rules} offers more challenges. For example, in a Bantu language, if a prefixal High tone spreads for three syllables, does one mark the initial prefix syllable alone, or the result of the spreading rule? Or, in west Africa, if underlying tones in a word are /HLH/, but surface as [H\textsuperscript{!}HH], what is the appropriate marking? The major question that involves both of these situation is: what \textit{depth} of phonological representation should be the basis for marking tone? This section addresses those questions.

Tone studies have advanced in the decades since the 1928 Rejaf conference, especially with Autosegmental Phonology \citep{Goldsmith1976} and Lexical Phonology \citep{Pulleyblank1986}. However, as \citet{Snider2014} notes, many people do not apply phonological theory more recent than \citet{Halle1968} to orthographies. Rather, the main distinctions that most orthographers have in mind are “deep” vs. “shallow” orthographies. However, as we will see, there are other options.

A shallow orthography is close to or identical with the surface pronunciation, after most or all of the rules have applied. This has certain consequences and raises the following issues.

\begin{itemize}
    \item The same word will appear with different tone marks \textit{depending on its context}. A “constant word image” (useful for quick word recognition) is not maintained.
    \item It tends to be cumbersome and hard to read. \citet{Bird1999a} showed that an exhaustive shallow tone marking was actually less readable than no marking in Dschang.
    \item How are multiple downsteps represented, when the tone can have several decreasing phonetic levels?
\end{itemize}


A deep orthography represents the sounds before the rules have applied. Very broadly, this is what linguists think of as the “underlying form.” This also has certain consequences. A deep orthography has certain characteristics.

\begin{itemize}
    \item It retains a constant word image, aiding quicker visual recognition of a word;
    \item It can sometimes be adapted better across dialects, since dialectal differences can be attributed to varying rule application;
    \item It can be significantly different than any person’s actual pronunciation, including pronunciation in isolation.
\end{itemize}

If a particular language has few tone processes, there will be little or no difference between a shallow and deep orthography. The above does not exhaust the possibilities; \citet{Bird1999b} and \citet{Roberts2013} give a number of other variations on marking tone.

\subsection{Lexical phonology as a useful framework}
\label{sec:PhonTheoryOrtho:LexicalPhonology:1}

I have mentioned “rules,” but what kind of rules do I mean? There is a rich history of types of rules and their interactions, and one would expect that a narrowing of types of rules would likely be helpful in determining tone orthographies. And so it is.

Lexical Phonology (e.g., \citealt{Pulleyblank1986}) is now disfavored as a comprehensive phonological theory, but the notion of \textit{lexical} vs. \textit{postlexical} processes is still invoked in contemporary theories such as Stratal Optimality Theory \citep{Kiparsky2000,Goldsmith2014}. I argue that Lexical Phonology offers a level of psycholinguistic realism that is helpful in determining which level to refer to in deriving orthographic representations. 

In Lexical Phonology, the output of the lexical level is the \textit{psychologically real} level. This level  is similar to but not precisely the same as the “phonemic level” of earlier theories. Following \citet{Snider2014}, I propose that this is the most appropriate phonological level for orthography in general. Specifically for this paper, it is proposed that this level is the most fruitful level in applying the results of tone rules to an orthography. 

\citet{Snider2014} is a major advocate of the above. One does not need to adopt the entire theory of Lexical Phonology to profit from its main benefits. The main question in dealing with a phonological rule that may make a difference in orthographic representation is whether a rule is lexical or postlexical. Several diagnostic questions can be fruitfully applied to determine this, which I have adapted with minor modification from \citet{Snider2014}. These questions are: 

\begin{itemize}
    \item Are there lexical exceptions to the process? 
    \item Does a given process lack phonetic motivation?
    \item Does the process have to apply across a \textit{morpheme} boundary? (not a word boundary)
\end{itemize}

If one or more answers to the above are “yes,” then the rule is a lexical rule; write the output of that rule. Other diagnostic questions:

\begin{itemize}
    \item Is the new sound the rule produces a \textit{non}{}-contrastive sound in the language? 
    \item When a given process has applied, do native speakers think that the sound that results is the same as the sound that underwent the process?
    \item Does the process apply across \textit{word} boundaries? 
\end{itemize}

If one or more answers to the above are “yes”, then the rule is postlexical; write the sound at the level \textit{before} the rule applies. Also, if there is no apparent reason to categorize a rule as lexical, Snider advises assuming it is postlexical.

    \begin{quote}
      The above questions are a starting point for tentative decisions that should be held somewhat loosely; all orthographic decisions need to be actually tested.\footnote{\citet[342--343]{Gudschinsky1958} gives an interesting example of a Mazatec man (Mexico) who was quite aware of the results of tone processes \textit{within} words (lexical rules), but insisted that the tones of two particular \textit{phrases} were different, though they were phonetically tonally identical (result of \textit{postlexical} rules).}
    \end{quote}
    
The experimental evidence from Kabiye (\citealt{Roberts2016neither}) on two tone processes which were marked differentially in test orthographies supports this. The authors tested what they termed the \textit{Lexical Orthography Hypothesis}, that is, that the lexical level (i.e., the output of the lexical phonology) offers the most promising level of phonological depth upon which to base a phonographic tone orthography that marks tone exhaustively.\footnote{“Exhaustive tone marking” is marking the tone on every syllable. There is reason to believe that this is not the most effective way to mark tone, but it was adopted for the purposes of having a more controlled experiment.}

They tested 97 tenth-graders with orthographies that represented two tonal processes in three different ways. The rules were:

\begin{description}
    \item[Lexical rule of L-spread:] in the Kabiye verb, the L tone of a prefix spreads rightwards onto a H verb root until it is blocked by a singly linked H tone. This is shown to be lexical because it applies only across a specific morpheme boundary and is limited to within a word. Results of this rule are illustrated in \tabref{tab:cahill:19}.
    \item[Postlexical rule of HLH plateauing:] a singly linked L between two H tones delinks, and the second H spreads left and has a downstepped register. This is shown to be postlexical by the fact that it applies across word boundaries as well as within words. Results of this rule are illustrated in \tabref{tab:cahill:20}.
\end{description}

\noindent The researchers tested 3 orthographies:

\begin{description}
    \item[Phonemic:] the pronunciation minus application of any allophonic processes
    \item[Lexical:] (output of lexical phonology) the phonemic level minus application of any postlexical processes
    \item[Deep:] (input of lexical phonology) the lexical level minus application of any lexical processes, a morphographic representation.
\end{description}

\noindent Examples of the orthographic output of these different systems are shown in Tables~\ref{tab:cahill:19} and~\ref{tab:cahill:20}. 

\begin{table}[p]
    \begin{tabularx}{\textwidth}{lp{2cm}Ql}
        \lsptoprule
         Speech       & Deep\newline orthography  & Lexical and phonemic orthographies & Gloss\\\midrule
        {[wélésí-Ø]} & 〈wélési〉 & 〈wélésí〉 & ‘listen!’\\
        listen-\textsc{imp} &          &          &          \\
        {[e-welesí-na]}       & 〈ewélésína〉 & 〈ewelesína〉 & ‘he listened’\\
        \textsc{3sg-nc1}-listen-\textsc{com}   &             &             &               \\
        {[te-welesí-na]} & 〈tewélésína〉 & 〈tewelesína〉 & ‘didn’t listen’\\
        \textsc{neg}-listen-\textsc{com} &                &               &               \\
        \lspbottomrule
    \end{tabularx}
    \caption{Low-spread and Kabiye orthographies}
    \label{tab:cahill:19}
\end{table}

%\begin{tabularx}{\textwidth}{XXXX}
%\lsptoprule
%Speech & deep orthography & lexical and phonemic orthographies & gloss\\
%[wélésí-Ø] \\
%listen-IMP & 〈wélési〉 & 〈wélésí〉 & ‘listen!’\\
%[e-welesí-na] \\
%3sgNC1-listen-COM & 〈ewélésína〉 & 〈ewelesína〉 & ‘he listened’\\
%{[te-welesí-na]} 

%NEG-listen-COM & 〈tewélésína〉 & 〈tewelesína〉 & ‘didn’t listen’\\
%\lspbottomrule
%\end{tabularx}


\begin{table}[p]
    \begin{tabularx}{\textwidth}{QQQl}
    \lsptoprule
    Speech & Deep and lexical orthographies & Phonemic orthography & Gloss\\\midrule
    {[sɛ-tʊ]} 

    thanks-\textsc{nc9} & 〈sɛtʋ〉 & 〈sɛtʋ〉 & ‘thanks’\\
    {[fɛyɪ]}

    there\_is\_no & 〈fɛyɩ〉 & 〈fɛyɩ〉 & ‘there is not’\\
    {[sɛ\textsuperscript{!}tʊ fɛyɪ]}

    thanks-\textsc{nc9} there\_is\_no & 〈sɛtʋ fɛyɩ〉 & 〈sɛ’tʋ fɛyɩ〉 & ‘don’t mention it!’\\
    \lspbottomrule
    \end{tabularx}
    \caption{HLH plateauing and Kabiye orthographies}
    \label{tab:cahill:20}
\end{table}

\begin{table}[p]
    \begin{tabularx}{\textwidth}{llQ}
    \lsptoprule

    Orthography & Lexical L tone spreading & Post-lexical HLH plateauing\\\midrule
    Phonemic & {Written as pronounced}     (easier) & {Written as pronounced}    (harder)\\\tablevspace
    {Lexical} &  & {Written without post-lexical processes} (easier)\\\tablevspace
    {Deep}  & Written morphographically \\  (harder) & \\
    \lspbottomrule
    \end{tabularx}
    \caption{Expected results from three experimental orthographies}
    \label{tab:cahill:21}
\end{table}

Note that because of the specific processes chosen, the results of the Low-Spread rule distinguish a Deep orthography from the others. Is the Deep or the Lexical/Phonemic representation better? The results of the HLH Plateauing rule distinguish the Phonemic orthography from the others; is the Phonemic or the Deep/Lexical representation better? \tabref{tab:cahill:21} shows the expected results if the Lexical Orthography Hypothesis is correct. Note that the experiment focused on the oft-neglected domain of \textit{writing} as well as reading.



The reader is referred to the paper for full results, but on the whole, the Lexical Orthography Hypothesis was supported. Lexical and Phonemic orthographies worked better in dealing with one tone process, and Lexical and Deep orthographies worked better in dealing with the other tone process. So the Lexical orthography fared well in both processes, while the others did worse in one orthography or the other. Specifically, those writing the Lexical orthography:

\begin{itemize}
    \item scored fewer errors writing an appropriate accent on a vowel than those writing the Deep orthography;
    \item scored fewer errors writing post-lexical non-automatic downstep than those writing the Phonemic orthography;
    \item experienced less degradation of performance on a later test than those writing the Deep and Phonemic orthographies;
    \item were more absorbed with the task of writing accents correctly than those writing the Deep and Phonemic orthographies (though this often caused them to write long vowels incorrectly).
\end{itemize}

One caveat for the experiment is that there is not universal acceptance by researchers what the underlying (deep) tones of Kabiye actually are. Also, as mentioned before, this experiment focuses on lexical tone, exhaustively marked. Other less exhaustive methods of tone marking were not explored.

\subsection{Language typology as a useful guide}
\label{sec:PhonTheoryOrtho:LanguageTypo:2}

Besides the largely theoretical insights of the Lexical Orthography Hypothesis, another promising tool for deciding how to mark tone is a more typological one. \citet{Kutsch2014} proposes two main types of tone languages. In her terminology, these are “stable tone languages” and “movable tone languages.”

\textit{Stable tone languages} are those in which tone rules do not change an underlying tone. They tend to have a cluster of properties:

\begin{itemize}
    \item These languages tend to have shorter words, and more tone levels. 
    \item Tone generally has a heavy functional load, both lexically and grammatically. 
    \item Grammatical tone can be looked at as tone replacement. 
    \item Writing tone on every syllable is possible and straightforward. 
    \item Teaching phonetic tone awareness is (relatively) easier, and a constant word image can be maintained.
\end{itemize}
Ndrulo and Attié, cited earlier, are examples of stable tone languages. 

\textit{Movable tone languages} are those in which the tones change according to the context, due to a variety of tone sandhi rules. These also tend to have a cluster of properties which differ from the stable tone languages:

\begin{itemize}
    \item These languages tend to have longer words and fewer tone levels. 
    \item They generally have a lighter load for lexical tone, but often a heavy functional load for grammatical tone. 
    \item Thus it may be less important to mark lexical tone, but it is important that \textit{grammatical} tone distinctions be differentiated.
    \item Teaching tone awareness could focus on grammatical notions rather than phonetics 
\end{itemize}

Sabaot, Lugungu, and many Bantu languages are examples of movable tone languages. 

Of course, these language types are prototypical. Many languages do not fall neatly into these categories. However, this can serve as a general first approximation and guide to the type of orthographic tone marking that may prove fruitful.

\section{Conclusions and recommendations}\label{sec:Conclusion:5}

I conclude this paper with several recommendations -- some definite and others more tentative -- and an open question on “selective tone marking.” 

\subsection{Recommendations}
\label{sec:Conclusions:Recommendations:1}
Some practices in orthography development have been confirmed enough by experienced people that I can definitely recommend these.

\begin{enumerate}[leftmargin=*]
\item First, work with the community! The emphasis in this paper has been on usability of the orthography, based on linguistic factors. However, if for any reason, the language community does not \textit{want} to use a particular orthography, linguistic perfection becomes irrelevant. Various sociopolitical factors that can be relevant in different situations are discussed in \citet{Cahill2014}.

\item All decisions on marking tone need to be tested. Unforeseen factors, including incomplete analysis, may result in one’s orthography not being as useable as anticipated. Whether the testing be formal or informal, one needs to check it with people who use the language (see \citealt{Karan2014} for details).

\item If it is decided to mark lexical tone in the orthography, mark the output of the lexical level, as discussed in \sectref{sec:PhonTheoryOrtho:LexicalPhonology:1} 
\item When marking grammatical tone of whatever sort, prioritize marking the \textit{meaning}, not the phonetics (in \citeauthor{Roberts2013}’s \citeyear{Roberts2013} term, “semiographically”). Readers and writers have meaning “in their heads” more than they do the abstract sound. Also, a particular grammatical meaning such as “recent past” may have several phonetic implementations. Figuring these out is a challenging task, but one which, as far as orthography goes, is unnecessary.
\item Consider how to \textit{teach} the orthography. Even if speakers know their language is tonal, they often do not have a high awareness of the specifics of tone, let alone how to represent this. Each tone mark should be taught in a separate lesson, just as any consonant or vowel. Also, lexical and grammatical tone should be taught separately. 

\begin{quote}
\ldots\ a tone orthography needs to be accompanied by a well thought-\linebreak through methodology for awareness raising of tonal contrasts and for teaching people to read with the symbols chosen to mark tone in a language. \citep[52]{Kutsch2014}
\end{quote}\largerpage[2]

\item Make the orthography compatible with electronic devices -- phones, tablets, internet, and computers in general. A Unicode-compatible orthography\footnote{Unicode is the international standard for encoding text in electronic data. Major software assumes user input uses Unicode characters and not a custom font. “Unicode-compatible” in our context means first, that only Unicode characters are used, and second, that they are used in accordance with their defined set of properties. One of those properties is whether it is treated as a “word-forming” character. The usual equals sign ({\ttfamily\char"003D}, Unicode U+003D) is not word-forming, but a shortened equals sign ({\ttfamily\char"A78A}, Unicode U+A78A) has been defined as word-forming.}  will be very helpful in the long run. Non-alphabetic symbols (e.g., {\ttfamily * = +}) are appealing for marking grammatical tone, but a warning here is appropriate. The advantages of these marks is that they are already present on the keyboard, they can be written in line with the other characters rather than going back to add a diacritic, and they can mark an easily recognized \textit{meaning} rather than the harder to process phonetics. However, the Unicode \textit{characteristics} of these symbols mean that many programs will not treat them as part of the word, but will split them off from the usual consonants and vowels. Publishing can potentially be hindered if this issue is neglected.

\item Finally, consider the writer as well as the reader. Active literacy in a language involves simplicity of writing as well as reading.

\end{enumerate}
The following are additional factors to consider as possibilities in orthography design, though I do not suggest them as firmly as the above recommendations. These seem reasonable, but have not been proven through practical experience to the extent that the definite recommendations above were.

When extra symbols are needed, consider writing them \textit{in line} with other letters, rather than accents above the letter (e.g., \textit{\^{}}\textit{baba}, not \textit{bába}). These are easier to write, since the pencil or pen does not have to be lifted to a separate tier (think of writing an English word like \textit{constitution}, which requires dotting 〈i〉s and crossing 〈t〉s.) More testing and experience is needed, but this may also be possibly easier to read.

\begin{quote}
Once the initial strangeness of such symbols in the orthography [Bokyi, see \tabref{tab:cahill:BokyiOrtho:14}] has been overcome, and their function is understood, teams learn to use them quite quickly and can get quite excited about them. But teaching phonetic tone-marking using accents is always a struggle here, and very, very few ever master it. (Harley, p.c.)\ia{Harley, Matthew@Harley, Matthew}
\end{quote}

If both grammatical and lexical tone are to be marked, mark them with different systems. Testing in Togo (Kabiyé language), Roberts marked lexical tone with accents, and grammatical tone with other characters. Roberts comments that readers seemed to “feel” the grammar more than the sound system. 

\subsection{A closing question}
\label{sec:Conclusion:ClosingQuestion:2}
One convention that has been fairly widely practiced, but also has been opposed for theoretical reasons, is “selective tone marking.” Selective tone marking applies tone marking only to one word of a minimal tone pair, leaving the other unmarked. Thus if a language has two words [bóbò] and [bóbó], with different meanings, they could be written as 〈bóbo〉 and 〈bobo〉. Selective tone marking thus contrasts with marking tone more extensively or exhaustively.

\citet[16]{Wiesemann1989} and \citet[132--133]{Longacre1953} assert that selective tone marking should be avoided. Wiesemann gives the following reason for rejecting selective tone marking:

\begin{quote}
It should be mentioned here that a system which marks tone where it is minimally different in individual words is not a good system. In such a system, for each individual word one must learn whether it carries a tone mark or not. To mark low tones only on words where there is a minimal tone pair makes the teaching of tone a matter of memory, rather than a matter of rules linked to pronunciation.
\end{quote}

\citet[133]{Longacre1953} adds the point that selective tone marking “presupposes that one has already made a list of all the words in the language to see which ones are minimal pairs. Such a claim is pretentious since most newly written languages do not have good dictionaries.” 

Thus two reasons for avoiding selective tone marking are 1) the memory load of having to know all the individual words which must be marked and 2) the improbability of the orthography designer knowing all such word pairs (or triplets, or more) that need to be marked.

However, dictionaries that include a large percentage of lexemes in a language are easier to produce now than in past years (\url{http://www.rapidwords.net/}). Also, the preference for rules rather than memorization a) is possibly a relic of Western education, with its bias against rote memorization, and b) ignores the fact that much of our successful (!) English orthography also depends on memorization rather than rules, as the examples in \tabref{tab:VariableEnglishPronunciation} show.

\begin{table}
    \begin{tabular}{llll}
        \lsptoprule
        Spelling of 〈ough〉 words & Phonetics &  Spelling of 〈ear〉 words & Phonetics\\
        \midrule
        \textit{cough} & [ɑf] &  \textit{hear} & [iɹ]\\
        \textit{though} & [o] &  \textit{heard} & [ɚ]\\
        \textit{through} & [u] &  \textit{heart} & [ɑɹ]\\
        \lspbottomrule
    \end{tabular}
    \caption{Variable English pronunciation of same spellings}
    \label{tab:VariableEnglishPronunciation}
\end{table}

English orthography is far from being an ideal model, but if such a widely-used orthography can depend so much on memorization, then the argument based on memory loses its force. So a better case can probably be made for selective tone marking than previous scholars have argued.

\section*{Acknowledgments}
This paper was first presented at the Academic Forum of the Graduate Institute of Applied Linguistics, Dallas, before its presentation at the 48\textsuperscript{th} Annual Conference on African Languages at Indiana University. I am grateful for comments and interesting discussion from both audiences. I also acknowledge valuable comments from two reviewers in preparation for the ACAL proceedings.


{\sloppy\printbibliography[heading=subbibliography,notkeyword=this]}
\end{document}

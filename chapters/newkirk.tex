\documentclass[output=paper]{langscibook} 
\ChapterDOI{10.5281/zenodo.3520597}
\author{Lydia Newkirk\affiliation{Rutgers University}}
\title{Logophoricity in Ibibio}
 
\abstract{This paper presents a description and analysis of logophoric pronouns in Ibibio. I show that Ibibio logophors, although they behave in most respects like typical logophoric pronouns in West African languages, obey Shift Together like shifted indexicals. In order to explain this data I propose that Ibibio logophors are sensitive to two operators in the left periphery of the embedded CP: a shifting operator and a logophoric binding operator. Ibibio indexicals (which do not shift) differ in that they are defined to be insensitive to shifting operators. Thus indexical shift requires cooperation between the semantics of the indexical and of the shifting operator. This proposal in turn expands the predicted typology of possible \emph{de se} pronouns cross-linguistically.}

\maketitle

\begin{document}
\section{Ibibio logophors}

Ibibio (Cross-River, Nigeria) logophors are distinct from both ordinary pronouns and reflexives (\ref{comp}), and can only occur in embedded clauses (\ref{embedonly}).\footnote{With the exception of the logophoric markers, on which I annotated the tone for clarity, I use the Ibibio orthography developed in \citet{Essien1990}, which does not mark tone.}

\begin{exe}
\ex\label{comp} \begin{xlist}
		\ex[]{\gll Ekpe\ix{i} a-bo ke	(\textbf{im\d{o}})\ix{i} \textbf{ì}-ma \textbf{í}-to Udo\\
			Ekpe \textsc{3sg}-say C \textsc{log} \textsc{log}-\textsc{pst} \textsc{log}-hit Udo\\ \hfill {\small logophor}
			\glt `Ekpe\ix{i} says that he\ix{i} hit Udo.'}
		\ex[]{\gll Ekpe\ix{i} a-bo ke \textbf{anye}\ix{i/k} \textbf{a}-diy\d{o}\={n}\d{o} ikwo ikwo mf\d{o}nmf\d{o}n\\
		Ekpe \textsc{3sg}-say C \textsc{3sg} \textsc{3sg}-know sing song well\\ \hfill {\small pronoun}
		\glt `Ekpe\ix{i} says that he\ix{i/k} sings well.'}\label{pro}
		\ex[]{\gll Ekpe\ix{i} a-bo ke ì-ma í-t\d{o} \textbf{idem}\ix{i} \\
		Ekpe \textsc{3sg}-say C \textsc{log-pst} \textsc{log}-hit self \\ \hfill {\small reflexive}
		\glt `Ekpe\ix{i} says that he\ix{i} hit himself\ix{i}'}\label{refl}
	\end{xlist}
	
	\ex\label{embedonly} \begin{xlist}
		\ex[*]{\gll Ekpe\ix{i} a-ma a-diya adesi \textbf{im\d{o}}\ix{i}\\
				Ekpe \textsc{3sg-pst} \textsc{3sg}-eat rice \textsc{log.poss}\\ \hfill {\small logophor}
				\glt Intended: `Ekpe\ix{i} ate his\ix{i} rice' }\label{possnomatter}
	\ex[]{\gll Ekpe\ix{i} a-ma a-diya adesi \textbf{am\d{o}}\ix{i/k}\\
				Ekpe \textsc{3sg-pst} \textsc{3sg}-eat rice \textsc{3sg.poss}\\ \hfill {\small pronoun}
				\glt `Ekpe\ix{i} ate his\ix{i/k} rice' }\label{nomat}
	\end{xlist}
\end{exe}
Logophoric pronouns differ in both agreement and morphology from the other pronominals in Ibibio (both the pronominals and logophors are listed in Table 1). However, like all of the other pronouns in the language, their independent forms are optional except in cases of emphasis or disambiguation. In (\ref{possnomatter}), the logophor cannot occur as a possessive in an unembedded clause, even though in the same context a regular third person pronoun is acceptable as corefering to the local subject.

\begin{table}
\centering
\begin{tabular}{>{\scshape}ccccc}
\lsptoprule
	& \multicolumn{2}{c}{Pronouns}	&	\multicolumn{2}{c}{Agreement}	\\\cmidrule(lr){2-3}\cmidrule(lr){4-5}
	&	Subject		&	Object		&	Subject 	&	Object			\\ \midrule
1sg	&	ami	&	mien	&	n- & n-	\\
2sg	&	afo	&	fien	& a-	& u- 		\\
3sg	&	anye	&	anye	& a-	&	a-	\\ 
1pl	&	\multicolumn{2}{c}{nny\d{i}n}	&	i-	&	i-	\\
2pl	&	\multicolumn{2}{c}{ndufo}	& e-	& e-	\\
3pl	&	\multicolumn{2}{c}{\d{o}mm\d{o}} & e-		& e- \\
\textsc{log.sg}	& \multicolumn{2}{c}{im\d{o}}	&	i-	& i-	\\
\textsc{log.pl}	&	\multicolumn{2}{c}{mmim\d{o}}	& i-	& i-	\\
\lspbottomrule
\end{tabular}
\caption{Ibibio pronouns and agreement markers}
\end{table}

Note from (\ref{pro}) that Ibibio patterns like Yoruba \citep[(\ref{yor})]{Adesola2005}, which has a strong pronoun \textit{oun} that must refer to the attitude holder, and a weak pronoun \textit{o} that is ambiguous between referring to the matrix attitude holder and taking some other third person referent. By that same token, Ibibio's pattern is distinct from Abe's \citep[(\ref{abe})]{Koopman1989}, in that regular Ibibio pronouns do not show anti-logophoricity; a regular pronoun can still corefer with the matrix subject.\footnote{In these and other examples from the literature, I preserve the glossing and capitalization from the cited work.}
\begin{exe}
	\ex{\citep[34]{Adesola2005}\label{yor}\\\gll Ol\'u\ix{i} ti k\'ede NO\ix{\set{i}[+LOG]} p\'e oun\ix{i}/\textbf{o}\ix{i/k} \`n b\s{\`o} l\s{\'o}la\\
	Olu \textsc{asp} announce {} that he \textsc{prog} come tomorrow\\
	\glt `Olu\ix{i} has announced that he\ix{i}/he\ix{i/k} is coming tomorrow.'}
	
	\ex{\citep[64a]{Koopman1989}\label{abe}\\\gll yapi\ix{i} hE kO \textbf{O}\ix{k}/n\ix{i,(k)} ye sE\\
	Yapi said kO he is handsome\\
	\glt `Yapi\ix{i} said that he\ix{i/*k} is handsome'}
\end{exe}
Assuming the facts from Abe are robust, this points to one potential variance in how logophors and regular pronouns tend to get their antecedents.

There are also separate plural logophoric pronouns in Ibibio:
\begin{exe}
	\ex{\gll \d{o}mm\d{o}\ix{i} e-ke e-bo ke \textbf{mmim\d{o}}\ix{i/*k} ì-ma í-kot \={n}wet\\
	\textsc{3pl} \textsc{3pl-pst} \textsc{3pl}-say C \textsc{log.pl} \textsc{log-pst} \textsc{log}-read book\\
	\glt `They\ix{i} said that they\ix{i/*k} read a book.'}\label{pl}
\end{exe}
Split antecedence is possible with a plural logophor (see also Yoruba; Adesola 2005), so long as the closest potential logophoric antecedent is included in the group taken to be the antecedent for the logophor.
\begin{exe}
	\ex{\gll Ekpe\ix{i} a-bo ke \textbf{mmim\d{o}}\ix{\set{i,k}} í-diya af\d{i}t adesi ad\d{o}\\
			Ekpe \textsc{3sg}-say C \textsc{log.pl} \textsc{log}-eat all rice \textsc{dem}\\
			\glt `Ekpe\ix{i} says that they\ix{\set{i,k}} ate all of the rice'}

	\ex{\gll Ekpe\ix{i} a-bo ke Udo\ix{j} a-kere ke ete \textbf{mmim\d{o}}\ix{\set{i,j}/\set{j,k}} a-ya ì-di í-w\d{o}\\
	Ekpe \textsc{3sg}-say C Udo \textsc{3sg}-think C father \textsc{log.pl} \textsc{3sg-fut} \textsc{log}-come \textsc{log}-visit\\
	\glt `Ekpe\ix{i} says that Udo\ix{j} thinks that their\ix{\set{i,j}/\set{j,k}} father will come visit.'}\label{split}
\end{exe}
The inverse relation, where a singular logophor has a plural antecedent, is impossible:
\begin{exe}
	\ex[*]{\gll [Akpan ye Udo]\ix{i} e-ma e-bo ke im\d{o}\ix{i} ì-ma í-diya sokoro\\
	Akpan \textsc{conj} Udo \textsc{3pl-pst} \textsc{3pl}-say C \textsc{log} \textsc{log-pst} \textsc{log}-eat orange\\
	\glt Intended: `[Akpan and Udo]\ix{i} said that he\ix{i} ate the orange.'}\label{badpl}
\end{exe}
This all is analogous to the facts of Yoruba as reported by \citet{Adesola2005}.


Logophors in Ibibio are subject-oriented, established in (\ref{hear}--\ref{tell}):
\begin{exe}
	\ex \begin{xlist}
	\ex{\gll Ekpe\ix{i} a-ma a-kop ke Udo\ix{k} a-ma í-k\d{i}t\\
	Ekpe \textsc{3sg-pst} \textsc{3sg}-hear C Udo \textsc{3sg-pst} \textsc{log-}see\\
	\glt `Ekpe\ix{i} heard that Udo\ix{k} saw him\ix{i}'}\label{hear}
	
	\ex{\gll Ekpe\ix{i} a-ma a-kop a-to Akpan\ix{j} ke Udo\ix{k} a-ma í\ix{i/*j}-k\d{i}t\\
	Ekpe \textsc{3sg-pst} \textsc{3sg}-hear \textsc{3sg}-from Akpan C Udo \textsc{3sg-pst} \textsc{log-}see\\
	\glt `Ekpe\ix{i} heard from Akpan\ix{j} that Udo\ix{k} saw him\ix{i/*j}}\label{hearfrom}
	\end{xlist}
	
	\ex{\gll Akpan\ix{i} a-ma a-d\d{o}kk\d{o} Ekpe\ix{k} ke a-kpe a-na nte (im\d{o}\ix{i/*k}) í-dep adesi mf\d{i}n\\
	Akpan \textsc{3sg-pst} \textsc{3sg}-tell Ekpe C \textsc{3sg-cond} \textsc{3sg-mod} C \textsc{log} \textsc{log}-buy rice today\\
	\glt `Akpan\ix{i} told Ekpe\ix{k} that he\ix{i/*k} should buy rice today.'}\label{tell}
\end{exe}
Rather than being tuned to the source in (\ref{hearfrom}), the embedded logophor can only refer to the syntactic subject, which `hear' licenses as a logophoric antecedent independently (\ref{hear}). Similarly, in (\ref{tell}), a logophor cannot refer to the addressee introduced in the matrix clause, but must refer to the subject.


Like what has been reported for Yoruba \citep{Adesola2005,Anand2006}, but unlike what has been reported for Ewe \citep{Pearson2015}, Ibibio logophors are obligatorily interpreted \textit{de se}:\footnote{\textit{De se} in this case refers to knowing self ascription of a property. For example, the \textit{de se} reading for ``John believes that he wrote the best paper" is the reading where John has the belief ``I (John) wrote the best paper." This differs from a \textit{de re} reading, where John may have read his paper without remembering that he wrote it, although he still comes to the conclusion that the paper that he wrote (but does not remember) is the best paper. Because in this case John does not identify himself as the writer of the best paper, this latter interpretation is \textit{de re}.}
\begin{context}
		Ekpe sings on occasion, but will never admit that he is any good. So one time, during one of his performances, you record him without his knowledge. Some time later, you play back the recording to him without telling him who is singing. Ekpe doesn't recognize himself in the recording, and comments ``he sings well."
		\end{context}
		\begin{exe}
		
		\ex \begin{xlist}
			\ex[]{\gll Ekpe\ix{i} a-bo ke \textbf{anye}\ix{i/k} a-diy\d{o}\={n}\d{o} ikwo ikwo mf\d{o}nmf\d{o}n\\
		Ekpe \textsc{3sg}-say C \textsc{3sg} \textsc{3sg}-know sing song well\\
		\glt `Ekpe\ix{i} said that he\ix{i/k} sings well.' }
			\ex[\#]{\gll Ekpe\ix{i} a-bo ke \textbf{im\d{o}}\ix{i} ì-me í-diy\d{o}\={n}\d{o} ikwo ikwo mf\d{o}nmf\d{o}n\\
		Ekpe \textsc{3sg}-say C \textsc{log} \textsc{log-pres} \textsc{log}-know sing song well\\
		\glt Intended: `Ekpe\ix{i} says that he\ix{i} sings well.'}
	\end{xlist}
	\end{exe}
In the above context, Ekpe does not knowingly attribute singing well to himself, but instead does so accidentally. That is, he only ascribes singing well to himself \textit{de re}, but not \textit{de se}. In such a context, only a regular Ibibio third person pronoun can be used, and the logophor is illicit.
	
	
When multiply embedded, Ibibio logophors can take antecedents more than one clause away:
\begin{exe}
	\ex{\gll Ekpe\ix{i} a-bo ke Udo\ix{k} a-ke a-kere ke (im\d{o}\ix{i/k}) ì-ke í-k\d{i}t Ima \\
	Ekpe \textsc{3sg}-say C Udo \textsc{3sg-pst} \textsc{3sg}-think C \textsc{log} \textsc{log-pst} \textsc{log}-see Ima \\
	\glt `Ekpe\ix{i} says that Udo\ix{k} thinks that he\ix{i/k} saw Ima.'}

\end{exe}
But when there is more than one logophor in the same clause, the coreference options are more limited:
\begin{exe}

	\ex[]{\gll Ekpe\ix{i} a-ma a-kop ke Udo\ix{k} a-ke a-bo ke {ayin-eka} \textbf{im\d{o}}\ix{k/\**i} a-ma a-k\d{i}t \textbf{im\d{o}}\ix{k/\**i} ke udua \\
			Ekpe \textsc{3sg-pst} \textsc{3sg}-hear C Udo \textsc{3sg-pst} \textsc{3sg}-say C brother \textsc{log.poss} \textsc{3sg-pst} \textsc{3sg}-see \textsc{log} at market \\
			\glt `Ekpe\ix{i} heard that Udo\ix{k} said that his\ix{k/*i} brother saw him\ix{k/\**i} at the market.'}\label{weakcross}
			
	\ex[*]{\gll Ekpe\ix{i} a-ma a-kop ke Udo\ix{k} a-ke a-bo ke \textbf{im\d{o}}\ix{i/k} ì-ma í-t\d{o} \textbf{im\d{o}}\ix{i/k} \\
	Ekpe \textsc{3sg-pst} \textsc{3sg}-hear C Udo \textsc{3sg-pst} \textsc{3sg}-say C \textsc{log} \textsc{log-pst} \textsc{log}-hit \textsc{log} \\
	\glt Intended: `Ekpe\ix{i} heard that Udo\ix{k} said that he\ix{i/k} hit him\ix{i/k}'}\label{condB}
\end{exe}
While (\ref{weakcross}) is grammatical, the only available interpretation is the one where both logophors take the same antecedent.\footnote{An 
    anonymous reviewer asked whether multiple multiply-embedded logophors must also take the closest antecedent. Unfortunately I do not have data to confirm whether this is in fact the case. However, if Ibibio logophors indeed behave like shifted indexicals (as I will claim), then it should be possible for the two logophors to take a more distant antecedent, as in Zazaki \citep{Anand2004}:
    \begin{exe}
    	\ex{\gll (Andrew): Ali\ix{A} mɨ\ix{U}-ra va kɛ  Hɛseni\ix{H} \textbf{to}\ix{U}-\textbf{ra} va ɛz\ix{\set{H,A,\**U}} braye Rojda-o \\
    	{} Ali me-to said that Hesen you-to said I brother Rojda-\textsc{gen} \\
    	\glt `Ali said to Andrew that Hesen said to Andrew that \set{\text{Hesen}, \text{Ali}, \**\text{Andrew}} is Rojda's brother.'}\label{complicated}
    \end{exe}
    This example is unfortunately complex, because there is potentially multiple shifting acts happening. But most importantly, although Hesen \textbf{can} be the antecedent for the deeply embedded first person pronoun, that is not the only reading of the sentence. Ali, the more distant attitude holder, is also eligible to antecede the shifted first person pronoun. I have no reason to expect Ibibio logophors to behave otherwise.
    
    The reviewer also asked whether reflexives have any impact on multiple logophors or long-distance antecedents. Logophoric reflexives are not themselves long-distance reflexives, but only local anaphors, and as such would have to have a logophor as a local antecedent, as in (\ref{refl})
}
More striking is that in (\ref{condB}) this effect is not ameliorated even to avoid a Condition B violation.\footnote{A
    Condition B violation occurs when a (non-reflexive) pronoun is bound locally. For example, \textit{John\ix{i} likes him\ix{k/\**i}} is unacceptable on an interpretation where \textit{John} and \textit{him} refer to the same person (that is, \textit{John likes him} cannot mean \textit{John likes himself}), because the pronoun would be bound locally, which violates Condition B of the Binding Theory, which states that a pronominal must be free within its clause. Note that if the pronoun is not in the same clause as its antecedent, binding is possible (e.g., \textit{John\ix{i} said that he\ix{i} is a genius}), because this does not violate Condition B.
}
Instead, the sentence is simply ungrammatical. Similarly the context in (\ref{markete}) fails to license two embedded logophors:
\begin{context}
	Udo and Akpan are two young schoolchildren, and are brothers. Ekpe is their friend, and is the same age as they are. One day, Udo and Akpan's father comes home and says that he saw Ekpe at the market when Ekpe was supposed to be in school. Word that he has been spotted skipping class gets back to Ekpe.
	\end{context}
	\begin{exe}	
		\ex[\#]{\gll Ekpe\ix{i} a-ma a-kop ke [Akpan ye Udo]\ix{k} e-ke e-bo ke ete mmim\d{o}\ix{k} a-ma í-k\d{i}t im\d{o}\ix{i} ke udua\\
		Ekpe \textsc{3sg-pst} \textsc{3sg}-hear C Udo \textsc{conj} Akpan \textsc{3pl-pst} \textsc{3pl}-say C father \textsc{log.pl.poss} \textsc{3sg-pst} \textsc{log}-see \textsc{log} \textsc{prep} market\\
		\glt Intended: `Ekpe\ix{i} heard that [Akpan and Udo]\ix{k} said that their\ix{k} father saw him\ix{i} at the market.'}\label{markete}	
\end{exe}
The context in (\ref{markete}) does not support partial/split antecedence for the plural logophor, and so the plural embedded logophor has to take a different antecedent than the singular logophor. But (\ref{markete}) is completely infelicitous in this context. In sum: Clausemate logophors (in Ibibio at least) have to refer together.\largerpage

This particular restriction on logophors is (to my knowledge) unattested in the \textit{de se} literature. Abe, Yoruba, and Ewe are all reported to allow multiple embedded logophors to take separate antecedents:
\begin{exe}
	\ex Abe \citep[41, 44a]{Koopman1989} \begin{xlist} 
		\ex{\gll \textbf{n}\ix{i} ceewu n kolo \textbf{n}\ix{i/\**k}\\
			\textit{n} friend Det likes \textit{n}\\
			\glt `his\ix{i} friend likes him\ix{i/*k}'}\label{abemat}
	\ex{\gll Api\ix{i} {bO wu} ye \textbf{n}\ix{i/k} kolo \textbf{n}\ix{i/k}\\
			Api believe \textit{ye} \textit{n} likes \textit{n}\\
			\glt `Api\ix{i} believes that he\ix{i/k} likes him\ix{i/k}'}\label{abemult}
			\end{xlist}
			
	\ex{Yoruba \citep[177]{Anand2006}\\
	\gll Olu\ix{i} so p\'e Ade\ix{k} ro p\'e b\`ab\'a \textbf{oun}\ix{i/k} ti r\`\bari {\`\bari}y\'a \textbf{\`oun}\ix{i/k}\\
			Olu say that Ade think that father oun.gen \textsc{perf} see mother oun.gen\\
			\glt `Olu\ix{i} said that Ade\ix{k} thought that his\ix{i/k} father had seen his\ix{i/k} mother.'}\label{yormult}
			
	\ex{Ewe \citep[73]{Clements1975}\\
	\gll Kofi\ix{i} xɔ-e se be Ama\ix{k} gblɔ be \textbf{y\`e}\ix{i/k}-ʄu \textbf{y\`e}\ix{i/k}\\
			Kofi receive-\textsc{pro} hear that Ama say that \textsc{log}-beat \textsc{log}\\
			\glt `Kofi\ix{i} believed that Ama\ix{k} said that he\ix{i} beat her\ix{k}' or\\
			`Kofi\ix{i} believed that Ama\ix{k} said that she\ix{k} beat him\ix{i}'}\label{ewemult}
\end{exe}
The Abe data requires a bit of explanation: \citet{Koopman1989} report that \textit{n}-series pronouns (logophors) are licit in matrix clauses, but if two of them occur they must have the same antecedent (\ref{abemat}). When embedded, however, they are able to receive disjoint interpretations. This puts (\ref{abemult}) in the same general pattern with the Yoruba (\ref{yormult}) and Ewe (\ref{ewemult}) examples, and all of them in contrast with Ibibio.

In order to explain this unusual property of Ibibio logophors, I must take a brief detour into the shifted indexicals literature, which will shed light on the difference between Ibibio and other logophoric languages.



\section{Ibibio logophors as shifted indexicals}


\subsection{Indexical shift cross-linguistically}

The leading analysis of shifted indexicals in the literature is that proposed by \citet{Anand2006}, and essentially followed by \citet{Sudo2012,Shklovsky2014,Deal2017}, {inter alia}. Descriptively speaking, shifted indexicals are cases of person, locative, or temporal indexicals (such as \textit{I}, \textit{here}, or \textit{yesterday}) which, when embedded under an attitude verb or verb of saying, do not refer to the utterance context, but instead refer to the context established by the embedding verb (in these examples, \textsc{auth}(c) denotes the speaker of the entire utterance, and \textsc{addr}(c) denotes the addressee of that utterance. That is, they are used for the English non-quotative senses of \textit{I} and \textit{you}).
\begin{exe}
	\ex{Zazaki (Indo-Iranian, Turkey), \citep[13]{Anand2004}\\
	\gll vɨzeri Rojda Bill-ra va kɛ {\textbf{ɜz}} \textbf{to}-ra miradi\u{s}a\\
	yesterday Rojda Bill-to said that \textbf{I} \textbf{you}-to angry.be-\textsc{pres}\\
	\glt `Yesterday Rojda said to Bill, ``I am angry at you."'\\
		`Yesterday Rojda said to Bill, ``\textsc{auth}(c) is angry at \textsc{addr}(c)."'\\
		\**`Yesterday Rojda said to Bill, ``\textsc{auth}(c) is angry at you."'\\
		\**`Yesterday Rojda said to Bill, ``I am angry at \textsc{addr}(c)."'}\label{zaz}
\end{exe}
One property of shifted indexicals cross-linguistically is that they obey Shift Together: Two indexicals embedded under the same attitude verb must either both shift or neither shift, as demonstrated by the possible interpretations of (\ref{zaz}), and illustrated schematically below.
\begin{exe}
	\ex{\textsc{Shift Together} Constraint \citep[297]{Anand2006}\\
		All shiftable indexicals within an \textit{attitude-context domain} must pick up reference from the same context. \begin{xlist}
		\ex[]{ \textbf{C\ix{A}} [ \ldots \textit{modal} \textbf{C\ix{B}} \ldots [ ind$_{{1}}^{{\textbf{A}}}$ \ldots ind$_{{2}}^{{\textbf{A}}}$ ]]}
		\ex[]{ \textbf{C\ix{A}} [ \ldots \textit{modal} \textbf{C\ix{B}} \ldots [ ind$_{{1}}^{{\textbf{B}}}$ \ldots ind$_{{2}}^{{\textbf{B}}}$ ]]}
		\ex[*]{ \textbf{C\ix{A}} [ \ldots \textit{modal} \textbf{C\ix{B}} \ldots [ ind$_{{1}}^{{\textbf{A}}}$ \ldots ind$_{{2}}^{{\textbf{B}}}$ ]]}
		\ex[*]{ \textbf{C\ix{A}} [ \ldots \textit{modal} \textbf{C\ix{B}} \ldots [ ind$_{{1}}^{{\textbf{B}}}$ \ldots ind$_{{2}}^{{\textbf{A}}}$ ]]}
		\end{xlist}}
\end{exe}

\citet{Anand2006} derives Shift Together by defining shifting operators that override the context values under attitude verbs. Where context parameters typically refer directly to the utterance context, these operators modify the context so that indexicals in their scope refer to the context set by the attitude verb, rather than the context set by the utterance.
% % Anand (2006) derives Shift Together by defining shifting operators that override the context values under attitude verbs. Where context parameters typically refer directly to the utterance context, these operators modify the context so that indexicals in their scope refer to the context set by the attitude verb, rather than the context set by the utterance.
\begin{exe}
	\ex\label{authshift} \den{OP\ix{auth} $\alpha$}$^{c,i}$ = \den{$\alpha$}$^{j,i}$, where $j$\,=\,\type{\textsc{auth}(\textbf{i}),\textsc{addr}(c),\textsc{time}(c),\textsc{world}(c)}
	\ex \den{OP\ix{per} $\alpha$}$^{c,i}$ = \den{$\alpha$}$^{j,i}$, where $j$\,=\,\type{\textsc{auth}(\textbf{i}),\textsc{addr}(\textbf{i}),\textsc{time}(c),\textsc{world}(c)}
\end{exe}
Because the operators overwrite the contextual information rather than simply adding to it, any indexical dependent on an overwritten value is forced to shift, and can never ``un-shift".

Another analysis present in the literature is that of \citet{Schlenker2003}. He proposes in indexical shift languages, the shiftable indexicals are lexically defined to optionally shift under the right sort of attitude verb. Non-shiftable indexicals, on the other hand, are defined to always take the utterance context, rather than any embedded context variable. In this sense, they act very much like bindees under an attitude verb (c\** refers to the utterance context).\pagebreak
\begin{exe}
	\ex \begin{xlist}
		\ex English `I': +indexical, +c\**
		\ex Amharic `I': +indexical, [underspecified]
	\end{xlist}
\end{exe}
There is one main objection to Schlenker's proposal: If shiftable indexicals are underspecified for a context variable, two embedded indexicals are expected to be able to take different context variables; Shift Together is left unexplained. As a result, the theory by \citet{Anand2006} summarized above is the analysis more commonly used in the literature. However, Schlenker's theory will come into play for my analysis of Ibibio logophors, which I now turn to.



\subsection{Logophors as shifted indexicals}

A tempting solution to the problem of Ibibio logophors is to propose that they are actually first-person indexicals, and that Ibibio has a shifting operator that shifts those indexicals like in Zazaki and Amharic. This is unfeasible however, because true Ibibio indexicals never shift, even if they occur clausemate with a logophor:
\begin{exe}
	\ex{\gll Ekpe\ix{i} a-kere ke (\textbf{im\d{o}}\ix{i}) i-ma i-n-k\d{i}t \textbf{mien} \\
	Ekpe \textsc{3sg}-think C \textsc{log} \textsc{log-pst} \textsc{log-1sg}-see \textsc{1sg.obj} \\
	\glt `Ekpe\ix{i} thinks that he\ix{i} saw me.'}\label{shift}
\end{exe}
If there is a shifting operator present in (\ref{shift}) that overwrites the \textsc{auth} value in the context (as would be expected for a pronoun that refers to the attitude-holder), then the first person indexical should also shift. But the true indexical stays constant to the utterance context.

In light of this, I claim that although an operator-based approach is essentially correct for shifted indexicals, the behavior of Ibibio logophors indicates that operators alone are not sufficient to account for indexical shift. I propose to integrate Schlenker's insight that the pronominals should be defined as shiftable, but with the adjustment that a pronominal's sensitivity to shifting is defined lexically, and no pronominal is underspecified for what context variable it takes. Either a pronominal will always shift in the presence of an operator, or it never will. For Ibibio, this means that its logophors are defined to shift, so that they take a shiftable context variable, while true Ibibio indexicals are defined as unshiftable: they take only the matrix context directly.
\begin{exe}
	\ex \begin{xlist}
		\ex{\denol{im\d{o}}$^{g,c}$ = \textsc{auth}(c) \hfill \textsc{shiftable}}
		\ex{\den{\textsc{1sg}}$^{g,c}$ = \textsc{auth}(c\**) \hfill \textsc{not shiftable}}
	\end{xlist}
\end{exe}
The actual shifting for the logophor cases is accomplished by the author shifter in (\ref{authshift}) above. Shiftable logophors will otherwise receive the same interpretation as unshiftable logophors; the distinction between the two will emerge only in deeply embedded clauses where multiple logophors appear; that is, in precisely the complex cases I discuss in this paper. In these examples, shifting logophors will obligatorily take the same antecedent (as is the case in Ibibio), while non-shifting logophors will not be obligated to take the same antecedent (as in Yoruba, Able, and other languages).

A relevant question at this point is whether Ibibio logophors are merely shifted indexicals, or whether there is also logophoric binding. A brief consideration of the \textit{De Re} Blocking Effect indicates that Ibibio logophors are also true logophors, involving binding by a logophoric operator.

The \textit{De Re} Blocking Effect \citep{Anand2006} states that a \textit{de se} pronominal (such as a logophor) cannot be c-commanded by a \textit{de re} pronominal. I illustrate this with the following examples from Yoruba \citep{Adesola2005}:
\begin{exe}
    \ex \langinfo{Yoruba}{}{\citealt{Adesola2005}}
    \begin{xlist}
	\ex{\gll Ad\'e\ix{i} so p\'e \textbf{oun\ix{i}} ti r{\`\bari} {\`\bari}w\'e \textbf{r\`e\ix{i,j}}\\
			Ade say that oun \textsc{perf} see book o-gen \\
			\glt `Ade\ix{i} said that he\ix{i} has seen his\ix{i,j} book.}\label{control}
	\ex{\gll Olu\ix{i} so p\'e \textbf{o\ix{\**i/j}} r{\`\bari} b\`ab\'a \textbf{\`oun\ix{i}}\\
			Olu say that o see father oun-gen\\
			\glt `Olu\ix{i} said that he\ix{\**i/j} has seen his\ix{i} father.'}\label{blocked}
	\ex{\gll Olu\ix{i} so p\'e b\`ab\'a \textbf{r\`e\ix{i/j}} r{\`\bari} {\`\bari}y\'a \textbf{\`oun\ix{i}}\\
			Olu say that father o-gen see mother oun-gen\\
			\glt `Olu\ix{i} said that his\ix{i/j} father has seen his\ix{i} mother.'}\label{unblocked}
	\end{xlist}
\end{exe}
In Anand's theory of logophoricity, (\ref{blocked}) does not allow the weak pronoun to refer to the logophoric center because in cases where it is co-indexed with the logophor, it is a competing binder for the more deeply embedded logophor, causing a condition B effect. This is ameliorated by interrupting c-command between the two pronouns, as in (\ref{unblocked}).

Crucially for determining the status of Ibibio logophors, shifted indexicals do not show \textit{De Re} Blocking Effects, demonstrated in Zazaki by (\ref{noblock}).

\begin{context}
At a friend's party, Hesen is shocked to see Ali, the boyfriend of his good friend Rojda, flirting with a woman in a big red dress and hat that obscures her face. After seeing her kiss Ali, Hesen rushes off to find Rojda. When he finds her, he tells her, ``The woman in the big red dress kissed your man." Of course, it was Rojda all along, only hidden under a costume!
\end{context}
\begin{exe}
	\ex{\citep[][333]{Anand2006}\\
	\gll Heseni va kɜ \textbf{Rojdaa} layik \textbf{tɨya} pach kerd\\
			Hesen\textsc{.obl} said that Rojda.\textsc{obl} boy your kiss did\\
			\glt `Hesen said (to Rojda\ix{i}) that Rojda\ix{i} kissed her\ix{i} man.'}\label{noblock}
\end{exe}
In the context, Hesen identifies Rojda only with the \textit{de re} relation ``the woman in the big red dress", making the occurance of \textit{Rojda} in the embedded clause \textit{de re}, while the embedded second person indexical is shifted to refer to Rojda. Despite the fact that the \textit{de se} indexical is c-commanded by the \textit{de re} name, the sentence is felicitous in the context. Anand takes this as evidence that shifted indexicals are not operator-bound in the same way that logophors are.

Therefore, if Ibibio logophors are in fact merely shifted indexicals that happen to look like logophoric pronouns, we can expect them to show no \textit{De Re} Blocking Effect, parallel to typical shifted indexicals. However, Ibibio logophors behave parallel to Yoruba logophors:\largerpage[2]
\begin{exe}
	\ex{\gll Ekpe\ix{i} a-ma a-bo ke \textbf{im\d{o}\ix{i}} ì-ma í-k\d{i}t ete \textbf{am\d{o}\ix{i/k}}\\
			Ekpe \textsc{3sg-pst} \textsc{3sg-}say C \textsc{log} \textsc{log-pst} \textsc{log}-see father \textsc{3sg.poss}\\
			\glt `Ekpe\ix{i} said that he\ix{i} saw his\ix{i/k} father.'}
	\ex{\gll Ekpe\ix{i} a-ma a-bo ke \textbf{anye}\ix{\**i/k} a-ma a-k\d{i}t ete \textbf{im\d{o}}\ix{i}\\
			Ekpe \textsc{3sg-pst} \textsc{3sg}-say C \textsc{3sg} \textsc{3sg-pst} \textsc{3sg}-see father \textsc{log.poss}\\
			\glt `Ekpe\ix{i} said that he\ix{\**i/k} saw his\ix{i} father.'}
	\ex{\gll Ekpe\ix{i} a-ma a-bo ke ete \textbf{am\d{o}}\ix{\**i/k} a-ma a-k\d{i}t eka \textbf{im\d{o}\ix{i}}\\
	Ekpe \textsc{3sg-pst} \textsc{3sg}-say C father \textsc{3sg.poss} \textsc{3sg-pst} \textsc{3sg}-see mother \textsc{log.poss}\\
	\glt `Ekpe\ix{i} said that his\ix{\**i/k} father saw his\ix{i} mother.'\footnote{Interestingly, interrupting the c-command relation in Ibibio does not seem to improve this case in Ibibio as it does in Yoruba. While this is an interesting distinction between Yoruba and Ibibio logophors that bears further investigation, it is orthogonal to the point that Ibibio logophors also show the \textit{De Re} Blocking Effect.}}
\end{exe}

According to \citet{Anand2006}, the \textit{De Re} Blocking Effect is due to binding competition between the \textit{de re} pronoun and the logophoric operator in the embedded left periphery. Given that Ibibio shows this effect, this is evidence that Ibibio logophors are not only shifted, but also logophorically bound. (Along with the fact that Ibibio logophors cannot appear in matrix clauses, cf.\ (\ref{nomat}).)

This means that Ibibio logophors are sensitive to both a shifting operator and a logophoric binding operator, a combination otherwise unattested in the \textit{de se} literature. Moreover, the fact that Ibibio logophors shift while Ibibio indexicals do not indicates that there is a conspiracy of factors required for indexical shift: Not only is there a shifting operator in the left periphery, but the indexicals of the language also have to be lexically sensitive to that shifting operator. This creates a new source of typological variance, which I elaborate on in the next section.

\section{Typological implications}
The above discussion of Ibibio logophoricity and its insight into indexical shift brings to light additional typological considerations; namely, it is now clear that languages can vary with regard to what pronominals are defined as shiftable, independent of what shifting operators (if any) are defined for the language.

This additional parameter only introduces minimal extra typological variance however, because for indexical shift to actually occur a language must have both shifting operators in its lexicon as well as some pronominal that is defined to shift under that operator. Similarly for logophors (as has been implicity assumed throughout the literature), a logophoric pronoun by itself is not sufficient for logophoric reference; it also must be bound by a logophoric operator and appropriately related to the attitude holder. Given this conspiracy of factors, the actual typology predicted is in Table \ref{type}, filled in with languages that (potentially) exemplify each typological option.\largerpage

\begin{table}[H]
\begin{tabularx}{\textwidth}{l>{\centering}Xcc}
          \lsptoprule
	&	&	\multicolumn{2}{c}{{Logophors}} \\\cmidrule(lr){3-4}
	&	{No Logophors}	&	{Shiftable}	&	{Unshiftable} \\ \midrule
{No Shifted Indexicals} &	English	&	Ibibio	&	Ewe,\footnote{\citet{Clements1975,Pearson2015}} Yoruba\footnote{\citet{Adesola2005}} \\
	{Shifted Indexicals}	& Zazaki\footnote{\citet{Anand2004,Anand2006}}\newline Amharic\footnote{\citet{Schlenker2003}}\newline Uyghur\footnote{\citet{Sudo2012,Shklovsky2014}}	&	\multicolumn{2}{c}{Aghem?\footnote{\citet{Hyman1979}}} \\
	\lspbottomrule
\end{tabularx}
\caption{Typology of logophors and shifted indexicals}\label{type}
\end{table}

I have already given examples of most of the languages types predicted, but Aghem requires some further comment. As described in \citet{Hyman1979}, Aghem might be an example of a language with both logophoric pronouns and shifted indexicals:
\begin{exe}
\ex\label{aghem}\langinfo{Aghem}{}{\citealt[14]{Hyman1979}}
\begin{xlist}%%% \renewcommand{\eachwordone}{\tipaencoding}
	\ex[?]{\gll w\`\i z\'\bari n 'v\'ʉ ndz\`ɛ \`a w\'\bari n \~{n}\'\bari'\'a \textbf{\'e} ŋg\'e 'l\'\bari gh\'a \textbf{w\`o}\\
	woman that said to him that \textbf{she/LOG} much like \textbf{you}\\
	\glt `The woman said to him that she liked him a lot.'\\
	`The woman said to him ``I like you a lot."'}
	
	\ex[]{\gll sǒog\`ɔʔ 'v\'ʉ m\'e \~{n}\'\bari'\'a \textbf{w\`o} l\`\bari gh\'a \textbf{m\`uɔ}, m\`ɔ \textbf{w\`o} mba\`aŋ l\'ɔ w\`\i{} b\`aʔt\`om$^{\circ}$\ldots\\
	soldier that (said) that \textbf{you} like \textbf{me} and \textbf{you} yet are {wife (of)} chief\\
	\glt `The soldier said, ``you like me, and yet you are the wife of the chief."'}
\end{xlist}
\end{exe}
These two examples are the only examples in \citet{Hyman1979} containing both logophors and embedded indexicals, or even potentially shifted indexicals at all. But to my knowledge Aghem indexicals have not been put through any tests to show that they are not quotation or partial quotation, nor are there are sentences with multiple embedded logophors, so there is no way to tell whether the logophoric pronouns behave like shifted indexicals either. Aghem's status as a indexical shift and logophoric language is therefore uncertain, but I mention is as an area of further investigation.


\section{Conclusion}\largerpage
In this paper I have described Ibibio logophors and situated them in the typology of \textit{de se} pronominals cross-linguistically. I have shown that they differ from other logophoric pronouns in that two clausemate logophors cannot take separate antecedents, but instead must refer together. This behavior, while unlike other logophoric languages, is reminiscent of a widely-attested restriction on shifted indexicals, which must Shift Together. I account for the Ibibio logophor behavior by proposing that they are sensitive to the same indexical shifting operator that is commonly proposed to account for indexical shift. True indexicals in Ibibio, which do not shift, are lexically defined as insensitive to this operator.
 
The introduction of lexical sensitivity to shifting operators expands the typology of \textit{de se} pronominals in a restricted way, allowing for the existence of languages like Ibibio, where logophors shift but regular indexicals do not, and potentially languages where both logophors and indexicals shift, as well as languages that have (unshiftable, but bound) logophors and shifted indexicals.


\section*{Abbreviations} 
\noindent\begin{tabularx}{.8\textwidth}{@{}>{\scshape}lQ@{}}
\textsc{1sg} & 1st person singular pronoun/marker\\
\textsc{1pl} & 1st person plural pronoun/marker\\
\textsc{2sg} & 2nd person singular pronoun/marker\\
\textsc{2pl} & 2nd person plural pronoun/marker\\
\textsc{3sg} & 3rd person singular pronoun/marker\\
\textsc{3pl} & 3rd person plural pronoun/marker\\
\textsc{auth} & author (or speaker)\\
\textsc{addr} & addressee\\
\textsc{asp} & aspect\\
C & complementizer\\
\textsc{conj} & conjunction\\
\textsc{log} & logophoric pronoun/marker\\
\textsc{obl} & oblique\\
\textsc{perf} & perfect\\
\textsc{poss} & possessive\\
\textsc{prep} & preposition\\
\textsc{pres} & present\\
\textsc{prog} & progressive\\
\textsc{pst} & past\\
\end{tabularx}

\section*{Acknowledgements}
Special thanks to Mfon Ud\d{o}inyang, Harold Torrence, the students of the Spring 2014 Field Methods class at the University of Kansas, Andrew McKenzie, and all the attendees of Syntactic Theory @ Rutgers (ST@R). This research was partially supported by NSF BCS-1324404.

{\sloppy \printbibliography[heading=subbibliography,notkeyword=this]}
\end{document}

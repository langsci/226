\documentclass[output=paper]{langscibook} 
\author{Khady Tamba\affiliation{Université Cheikh Anta Diop}}
\title{The Syntax of experiencers in Sereer-siin}

% \chapterDOI{} %will be filled in at production
 
\abstract{The special grammatical status of experiencers has been at the center of various studies within  linguistics for several decades \citep{Belletti1988,Ameka1990,Pesetsky1995}. This attention is due to the fact that there are specific syntactic features that are only associated with experiencers \citep{Landau2010}. This study investigates object experiencers in Sereer-Siin -- a dialect of Sereer, a West Atlantic language of the Niger Congo family spoken in Senegal. I show that object experiencers in Sereer can be classified in two groups according to their syntactic behavior. In this study I provide evidence that Sereer experiencers can be used to extend Belletti and Rizzi’s traditional classification of experiencers. Using data from Italian, Belletti and Rizzi have classified object experiencers in two groups. The first one they refer to as class II,  has a nominative theme and an accusative experiencer whereas the second one, class III, has a nominative theme and a dative experiencer. In Sereer, like in Italian, different properties of object experiencers account for asymmetries noted with respect to constructions like passive, antipassive and nominalization.}

\begin{document}
 
\maketitle 

% \href{mailto:khady.tamba@ucad.edu.sn}{khady.tamba@ucad.edu.sn}
 

\section{Introduction}
\subsection{Experiencers in Sereer}

Sereer-siin (Sereer) is a West Atlantic language of the Niger Congo family spoken in Senegal. It is genetically  related to Wolof  and Pulaar (\citealt{Simons2017}). There are various dialects of Sereer. This variety, considered the standard, is spoken in the areas traditionally known as Siin and Saalum.  The basic word order of Sereer is SVO. Sereer is an agglutinative language; this is exemplified in \REF{ex:tamba:1}.

\ea \label{ex:tamba:1}
\ea
\gll o          njaaƈ   onqe                            ɓog     -u            -ir               {}-k    {}-at  {}-ir    {}-an\\            
    \textsc{cl}  child  \textsc{cl\textsubscript{def}} wash   -\textsc{ref}  -\textsc{inst}    {}-\textsc{ipfv} {}-\textsc{it}  {}-\textsc{neg}   {}-\textsc{3obj}\\
\glt `The child will not wash him/herself with this again'
\ex
\gll xa   caaƈ   axe   ɓog  {}-u  {}-ir       {}-k  {}-at    -ir      {}-an \\
\textsc{cl\textsubscript{pl}}  child  \textsc{cl\textsubscript{def-pl}} wash   -\textsc{ref}  -\textsc{inst}  {}-\textsc{ipfv}   -\textsc{it}    {}-\textsc{neg}   {}-\textsc{obj}\\
\glt `The children will not wash him/herself with this again'
\z
\z
    
There are various studies dealing with the noun class system in Sereer \citep{Fal1980, Faye1979, Faye2013, Renaudier2012};  however there is no agreement about the number of noun classes found in Sereer. For instance, \citet{Faye2013} argues that Sereer has nine noun classes, whereas for \citet{Fal1980} and \citet{McLaughlin1992}, Sereer has thirteen noun classes. \tabref{tab:tamba:1} from \citet[284]{McLaughlin1992} gives a list of the different noun classes found in Sereer. 

\begin{table}
\caption{\label{tab:tamba:1}Sereer noun classes}

\begin{tabular}{llll}
\lsptoprule 
Class & Prefix & Clitic determiner & Class content\\\midrule
1 & o- & oxe & human singular\\
2 & ø{}- & we & human plural\\
3a & a- & ale & singular\\
3b & a- & ale & augmentative singular\\
4 & a- & ake & plural\\
5 & ø - & le & singular\\
6 & ø- & ne & singular\\
7 & ø - & fee & singular\\
8 & fo- & ole & plural\\
9 & ø- & ke & plural\\
10 & o- & ole & singular\\
11 & xa- & axe & plural\\
\lspbottomrule
\end{tabular}
\end{table}


  Indefiniteness is shown with the presence of a noun class marker (\textit{prefix} in \tabref{tab:tamba:1}) before the noun whereas definiteness is shown through the presence of a prenominal and a postnominal class marker (\textit{clitic determiner} in \tabref{tab:tamba:1}). 

  Consonant mutation is another important characteristic of Sereer. It is used to show subject-verb  agreement with regard to number; it is also used for inflectional and derivational processes (see McLaughlin for a detailed analysis on consonant mutation in Sereer). Finally, in Sereer, object pronouns are suffixes incorporated to the noun \citep{BaierTA}.

\subsection{Experiencers in Sereer}

There are different types of experiencers. An example of subject experiencer is shown in \REF{ex:tamba:2a} with the verb \textit{bug} `love'. Object experiencers are shown in (\ref{ex:tamba:2}b--d) with the verbs \textit{diidlat} `be scared', \textit{bet} `surprise' and \textit{sooɓ}  ‘miss’ respectively.


\ea \label{ex:tamba:2}
\ea Subject Experiencer \label{ex:tamba:2a}\\
\gll    Faatu    a              bug  {}-a              Maamkoor\\
        Faatu   \textsc{3sg}  love  {}-\textsc{perf}    Maamkoor\\
\glt    `Faatu loves Mamkoor'

\ex Object Experiencer \label{ex:tamba:2b}\\
\gll Awa a diid -lat -a o fes ole\\
    Awa \textsc{3sg} be.scared -\textsc{caus} \textsc{perf} \textsc{cl} young.man \textsc{cl\textsubscript{def}}\\
\glt `Awa frightened the young man'

\ex Object Experiencer \label{ex:tamba:2c}\\
\gll Faatu  a               bet         -a       o               njaaƈ       onqe\\
    Faatu   \textsc{3sg}    surprise    -\textsc{perf}  \textsc{cl}    child   \textsc{cl\textsubscript{def}} \\
\glt `Faatu surprised the child'


\ex Object Experiencer \label{ex:tamba:2d}\\
\gll o     fes             ole        a       sooɓ  a        o  tew   oxe\\
 \textsc{cl} young.man  \textsc{cl\textsubscript{} \textsc{def}}     \textsc{3sg}       miss  \textsc{perf}   \textsc{cl}  woman \textsc{cl\textsubscript{def}}\\
\glt   `The woman  misses the young man'
\z
\z 

In \REF{ex:tamba:2b} the psych verb appears with a causative suffix whereas in \REF{ex:tamba:2c}  and \REF{ex:tamba:2d} the verb does not bear any extra morphology. The rest of this paper is on object experiencers (\ref{ex:tamba:2}b--d).

\section{Previous Studies}

In their seminal work on experiencers, \citet{Belletti1988} use Italian to posit three classes of experiencers. They are illustrated in \REF{ex:tamba:3} from \citet[291--292]{Belletti1988}.

\ea \label{ex:tamba:3}
\langinfo{Italian}{Indo-European}{\citealt{Belletti1988}}\\
\ea \label{ex:tamba:3a}
 Gianni teme questo\\
\glt `Gianni fears this'

\ex \label{ex:tamba:3b}
 Questo preoccupa Gianni\\
\glt `This worries Gianni'

\ex \label{ex:tamba:3c}
 A Gianni piace questo\\
\glt `Gianni pleases this'

\ex \label{ex:tamba:3d}
 Questo piace a Gianni\\
\glt `This pleases to Gianni'
\z
\z 

Belletti \& Rizzi (B\,\&\,R)\ia{Belletti, Adriana@Belletti, Adriana|(}\ia{Rizzi, Luigi@Rizzi, Luigi|(} make the claim that \REF{ex:tamba:3a} and \REF{ex:tamba:3b} are  syntactically similar as they can both be related to transitive structures.  Thus verbs like \textit{temere} in \REF{ex:tamba:3a}  belong to a class of  experiencer verbs which have an experiencer as a subject and a theme as an object. In contrast, in \REF{ex:tamba:3b} with \textit{preoccupare} type verbs, the experiencer is in object position whereas the theme is in subject position.  Finally, in \REF{ex:tamba:3c} with experiencer verbs of the \textit{piacere} class, the verb appears with a dative experiencer as a subject and a nominative theme, however these arguments of the verb can appear in a different order as shown in \REF{ex:tamba:3d}. 

In his study on the syntax of experiencers, \citet{Pesetsky1995} puts aside the expression “theme” in B\,\&\,R  and argues for the following: 

\begin{itemize}
\item Subject argument with object experiencers are always \textit{Causer.}
\item Object experiencers always have one of these roles: Target of Emotion or Subject Matter of Emotion.
\end{itemize}


\citet[3]{Landau2010} builds on B\,\&\,R and uses the three classes of experiencers to classify English experiencers.  

\begin{itemize}
\item Class I: Nominative experiencer, accusative theme.\\
John loves Mary.

\item Class II: Nominative theme, accusative experiencer.\\
The show amused Bill.

\item Class III: Nominative theme, dative experiencer.\\
The idea appealed to Julie.
\end{itemize}

\citeauthor{Landau2010} argues that experiencers are mental locations (locatives) and that universally “object experiencers behave like oblique arguments, whether their governing preposition is overt or not” \citet[127]{Landau2010}. He further makes the claim that in some languages object experiencers are overtly quirky and for that reason can occur in subject position.


Throughout this paper I use B\,\&\,R's classification of Italian experiencers and adapt it to classify Sereer experiencers.

\subsection{Classifying Sereer experiencers using B\,\&\,R}

In this section, I show that Sereer experiencer verbs come in three classes (adap\-ted from B\,\&\,R). Verbs in the first class come with a nominative experiencer and an accusative theme whereas the verbs in the second class appear with a nominative causer, accusative experiencer. Finally, the last class of experiencer verbs have a nominative theme and a dative experiencer. These different classes of experiencers are illustrated next. 



\subsubsection{Class I: Nominative experiencer, accusative theme}

In this class, verbs like \textit{bug} ‘love’ and  \textit{and} ‘know’ are found, they have the structure of a regular transitive verb.

\ea \label{ex:tamba:4}
\ea \label{ex:tamba:4a}
\gll Faatu  a      bug    {}-a         Maamkoor\\      
Faatu    \textsc{3sg}  love    {}-\textsc{perf}   Maamkoor\\
\glt `Faatu loves Mamkoor'

\ex \label{ex:tamba:4b}
\gll o  tew  oxe    a     and      a      Maamkoor \\
\textsc{cl}  lady  \textsc{cl\textsubscript{def}}   \textsc{3sg}  know  \textsc{perf}   Maamkoor\\
\glt `The lady knows Mamkoor'
\z
\z
    

In \REF{ex:tamba:4a} and \REF{ex:tamba:4b} the entity undergoing a psychological experience is in subject position. The next two classes describe object experiencers.

\subsubsection{Class II: Nominative causer, accusative experiencer} 

These experiencer verbs come in two types. The first type must appear with a causative suffix whereas the second type does not occur with a causative suffix. Note that this is different from B\,\&\,R's Class II. Even though they do not use “causer”, my assumption is that a causer role is added to the subject of experiencer verbs of Class II. In \REF{ex:tamba:5a} and \REF{ex:tamba:5b} the verbs \textit{diidlat} ‘frighten’ and \textit{jaaxɗat} ‘worry’ are used transitively and are morphological complex. 

\ea \label{ex:tamba:5}
\ea \label{ex:tamba:5a}
\gll Awa  a    diid       {}-lat    -a      o     fes              ole\\
    Awa  \textsc{3sg} be.scared -\textsc{caus}   \textsc{perf}  \textsc{cl}    young.man  \textsc{cl\textsubscript{def}}\\
\glt `Awa frightened the young man'

\ex \label{ex:tamba:5b}
\gll Faatu   a       jaax       {}-ɗat    {}-a       o       tew  oxe\\
    Faatu    \textsc{3sg}   be.worried   {}-\textsc{caus}   \textsc{perf}    \textsc{cl}  lady  \textsc{cl\textsubscript{def}}\\
\glt `Faatu made the woman worried'
\z
\z

In \REF{ex:tamba:5a} and \REF{ex:tamba:5b}, the experiencers are in  object position. The other type of object experiencer in this class is shown in \REF{ex:tamba:6}.\ia{Belletti, Adriana@Belletti, Adriana|)}\ia{Rizzi, Luigi@Rizzi, Luigi|)}


\ea \label{ex:tamba:6}
\ea \label{ex:tamba:6a}
\gll Faatu   a        bet         a        o  njaaƈ   onqe  \\
    Faatu   \textsc{3sg}    surprise     \textsc{perf}  \textsc{cl} child   \textsc{cl\textsubscript{def}} \\
\glt `Faatu suprised the child'

\ex \label{ex:tamba:6b}
\gll Faatu  a  weg      a  o  tew    oxe\\
    Faatu  \textsc{3sg}  be.unlucky    \textsc{perf}  \textsc{cl}  lady      \textsc{cl\textsubscript{def}}\\
\glt `Faatu brought bad luck to the woman'
\z
\z 

In \REF{ex:tamba:6a} and \REF{ex:tamba:6b} there is no overt causative suffix on the verb. My assumption is  that there is a silent causative suffix. As will be show in \sectref{sec:tamba:3} these verbs behave the same in some syntactic environments. Object experiencers, according to \citet{Pesetsky1995}, add an additional causer argument. In addition, the argument related to the “cause” must be realized as the subject \citep{Grimshaw1990}.

Verbs of this class (Class II) behave like regular transitive verbs projecting a light verb headed by an overt or a silent causative.

\subsubsection{Class III: Nominative theme, dative experiencer}

Arguments of verbs of this class, shown in \REF{ex:tamba:7}, appear in the same order as the ones of verbs of Class II described \REF{ex:tamba:6}.

\ea \label{ex:tamba:7}
\ea \label{ex:tamba:7a}
\gll o  fes              ole         a          sooɓ  a        o  tew        oxe\\
    \textsc{cl} young.man  \textsc{cl\textsubscript{} \textsc{def}}    \textsc{3sg}       miss  \textsc{perf}   \textsc{cl}  woman \textsc{cl\textsubscript{def}}\\
\glt `the woman  misses the young man'
\ex \label{ex:tamba:7b}
\gll Awa a     fel         a       o     njaaƈ onqe\\
Awa \textsc{3sg} appeal \textsc{perf}  \textsc{cl}   boy   \textsc{cl\textsubscript{det} }\\
\glt `Awa appeals to the boy'
\z
\z

In \REF{ex:tamba:7a} and \REF{ex:tamba:7b} the verbs do not have a causative component associated with them. These verbs have been argued to be stative/unaccusative. My assumption is that the object experincers of these verbs are introduced by a silent preposition making them oblique. As will be shown, these objects behave like typical datives. The different classes of experiencers in Sereer are summarized in \tabref{tab:tamba:2}.
 
\begin{table}
\caption{Sereer experiencers}
\label{tab:tamba:2}
\begin{tabular}{lll}
\lsptoprule
{Class I} & Nominative experiencer & Accusative theme\\
{Class II} & Nominative causer & Accusative experiencer\\
{Class III} & Nominative theme & Dative experiencer\\
\lspbottomrule
\end{tabular}
\end{table}

The remainder of this paper focuses on Class II and Class III experiencers by discussing syntactic differences between the two types of object experiencers that account for the classification in \tabref{tab:tamba:2}. More specifically, I show that they behave differently with respect to passivization, nominalization and antipassivization.


\section{Distinguishing between the two object  experiencers in Sereer}
\label{sec:tamba:3}
\subsection{Passivization test}
\label{sec:tamba:3.1}

In Sereer, passive is shown through the promoting of the verb internal argument to subject position whereas the external argument is demoted through suppression. Passive construction is marked through the use of the suffix -\textit{el}\footnote{This suffix is referred to in \citet{FayeMous2006} as anticausative.} on the infinitive verb \citep{Faye1979,Renaudier2012,Faye2013}, however, this suffix has various allomorphs conditioned by aspect, tense and/or negation.

\ea \label{ex:tamba:8}
\ea \label{ex:tamba:8a}
\gll Awa   a       ñaam   a         maalo fe\\
Awa   \textsc{3sg}   eat      \textsc{perf}    rice     \textsc{cl\textsubscript{def}}\\
\glt `Awa ate the rice'
\ex \label{ex:tamba:8b}
\gll maalo fe     a      ñaam-eɁ       (*Awa)\\
rice    \textsc{cl\textsubscript{def}} \textsc{3sg}  eat-\textsc{pass}   Awa\\
\glt `the rice was eaten'
\z
\z

As seen in \REF{ex:tamba:8}, with a regular transitive verb, the demoted external argument \textit{Awa}  cannot appear in passive constructions. Passivizing experiencer verbs yields different results according to the nature of the object. 

Sereer Class II object experiencers can successfully undergo passivization. This is illustrated in \REF{ex:tamba:9} and \REF{ex:tamba:10} with the verbs \textit{diidlat} ‘frighten’ and \textit{bet} ‘surprise’. 

\ea \label{ex:tamba:9}
\ea \label{ex:tamba:9a}
\gll Awa  a      diid         {}-lat    a       o    fes            ole\\
Awa  \textsc{3sg} be.scared -\textsc{caus}  \textsc{perf}    \textsc{cl} young.man  \textsc{cl\textsubscript{} \textsc{def}}\\
\glt `Awa frightened the young man'
\ex \label{ex:tamba:9b}
\gll o  fes                ole     a      diid         {}-lat    {}-eɁ\\
\textsc{cl} young.man  \textsc{cl\textsubscript{} \textsc{def}}  \textsc{3sg}   be.scared -\textsc{caus}    \textsc{{}-pass}\\
\glt `The young man was frightened'
\z
\z

In \REF{ex:tamba:9b} derived from \REF{ex:tamba:9a} the object experiencer is promoted to subject position and the verb, which is morphologically complex, must appear with a passive morpheme. A similar situation can be observed in (\ref{ex:tamba:10}a--b) with the verb \textit{bet} ‘surprise’ which is morphological simple. Note however that I mentioned earlier that this verb is semantically similar to \textit{diidlat} ‘frighten’ as they both have the “cause” component. 

\ea \label{ex:tamba:10}
\ea \label{ex:tamba:10a}
\gll Faatu      a      bet      {}-a        o      njaaƈ onqe\\
Faatu      \textsc{3sg}  surprise \textsc{perf}   \textsc{cl}  child   \textsc{cl\textsubscript{def}}\\
\glt `Faatu surprised the child'
\ex \label{ex:tamba:10b}
\gll  o    njaaƈ  onqe  a      bet       {}-eɁ\\
\textsc{cl}  child   \textsc{cl}     \textsc{3sg}  surprise \textsc{{}-pass}\\
\glt `the child was surprised'
\z
\z

In \REF{ex:tamba:10} the verb behaves as expected since it allows passivization. The object experiencer can move to subject position along with a demotion of the original subject through suppression. 

Next I show that Class III object experiencers cannot undergo passivization. This is illustrated in \REF{ex:tamba:11} with the verb \textit{sooɓ} ‘miss’.

\ea \label{ex:tamba:11}
\ea[] 
{\gll o     fes               ole        a          sooɓ  a      o      tew      oxe \\
    \textsc{cl}    young.man  \textsc{cl\textsubscript{def}}      \textsc{3sg}              miss  \textsc{perf} \textsc{cl}    woman \textsc{cl\textsubscript{def}}  \\
\glt `the woman misses the young man'} \label{ex:tamba:11a}
\ex[*]
{\gll o      tew      oxe  a          sooɓ -eɁ\\
\textsc{cl}  woman \textsc{cl\textsubscript{def}}  \textsc{3sg}      miss -\textsc{pass}\\
\glt `The woman was missed' (intended)
} \label{ex:tamba:11b}
\z
\z

\REF{ex:tamba:11b} shows that passive morphology is incompatible with verbs of this class, that is the object experiencers cannot be promoted to subject position. This is evidence that they are different from the ones in Class II.  \REF{ex:tamba:12} follows a similar pattern  with the verb \textit{fel} `appeal to'.

\ea \label{ex:tamba:12}
\ea[]
{\gll Awa a     fel  {}-a       o      njaaƈ onqe\\
Awa \textsc{3sg} appeal   \textsc{perf}  \textsc{cl} child   \textsc{cl\textsubscript{def}}\\
\glt `Awa appeals to the boy'} \label{ex:tamba:12a}
\ex[*] 
{\gll o    njaaƈ onqe     {}-a      fel     {}-eɁ\\
\textsc{cl\textsubscript{} } chile   \textsc{cl\textsubscript{def}} \textsc{3sg}  appeal.to  {}-\textsc{pass}\\
\glt `The child was appealed to' (intended)
} \label{ex:tamba:12b}
\z
\z

After this passivization test, I use another test which consists of nominalizing the clause containing an object experiencer. 

\subsection{Nominalization test}

\citet{Grimshaw1990} argues that nominalization and passivization are related in that in both cases the external argument is optional and as such, can be suppressed. Indeed in English, for instance, the external argument in such constructions is optional as \REF{ex:tamba:13} shows.

\ea \label{ex:tamba:13}
\ea
The door was opened (by John)
\ex 
The opening of the door (by John)
\z
\z

These sentences show that in English the external argument can be suppressed in nominalization and passivization. In Sereer, a similar situation can be observed, however the suppression of the external argument in these constructions is mandatory as mentioned earlier with passsives. If these two constructions (i.e. passivization and nominalization) are related, one should expect to see results similar to the ones observed with the passivization test. 

In Sereer, Class II experiencer verbs can successfully undergo nominalization as \REF{ex:tamba:14} shows. 

\ea \label{ex:tamba:14}
\ea \label{ex:tamba:14a}
\gll Awa   a    diid          {}-lat    -a      o   fes            ole\\
Awa  \textsc{3sg} be.scared -\textsc{caus} \textsc{perf}  \textsc{cl} young.man \textsc{cl\textsubscript{def}}\\
\glt `Awa frightened the young man'
\ex Nominalization\\ \label{ex:tamba:14b}
\gll o   diid   -lat   ole      no o    fes       ole\\
\textsc{cl} be.scared   -\textsc{caus}  \textsc{cl\textsubscript{def}}  \textsc{p}  \textsc{cl} young.man    \textsc{cl\textsubscript{def}}\\
\glt `The frightening of the young man'
\z
\z

In \REF{ex:tamba:14b} the  nominal derived from \REF{ex:tamba:14a} appears with noun class markers. In addition, the internal argument of the verb, the object experiencer, is introduced by the preposition \textit{no}.  A similar pattern can be observed in \REF{ex:tamba:15} with the verb \textit{bet} `surprise'.

\ea \label{ex:tamba:15}
\ea \label{ex:tamba:15a}
\gll Faatu a      bet    {}-a        o        njaaƈ onqe\\
Faatu \textsc{3sg}  surprise \textsc{perf}   \textsc{cl}     child   \textsc{cl\textsubscript{def}} \\
\glt `Faatu surprised the child'
\ex Nominalization\\ \label{ex:tamba:15b}
\gll o     bet        ole      no o    njaaƈ   onqe\\
\textsc{cl}   surprise  \textsc{cl\textsubscript{} \textsc{def}}   \textsc{p}  \textsc{cl}  child     \textsc{cl\textsubscript{} \textsc{def}}\\
\glt `the surprising of the child'
\z
\z

Just like in the previous example, in \REF{ex:tamba:15} the object experiencer appears in a prepositional phrase whereas the nominalized verb occur with nominalizers (i.e. noun class markers). 

 Class III experiencer verbs fail to undergo nominalization. This  is illustrated in \REF{ex:tamba:16}.


\ea \label{ex:tamba:16}
\ea[]
{\gll Awa a     sooɓ  {}-a       o      tew           oxe  \\
Awa \textsc{3sg}  miss   \textsc{perf}  \textsc{cl\textsubscript{}      } woman   \textsc{cl\textsubscript{} \textsc{def}}\\
\glt `Awa misses the woman'}\label{ex:tamba:16a}
\ex[*] 
{\gll o    sooɓ   ole     no   no  tew       oxe\\    
\textsc{cl\textsubscript{}}  miss     \textsc{cl\textsubscript{def}}   \textsc{p}      \textsc{cl} woman \textsc{cl\textsubscript{def}}\\
\glt `The woman being missed' (intended)}\label{ex:tamba:16b}
\z
\z
 
In \REF{ex:tamba:16b}, derived from \REF{ex:tamba:16a}, nominalizing the verb results in ungrammaticality. This is expected since the verb does not assign accusative case to the object experiencer. A similar situation is can be noted in \REF{ex:tamba:17}.

\ea \label{ex:tamba:17}
\ea[]
{\gll Awa a      fel    {}-a       o      tew     oxe\\  
Awa \textsc{3sg}  appeal.to   \textsc{perf}  \textsc{cl\textsubscript{}} woman     \textsc{cl\textsubscript{def}}\\
\glt `Awa appeals to the woman'}\label{ex:tamba:17a}
\ex[*] 
{\gll o     fel       ole     no   no  tew       oxe\\
\textsc{cl}  appeal.to  \textsc{cl\textsubscript{def}}    \textsc{p}      \textsc{cl} woman  \textsc{cl\textsubscript{def}}\\
\glt `The woman being appealed to' (intended)
}\label{ex:tamba:17b}
\z
\z

In \REF{ex:tamba:17} the verb \textit{fel} ‘appeal to’ cannot be nominalized as the ungrammaticality of  \REF{ex:tamba:17b} shows. Surprisingly, if the passive morphology -\textit{el} appears with a verb of this type (Class III) nominalization is possible as shown in \REF{ex:tamba:18}.

\ea \label{ex:tamba:18}
\ea[]
{\gll Awa a     fel   a       o      njaaƈ onqe\\
Awa \textsc{3sg} appeal   \textsc{perf} \textsc{cl}\textsc{\textsubscript{}        }boy    \textsc{cl\textsubscript{} \textsc{def}}\\
\glt `Awa appeals to the boy'}
\ex[] 
{\gll o      pel      {}-el      ole    no  o    njaaƈ onqe\\
\textsc{cl}    appeal \textsc{-pass}   \textsc{cl\textsubscript{def}}   \textsc{p}   \textsc{cl}  boy   \textsc{cl\textsubscript{def}}\\
\glt `The appealing to the young man' (intended)}
\ex[*] 
{\gll o      pel       ole    no  o    njaaƈ onqe\\
\textsc{cl}    appeal   \textsc{cl\textsubscript{def}} \textsc{p}    \textsc{cl} boy     \textsc{cl\textsubscript{def}}\\
\glt `The  young man being appealed to'}
\z
\z

\citet{Faye2013} argues that another use of the passive marker -\textit{el}  is to derive nominals from stative verbs. The behavior of Class III experiencer verbs with respect to nominalization  shows that these verbs are different from the ones of Class II. 

These types of experiencer verbs are superficially transitive but underlyingly unaccusative \citep{Belletti1988, Pesetsky1995, Landau2010}.  

In the next subsection, I use antipassivation to further distinguish between the two types of object experiencers.

\subsection{Antipassivization test}

The term “antipassive” is generally used to refer to a characteristic of voice in ergative languages \citep{Crystal2008}. In antipassive constructions, the verb is semantically transitive but does not project a direct object \citep{Polinsky2017}. 

Polinsky further argues that crosslinguistically, antipassive can be diagnosed through case marking, noun incorporation, agreement, word order, verbal affixation. She also provides evidence that antipassive can be found in accusative languages (see also \citealt{Heaton2017}). 

  In Sereer \citep{Renaudier2011}, and related languages like Wolof \citep{Creissels2008} antipassive is marked through verbal suffixation. 

\ea \label{ex:tamba:19}
\ea
\gll o     ɓox   ole   a     ŋat         {}-a       o   njaaƈ  onqe\\
 \textsc{cl} dog      \textsc{cl\textsubscript{def}}   \textsc{3sg}  bite         \textsc{perf}  \textsc{cl} boy     \textsc{cl\textsubscript{def}}\\
 \glt `The dog bit the boy'
\ex 
\gll o     ɓox      ole                             kaa                         ŋat     {}-a                    {}-a       (*o      njaaƈ onqe)\\
\textsc{cl} dog     \textsc{cl\textsubscript{def}} \textsc{ipfv.}\textsc{3sg}    bite   {}-\textsc{antip}       {}-\textsc{perf} {} {}\\
\glt `The dog bit'
\z
\z
              
In Sereer, the suffix -\textit{a} is used on the verb to mark the antipassive construction. The antipassive morpheme is very productive (not only related to verbs of transfer and ditransitives as argued in Renaudier).

With respect to experiencer verbs, different results are noticed according to the type of object being dealt with. Verbs belonging to Class II can undergo antipassivization, that is they can appear with the passive marker along with a suppression of the object experiencer. This is shown in \REF{ex:tamba:20} and in \REF{ex:tamba:21}.

\ea \label{ex:tamba:20}
\ea \label{ex:tamba:20a}
\gll Awa  a      diid              {}-lat    a         o    fes            ole \\
Awa  \textsc{3sg} be.scared      {}-\textsc{caus}   \textsc{perf}      \textsc{cl}   young.man  \textsc{cl\textsubscript{def}}\\
\glt `Awa frightened the young man'
\ex \label{ex:tamba:20b}
\gll Awa    kaa            diid           {}-lat       {}-a   -a\\
Awa    \textsc{ipfv.}\textsc{3sg}    be scared   {}-\textsc{caus}    \textsc{{}-antip - perf}\\
\glt `Awa frightened'
\z
\z
           
\ea \label{ex:tamba:21}
\ea \label{ex:tamba:21a}
\gll Faatu   a      bet  {}-a        o      njaaƈ  onqe\\
Faatu      \textsc{3sg}  surprise  \textsc{perf}    \textsc{cl}      child   \textsc{cl\textsubscript{def}}\\
\glt `Faatu surprised the child'
\ex \label{ex:tamba:21b}
\gll Faatu kaa bet {}-a      {}-a \\
Faatu          \textsc{ipfv.}\textsc{3sg}        surprise     \textsc{-antip}    \textsc{perf} \\
\glt `Faatu surprised'
\z
\z

These examples show that experiencer verbs of Class II behave like regular transitive verbs in that they can undergo antipassivization. In both \REF{ex:tamba:20b} and in \REF{ex:tamba:21b} the object experiencer argument is suppressed.

Contrary to Class II verbs, verbs of Class III cannot occur with the antipassive marker -\textit{a}.

\ea \label{ex:tamba:22}
\ea[]
{\gll Awa a      fel   -a       o      tew       oxe  \\
Awa \textsc{3sg} appeal    \textsc{perf} \textsc{cl} woman  \textsc{cl\textsubscript{def}}\\
\glt `Awa appeals to the woman'}
\ex[*]
{\gll Awa      kaa           fel     {}-a        {}-a \\
 Awa \textsc{ipfv.}\textsc{3sg}       appeal     \textsc{-antip} \textsc{-perf}\\
\glt `Awa appealed to' (intended)} \label{ex:tamba:22b}
\z
\z
      
In \REF{ex:tamba:22b} the verb  \textit{fel} `appeal to'  appear with the antipassive marker and this yields ungrammaticality. The same situation can be observed in \REF{ex:tamba:23b}.

\ea \label{ex:tamba:23}
\ea[]
{\gll \textit{o}    \textit{njaaƈ}   \textit{onqe}     \textit{a}    \textit{sooɓ}   \textit{a}       \textit{o}      \textit{tew}       \textit{oxe}\\
\textsc{cl}   boy    \textsc{cl\textsubscript{def}}   \textsc{3sg} miss   \textsc{perf} \textsc{cl} woman  \textsc{cl\textsubscript{def}}\\
\glt `The woman misses the boy'} \label{ex:tamba:23a}
\ex[*]
{\gll o    njaaƈ   onqe      kaa                sooɓ      {}-a\\
o     njaaƈ  \textsc{cl\textsubscript{def}}     \textsc{ipfv.}\textsc{3sg}  miss {}-antip\\
\glt }\label{ex:tamba:23b}
\z
\z
              
In this section I have used various tests (i.e. passivization, nominalization and antipassivization) to substantiate the claim that Sereer object experiencers come into two classes, Class II and Class III. 

\section{Conclusion}

The main aim of this study was to describe object experiencers in Sereer in light of \citet{Belletti1988}.  I have shown that they come in two types, Class II and Class III. Contrary to \citet{Landau2010}, I have shown that Class II object experiencers are not oblique and behave like regular transitive verbs. In contrast, Class III object experiencers are oblique and as such do not display typical object properties. It is my assumption that these objects are introduced by a silent preposition. This is in line with B\,\&\,R's analysis of experiencers of this type as being assigned an inherent dative case. This study is not only a contribution to the literature of experiencers but is also a contribution to the study of argument structure in Sereer. \tabref{tab:tamba:3}, repeated from above, summarizes the different properties of the experiencer verbs found in Sereer.

\begin{table}
\caption{\label{tab:tamba:3}Sereer experiencers}
\begin{tabular}{lll}
\lsptoprule
{Class I} & Nominative experiencer & Accusative theme\\
{Class II} & Nominative causer & Accusative experiencer\\
{Class III} & Nominative theme & Dative experiencer\\
\lspbottomrule
\end{tabular}
\end{table}

 
%\begin{verbatim}%%move bib entries to  localbibliography.bib
%@misc{BaierTA,
%	author = {Baier, Nico},
%	title = {Object Suffixes as Incorporated Pronouns in {Seereer}. {{T}}o appear in the},
%	year = {2015}
%}

%@misc{Proceedings2015,
%	author = {Proceedings of ACAL 45, \citealt{June},
%	title = {}.},
%	year = {2015}
%}


%@misc{Belletti1988,
%	author = {Belletti, Adriana and Rizzi Luigi},
%	title = {Psych Verbs and Theta Theory. \textit{In} \textit{Natural} \textit{Language}},
%	year = {1988}
%}


%\textit{and} \textit{Linguistic} \textit{Theory}. 6:291-352.

%Creissels, Denis and Nouguier-Voisin, Sylvie. 2008. The verbal suffixes of Wolof coding valency changes and the notion of coparticipation.In Konig, E. and Gast, V. (eds) \textit{Reciprocals} \textit{and} \textit{reflexives:} \textit{Crosslinguistic} \textit{and} \textit{Theoretical} \textit{Exporations} \textit{289-306.} Berlin: Mouton de Gruyter.

%@book{Crystal2008,
%	address = {Maldan},
%	author = {Crystal, David.},
%	publisher = {Blackwell},
%	sortname = {Crystal, David.},
%	title = {\textit{A} \textit{dictionary} \textit{of} \textit{Linguistics} \textit{and} \textit{Phonetics.} 6\textsuperscript{th} Ed},
%	year = {2008}
%}


%Publishing.

%@book{Fal1980,
%	address = {\textit{Grammaire} \textit{dialectaledu} \textit{Seereer.} Dakar},
%	author = {Fal, Arame.},
%	publisher = {La Maison du Livre Universel},
%	sortname = {Fal, Arame.},
%	title = {Les nominaux en sereer-siin: {{P}}arler de Jaxaw. {{D}}akar: {{L}}es Nouvelles Editions Africaines. {{F}}aye, Souleymane. 2013},
%	year = {1980}
%}


%@book{Faye1979,
%	address = {Dakar },
%	author = {Faye, Waly Coly.},
%	publisher = {Université Cheikh Anta Diop},
%	sortname = {Faye, Waly Coly.},
%	title = {Etude morphosyntaxique du sereer singandum (région de Jaxaaw-     \{MakeU}ppercase{ñ}aaxar), Thèse de 3\textsuperscript{è}\textsuperscript{me} cycle},
%	year = {1979}
%}


%@book{Grimhaw1990,
%	address = {Cambridge},
%	author = {Grimhaw, Jane.},
%	publisher = {MIT press},
%	sortname = {Grimhaw, Jane.},
%	title = {\textit{Argument} \textit{Structure}},
%	year = {1990}
%}


%@misc{Heaton2017,
%	author = {Heaton, Raina},
%	title = {\textit{A} \textit{Typology} \textit{of} \textit{antipassives,} \textit{with} \textit{special} \textit{reference} \textit{to} \textit{{Mayan}.}},
%	year = {2017}
%}


%Honololu:University of Hawai at Manoa (Doctoral dissertation)

%@book{Landau2010,
%	address = {Cambridge},
%	author = {Landau, Idan.},
%	publisher = {MIT press},
%	sortname = {Landau, Idan.},
%	title = {\textit{The} \textit{Locative} \textit{syntax} \textit{of} \textit{experiencers}},
%	year = {2010}
%}


%@misc{McLaughlin1992,
%	author = {McLaughlin, Fiona},
%	title = {Consonant mutation in {Seereer}-Siin. {{I}}n \textit{Studies} \textit{in} \textit{African} \textit{Linguistics} 23 \REF{ex:tamba:3}.},
%	year = {1992}
%}


%@book{Pesetsky1995,
%	address = {Cambridge},
%	author = {Pesetsky, David.},
%	publisher = {MIT press},
%	sortname = {Pesetsky, David.},
%	title = {Zero Syntax Experiencers and Cascades },
%	year = {1995}
%}


%Polinsky Maria.~2017. Antipassive. In  \textit{The} \textit{Oxford} \textit{Handbook} \textit{of} \textit{ergativity.}In Coon, Jessica \& Massam, Diane \& Travis, Lisa Demena (eds).Oxford: OUP.

%@book{Renaudier2011,
%	address = {The antipassive in accusative languages},
%	author = {Renaudier, Marie.},
%	publisher = {The case of Sereer  (Atlantic,Senegal  Université Lyon 2 – Dynamique du Language (Unpublished Handout)},
%	sortname = {Renaudier, Marie.},
%	title = {\biberror{no title}},
%	year = {2011}
%}


%@book{Renaudier2012,
%	address = {\textit{Dérivation} \textit{et} \textit{valence} \textit{en} \textit{sereer} (Parler de Mar Lodj) Lyon},
%	author = {Renaudier, Marie.},
%	publisher = {Université   Lyon 2 (Doctoral dissertation)},
%	sortname = {Renaudier, Marie.},
%	title = {\biberror{no title}},
%	year = {2012}
%}

%@book{Simons2017,
%	address = {Dallas},
%	booktitle = {~\textit{{Ethnologue}:} \textit{Languages} \textit{of} \textit{the} \textit{World.}20\textsuperscript{th} Ed},
%	editor = {Simons, Gary F. and Charles D. Fennig},
%	publisher = {SIL International},
%	sortname = {Simons, Gary F. and Charles D. Fennig},
%	title = {~\textit{{Ethnologue}:} \textit{Languages} \textit{of} \textit{the} \textit{World.}20\textsuperscript{th} Ed},
%	year = {2017}
%}


%\end{verbatim}
\section*{Abbreviations}
\begin{multicols}{2}
\begin{tabbing}
\textsc{antip}\hspace{.5cm} \= third person object pronoun\kill
\textsc{antip} \> antipassive\\
\textsc{caus} \> causative \\
\textsc{cl} \> noun class marker\\
\textsc{cl\textsubscript{def}} \> definite\\
\textsc{foc} \> focus\\
\textsc{inst} \> instrument\\
\textsc{ipfv} \> imperfective\\
\textsc{it} \> iterative\\
\textsc{neg} \> negation \\
\textsc{pass} \> passive marker \\
\textsc{p} \> preposition\\
\textsc{perf} \> perfective\\
\textsc{ref} \> reflexive \\
\textsc{3obj} \> third person object pronoun\\
\textsc{3sg} \>  third person singular
\end{tabbing}
\end{multicols}
 
\section*{Acknowledgements}

Special thanks to my consultant Dr Mamecor Faye for his patience and availability. I would also like to thank the following persons for their support: Pr Enoch Aboh (University of Amsterdam), Dr Mamadou Bassene (UCAD), Dr Kelly Berkson (Indiana University) and Pr Amadou Abdoul Sow (Dean of UCAD School and Arts and Humanities) for the financial support. Finally, I would like to thank two anonymous reviewers for their comments that helped improve the quality of this paper.

{\sloppy
\printbibliography[heading=subbibliography,notkeyword=this] 
}
\end{document}

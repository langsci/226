\documentclass[output=paper]{langsci/langsci}
\title{Optional past tense in Wolof} 
\author{M.~Ryan Bochnak\affiliation{University of Konstanz} \lastand  Martina Martinović\affiliation{University of Florida}}
\abstract{In this paper, we discuss the interpretation of past temporal marker \textit{oon} in Wolof (Niger-Congo; Senegal), in light of recent claims in the literature regarding its status as a so-called ``discontinuous past." We show that the cessation inference associated with \textit{oon} is a conversational implicature. Thus, \textit{oon} can receive an analysis as a plain semantic past tense.}

\IfFileExists{../localcommands.tex}{%hack to check whether this is being compiled as part of a collection or standalone
  % add all extra packages you need to load to this file  
\usepackage{tabularx} 

%%%%%%%%%%%%%%%%%%%%%%%%%%%%%%%%%%%%%%%%%%%%%%%%%%%%
%%%                                              %%%
%%%           Examples                           %%%
%%%                                              %%%
%%%%%%%%%%%%%%%%%%%%%%%%%%%%%%%%%%%%%%%%%%%%%%%%%%%% 
%% to add additional information to the right of examples, uncomment the following line
% \usepackage{jambox}
%% if you want the source line of examples to be in italics, uncomment the following line
% \renewcommand{\exfont}{\itshape}
% \usepackage{lipsum}
% \usepackage[normalem]{ulem}
\usepackage{tikz}
\usepackage{tikz-qtree}
\usepackage{tikz-qtree-compat}
\usepackage{tabularx}
\usepackage{verbatim}
\usepackage{pifont}    %for pointing hand, checkmarks, crosses
\usepackage{tipa}
\usepackage{amssymb}
\usepackage{amsmath}
\usepackage{subcaption}
\usepackage{csquotes}
\usepackage{multirow}
\usepackage{multicol}
\usepackage{outlines} 
%\usepackage{wrapfig}
\usepackage{enumitem}
\usepackage{rotating}  
%\usepackage{ling-macros-custom}
\usepackage{stmaryrd}
%\usepackage{qtree}
\usepackage{./langsci/styles/langsci-optional}
\usepackage{./langsci/styles/langsci-lgr}
\usepackage{hhline}
\usepackage{langsci-gb4e}
\usepackage[linguistics]{forest}
 



   
\newcommand{\mc}[1]{\textsc{#1}}	% morpheme glossing as small caps: does not clash with fontspec
\def\Id#1{\vskip-.cm\hskip.01cm\vtop{#1}} % for controlling where the example breaks onto the next line


\setlength{\emergencystretch}{2pt}
\newcolumntype{s}{>{\hsize=.5\hsize}X}
\newcolumntype{m}{>{\hsize=.25\hsize}X}


 

%makes subscripts
\newcommand{\sub}[1]{\textsubscript{#1}}
\newcommand{\tikzmark}[1]{\tikz[overlay,remember picture]\node(#1){};}

%draws normal arrow
\newcommand*{\DrawArrow}[3][]{%
	\begin{tikzpicture}[overlay,remember picture, >=latex']
	\draw[rounded corners, -latex,<-,semithick,#1]  (#2) -- ++(0,-1.2em)coordinate (a) -- 
	($(a-|#3)$) -- (#3);
	\end{tikzpicture}%
}

%draws dashed arrow
\newcommand*{\DrawXArrow}[3][]{%
    % #1 = draw options
    % #2 = left point
    % #3 = right point
    \begin{tikzpicture}[overlay,remember picture,>=latex']
       % \draw [-latex, #1] ($(#2)+(0.1em,0.5ex)$) to ($(#3)+(0,0.5ex)$);
		\draw[rounded corners, -latex,<-,semithick,#1]  (#2) -- ++(0,-1.2em)coordinate (a) -- 
		node%[font=\footnotesize]
		{\ding{56}}
		($(a-|#3)$) -- (#3);
    \end{tikzpicture}%
}% 

\newcommand*{\DrawArrowok}[4][]{%
	% #1 = draw options
	% #2 = left point
	% #3 = right point
	\begin{tikzpicture}[overlay,remember picture, >=latex']
	% \draw [-latex, #1] ($(#2)+(0.1em,0.5ex)$) to ($(#3)+(0,0.5ex)$);
	\draw[rounded corners, -latex,<-,semithick,#1]  (#2) -- ++(0,-1.2em)coordinate (a) -- 
	%node[below,font=\footnotesize]{#4} 
	node[midway,fill=white,font=\normalsize]{#4}
	($(a-|#3)$) -- (#3);
	\end{tikzpicture}%
}% 

\newcommand*{\DrawLXArrow}[3][]{%
	% #1 = draw options
	% #2 = left point
	% #3 = right point
	\begin{tikzpicture}[overlay,remember picture,>=latex']
	% \draw [-latex, #1] ($(#2)+(0.1em,0.5ex)$) to ($(#3)+(0,0.5ex)$);
	\draw[rounded corners, -latex,<-,semithick,#1]  (#2) -- ++(0,-1.8em)coordinate (a) -- 
	node%[font=\footnotesize]
	{\ding{56}}
	($(a-|#3)$) -- (#3);
	\end{tikzpicture}%
}% 
\usetikzlibrary{arrows,shapes,positioning,shadows,trees,calc}
\usetikzlibrary{decorations.text}

 
\newcommand{\all}[1]{\ensuremath{\forall #1}}

\newcommand{\bari}{\ipabar{\i}{.5ex}{1.1}{}{}} 

\let\oldemptyset\emptyset
\let\emptyset\varnothing
 


\usetikzlibrary{calc,positioning,tikzmark}

%hyman
\newcommand{\higr}[1]{{\color{red}#1}}
 
 
\definecolor{lsDOIGray}{cmyk}{0,0,0,0.45}

%newkirk
\newcommand{\ix}[1]{{\color{red}\textsubscript{#1}}} %probably i.nde.x
\newcommand{\ol}[1]{{\color{red}\textit{#1}}} %probably o.bject l.anguage
\newcommand{\alert}[1]{{\color{red}\textbf{#1}}}
\newenvironment{context}{\color{red}
\smallskip 

\scriptsize 
}{\color{black}\normalsize }
\newcommand{\m}[1]{{\color{red}\textsc{#1}}}
\newcommand{\den}[1]{{\color{red}#1}}
\newcommand{\type}[1]{{\color{red}#1}}
\newcommand{\denol}[1]{{\color{red}#1}}
\newcommand{\bex}{\ea}
\newcommand{\fex}{\z}
\newcommand{\set}{{\color{red} SET}}
 
\def\denotes#1{$\lbrack\!\lbrack${#1\/}$\rbrack\!\rbrack$}      % denotes
\newcommand{\citeNP}{\citealt}


\makeatletter
\let\thetitle\@title
\let\theauthor\@author 
\makeatother


\newcommand{\togglepaper}[1][0]{ 
  \bibliography{../localbibliography}
  \papernote{\scriptsize\normalfont
    \theauthor.
    \thetitle. 
    To appear in: 
    Samson Lotven,   Silvina Bongiovanni,   Phillip Weirich,   Robert Botne \&  Samuel Gyasi Obeng (eds.),  
    African linguistics across the disciplines: Selected papers from the 48th Annual Conference on African Linguistics 
    Berlin: Language Science Press. [preliminary page numbering]
  }
  \pagenumbering{roman}
  \setcounter{chapter}{#1}
  \addtocounter{chapter}{-1}
}

 
  \togglepaper[13]
}{}


\begin{document}

\maketitle
\section{Introduction}

There has been some debate in the recent literature regarding the semantic nature of so-called ``discontinuous past'' markers. On the one hand, \citet{Plungian2006}, to whom the term ``discontinuous past'' is due, characterize its meaning as ``past and not present'' or ``past with no present relevance.'' 

On the other hand, \cite{Cable2017a} 
argues that the apparently discontinuous semantics of the Tlingit (Na Dene, Alaska and British Columbia) decessive form \citep{Leer1991} is actually a defeasible implicature, i.e., not part of the conventional semantics of the tense form. Cable assigns a plain past tense semantics, and the implicature of ``not present'' or ``no present relevance" is derived via competition with temporally unmarked clauses, which can receive either a past or present interpretation. Cable further observes that discontinuous pasts are found exclusively in languages where overt past marking is optional,\footnote{By ``optional'' past, we are referring to the fact that past temporal reference can also grammatically be achieved with temporally unmarked clauses. Speakers may nevertheless make use of optional past markers for specific rhetorical purposes, as highlighted by \citet{Plungian2006}.} and thus calls into question whether the category of discontinuous past exists at all in natural language.


In this paper, we contribute to this discussion by examining the past temporal marker
\textit{\textbf{oon}}\footnote{The past marker surfaces as
  \textit{oon} if the preceding element ends in a consonant, and as
  \textit{woon} if it ends in a vowel, as a result of a phonological hiatus repair.} in Wolof (Niger-Congo; Senegal; see \citealt{church81systeme, robert91approche}). This tense marker was identified by \citet{Plungian2006} as a discontinuous past, and the Wolof data formed an important part of their argument for the existence of discontinuous pasts in the world's languages. The main evidence for this claim comes from a cessation inference associated with the use of \textit{oon} in contrast with temporally unmarked past-referring sentences.\footnote{Like Cable, we follow \citet{altshuler12moment} in using the terminology of ``cessation'' to describe this inference.} For instance, comparing (\ref{ex:bochnak:Robert1}) and (\ref{ex:bochnak:Robert2}), the addition of \textit{oon} in (\ref{ex:bochnak:Robert2}) gives rise to a cessation inference that the result state of the event (here, the subject being gone) no longer holds at present.\footnote{In examples taken from other sources, we modify the morpheme glosses according to the analysis of \citet{Martinovic2015b}. For examples from \citealt{robert91approche}, we keep the original French translation and add our own colloquial English translation. The translations do not represent our analysis of the Wolof forms.}



\ea
\gll Dem-na-$\emptyset$ Ndar.\\
go-C-\textsc{scl.3sg} Saint-Louis\\
\glt \textit{``He left for Saint-Louis (and is still there).''}\\
\textit{``Il est parti \`a Saint-Louis (c'est toujours vrai, il n'est
  pas l\`a).''} \hfill \citep[p.~279]{robert91approche}\label{ex:bochnak:Robert1}
\z

\ea
\gll Dem\textbf{-oon}-na-$\emptyset$ Ndar.\\
go-\textsc{pst}-C-\textsc{scl.3sg} Saint-Louis\\
\glt \textit{``He had left for Saint-Louis (and has since come
  back).''}\\\textit{``Il \'etait parti \`a Saint-Louis (et en T$_0$,
  il est revenu).''} \hfill \citep[p.~279]{robert91approche}\label{ex:bochnak:Robert2}
\z


The use of \textit{oon} with stative predicates gives rise to the inference that the state no longer holds in the present, as illustrated by the translations in (\ref{ex:bochnak:Tor1})-(\ref{ex:bochnak:Tor2}).

\ea
\gll Tiit-na-a.\\
afraid-C-\textsc{scl.1sg}\\
\glt \textit{``I am afraid.''}\hfill\citep[p.~25]{torrence12clause}\label{ex:bochnak:Tor1}
\z

\ea
\gll Tiit-\'oon-na-a.\\
afraid-\textsc{pst}-C-\textsc{scl.1sg}\\
\glt \textit{``I was afraid (but I am not now).''}\hfill\citep[p.~26]{torrence12clause}\label{ex:bochnak:Tor2}
\z

The analysis of \textit{oon} as a discontinuous past contrasts with that of the past tense in languages like English, which makes no claim about the state of affairs at present. For instance, in (\ref{ex:bochnak:russian1}), we have a discourse about a past time, and the past tense is used in each clause. These uses of the past tense in English simply refer to the topical past time, and make no claims about the state of affairs at the speech time. For instance, the sentence in (\ref{ex:bochnak:russian}) only makes a claim about the past topic time (the time of looking in the room), and not about the present; intuitively, the book is still in Russian at the speech time (if it still exists).

\begin{exe}
\ex\label{ex:bochnak:russian1} Context: A judge poses question (a) to a witness, who replies with (b)-(c):
\begin{xlist}
\ex What did you notice when you looked in the room?
\ex The light was on. There was a book on the table.
\ex\label{ex:bochnak:russian} It was in Russian. \hfill \citep{klein94time}
\end{xlist}
\end{exe}


In this paper, we argue that \textit{oon} is in fact not a discontinuous past, but rather a past marker with a conventional meaning parallel to the past tense in English. Using diagnostics similar to those that \citet{cable16implicatures} used in his study of Tlingit, we show that the cessation inference of \textit{oon} is not part of its conventional meaning, but rather is a conversational implicature, arising due to competition with temporally unmarked clauses (see also \citealt{bochnak16past} on optional past in Washo (Hokan/isolate, California and Nevada)). In this respect, Wolof \textit{oon} is similar to other optional past markers in other languages, as has been argued in the recent literature. In this respect, we concur with \citet{church81systeme}, who also showed that \textit{oon} does not always have a ``discontinuous" interpretation (though with different terminology and analytical tools). %, e.g., \cite{bochnak16past} for Washo (Hokan/isolate, California and Nevada), \cite{cable16implicatures} for Tlingit (Na Dene, Alaska and British Columbia).
 
The paper proceeds as follows. In section 2, we discuss the temporal
interpretation of tenseless clauses in Wolof, while in section 3 we
turn to the interpretation of \textit{oon} and show that it behaves
like an ordinary past tense marker. Section 4 contains our analysis, including a proposal for
deriving the cessation implicature associated with \textit{oon}. In
section 5 we survey some syntactic evidence that suggests that
\textit{oon} does not behave syntactically like a tense head. Section 6 concludes.

Unless noted otherwise, all the data in the paper were obtained by the
second author in Saint-Louis, Senegal, during March 2016 and
April-May 2017. All speakers were native speakers of Wolof, and Wolof
was their first language. The data represent judgments of 9 speakers,
age 30 to 68.  We use direct elicitation in order to replicate as close as possible the data used in previous works on so-called discontinuous pasts in other languages \citep{bochnak16past, cable16implicatures}.


\section{Temporal interpretation of tenseless clauses}

Wolof finite indicative clauses have an obligatory complementizer
layer \citep{Martinovic2015b}. There are several types of
complementizers with different syntactic and information-structural
properties; these differences do not concern us here as they do not
affect the temporal interpretation. We therefore gloss all
complementizers as C.

 Tense marking and negation in Wolof are only
possible in the presence of a complementizer \citep{Njie1982}. Wolof
also has \textit{minimal clauses} \citep{Sauvageot1965,
  church81systeme,  Dialo1981, robert91approche,
  Zribi-HertzDiagne2003}, which can be used in a narrative context and
appear to be smaller than TPs.\footnote{\cite{Zribi-HertzDiagne2003} consider
them to be \textit{v}Ps, but they can contain imperfective aspect,
which suggests they are at least as big as an AspP.} The temporal interpretation of such
clauses is determined with respect to a previously introduced
temporal anchor. In this paper we
are therefore only concerned with clauses that contain the CP and TP layers.

In clauses with no overt tense/aspect marking, stative predicates receive a present interpretation by default, as in (\ref{ex:bochnak:good1})-(\ref{ex:bochnak:angry1}).

\ea
\gll Baax-na-$\emptyset$.\\
good-C-\textsc{scl.3sg}\\
\glt \textit{``It is good.''/ $^{\#}$``It was good.''}\label{ex:bochnak:good1}
\z

\ea
\gll Da-ma mer.\\
do.C-\textsc{scl.1sg} angry\\
\glt \textit{``I am angry.''/ $^{\#}$``I was angry.''}\label{ex:bochnak:angry1}
\z

Meanwhile, eventive predicates receive a default past interpretation, as in (\ref{ex:bochnak:eat1})-(\ref{ex:bochnak:leave1}). As shown in (\ref{ex:bochnak:dance1}), activities pattern with other eventive predicates, rather than states.

\ea
\gll Xale yi lekk-na-\~nu ceeb.\\
child the.\textsc{pl} eat-C-\textsc{scl.3pl} rice\\
\glt \textit{``The children ate rice.''/ $^{\#}$``The children are eating
  rice.''}\label{ex:bochnak:eat1}
\z
 
\ea
\gll Musaa dem-na-$\emptyset$.\\
Moussa leave-C-\textsc{scl.3sg}\\
\glt \textit{``Moussa left.''/ $^{\#}$``Moussa is leaving.''}\label{ex:bochnak:leave1}
\z

\ea
\gll Musaa f\'ecc-na-$\emptyset$.\\
Moussa dance-C-\textsc{scl.3sg}\\
\glt \textit{``Moussa danced.''/ $^{\#}$``Moussa is dancing.''}\label{ex:bochnak:dance1}
\z

However, these defaults are not tied to the aspectual class of the predicate
per se. Derived statives (e.g., eventive predicates co-occurring with
`imperfective'  \textit{di}) can also have present temporal reference,
as in (\ref{ex:bochnak:see1})-(\ref{ex:bochnak:eat2}).\footnote{In the examples, \textit{-y} is
an allophonic realization of \textit{di}; see Dunigan 1994, Torrence
2005, 2012, Martinovi\'c 2015.}



\ea
\gll Usmaan-a {di ($>$Usmaanay)} gis Musaa. \\
Oussman-C \textsc{impf} see Moussa \\
\glt \textit{``It's Oussman who sees Moussa.''}\label{ex:bochnak:see1}
\z

\ea
\gll Daf-a-$\emptyset$ {di ($>$dafay)} a\~n, m\"en-ul \~n\"ew.\\
do-C-\textsc{scl.3sg} \textsc{impf} eat.lunch, can-\textsc{ex:bochnak:neg} come\\
\glt \textit{``He is eating lunch, he cannot come.''}\\
\textit{``Il est en train de manger, il ne peut pas venir.''}\hfill\citep[p.~263]{robert91approche}\label{ex:bochnak:eat2}
\z




To account for these facts, we follow the principles of \citet{smith05temporal, smith07time} for default temporal interpretation of tenseless clauses. These principles were developed to account for temporal interpretation of the tenseless language Mandarin \citep{smith05temporal}, and have been applied to other tenseless languages, such as Navajo \citep{smith07time} and Hausa \citep{mucha13temporal}. The three main principles---the Deictic Principle, the Simplicity Principle of Interpretation, and the Bounded Event Constraint---are given in (\ref{ex:bochnak:deictic})-(\ref{ex:bochnak:bounded}):

\ea\label{ex:bochnak:deictic} \textbf{Deictic Principle} \\ Situations (events) are located with respect to UT \\(i.e., utterance time is the default reference point)
\z

\begin{exe}
\ex\label{ex:bochnak:simplicity} \textbf{Simplicity Principle of Interpretation} \\
Choose the interpretation that requires the least information added or
inferred. \smallskip \\
\textbf{Hierarchy of Simplicity:} 
\begin{xlist}
\ex\label{ex:bochnak:simple1} RT = UT: Present time reference is the simplest kind of temporal reference since (i) an utterance event always provides a time interval to which an RT variable can be anchored, namely UT; (ii) present interpretation requires no displacement of either the time or world of evaluation
\ex\label{ex:bochnak:simple2} RT $<$ UT: Past time reference is more complex since it requires the displacement of RT from the concrete utterance event
\ex\label{ex:bochnak:simple3} RT $>$ UT: Future time reference involves both RT shifting but also modal displacement, and thus increases the level of abstraction
\end{xlist}
(ensures that present is preferred over past, which is in turn preferred over future)
\end{exe}

\ea\label{ex:bochnak:bounded} \textbf{Bounded Event Constraint} \\ Bounded events are not located in the present. Speakers follow a tacit convention that communication is instantaneous. The present perspective is incompatible with the report of a bounded event, because the bounds would go beyond that moment. \\
(bounded events cannot be located in the present)
\z

The Deictic Principle states that the utterance time is the default reference point for temporal interpretation. Together with (\ref{ex:bochnak:simple1}), this predicts a present interpretation as a default. However, by (\ref{ex:bochnak:bounded}), bounded events -- which covers (perfective) eventive predicates -- cannot be located in the present. These are then shifted to a past interpretation, given (\ref{ex:bochnak:simple2}). This setup also predicts that future reference with tenseless clauses is dispreferred. In many tenseless languages, additional morphology must be used to achieve future time reference \citep{matthewson06temporal, tonhauser11temporal, bochnak16past}. This is indeed also the case for Wolof, where the imperfective marker \textit{di} is used for future reference, as in (\ref{ex:bochnak:future1})\footnote{See \citealt{bochnak17deriving} for discussion and an analysis of imperfective \textit{di} and its future uses.}

\ea
\gll Di-na-a toog ceeb-u-j\"en.\\
\textsc{impf}-C-\textsc{scl.1sg}  cook rice-\textsc{gen}-fish\\
\glt \textit{``I will cook ceebuj\"en.''}\label{ex:bochnak:future1}
\z


\section{The interpretation of \textit{oon}}

Turning to the semantics of \textit{oon}, we argue that it is a
regular past tense marker. The main pieces of evidence for this claim are that clauses with \textit{oon} are obligatorily interpreted as
past, and the cessation inference (i.e., ``discontinuous'' interpretation) does not always occur with \textit{oon}. We also show that
\textit{oon} is not an English-style perfect, and that \textit{oon} is
found in counterfactual conditionals.

First, we find that \textit{oon} is only compatible with past time reference. In addition to the examples we have already seen, we add (\ref{ex:bochnak:good2})-(\ref{ex:bochnak:eat3}) below.

\ea
\gll Baax-\textbf{oon}-na-$\emptyset$.\\
good-\textsc{pst}-C-\textsc{scl.3sg}\\
\glt \textit{``It was good.''/ $^{\#}$``It's good.''}\label{ex:bochnak:good2}
\z

\ea
\gll Xale yi lekk-\textbf{oon}-na-\~nu ceeb.\\
child the.\textsc{pl} eat-\textsc{pst}-C-\textsc{scl.3pl} rice\\
\glt \textit{``The children ate rice.''/ $^{\#}$``The children are eating
  rice.''}\label{ex:bochnak:eat3}
\z

Second, the cessation inference associated with \textit{oon} is not
always present for all speakers.\footnote{There is both interspeaker
  and intraspeaker variation in this. Some speakers insist on the
  cessation inference in some contexts but not in others, and for some
  speakers it is never present. We have not found any speakers for
  whom the cessation inference is obligatory in all contexts that we tested.} Recall the data in (\ref{ex:bochnak:Robert1})-(\ref{ex:bochnak:Robert2}), repeated here, which show that the use of \textit{oon} can
trigger a cessation inference.

\ea
\gll\label{leave1-rep}Dem-na-$\emptyset$ Ndar.\\
go-C-\textsc{scl.3sg} Saint-Louis\\
\glt \textit{``He left for Saint-Louis (and is still there).''}\\\textit{``Il est parti \`a Saint-Louis (c'est toujours vrai, il n'est
  pas l\`a).''} \hfill \citep[p.~279]{robert91approche}
\z

\ea
\gll\label{leave2-rep}Dem\textbf{-oon}-na-$\emptyset$ Ndar.\\
go-\textsc{pst-C-scl.3sg} Saint-Louis\\
\glt \textit{``He had left for Saint-Louis (and has since come back).''}\\\textit{``Il \'etait parti \`a Saint-Louis (et en T$_0$,
  il est revenu).''} \hfill \citep[p.~279]{robert91approche}
\z

However, when a past topic time is overtly specified, e.g., by a time adverbial as in (\ref{ex:bochnak:car}), there is no cessation implicature.\footnote{We place \textit{oon} in brackets to indicate that the sentence with and without \textit{oon} are accepted by speakers in the context provided.}

\ea\label{ex:bochnak:car}
\gll Musaa j\"end(\textbf{-oon})-na-$\emptyset$ oto bu bees at bi j\'all, waye mu-{angi ($>$ mungi)} ko {di ($>$ koy)} dawal ba l\'eegi.\\
Moussa buy(-\textsc{pst})-C-\textsc{scl.3sg} car C new year C past but \textsc{scl.3sg}-C \textsc{ocl.3sg} \textsc{impf} drive until now\\
\glt \textit{``Moussa bought a new car last year, but he is still driving it.''} 
\z

With predicates such as \textit{xaru} `kill oneself', many speakers
report that the use of \textit{oon} implies that Moussa is now alive
again, or that the suicide was unsuccessful (i.e., a cessation
inference is detected). However, this effect is reported to go away
for some speakers in
particular contexts; e.g.~if
(\ref{ex:bochnak:suicide}) is said as part of the story of Moussa's life, or if
we are retelling the events of, for example, last week.

\ea\label{ex:bochnak:suicide}
\gll Musaa xaru(-\textbf{woon})-na-$\emptyset$ ayub\'es bi weesu.\\
Moussa kill.oneself\textsc{(-pst)-C-scl.3sg} week C past\\
\glt \textit{``Moussa killed himself last week.''}
\z

The example (\ref{ex:bochnak:russian1}) from \cite{klein94time} is also felicitous in Wolof for most
of our speakers, as shown in (\ref{ex:bochnak:wolof1}). Even though the book
presumably still exists and is still in Wolof,  \textit{oon} can be
used in the answer in (\ref{ex:bochnak:wolof2}).

  \begin{exe}
\ex\label{ex:bochnak:wolof1} Context: A judge poses question (a) to a witness, who replies with (b)-(c):
\begin{xlist}
\ex 
\gll Lan nga gis bi nga xool-e neeg bi?\\
what C.\textsc{scl.2sg} see when \textsc{scl.2sg} look.at-\textsc{ant} room the.\textsc{sg}\\
\glt\textit{``What did you see when you looked at the room?''}
\ex
\gll L\`amp bi t\`akk-oon-na-$\emptyset$. Am-oon-na-$\emptyset$ benn t\'eer\'e bu ubbeeku si kaw taabal bi.\\
lamp the.\textsc{sg}
be.alight-\textsc{pst}-C-\textsc{scl.3sg}. have-\textsc{pst}-C-\textsc{scl.3sg}
one book C be.open on top table the.\textsc{sg}\\
\glt\textit{``The light was on. There was an open book on the table.''}
\ex
\gll T\'eer\'e wolof la-$\emptyset$ (woon).\\
book Wolof C-\textsc{scl.3sg} (\textsc{pst}).\\
\glt \textit{``It was/is in Wolof.''}\label{ex:bochnak:wolof2}

\end{xlist}
\end{exe}



Since the cessation inference is not always present, we conclude that it is not part of the lexical semantics of \textit{oon}. Therefore, we do not consider it a ``discontinuous past'' in the sense of \citet{Plungian2006}, since it does not assert that a state of affairs fails to hold at speech time.


Third, we observe that \textit{oon} does not behave like an English-style perfect.\footnote{We acknowledge that perfects in many languages do not have these properties.} The English perfect does not co-occur with temporal frame adverbials \citep{klein92present}, see (\ref{ex:bochnak:english}). However, we have already seen in (\ref{ex:bochnak:car}) that \textit{oon} can co-occur with temporal adverbials.

\ea\label{ex:bochnak:english}
$^{\#}$I have bought a car yesterday/last year/on December 1, 2010.
\z

The English perfect also displays so-called lifetime effects \citep{mccawley71tense}. The sentence in (\ref{columbus2}) is apparently infelicitous because Christopher Columbus is no longer living. However, as shown in (\ref{ex:bochnak:columbus}), Wolof \textit{oon} does not have this property.

\ea\label{columbus2}
$^{\#}$Christopher Columbus has discovered America.
\z

\ea\label{ex:bochnak:columbus}
\gll Colombo f\'ee\~nal(\textbf{-oon})-na-$\emptyset$ Amerik.\\
Columbus find-(\textsc{pst)-C-scl.3sg} America\\
\glt \textit{``Columbus found America.''}
\z

Another property of the English perfect (and of so-called terminative aspects more generally, see \citealt{bohnemeyer02grammar}), is that they assert that the result state of the perfect-marked event still holds. This means continuations like in (\ref{ex:bochnak:glasses}) are infelicitous. These types of examples are nevertheless felicitous in Wolof with \textit{oon}, as shown in (\ref{glasses2}).

\ea\label{ex:bochnak:glasses} 
I have lost my glasses, $^{\#}$and now I (have) found them. 
\z

\ea\label{glasses2}
\gll Sama lunettes r\'eer(-\textbf{oon})-na-$\emptyset$-ma, waye gis(-\textbf{oon})-na-a-ko. \\
\textsc{poss.1sg} glasses lose\textsc{(-pst)-C-scl.3sg-ocl.1sg} but see\textsc{(-pst)-C-scl.1sg-ocl.3sg} \\
\glt \textit{``I lost my glasses, but I found them.''} 
\z

Fourth, we find \textit{oon} in counterfactual conditionals. Past tense marking is common cross-linguistically in counterfactual conditionals \citep{iatridou00grammatical, halpert12aspect}, including in English. The sentence in (\ref{ex:bochnak:paris}) has a present topic time (by the presence of \textit{right now}), but has past morphology in the antecedent. We also see this in Wolof, where \textit{oon} appears in counterfactual conditionals, as in (\ref{ex:bochnak:counterfact}).\footnote{We do not intend this point as an argument against a discontinuous past analysis, but rather as evidence that \textit{oon} behaves quite similarly to non-discontinuous pasts in more familiar languages.}  

\ea\label{ex:bochnak:paris} 
If I was in Paris right now, I would be eating a croissant. 
\z

\ea\label{ex:bochnak:counterfact}
\gll Su-ma ragal\textbf{-oon} rabi, di-na-a tiit l\'eegi.\\
if-\textsc{scl.1sg} be.afraid.of-\textsc{pst} spirit,
\textsc{impf-C-scl.1sg} be.afraid now\\
\glt \textit{``If I was afraid of spirits, I would be afraid now.''}
\z

Although the role of the past tense in counterfactuals is a matter still very much under debate, it is certainly striking that Wolof uses this marker in counterfactuals, just like in many other typologically unrelated languages.

In sum, apart from its optionality, \textit{oon} behaves like a regular past tense, where the apparent discontinuous semantics are not part of its conventional meaning. We therefore propose to analyze it semantically as a regular past tense, just like other optional pasts in Washo \citep{bochnak16past} and Tlingit \citep{cable16implicatures}.




\section{Analysis}

We define tense as a morpheme whose conventional meaning relates a reference or topic time with the speech time, or possibly another evaluation time \citep{reichenbach47tenses, klein94time}. We assume a pronominal or referential theory of tense, whereby the reference time of a clause is represented as a temporal pronoun located in the T head (e.g., \citealt{abusch97sequence, heim94comments, partee73some}, among many others). Like other pronouns, it bears an index, and receives its value from an assignment function $g$. Every finite clause contains a reference time pronoun, regardless of whether there is an overt tense morpheme or not. We treat \textit{oon} as a tense feature which modifies the temporal pronoun, placing a presupposition on its possible values (i.e., restricting it to times in the past of the speech time).

We propose that a sentence such as (\ref{ex:bochnak:tree1}) has the clause
structure given in Figure \ref{ex:bochnak:tree-cs}.\footnote{The example
  (\ref{ex:bochnak:tree1}) is of a predicate focus sentence with
  \textit{do}-support in C. We choose this clause type for
  illustration as the verb here stays low, unlike in some other cases
  when it raises to C taking \textit{oon} with it. The clause structure
 in Figure \ref{ex:bochnak:tree-cs} is somewhat simplified from what \cite{Martinovic2015b} assumes; any
differences are not relevant for our analysis here. For example, the
non-pronominal subject in these clauses is in the left periphery
(Spec,CP), and it is doubled by a subject clitic which is here
represented in Spec,TP (the clitics all move to adjoin above TP at a
late stage in the syntax). The details of the doubling are not
relevant here; we assume that the non-pronominal argument is generated
in the subject position in Spec,\textit{v}P (omitted for simplicity). Additionally, the verb also
raises through the Asp head and carries it on to T, but we also omit this
here. \textsc{scl} = subject clitic.} Semantically, AspP denotes a predicate of times as in (\ref{ex:bochnak:AspP}), where we assume arguments of the verb are interpreted in their base position. The temporal argument slot is filled in by the reference time pronoun. When  \textit{oon}, defined in (\ref{ex:bochnak:woon1}), surfaces, it adds the presupposition that the reference time is located in the past of the speech time $t_c$. (In the absence of a morphological tense, we assume the value of the reference time pronoun in T is restricted by the principles outlined in section 2.) The result is a proposition meaning, given in (\ref{ex:bochnak:TP}).\footnote{\citet{robert91approche} analyzes \textit{oon} as a relative past, in which case the reference time can be related to an evaluation time other than the speech time, i.e., $t_c$ in (\ref{ex:bochnak:woon1}) can be distinct from the speech time.}

\ea\label{ex:bochnak:tree1} \gll Colombo daf-a-$\emptyset$ f\'e\~naal-oon Amerik.\\
Columbus do-C-\textsc{scl.3sg} discover-\textsc{pst} America\\
\glt\textit{``Columbus DISCOVERED America''}
\z

%\begin{figure}
%\Tree [.CP [.DP\\\textit{Colombo} ] [.C$'$ [.C [.V\\\textit{daf} ] [.C\\\textit{a} ] ] [.TP [.\textsc{scl}\\$\emptyset$ ] [.T$'$ [.T [.V\\\textit{f\'e\~naal} ] [.T [.\textit{oon} ] [.T$_1$ ] ] ] [.AspP [.Asp ] [.VP [.$t_V$ ] [.DP\\\textit{Amerik} ] ] ] ] ] ] ]
%\caption{Wolof clause structure}
%\label{ex:bochnak:tree-cs}
%\end{figure}

\begin{exe}
\ex 
\begin{xlist}
\ex\label{ex:bochnak:AspP} $\llbracket$ AspP $\rrbracket^{g,c} = \lambda t \lambda w.\textbf{discover}(a)(c)$ at $t$ in $w$
\ex\label{ex:bochnak:T1} $\llbracket$ T$_1$ $\rrbracket^{g,c} = g(1) $
\ex\label{ex:bochnak:woon1} $\llbracket$ \textit{oon} $\rrbracket^{g,c} = \lambda t.t$ ; defined only if $t < t_c$  
\ex\label{ex:bochnak:TP} $\llbracket$ TP $\rrbracket^{g,c} = \lambda w.\textbf{discover}(a)(c)$ at $g(1)$ in $w$ ; defined only if $g(1) < t_c$ 
\end{xlist}
\end{exe}


Under this analysis, cessation is not part of the lexical semantics of \textit{oon}, contra \citet{Plungian2006}. Instead, \textit{oon} only adds a plain past presupposition, just like the past tense in English. The question, then, is how to account for the robust intuition, both by native speakers and previous authors, that the use of \textit{oon} in many contexts generates a cessation inference.

We suggest that the cessation inference is a conversational implicature derived by the Gricean Maxim of Manner (cf.~\citealt{altshuler12moment, cable16implicatures}, for whom cessation inferences are analyzed as \textit{scalar} implicatures). Following \citet{levinson00presumptive}, a marked message indicates a marked situation. We assume that a past-referring clause containing \textit{oon} is ``marked" in comparison to a past-referring tenseless clause. A Gricean chain of reasoning proceeds as follows.\footnote{A reviewer points out the possibility that Gricean conversational maxims may not be followed in all cultures. At this point, we have no reason to believe they do not apply in Wolof, but leave further investigation into this question for later work.} A sentence with \textit{oon} is morphologically more marked than a sentence without \textit{oon}. Given that past marking is not required by the grammar of Wolof, the speaker (in most situations) could have used an unmarked form for past temporal reference. Since the speaker used a marked form, the hearer infers that the speaker must believe that the situation is marked. That is, more than just a plain past meaning is intended by the speaker. The hearer infers that the state of affairs described does not hold at present, otherwise the simpler form could have been used. Therefore, cessation is an inference derived from the fact that \textit{oon} is optional, and temporally unmarked clauses can also have past interpretations. 

If the implicature calculation is based on Manner, whereby a marked message leads to pragmatic enrichment, the question arises as to why a \textit{cessation} implicature in particular is calculated. Why  could some other inference not be calculated? Apparently, other inferences are in fact possible and attested. For instance, \citet{church81systeme} claims that the use of \textit{oon}
often gives rise to a \textit{remoteness} inference as well. Our speakers also seem to prefer \textit{oon} when talking about a time in a more distant past. Some speakers even report they \textit{must} use \textit{oon} in those cases (e.g.~when speaking about an event that occurred last year). This inference would also be a conversational implicature, and not part of the lexical semantics of \textit{oon}, given examples like (\ref{ex:bochnak:car2}), where \textit{oon} is possible with \textit{demb} `yesterday' (assuming one day ago does not count as `distant').

\ea\label{ex:bochnak:car2}
\gll Musaa j\"end(-\textbf{oon})-na-$\emptyset$ oto bu bees d\'emb \\
Moussa buy\textsc{-pst-C-scl.3sg} car C new yesterday. \\
\glt \textit{``Moussa bought a new car yesterday."} 
\z



\section{Is \textit{oon} a tense head?}

There is some indication that \textit{oon} is syntactically not a
head. First, in clauses in which the verb raises to C, \textit{oon} is
affixed onto it in affirmative cases, as in (\ref{ex:bochnak:affirm}) but skipped over in the presence
of negation, shown in (\ref{ex:bochnak:neg}). \cite{Martinovic2015b, Martinovic2016a} proposes an
analysis of the affixation of \textit{oon} in which she argues that
\textit{oon} affixes onto
the verb postsyntactically (at PF) in a certain syntactic
configuration, but not in others. Crucially, for her analysis to go
through, \textit{oon} cannot be a head, as it would then always be
picked up by head movement. She therefore suggests that \textit{oon}
is a phrasal morpheme in Spec,TP.


  
\ea
\gll Xale yi lekk-\textbf{oon}-na-\~nu j\"en.\\
child the.\textsc{pl} eat-\textsc{pst-C-scl.3pl} fish\\
\glt \textit{``The children ate fish.''}\label{ex:bochnak:affirm}
\z

\ea
\gll Xale yi lekk-{u(l)}-$\emptyset$-\~nu \textbf{woon} j\"en.\\
child the.\textsc{pl} eat-\textsc{neg-C-scl.3pl} \textsc{pst} fish\\
\glt \textit{``The children didn't eat fish.''}\label{ex:bochnak:neg}
\z


The second piece of evidence that casts doubt on the treatment of 
\textit{oon} as a T head is that it can occur apparently affixed onto
non-verbal elements, as reported by \cite{torrence12clause}. 

\ea
\gll Kan-ati-\textbf{woon} la-\~nu d\'oor?\\
who-again-\textsc{pst} C-\textsc{scl.3pl} hit\\
\glt \textit{``Who did they hit again?''}\hfill\citep[p.24]{torrence12clause}
\z

Given the examples above, it is possible that \textit{oon} is not a
tense morpheme, but a particular type of a temporal adverbial (cf.~\citealt{tonhauser06temporal} on an optional past adverbial in Paraguayan Guaran\'\i). We
leave this only as a speculation at this point, as the matter requires
further research. Given that our definition of tense in section 4 only makes reference to semantic notions, the question of whether \textit{oon} behaves syntactically like a T head is orthogonal to the core aspects of our semantic analysis, where \textit{oon} is treated as a tense (i.e., relating a reference time to speech time).


\section{Conclusion}

We have argued that \textit{oon} in Wolof marks past temporal reference, and can be given an analysis of a semantic past tense. The cessation inference detected by \citet{church81systeme} and
\citet{robert91approche}, and analyzed as discontinuous past by
\citet{Plungian2006}, is not present in all uses of \textit{oon},
and we argue this is a conversational implicature. Given that the same conclusion was reached for the optional tense languages Washo \citep{bochnak16past} and Tlingit \citep{cable16implicatures}, the status of discontinuous past as a grammatical category is therefore called into question. 


\section*{Abbreviations}

\textsc{ant} = anterior; C= complementizer; \textsc{gen} = genitive; \textsc{impf} = imperfective; \textsc{ex:bochnak:neg} =
negation; \textsc{ocl} = object clitic; \textsc{pl} = plural; \textsc{poss} = possessive;
\textsc{rel} = relative; \textsc{scl} = subject clitic; \textsc{sg} = singular


\section*{Acknowledgements}

We wish to thank our Wolof consultants Abdou Aziz Djakhate, Magatte Bocar Ndiaye, Louis Camara, Mbaye
Diop, Jean L\'eo Diouf, Ibrahim Gise,  Alioune Kebe, Ismail Kebe,
Tapha Ndiaye for their time and dedication. We also thank the audience
at ACAL 48 for their helpful comments. All errors are our own.  

\sloppy
\printbibliography[heading=subbibliography,notkeyword=this]

\end{document}
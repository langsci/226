\documentclass[output=paper,
modfonts
]{langscibook} 

\title{Focus in Limbum} 

\author{%
Laura Becker\affiliation{Leipzig University}\and 
 Imke Driemel\affiliation{Leipzig University}\lastand 
 Jude Nformi Awasom\affiliation{Leipzig University}
}

% \chapterDOI{} %will be filled in at production
% \epigram{}

\abstract{In this paper, we discuss the realization of focus in Limbum (Grassfields Bantu, Cameroon), a language which shows a so-far unattested pattern of focus marking, where two distinct focus constructions are realized by two different particles, {\em \'a} and {\em b\'a}, which express information focus on the one hand and contrastive focus on the other. Strikingly, the former is realized by a structurally more complex construction (particle\,+\,fronting) -- the inverse pattern of what is attested cross-linguistically \citep{Fiedler2010,Skopeteas2009}. A biclausal cleft structure underlying the {\em \'a} strategy can be argued to be implausible. Instead, we adopt a Q/F particle analysis \citep{Cable2010} which proposes the existence of a particle independent of a higher functional head mediating between that head and the focused phrase. Limbum provides overt evidence for both, the head and the particle.}

\begin{document}
\maketitle

\section{Introduction} 

The present paper discusses two focus strategies in Limbum (Grassfields Bantu, Cameroon) that can be distinguished on the basis of different focus markers and the types of focus they convey. The constructions and their respective focus markers are shown in (\ref{ex:becker:intro})\footnote{Translations are modeled after the interpretations the focus strategies come with. Small capitals, as in (\ref{ex:becker:intro1}), signals pitch accent, an intonation strategy English makes use of. A cleft structure is chosen as a translation if the sentence conveys an exhaustive meaning, see (\ref{ex:becker:intro2}).} below.

\ea \label{ex:becker:intro}
\ea \label{ex:becker:intro1}
\gll \textbf{á}  Nfor (cí) m\`ɛ bí t\=u \\
\textsc{foc} Nfor \hspaceThis{(}\textsc{comp} \textsc{1sg} \textsc{fut1} send \\
\glt `I will send \textsc{\MakeLowercase{NFOR}}.'
\ex \label{ex:becker:intro2}
\gll m\`ɛ bí t\=u bá Nfor\\
\textsc{1sg} \textsc{fut1} send \textsc{foc} Nfor\\
\glt `It is Nfor whom I will send.'\footnote{All Limbum data in this paper are our own. They are based on the judgement of two native speakers of Limbum from Nkambe.}
\z
\z 


\noindent
The sentence in (\ref{ex:becker:intro1}) shows the focus marker \textit{á}, consistently followed by the fronted constituent that is focused, in turn followed by an optional element, which we label complementizer for now. Note that this element occurs exclusively with this type of focus construction, which we will address in more detail in \sectref{sec:becker:subsec:analysis}. The second strategy, shown in (\ref{ex:becker:intro2}), involves the marker \textit{bá}, which consistently occurs left adjacent to the focused constituent.
 
We will show that the \textit{á} construction, although appearing similar to cleft constructions, does not mark contrastive/exhaustive focus, but rather information focus.\footnote{\label{footnote:becker:fn1}Note that Limbum also has the option of leaving focus completely unmarked. This strategy mostly patterns with {\em\'a}. The {\em \'a} strategy, however, imposes an existence assumption \citep{Dryer1996} on the context which is not required in the absence of a focus marking particle. For reasons of space, we cannot go into detail here, but see \citet{Driemeletal2018}.} For exhaustive focus, only the \textit{bá} strategy is felicitous.\footnote{While {\em b\'a} necessarily expresses exhaustivity, it is not the only strategy Limbum offers to express such a type of focus. Example (\ref{ex:becker:s3}) presents a cleft structure which is also able to trigger an exhaustive interpretation.} This is rather surprising, since the structurally more complex construction with \textit{á} and fronting of the focused constituent is used to convey the ``simpler'' kind of focus, i.e.\ focus without any additional semantic or pragmatic restrictions. This goes against the trend of focus marking observable cross-linguistically, where information focus is expressed with a canonical focus structure and contrastive focus with a relatively more marked structure \citep{Zimmermann2011a}. \sectref{sec:becker:semantics} briefly discusses semantic evidence for the focus constructions to necessarily express different types of focus. In \sectref{sec:becker:syntax}, we turn to the syntactic analysis of the {\em \'a} strategy where we argue against an underlying cleft structure and eventually adopt a feature-driven focus movement analysis along the lines of \citet{Cable2010}.



\section{Focus in Limbum: Interpretation}\label{sec:becker:semantics}

Before we turn to the two constructions at stake, this section provides a brief overview of focus in general. Following \citet[2]{Zimmermann2010}, focus is ``a classical semantic notion expressing that a focused linguistic constituent is selected from a set of alternatives'', i.e.\ focus marks the presence of alternatives \citep{Rooth1992a,Krifk2008}. Focus is generally said to be involved in question-answer congruence, correction, and the marking of contrast, among other contexts.

The literature often distinguishes two main types, namely information focus and contrastive focus. The former signals the presence of contextual alternatives and often introduces new information. Therefore, we will use question-answer pairs to test for information focus. The latter type of focus comprises a number of subtypes, all of which add semantic and/or pragmatic conditions on the alternatives laid out by the presence of focus. In this paper, we will consider:\footnote{For reasons of brevity, we cannot discuss all possible types with respect to the focus strategies in Limbum in the present paper. To just name a few other important types, \textit{selection} features an explicit set of alternatives, from which one or more alternatives can be chosen; \textit{exclusivity} has one (set of) alternative(s) selected, where at least one of the non-selected (set of) alternative(s) is false \citep{vanderWal2011,vanderWal2014}, or only stronger alternatives on some scale are false \citep{Beaver2008,Coppock2012}; \textit{unexpectedness} involves the selected alternative to stand out \citep{Zimmermann2008,Zimmermann2011a,Hartmann2008,Skopeteas2009,Skopeteas2011,Frey2010,Zimmermann2011a,Destruel2014}. }
\begin{description}
\item[Information focus:] marks the presence of alternatives
\ea Who\textsubscript{F} stole the cookie?\\ \relax[\textsc{\MakeLowercase{PE}}ter]\textsubscript{F} stole the cookie. \z
\item[Contrast:] an explicit alternative is present; often within the same utterance
\ea An [\textsc{\MakeLowercase{AME}}rican]\textsubscript{F} farmer talked to a [Ca\textsc{\MakeLowercase{NA}}dian]\textsubscript{F} farmer. \z
\item[Correction:] an explicit alternative from a previous utterance is rejected by giving a new explicit alternative
\ea \relax[\textsc{\MakeLowercase{PE}}ter]\textsubscript{F} stole the cookie.\\ No, [\textsc{\MakeLowercase{MA}}ry]\textsubscript{F} did it.\z
\item[Exhaustivity:] one (set of) alternative(s) is selected; all non-selected alternatives are false \citep{Szabolcsi1981,Kiss1998,Vallduvi1998,Horvath2010,Horvath2013}, e.g.
\ea
Hungarian\\
\gll Anikó a templomba ment be, (máshová nem ment be).\\
Anikó the church.into went in.\textsc{prv} elsewhere not went in.\textsc{prv}\\
\glt `It was the \textsc{\MakeLowercase{CHURCH}} that Anikó entered (and nowhere else).'
\z 
\end{description}

\noindent
In this section, we will look at three context tests that show how the two focus markers are felicitous in different contexts in Limbum. Then, we will address exhaustivity in more detail and provide evidence for \textit{bá} involving exhaustivity, while \textit{á} does not.

New information can be modeled with the help of an inquisitive context. Imagine the following scenario:

\begin{exe}
\ex
\textit{Context:} A and B are talking on the phone, the connection is really bad. A was telling B that she was going to meet someone, but B could not understand the person's name. B asks A to repeat whom she is going to meet.\label{ex:becker:context1}
\begin{xlist}
\exi{A:}[]{
\gll á Ngàlá  (cí) m\`ɛ bí k\=ɔn\={\i} \\
\textsc{foc} Ngala \hspaceThis{(}\textsc{comp} \textsc{1sg} \textsc{fut1} meet \\
\glt `I will meet \textsc{\MakeLowercase{NGALA}}.'}
\exi{A$'$:}[\#]{
\gll m\`ɛ bí kɔn\={\i} bá Ngàlá\\
\textsc{1sg} \textsc{fut1} meet \textsc{foc} Ngala \\
\glt `It is Ngala whom I will meet.'}
\end{xlist}
\end{exe}

\noindent 
In such a context, A can clarify who she is going to meet with the \textit{á} marker, but not use \textit{bá}. The latter, as will be shown in (\ref{ex:becker:context2}) and (\ref{ex:becker:burgers}), requires an additional contrastive component.

Corrective contexts require an utterance with an explicit alternative, which is followed by another alternative in a second utterance, automatically canceling the first one. In such contexts, the \textit{bá} strategy is obligatory:

\begin{exe}
\ex
\textit{Context:} A bought a pair of shoes. B does not remember correctly and tells someone that A bought a dress. A corrects B saying that she bought shoes (instead). 
\label{ex:becker:context2}
\begin{xlist}
\exi{B:}[] {
\gll \'i b\'a y\=u c\`ɛʔ\\
 	\textsc{3sg} \textsc{pst2} buy dress\\
\glt `She bought a dress.'\label{ex:becker:7B}}
\exi{A:}[\#] {
\gll \'a bl\'ab\'aʔ (c\'i) m\`ɛ b\=a y\'u \\
\textsc{foc} shoes \textsc{comp} \textsc{1sg} \textsc{pst2} buy\\
\glt `I bought \textsc{\MakeLowercase{SHOES}}.'}
\exi{A$'$:}[] {
\gll m\`ɛ b\=a y\=u b\'a bl\'ab\'aʔ\\
\textsc{1sg} \textsc{pst2} buy  \textsc{foc} shoes\\
\glt `It is shoes that I bought.' }
\end{xlist}
\end{exe}
In order to correct B's statement, example (\ref{ex:becker:context2}) shows that \textit{bá} now becomes licit, while \textit{á} cannot be used to mark focus any longer in the presence of correction.

A similar effect can be observed with the expression of contrast. Again, only \textit{bá} is felicitous for contrasting two arguments, \textit{á} being not acceptable in this context.

\ea \label{ex:becker:burgers}
\ea[]{
\gll T\'ank\'o k\'i n\=ɔ {mndz\=\i p}, Ng\`al\'a c\'i n\=ɔ {b\'a} {bl\=e\=e}\\
Tanko \textsc{hab} drink {water} Ngala but drink \textsc{foc} blood\\
\glt `Tanko drinks water but it is blood that Ngala drinks.'}
\ex[*]{
\gll T\'ank\'o k\'i n\=ɔ {mndz\=\i p}, {\'a} {bl\=e\=e} c\'i Ng\`al\'a  n\=ɔ\\
Tanko \textsc{hab} drink {water} \textsc{foc} {blood} but Ngala  drink\\
\glt `Tanko drinks water but Ngala drinks \textsc{\MakeLowercase{BLOOD}}.'}
\z\z
To test for exhaustivity, we will apply tests that have been proposed by  \citet{Kiss1998}: combining exhaustively focused constituents with \textit{also} or universal quantifiers is infelicitous since they both semantically contradict exhaustivity. 
As examples (\ref{ex:becker:also2}B)\footnote{While {\em \'i} encodes a \textsc{3sg} pronoun, both {\em \`a} and {\em \'i} seem to function as \textsc{3sg} subject markers, i.e.\ they can optionally co-occur with NP subjects. \textsc{3sg} pronoun {\em \'i} can be seen in (\ref{ex:becker:7B}B), (\ref{ex:becker:also4}B-B$'$), and (\ref{ex:becker:s12}). \textsc{3sg} subject markers are realized either as {\em \`a}, see (\ref{ex:becker:also1}A-B), (\ref{ex:becker:also4}A), (\ref{ex:becker:s10}), (\ref{ex:becker:s2}), (\ref{ex:becker:s3}), and (\ref{ex:becker:s23}), or as {\em \'i}, see (\ref{ex:becker:s7}), (\ref{ex:becker:s4}), and (\ref{ex:becker:s5}).} and  (\ref{ex:becker:also5}B) show for the subject and object,  respectively, the {\em \'a} strategy is able to occur with {\em also} scoping over the constituent in focus. The marker \textit{bá}, on the other hand, does not allow for focused constituents including an \textit{also} phrase, see (\ref{ex:becker:also1}B$'$)\footnote{As can be observed in (\ref{ex:becker:also1}), subject focus comes with an additional restriction for {\em b\'a} focused constituents, in that they can only occur postverbally. Glossing {\em \`a} as \textsc{expl} is only one option and might not be the most convincing one, since typical expressions involving expletives such as weather verbs, locative inversions, or existential constructions do not occur with {\em \`a}. An alternative is to analyze {\em \`a} as a default marker since it is identical to the \textsc{3sg} subject marker.} and (\ref{ex:becker:also4}B$'$). This behaviour is consistent with contrast and correction scenarios, given in (\ref{ex:becker:burgers}) and (\ref{ex:becker:context2}), i.e. contexts that involve exhaustivity.

\begin{exe}
\ex
\begin{xlist}
\exi{A:}[] {{\label{ex:becker:also1}}
\gll Nf\`o \`a m\=u y\=u rk\=ar. \\
Nfor \textsc{3sg} \textsc{pst2} buy car\\
\glt `Nfor bought a car.'}
\exi{B:}[]{ \label{ex:becker:also2}
\gll \'a Ng\`al\'a  f\'ɔŋ \`a m\=u y\=u rk\=ar. \\
\textsc{foc} Ngala also \textsc{3sg} \textsc{pst2} buy car  \\
\glt `\textsc{\MakeLowercase{NGALA}} bought a car, too.'}
\exi{B$'$:}[\#] {\label{ex:becker:also3}
\gll \`a m\=u y\=u {b\'a} Ng\`al\'a rk\=a f\'ɔŋ. \\
 \textsc{expl} \textsc{pst2} buy \textsc{foc} Ngala car also\\
\glt `It was also Ngala who bought a car.'}
\end{xlist}
\end{exe}



\begin{exe}
\ex
\begin{xlist}
\exi{A:}[] {\label{ex:becker:also4} 
\gll Nf\`o \`a m\=u y\=u rk\=ar. \\
Nfor \textsc{3sg} \textsc{pst2} buy car\\
\glt `Nfor bought a car.'}
\exi{B:}[]{ \label{ex:becker:also5}
\gll \'a nt\`umnt\`um  f\'ɔŋ \'i m\=u y\'u. \\
\textsc{foc} motorbike also \textsc{3sg} \textsc{pst2} buy  \\
\glt `He bought a \textsc{\MakeLowercase{MOTORBIKE}}, too.'}
\exi{B$'$:}[\#] {\label{ex:becker:also6}
\gll \'i m\=u y\=u {b\'a} nt\`umnt\`um f\'ɔŋ. \\
 \textsc{3sg} \textsc{pst2} buy \textsc{foc} motorbike also\\
\glt `It was also a motorbike he bought.'}
\end{xlist}
\end{exe}
Using a universal quantifier inside of the focused constituent, we get the same effect: the universal quantifier is incompatible with exhaustivity because it inherently makes reference to all alternatives from a set, whereas exhaustivity entails that some alternative is selected from the set, excluding others. Again, examples (\ref{ex:becker:uni1}) and (\ref{ex:becker:uni2})  illustrate for focused subjects and objects that \textit{á}, as predicted, is compatible with universal quantifiers, while \textit{bá} is not:

\ea  \label{ex:becker:uni1}
\settowidth\jamwidth{object focus}
\ea[] { 
\gll á ŋw\`ɛ ns\`ip (c\'i) \`a b\=a zh\=e b\=a\=a  \\
\textsc{foc} person all \hspaceThis{(}\textsc{comp} \textsc{3sg} \textsc{pst1} eat fufu\\ \jambox{{subject focus}}
\glt `\textsc{\MakeLowercase{EVERYBODY}} ate fufu.'}
\ex[*]{
\gll \`a b\=a zh\=e bá ŋw\`ɛ ns\`ip b\=a\=a  \\ 
\textsc{expl} \textsc{pst1} eat \textsc{foc}  person all fufu \\ 
\glt `It is everybody who ate fufu.'}
\z \z 

\ea  \label{ex:becker:uni2}
\settowidth\jamwidth{object focus}
    \ea[] { 
\gll {\'a} {ŋw\`ɛ} {ns\`ip} (c\'i) m\`ɛ b\'i k\=ɔn\={\i}  \\
	\textsc{foc} {person} {all} (\textsc{comp}) I \textsc{fut1} meet \\ \jambox{{object focus}}
    \glt `I will meet \textsc{\MakeLowercase{EVERYBODY}}.'} 
\ex[*] {
\gll m\`ɛ b\'i kɔn\={\i} {b\'a} {ŋw\`ɛ} {ns\`ip}  \\
 	I \textsc{fut1} meet \textsc{foc}  {person} {all} \\ 
\glt `It is everybody that I will meet.'}
\z \z 


\section{The syntax of \textit{\'a}}\label{sec:becker:syntax}
Focused constituents that are preceded by the focus marker {\em \'a} have to occur clause-initially. They can be followed by what we have so far glossed as the complementizer {\em c\'i}.
\ea
\settowidth\jamwidth{object focus}
\ea \label{ex:becker:s7}
\gll \'{a} {Nf\`{o}} (c\'{i}) \'{i} b\=a zh\=e b\=a\=a\\  
     \textsc{foc} {Nfor} \hspaceThis{(}\textsc{comp} \textsc{3sg} \textsc{pst1} eat fufu\\ \jambox{{subject focus}}
\glt `\textsc{\MakeLowercase{NFOR}} ate fufu.'
\ex \label{ex:becker:s8}
\gll \'{a} {Ng\`{a}l\'{a}} (c\'{i}) m\`ɛ b\'{i} k\=ɔn\={\i}\\  
     \textsc{foc} {Ngala} \hspaceThis{(}\textsc{comp} \textsc{1sg} \textsc{fut1} meet\\ \jambox{{object focus}}
\glt `I will meet \textsc{\MakeLowercase{NGALA}}.'
\ex \label{ex:becker:s9}
\gll \'a {\`ay\`aŋs\`e} (c\'i) s\`i b\'if\=u y{\'{ε}} Shey\\  
     \textsc{foc} {tomorrow} \hspaceThis{(}\textsc{comp} \textsc{1pl.incl}  \textsc{fut2} see Shey\\ \jambox{{adverbial focus}}
\glt `We will see Shey \textsc{\MakeLowercase{TOMORROW}}.'
\z \z

Similar to many West African languages \citep{Koopman1984,Ameka1992,Manfredi1997,Biloa1997,Aboh1998,Aboh2006}, verb focus in Limbum is realized by doubling of the verb. Note that the higher copy of the verb differs from the lower copy in that it is prefixed with a noun class marker.\footnote{Nouns which are formed from verbs via prefixing of the noun class 5 marker \textit{r-} are generally the gerundive form of the verb \citep{Nformi2017}. In such derivations, the tone of the noun class prefix lowers the tone of the verb root if it is a H tone verb (\ref{ex:becker:s11}). The infinitive form of the verb in the language also looks similar to the gerundive but differs in that it has the infinitive marker \textit{\`a}.}
\ea {Verb focus:}\footnote{This focus construction cannot be used to express \textsc{tam} focus. It can, however, express \textsc{verum} focus.}
\settowidth\jamwidth{transitive}
\ea \label{ex:becker:s10}
\gll \'a {r-gw\`e} (c\'i) nd\=ap f\=ɔ \`a $\varnothing$ {gw\`e}\\  
     \textsc{foc} {\textsc{5}-fall} \hspaceThis{(}\textsc{comp} house \textsc{det}  \textsc{3sg} \textsc{perf} {fall}\\ \jambox{intransitive}
\glt `The house \textsc{\MakeLowercase{FELL}}.'
\ex \label{ex:becker:s11}
\gll \'a {r-y\=u} (c\'i) nj\'iŋw\`ɛ f\=ɔ b\'i {y\'u} ms\=aŋ\\  
     \textsc{foc} {\textsc{5}-buy} \hspaceThis{(}\textsc{comp} woman \textsc{det} \textsc{fut1} {buy} rice\\ \jambox{transitive}
\glt `The woman will \textsc{\MakeLowercase{BUY}} rice.'
\z \z


\subsection{Against a biclausal structure}
As was shown in the previous section, the \textit{\'a} strategy contrasts with the \textit{b\'a} strategy in that it is compatible with non-exhaustive contexts. This provides our first argument against an underlying biclausal cleft structure, as those are typically found with an exhaustive meaning component \citep{Horn1981,Percus1997}. In this section, we provide three syntactic arguments against a cleft structure. 

Based on sentences like (\ref{ex:becker:s1}) in which \textit{\'a} seems to act like a copula, \citet[][301]{Fransen1995} concludes that the high focus marker strategy constitutes a cleft.
\ea \label{ex:becker:s1}
\gll \'a rt\=e\=e\\  
     ? palm.tree\\ 
\glt `It is a palm tree.'
\z
An alternative analysis of (\ref{ex:becker:s1}) takes copulas to be silent while \textit{\'a} acts as a focus particle. This idea predicts copulas to show up as soon as they have to act as hosts for negation and/or tense affixes. Adding an overt tense marker to \textit{\'a} is ungrammatical, see (\ref{ex:becker:s2}). As predicted, the only way to save the structure is by using a copula and an expletive, see (\ref{ex:becker:s3}).
\ea \label{ex:becker:s2}
\gll (*m\=u) \'{a} (*m\=u) b\=a\=a (c\'i) Nf\`{o} \`{a} b\=a zh\=e \\  
     \hspaceThis{(*}\textsc{pst2} \textsc{foc} \hspaceThis{(*}\textsc{pst2} fufu \hspaceThis{(}\textsc{comp} Nfor  \textsc{3sg} \textsc{pst1} eat \\ 
\glt `Nfor ate \textsc{\MakeLowercase{FUFU}}.'
\z
\ea \label{ex:becker:s3}
\gll à m\=u b\=a b\=a\=a Nf\`{o} \`{a} m\=u zh\=e \\  
     \textsc{expl} \textsc{pst2} \textsc{cop} fufu Nfor  \textsc{3sg} \textsc{pst2} eat \\ 
\glt `It was a fufu that Nfor ate.'
\z
Our second and third argument concern the cleft clause. \textit{Extraposition} \citep{Akmajian1970,Gundel1977,Percus1997} as well as \textit{predicative approaches} \citep{Svenious1998,Hedberg2000,Reeve2011} uncontroversially take cleft clauses to be embedded relative clauses. In Limbum, there is ample reason to doubt the existence of a relative clause in an {\em \'a} construction. While the complementizer \textit{c\'i} is optionally spelled out following the focused constituent, it cannot, however, act as a relative pronoun. 
\ea \label{ex:becker:s4}
\gll m\=u  {zh\sout{\v{\i}}} / *c\'i \'i m\=u zh\'e\'e mŋg\`ɔmb\'e \\  
     child \textsc{rel} / \textsc{comp} \textsc{3sg} \textsc{pst2} eat plantains \\ 
\glt `the child who ate plantains'
\z
Furthermore, relative clauses can optionally co-occur with the right-headed demonstrative marker \textit{n\`a} \citep{Fransen1995,Mpoche1993}, shown in (\ref{ex:becker:s5}). Crucially, the demonstrative is prohibited in the \textit{\'a} strategy, see (\ref{ex:becker:s6}).
\ea \label{ex:becker:s5}
\gll m\=u zh\sout{\v{\i}} \'i m\=u zh\'e\'e mŋg\`ɔmb\'e ({n\`a}) \\  
     child \textsc{rel} \textsc{3sg} \textsc{pst2} eat plantains \hspaceThis{(}\textsc{dem} \\ 
\glt `the child who ate plantains'
\z
\ea \label{ex:becker:s6}
\gll \'{a} {ŋkf\'ʉ\'ʉ} (c\'{i}) m\`ɛ b\'{i} k\=ɔn\={\i} (*{n\`a}) \\  
     \textsc{foc} {chief} \hspaceThis{(}\textsc{comp} \textsc{1sg} \textsc{fut1} meet \hspaceThis{(*}\textsc{dem} \\ 
\glt `I will meet the \textsc{\MakeLowercase{CHIEF}}.'
\z
To sum up, a biclausal cleft structure requires a copula and and a relative clause, neither of which seems to be present in the {\em \'a} construction.

\subsection{Focus movement analysis} \label{sec:becker:subsec:analysis}
In line with what has been argued for question particles in Japanese \citep{Hagstrom1998}, Sinhala \citep{Kishimoto2005}, and Tlingit \citep{Cable2010} on the one hand and focus fronting in Hungarian \citep{Horvath2007,Horvath2010,Horvath2013} on the other, we propose that the focus particle {\em \'a} merges with a constituent that is focused (or at least contains a constituent that is focused). The particle heads its own projection FP and bears an {\small $\bullet F \bullet$} feature. This feature projects up to FP enabling the contained constituent to be focused. A higher functional head, optionally spelled out as {\em c\'i}, probes for the feature, finds it on FP and, as a consequence, attracts FP (and everything contained in it) to its specifier, see \figref{fig:syntax:f1}.\footnote{The exact nature of feature {\em F} and {\em FocP} and how they differ from focus on the contained constituents that needs to be interpreted is not entirely worked out in this paper. Based on the claim in footnote \ref{footnote:becker:fn1}, it is possible to reanalyze {\em F} and {\em FocP} as triggers for movement that have a semantic impact, in the spirit of \citet{Horvath2007,Horvath2013}. This analyzes will have consequences for the information focus status of the {\em \'a} strategy and its relation to {contrastive focus}, both of which are explored in \citet{Driemeletal2018, DriemelNformi2018}.}

The alternative proposal in which {\em \'a} itself spells out the focus head and attracts the focused constituent to its specifier, sketched in (\ref{ex:becker:s13}), can be refuted based on the linear order of the structures: {\em \'a} would be predicted to follow the focused constituent, contrary to fact. An ad-hoc movement step of {\em \'a} to a higher (possibly) C or Force head is ruled out based on the behaviour of focused constituents in embedded clauses.
\ea
\ea[]{ \label{ex:becker:s12}
\gll \'i b\=a l\'a n\`ɛ \'a rk\'ar f\=ɔ (c\'i) {nd\=u} {zh\`i} \`a \`m y\'u \\  
     \textsc{3sg} \textsc{pst1} say \textsc{comp} \textsc{foc} car \textsc{det} \hspaceThis{(}\textsc{comp} {husband} her \textsc{3sg} \textsc{pst3} buy  \\ 
\glt `She said that her husband bought the \textsc{\MakeLowercase{CAR}}.'}
\ex[*]{...[\textsubscript{VP} [\textsubscript{V} l\'a][\textsubscript{CP} [\textsubscript{C} n\tikzmark{s13f}\`ε] [\textsubscript{FocP} rk\'ar f\=ɔ [\textsubscript{Foc} \'a\tikzmark{s13t}] [\textsubscript{FinP} [\textsubscript{Fin} c\'i ]]]]] \label{ex:becker:s13}
	\DrawXArrow[dashed]{{pic cs:s13f}}{{pic cs:s13t}}}
\z \z
\vspace{3mm}
\noindent The complementizer {\em n\`ɛ} would block movement of {\em \'a} to C, nevertheless {\em \'a} precedes the focused constituent. Hence, we assume the left periphery of the embedded clause in (\ref{ex:becker:s12}) to be composed as shown in \figref{fig:syntax:f1}.
\begin{figure}
\begin{tikzpicture}[baseline,scale=0.8,sibling distance=30pt]
\tikzset{every tree node/.style={align=center,anchor=north}}
\Tree  [.{...} [.{V\\l\'a} ][.CP [.C\\{n\`ε} ][.\node{FocP}; [.\node(j2){FP\textsubscript{F}}; ] [.\node(j1){Foc\1}; [.\node{Foc$_{\small{ \bullet F \bullet}}$\\(c\'i)}; ] [.\node(tp){TP}; [.... ] [.... [.\node(fp){FP\textsubscript{F}}; [.\node{F\textsubscript{F}\\\'a}; ] [.\node(tp){XP}; \edge[roof]; {...} ] ] [.... ] ] ] ]] ]]
\draw[<-] (j2) ..controls +(south:3) and +(west:0.5)..  (fp.west);
			\end{tikzpicture}
\caption{Focus movement of FP}
\label{fig:syntax:f1}
\end{figure}
Support for the FP analysis comes from the fact that {\em c\'i} can only occur in clauses realizing the {\em \'a} strategy. Thus, {\em c\'i} seems to be tied to the presence of {\em \'a} focus. Under the account, presented in (\ref{ex:becker:s13}), this obligatory co-occurrence would be a coincidence. Limbum, therefore, is strikingly different from Japanese, Sinhala, Tlingit, and Hungarian in that it provides overt evidence for both the locally merged particle as well as the higher functional head which causes overt movement. The functional head must be different from C, since an additional complementizer can co-occur with {\em c\'i}, as (\ref{ex:becker:s12}) shows. Moreover, {\em c\'i} can never act as a complementizer on its own, it is dependent on the occurrence of {\em \'a}.

Limbum patterns with Tlingit, in that the particle takes the focused phrase as a complement rather than adjoining to it. FP as a projection of F bears the F-feature probed for by the Foc head. Since FP properly contains the focused phrase, the entire FP is expected to move to spec,FP, including possibly non-focused material. In other words, focus movement is predicted to pied-pipe. (\ref{ex:becker:s14}) shows the inability of possessors to be extracted by themselves, they obligatorily have to pied-pipe the possessum.
\pagebreak
\begin{exe}
\ex \textit{Context:} A heard B telling someone on the phone that B would pick up someone's brother from the bus station. A couldn't properly understand whose brother B will pick up.\label{ex:becker:s14}
\begin{xlist}
\exi{A:}[]{
\gll \'a nd\'ur nd\=a (c\'i) \`a b\'i l\`ɔr\={\i} \\  
     \textsc{foc} brother who \hspaceThis{(}\textsc{comp} \textsc{2sg} \textsc{fut1} pick.up  \\ 
\glt `Whose brother will you pick up?'}
\exi{B:}[]{
\gll \'a {nd\'ur} T\'ank\'o  (c\'i) m\`ɛ b\'i l\`ɔr\={\i} \\  
     \textsc{foc} {brother} Tanko \hspaceThis{(}\textsc{comp} \textsc{1sg} \textsc{fut1} pick.up  \\ 
\glt `I will pick up \textsc{\MakeLowercase{TANKO}}'s brother.'}
\exi{B$'$:}[*]{
\gll \'a {Tánkó} (c\'i) m\`ɛ b\'i l\`ɔr\={\i} {nd\'ur} \\  
     \textsc{foc}  {Tanko} \hspaceThis{(}\textsc{comp} \textsc{1sg} \textsc{fut1} pick.up {brother}  \\ 
\glt `I will pick up \textsc{\MakeLowercase{TANKO}}'s brother.'}
\end{xlist}
\end{exe}
An alternative account like the one shown in (\ref{ex:becker:s13}) cannot predict pied-piping without assuming further constraints on movement. Whichever phrase is focused, and thus bears an F feature, would be predicted to move to spec,FocP, see (\ref{ex:becker:s15}) for an illustration.
\ea[*]{[... \'a\sub{1} [\textsubscript{FocP} [\textsubscript{DP} Tán\tikzmark{s15f}kó\textsubscript{F}]\sub{2} [\textsubscript{Foc} t\sub{1}] [\textsubscript{TP} ... [\textsubscript{DP} [\textsubscript{D$'$} [\textsubscript{NP} nd\'ur] $\varnothing_D$] t\tikzmark{s15t}\sub{2}]...]]
	\DrawArrowok{{pic cs:s15f}}{{pic cs:s15t}}{\ding{52}} \label{ex:becker:s15}}
\z\vspace{6mm}
In the current analysis the FP is the closest goal the Foc head sees. It is therefore the entire FP that gets attracted to the specifier of FocP, making it impossible for a focused phrase contained in an FP to move to spec,FocP on its own, see (\ref{ex:becker:s16}).
\ea \relax[\textsubscript{FocP} \tikzmark{s16f2}\hspace{2mm}\tikzmark{s16f} [\textsubscript{Foc} c\'i] [\textsubscript{TP} ... [\textsubscript{F\tikzmark{s16t}P} \'a [\textsubscript{DP} [\textsubscript{D$'$} [\textsubscript{NP} nd\'ur] $\varnothing_D$] Tán\tikzmark{s16t2}kó\textsubscript{F}]]...]]
	\DrawArrow{{pic cs:s16f}}{{pic cs:s16t}}
    \DrawLXArrow[dashed]{{pic cs:s16f2}}{{pic cs:s16t2}} \label{ex:becker:s16}
\z
\vspace{6mm}
Extractions of the type shown in (\ref{ex:becker:s15}) can potentially be ruled out by general constraints on movement since they seem to be marked cross-linguistically \citep{Corver1990a,Zeljko2005}. We would like to point out, however, that possessor extraction is not banned per se, since it is allowed in topic configurations, shown in (\ref{ex:becker:s17}), albeit with a resumptive pronoun. A base-generation approach seems implausible since topicalization is less acceptable out of islands, shown e.g.\ in (\ref{ex:becker:s17b}) for a complex noun phrase.
\ea
\ea[] {
\gll \`a mb\`o Tanko, m\`ε{} m\=u y\=ε{} nf\sout{\=\i} zh\sout{\`i} \\  
     as for Tanko \textsc{1sg} \textsc{pst2} see brother his \\ 
\glt `As for Tanko, I met his brother.' \label{ex:becker:s17}}
\ex[?]  {
\gll \`a mb\`o Tanko, m\`ε{} r\`iŋ ŋwe zh\sout{\v i} m\=u k\'ɔn\'i nf\sout{\=\i} zh\sout{\`i} \\  
as for Tanko \textsc{1sg} know man \textsc{rel} \textsc{pst2} meet brother his \\ 
\glt `As for Tanko, I know a man who met his brother.' \label{ex:becker:s17b}}
\z \z
Since the possessor can, in principle, move out of the DP it is contained in, we conclude that it must be the focus particle {\em \'a} merged with the entire DP that prevents the possessor from moving to spec,FocP alone.

Another environment in which we can observe the pied-piping property of focus movement concerns prepositional phrases, shown in (\ref{ex:becker:s18}). Prepositions cannot be stranded if the phrase they merge with is narrowly focused.
\begin{exe}
\ex \textit{Context:} A heard B telling someone on the phone that B shot an animal with something but it is not clear to A with what.\label{ex:becker:s18}
\begin{xlist}
\exi{A:}[]{
\gll \'a n\sout{\`i} k\=ε{} (c\'i) w\`ε{} m\=u t\=a ny\`a \`a? \\  
   \textsc{foc} with what \hspaceThis{(}\textsc{comp} \textsc{2sg} \textsc{pst2} shoot animal \textsc{q}    \\ 
\glt `With what did you shoot the animal?'}
\exi{B:}[]{
\gll \'a n\sout{\`i} ŋg\=ar (c\'i) m\`ε{} m\=u t\=a ny\`a \\  
     \textsc{foc} with gun \hspaceThis{(}\textsc{comp} \textsc{1sg} \textsc{pst2} shoot animal  \\ 
\glt `I shot the animal with a \textsc{\MakeLowercase{GUN}}.'}
\exi{B$'$:}[*]{
\gll \'a ŋg\=ar (c\'i) m\`ε{} m\=u t\=a ny\`a n\sout{\`i} \\  
     \textsc{foc} gun \hspaceThis{(}\textsc{comp} \textsc{1sg} \textsc{pst2} shoot animal with \\ 
\glt `I shot the animal with a \textsc{\MakeLowercase{GUN}}.'}
\end{xlist}
\end{exe}
Similar to the possessor case, the alternative account in which the focus particle {\em \'a} spells out the FOC head would predict the complement of P to be attractable to spec,FocP, in case it is the constituent that carries the F feature.
\ea[*]{[... \'a\sub{1} [\textsubscript{FocP} [\textsubscript{DP} ŋg\tikzmark{s19f}\=ar\textsubscript{F}]\sub{2} [\textsubscript{Foc} t\sub{1}] [\textsubscript{TP} ... [\textsubscript{PP} [\textsubscript{P} n\sout{\`{\i}}] t\tikzmark{s19t}\sub{2}]...]]
	\DrawArrowok{{pic cs:s19f}}{{pic cs:s19t}}{\ding{52}} \label{ex:becker:s19}}
\z
\vspace{3mm}
In contrast, the FP analysis predicts FP to be the goal that checks the F feature on the FOC head. Hence, the entire PP has to move to spec, FocP.
\ea \relax[\textsubscript{FocP} \tikzmark{s20f2}\hspace{2mm}\tikzmark{s20f} [\textsubscript{Foc} c\'i] [\textsubscript{TP} ... [\textsubscript{F\tikzmark{s20t}P} \'a [\textsubscript{PP} [\textsubscript{P} n\sout{\`{\i}}] ŋg\tikzmark{s20t2}\=ar\textsubscript{F}]...]]
	\DrawArrow{{pic cs:s20f}}{{pic cs:s20t}}
    \DrawLXArrow[dashed]{{pic cs:s20f2}}{{pic cs:s20t2}} \label{ex:becker:s20}
\z
\vspace{6mm}
Again, conditions on preposition stranding can be independently motivated, since this kind of movement seems to be banned in a number of languages \citep{Abels2003,Heck2008}. The FP analysis, however, offers an explanation for the lack of preposition stranding and possessor extraction simultaneously.

At this point, it is important to answer the question why the focus particle cannot merge directly with the narrowly focused constituent in (\ref{ex:becker:s16}) and (\ref{ex:becker:s20}). Here we follow \citet{Cable2010} by adopting the {\em QP-Intervention Condition} reformulated for FPs.
\ea {\em FP-Intervention Condition:} (adapted from \citealt[57]{Cable2010})\label{s21}\\
An FP cannot intervene between a functional head $\alpha$ and a phrase selected by $\alpha$. (Such an intervening FP blocks the selectional relation between $\alpha$ and the lower phrase.)
\z
By assumption, functional heads {\em c-select} for their arguments, while lexical heads {\em s-select} for their arguments \citep[][62]{Cable2010}. An FP can intervene between a lexical head and the phrase selected by that head because the F particle does not change the semantic type of the phrase it merges with. An FP cannot, however, intervene between a functional head and the phrase it selects for since the F particle indeed changes the category of the phrase it merges with. Hence, {\em \'a} cannot merge with the embedded XP of a prepositional phrase because it would intervene between the functional head P and XP. Neither can {\em \'a} directly merge with a possessor because the functional element D c-selects its possessor and {\em \'a} would act as an intervener.

Further support for (\ref{s21}) comes from VP-fronting, here analyzed as remnant $v$P-fronting. If {\em \'a} were to take $v$P as its complement, the particle would intervene between $v$P and the higher functional head T. As a consequence, VPs cannot (per se)\footnote{There is, however, a way to repair the structure using do-support:
\ea
\gll \'a {r-y\=u} {ms\=aŋ} (c\'i) nj\'iŋw\`ɛ f\=ɔ b\'i {g\={\i}} \\ \textsc{foc} {{5}-buy} {rice} \hspaceThis{(}\textsc{comp} woman \textsc{det} \textsc{fut1} {do} \\ 
\glt `The woman will \textsc{\MakeLowercase{BUY}} \textsc{\MakeLowercase{RICE}}.'
\z
At this point, it is unclear to us why do-support is able to save the construction. %Note that it is not enough to substitute the lower VP copy with {\em b\'i}, the noun class prefix has to be present as well.
} be focused with the {\em \'a} strategy.
\begin{exe}
\ex \label{ex:becker:s22}
	\begin{xlist}
		\exi{A:}[]{ 
			\gll \'a k\=ε{} (c\'i) nj\'iŋw\`ɛ f\=ɔ b\'i \`a \\ 
			\textsc{foc} what \hspaceThis{(}\textsc{comp} woman \textsc{det} \textsc{fut1} \textsc{q}\\ 
			\glt `What will the woman do?'}
		\exi{B:}[*]{ 
			\gll \'a {(r-)y\=u} {ms\=aŋ} (c\'i) nj\'iŋw\`ɛ f\=ɔ b\'i \\ 
			\textsc{foc} \hspaceThis{(}5-buy {rice} \hspaceThis{(}\textsc{comp} woman \textsc{det} \textsc{fut1}\\ 
			\glt `The woman will \textsc{\MakeLowercase{BUY}} \textsc{\MakeLowercase{RICE}}.'}
	\end{xlist}
\end{exe}
While the ban on P-stranding and possessor extraction might be reducible to the interplay of the {\em \textsc{\MakeLowercase{PIC}}} \citep{Chomsky2000} and {\em Anti-locality} \citep{Abels2003,Erlewine2016}, this crucially does not apply to the lack of VP fronting because TPs are uncontroversially denied phasehood status. The impossibility to front a VP in (\ref{ex:becker:s22}), thus, requires an independent explanation. In contrast, the FP analysis can capture all three properties of the {\em \'a} strategy.

Finally, a note on verb focus is in order. As (\ref{ex:becker:s10}) and (\ref{ex:becker:s11}) show, verb focus requires doubling on the one hand and a noun class marker prefixing the higher copy on the other hand. The latter suggests that the focus particle {\em \'a} c-selects for nominal phrases, so that verbs have to be nominalized in order to be merged with {\em \'a}. The behaviour of the focus particle is not unusual for Limbum since coordinators seem to make the same kind of distinction. As (\ref{ex:becker:s23}) shows, the choice of coordinator correlates with the categories of the conjuncts.\footnote{Limbum shows a great deal of homophony (compare also the use of {\em c\'i} as a sentence coordinator in (\ref{ex:becker:burgers}) vs.\ the general use of {\em c\'i} with respect to focused {\em \'a} phrases), which could account for the fact that {\em \'a} and {\em b\'a} can act as coordinators as well as focus particles. Alternatively, coordinators and focus particles could also be related diachronically. This issue must be left open for now.}
\ea \label{ex:becker:s23}
\ea
\gll Shey \`a m\=u r\'o Njobe \underline{b\'a} Shey \\ 
Shey \textsc{3sg} \textsc{pst2} search Njobe and Shey \\ 
\glt `Shey searched for Njobe and Shey.'
\ex
\gll Shey \`a m\=u r\'o Njobe m\`a nt\=a\=a \underline{b\'a} k\`o l\`aʔ \\ 
Shey \textsc{3sg} \textsc{pst2} search Njobe at market and at home \\ 
\glt `Shey searched for Njobe at the market and at home.'
\ex
\gll Shey \`a m\=u c\=aŋ \underline{\'a} gw\`e \\ 
Shey \textsc{3sg} \textsc{pst2} run and fall \\ 
\glt `Shey ran and fell.'
\z \z 
Since categorical sensitivity shows up elsewhere in the language, we tentatively conclude that the noun class prefix in verb focus constructions is due to a selectional restriction of {\em \'a}. Attaching a noun class prefix to one of the copies could potentially serve as a reason for multiple spell out, i.e.\ doubling. A detailed analysis, however, is still missing and left for future research.

\section{Summary and future work}

In this paper, we have shown that the two focus strategies in Limbum, involving two different markers, also clearly differ in their functions: the marker \textit{á} is linked to information focus (i.e. focus with no further semantic/pragmatic conditions), while \textit{bá} occurs in contexts that involve contrast and exhaustivity. The interpretation effects that the {\em \'a} strategy triggers are compatible with the syntactic analysis: the lack of tense marking on copulas, the behaviour of the complementizer \textit{cí}, and the ban on right peripheral demonstrative markers provide evidence against an underlying cleft structure. The current proposal, therefore, models {\em\'a} focus as focus movement, where the focus particle is directly merged with the focused phrase and attracted to the left periphery by a higher functional head, pied-piping the focused constituent. While this type of analysis has been proposed for other languages \citep{Hagstrom1998,Cable2010}, albeit for questions, Limbum crucially provides morphological evidence for the existence of a particle ({\em \'a}) as well as the higher functional head ({\em c\'i}).

Even though a cleft analysis is ruled out, the Limbum patterns, shown in this paper, nevertheless present a so-far unattested opposition of focus strategies: information focus, being less marked semantically, is expressed by a complex strategy consisting of a particle and fronting, whereas contrastive/exhaustive focus, although imposing additional semantic restrictions on the focus alternatives, is realized by a particle only. The reasons why Limbum shows the reverse picture in terms of structural markedness and complexity of interpretation need to be explored further in future work.\footnote{Although see \citet{Driemeletal2018} for a possible explanation.}

One last point concerns the syntax of {\em b\'a}. In contrast to the {\em \'a} strategy, the {\em b\'a} construction does not seem to provide overt evidence for the existence of a higher functional head. The behaviour of focused subjects, however, indicates certain positional restrictions a focused phrase has to obey. Future work will explore whether the FP analysis can be extended to the {\em b\'a} strategy.

\section*{Acknowledgements}
We would like to thank the audience at ACAL 48, Indiana University, and two anonymous reviewers for helpful comments.


\section*{Abbreviations}
\begin{multicols}{2}
\begin{tabbing}
1-,2-,5-,\hspace{1cm} \= 1st, 2nd, 3rd person\kill
1,2,3          \> 1st, 2nd, 3rd person\\
1-,2-,5-,      \> Noun classes        \\
\textsc{comp}  \> Complementizer      \\
\textsc{cop}   \> Copula              \\
\textsc{det}   \> Determiner          \\
\textsc{dem}   \> Demonstrative       \\
\textsc{expl}  \> Expletive           \\
\textsc{foc}   \> Focus marker        \\
\textsc{fut1}  \> Near future tense   \\
\textsc{hab}   \> Habitual            \\
\textsc{incl}  \> Inclusive           \\
\textsc{perf}  \> Perfective          \\
\textsc{pl}  \> Plural   \\ 
\textsc{prep}  \> Preposition     \\
\textsc{prv}  \> Preverb   \\
\textsc{pst1}  \> Recent past tense \\
\textsc{pst2}  \> Distant past tense  \\
\textsc{pst3}  \> Remote past tense  \\
\textsc{rel}  \> Relative pronoun\\
\textsc{sg}  \> Singular  \\
\char"02CA \> High tone \\
\char"02CB \> Low tone \\
\hspace{1ex}\char"0304\> Mid tone
\end{tabbing}
\end{multicols}

\printbibliography[heading=subbibliography,notkeyword=this]

\end{document}

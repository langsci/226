\documentclass[output=paper,modfonts]{langsci/langscibook} 
\title{A closer look at the Akan determiner bi: An epistemic indefinite analysis} 

\author{ Augustina Pokua Owusu\affiliation{Rutgers University}}

\abstract{This study aims to shed light on the epistemic indefinite interpretation (EI) of the Akan (Asante Twi) determiner \emph{bi} which hitherto had not been discussed in the Akan literature. In previous studies, Amfo (2009) and Arkoh (2011) review its referential or specific indefinite interpretation. The current study shows that in addition to the above interpretation when \emph{bi} is used, the speaker signals that she does not have access to all the information about who or what satisfies the existential claim they are making. I employ  Aloni (2001) and Aloni and Port's (2015) theory of conceptual covers and methods of identification to determine ``knowledge" of a referent in a particular context. Conceptual covers are sets of individual concepts which exclusively and exhaustively covers the domain of individuals, (Aloni, 2001). I show that the epistemic indefinite analysis and the specificity or referential analysis are compatible. When \emph{bi} is used, the speaker asserts she can name a noteworthy or identifying property about the referent of the NP, which is the referential or specificity interpretation. Additionally, she presupposes that she is ignorant about further characterizing information about the referent, the epistemic indefinite interpretation.}

\begin{document}
\maketitle

\section{Introduction} 
The study of epistemic indefinites has become popular as the interest in non-verbal modality rises (see Aloni \& Port 2015, Alonso-Ovalle \& Menendez-Benito 2003 a.o for studies on epistemic indefinites). Epistemic determiners or pronouns, as their name suggests, signal the epistemic state of a speaker, i.e., whether the speaker knows the referent of an NP or not. The study of epistemic indefinites is usually divorced from the study of specific indefinites or the wide scope interpretation of the indefinite for two main reasons. First, in most of the well-studied languages, the specific indefinite determiner and the epistemic indefinite determiner are expressed by different morphemes. For instance, in English, specificity is encoded by either a wide scope reading of the determiner \emph{a} or the morpheme \emph{certain}, while epistemic indefiniteness is encoded by the quantifiers \emph{some} or \emph{some or other}. In German, specificity is encoded by determiners \emph{bestimmt} and \emph{gewiss} while the epistemic indefinite determiner is \emph{irgendein}. Secondly, these markers appear to express opposing concepts. Specificity requires that the speaker have a particular referent in mind; these markers are often used referentially, i.e., there is a particular referent that can be identified by the speaker or some other salient individual. Epistemic indefinites, on the other hand, require that the speaker be ``ignorant'' about the referent of the indefinite. These indefinites are, however, not mutually exclusive. For instance, the German epistemic indefinite \emph{irgedein}, and specific indefinite \emph{bestimmt}, are felicitous in the same sentence (Aloni and Port 2015).

This paper aims to show that these two types of indefinites can be encoded in the same lexical item in a language. The Akan determiner \emph{bi}, which has been argued to have a specific indefinite use (Amfo 2009, Arkoh 2011) also has an epistemic indefinite interpretation. The specificity interpretation requires that the speaker have a mental representation of a particular individual which he can characterize by a noteworthy or identifying property. The epistemic indefinite interpretation is a presupposition of ignorance about further characterizing information about the referent. These pieces of missing information -- what the speaker does not know -- are the critical properties that determine whether the speaker ``knows" the referent. Context determines what a speaker has to know about a referent to claim he ``knows" the referent.

\ea 
\langinfo{Akan}{}{Arkoh 2011:37}\\
\ea\label{ex1}

\gll Maame Ama yε-ε edziban \textbf{bi}. \\
     woman Ama do-\textsc{past} food  \textsc{ind}   \\
\glt `Madam Ama cooked (some specific) food.'

\ex\label{ex2}
\langinfo{Akan}{}{personal knowledge}\\
\gll  Maame Ama yε-ε aduane \textbf{bi} nanso me-n- nim aduane kro. \\
     woman Ama do-\textsc{past} food  \textsc{ind} but \textsc{1sg.}-\textsc{neg}- know food one \\ 
     \glt `Madam Ama cooked (some specific) food, but I don't know what food it is.'
     
\z \z 

The food in (\ref{ex1}) is specific; i.e., there is a particular food being cooked by Maame Ama that the speaker has in mind. \emph{Bi} is felicitous since the fundamental way to identify food is by its name, and the speaker does not know this. In other words, the ignorance presupposition is satisfied. The fact that the speaker does not know the name is stated explicitly in (\ref{ex2}). Adopting Aloni and Port (2015) methods of identification and conceptual covers, I account for what counts as ``knowing" in a context.

The paper is structured as follows: In \sectref{sec:owusu:2}, I review the specific indefinite analysis of \emph{bi}, paying attention to its scope taking properties and its felicity conditions. In \sectref{sec:owusu:3}, I discuss epistemic indefinites, their functions and the methods of identifications.  \sectref{sec:owusus:4} is the analysis and \sectref{sec:owusu:5}, the conclusion.  


\section{The specificity analysis}
 
The specificity or referential interpretation of \emph{bi} has been discussed by Amfo (2009) and Arkoh (2011). Amfo (2009) argues that \emph{bi} is an existential quantifier that has the cognitive status of referential. The determiner \emph{bi} is used to introduce new referents into the discourse. In the spirit of Heim's (1983) File Change semantics metaphor, when \emph{bi} is used a new file for the NP is created. An addressee constructs a referent for the NP by identifying properties that exemplify the head noun in question. Arkoh (2011), on the other hand, argues that when \emph{bi} is used as a determiner, it is interpreted as referential along the lines of Fodor and Sag (1982) and Kratzer (1998). She argues against Amfo's (2006) claim that it is an existential quantifier. She claims that \emph{bi} does not have a quantificational interpretation. Amfo (2009) and Arkoh (2011) make different predictions about the scope of \emph{bi}. A referential determiner always has wide scope while a specific existential indefinite can get an intermediate scope. For Arkoh \cite{Arkoh2011}, one piece of evidence against the existential quantifier analysis is the fact that \emph{bi}+NP can make a discourse referent that can be referred to as in (\ref{02}). The bare NP, which is interpreted as an existential quantifier does not have this property, as a result (\ref{06}) is infelicitous.    
\ea
\langinfo{Akan}{}{Arkoh 2011:35}\\
\ea\label{01} 
\gll Kwame  hwe-e  abɔfra (tuntum) bi.\\
 Kwame cane-\textsc{past} child black \textsc{ref}\\
 
\glt ‘Kwame caned a certain (dark) child.'

\ex\label{02}
\gll ɔ-yε bubuafɔ.\\
3\textsc{sg.subj}-be cripple\\
 
\glt ‘S/he is a cripple’
\z \z

\ea
\langinfo{Akan}{}{Arkoh 2011:35}\\
\ea\label{05} 
\gll Kwame  hwe-e  abɔfra.\\
 Kwame cane-\textsc{past} child \\
 
\glt ‘Kwame caned a child.'

\ex\label{06}
\gll \# ɔ-yε bubua-fɔ.\\
{} 3\textsc{sg.subj}-be cripple-\textsc{nom}\\
 
\glt ‘S/he is a cripple’.'
\z \z 

The English indefinite determiner \emph{a} is ambiguous between quantificational and referential readings as shown below.

\ea
\langinfo{English}{}{Abusch and Rooth (1997)}\\
\ea\label{ex3}

John overheard the rumor that a student of mine had been called before the dean
\ea \label{ex4} John overheard the rumor that a particular student of mine namely Bill has been called before the dean.
\ex \label{ex5} John overheard the rumor that some student of mine has been called before the dean.
\z\z 
\z 

In (\ref{ex4}), the indefinite is interpreted as referring to a specific student, one that the John is aware of. In (\ref{ex5}), on the other hand, John does not appear to know the referent of the indefinite. In English, the referential and quantificational interpretation results from a wide scope or narrow scope reading of the indefinite. 
The aim of this section is not to choose between these two analyses, but to show that a referential or specificity interpretation of \emph{bi} has been explored. What is essential for this study is that both analyses agree that when a speaker uses  N + \emph{bi}, he intends to refer to a particular referent which he has in mind.            


Even as a referential determiner, \emph{bi} does not have the same interpretation or distribution as the referential definite determiner. Heim (1983) argues that specific indefinite, like other indefinites, can introduce a new file. A definite, on the other hand, can only be used when updating an existing file; you cannot introduce a new discourse referent with a definite marker. You can begin narratives with a specific indefinite, but not a definite determiner. Another difference is that a definite expression is used when a speaker presupposes that the referent of the expression is also accessible to the hearer. That is, the speaker assumes that there is a unique referent that the hearer can identify. Either (i) because the referent was previously mentioned in the context of discourse, or (ii) because the referent is part of the interlocutors’ shared knowledge, or (iii) because there is enough descriptive content in the sentence to identify the referent.  The referent becomes identifiable as the sentence is processed (Comrie 1989:135; Giv\'on 2001:450; Gundel, Hedberg, and Zacharski 1993:277; Hawkins 1978:167-68; Payne 1997:263). For specific indefinites, specificity lies in the fact that the speaker has a particular referent in mind. The addressee is just expected to be able to form a representation of this referent provided there are enough clues in the utterance itself and an accessible context. The speaker does not assume that the listener/addressee knows the referent of the NP. The difference between \emph{bi} and the definite determiner \emph{no} is that of assumed addressee ignorance. When a referent is familiar to both addressee and speaker, the definite determiner is used. This is the difference between (\ref{ex8}) and (\ref{ex9}). 

\ea \langinfo{Akan}{}{Amfo 2009}\\
\ea\label{ex8}
\gll  Kwame dze edziban no maa Ama.\\
   Kwame take food Fam give-\textsc{past} Ama \\ 
\glt `Kwame gave the food to Ama.'\\

\ex\label{ex9} 
\gll Kwame dze edziban bi maa Ama.\\
   Kwame take food \textsc{ind} give-\textsc{past} Ama \\ 
\glt `Kwame gave a certain food to Ama.'\\
\z \z 

In (\ref{ex8}), the food that Kwame gave to Ama is discourse old; the referent is familiar to both the speaker and the addressee. This familiarity may be due to one of the reasons mentioned above that makes a referent of NP accessible to an addressee. The referent of food in (\ref{ex9}) is, however, discourse new and only familiar to the speaker.
	 
\subsection{Scoping-Taking properties of \emph{bi}}
One of the unique and uncontroversial characteristics of specific indefinites is their ability to escape scope islands. Scope islands are syntactic configurations which disallow wide scope for most quantifiers; these include relative clauses and antecedents of conditionals (see Fodor and Sag 1982, and much subsequent literature). In addition to taking wide scope, indefinites have been observed to take an intermediate scope, outside of the scope island but underneath a higher quantifier. In this section, I explore the scope properties of \emph{bi} with intensional predicates, in the context of negation, with other nominal quantifiers, and in the antecedent of a conditional. We begin with intensional predicates.

   When \emph{bi} is embedded under intensional predicates, it always receives a wide scope reading. It scopes over the intensional predicate. In this way \emph{bi} is similar to the German specificity markers \emph{bestimmt} and \emph{gewiss}, (Ebert et al 2012).
   
 \ea 
 \langinfo{Akan}{}{Personal knowledge}\\
 \ea \label{006}
\gll Kofi re- hwεhwε CD \textbf{bi}.\\
    Kofi -\textsc{prog}-  search CD \textsc{ind}\\
\glt `Kofi is looking for a certain CD.'\\
$\exists$y [CD(y) $\wedge$ search(K,y)]]
\ex\label{006b}
\gll Kofi re- hwεhwε CD.\\
 Kofi -\textsc{prog}-  search CD  \\
\glt `Kofi is looking for a CD.'
\z \z 

In (\ref{006}), Kofi is not going to be happy when he finds just any CD; he will only be happy if he finds his \emph{Thriller} CD. There is no such restriction in (\ref{006b}), finding any CD will make Kofi happy.

\emph{Bi} is infelicitous in a negative context. Neither a wide scope nor narrow scope interpretation is available in (\ref{ex12}). This sentence can only be saved when the indefinite determiner is replaced by the NPI \emph{biara} 'any' as in (\ref{ex13}). \footnote{the NPI is derived from a combination of the indefinite determiner \emph{bi} and the emphatic particle \emph{ara}. \emph{biara} like \emph{any} has a free-choice interpretation that is licensed in positive sentences.

\ea
\langinfo{Akan}{}{Personal knowledge}\\
\label{ex120}
\gll  Kofi  bε- gye CD biara.\\
Kofi \textsc{fut}-  take CD any \\
\glt `Kofi will take any CD.'
\z 
}  
\ea
\langinfo{Akan}{}{Personal knowledge}\\

  \ea[*]{ \label{ex12}
\gll  Kofi n- hwεhwε CD \textbf{bi}. \\
    Kofi \textsc{neg}-  search CD \textsc{ind} \\}

\ex\label{ex13}
\gll  Kofi n- hwεhwε CD biara.\\
Kofi \textsc{neg}-  search CD any \\
\glt `Kofi is not looking for any CD.'
\z \z 

Some specific indefinites like the German \emph{bestimmt} can in principle scope under negation. When it is licensed under negation, both the wide scope and the narrow scope interpretations are technically possible, though speakers disprefer the wide scope interpretation. \emph{Gewiss}, on the other hand, is not licensed under negation.

The indefinite determiner also interacts with other nominal quantifiers like the universal quantifier. 
\ea 
\langinfo{Akan}{}{Personal knowledge}\\
  \ea\label{ex100}
\gll Obiara hyia -a presidential candidate  \textbf{bi}. \\
    Everyone meet -\textsc{pst} presidential candidate \textsc{ind} \\
\glt `Everyone met a presidential candidate.'

\ex\label{ex15}
\gll  Yε- kɔ -e no obiara tɔ -ɔ  nwoma \textbf{bi}.\\
3{\pl}- go -\textsc{pst}  CFM everyone buy -\textsc{pst} book \textsc{ind} \\
\glt `When we went, everyone bought a certain book.'
\z \z
 
 \ea
 \ea
$\exists$y$\forall$x[presidential candidate(x)[$\rightarrow$]met(y,x)]
\ex $\forall$x$\exists$y[presidential candidate(x)[$\rightarrow$]met(y,x)] 
\z\z

(\ref{ex100}) is ambiguous between a wide scope and a narrow scope reading, just like \emph{certain} in English  (see Farkas 2002, ex. 54). The wide scope interpretation is that there is a particular presidential candidate, for instance, Hillary Clinton, such that everyone met her. The narrow scope reading expresses that for everyone there is a unique presidential candidate that they met, Kofi met Trump, Ama met Hillary, and Kwame met Bernie. In this way, \emph{bi} is similar to the German specificity marker \emph{bestimmt}, which is also ambiguous between a wide and narrow scope interpretations with nominal quantifiers. It, however, differs from the other specificity marker \emph{gewiss} which only has a wide scope reading with nominal quantifiers.               
  
  Indefinites scope outside of conditionals despite the fact that conditionals constitute scope islands for other quantifiers (cf. Fodor and Sag 1982; Endriss 2009, and the references cited therein). The example taken from Farkas (2002) has two readings.

\ea\label{ex14}\langinfo{German}{}{Farkas 2002}\\
\gll Wenn Ben ein Problem von der Liste lost, wird Mr.Koens ihn loben. \\
    If Ben a problem from the list solves will Mr.Kiens him prais\\
\glt `If Ben solves a problem from the list, Mr.Koens will praise him.'
\z

First, there is a narrow scope reading for the indefinite that says that  Mr. Koens will praise Ben if he solves some problem or other from the list; any question that he answers will earn him praise. But there is also an exceptional wide-scope reading where the indefinite takes scope over the conditional, stating that there is some specific problem on the list such that Mr. Koens will praise Ben if he solves that problem. 

In Akan, only the wide scope meaning is available in this context. The narrow scope interpretation is not available; it is only possible when the indefinite is replaced by the free choice item \emph{biara}.



\ea\label{ex15}\langinfo{Akan}{}{Suggested by reviewer} \\
\gll  Sε Kofi tumi bua nsεm \textbf{bi} ano wɔ nsohwε no mu a mε- kyε no adeε.\\
 if Kofi be.able answer questions \textsc{ind} mouth be.located exam \textsc{def} in \textsc{rel} \textsc{1sg.fut}- give 3\textsc{sg}.\textsc{obj} thing \\
\glt `If Kofi answers some questions on the test/in the exam, I will give him a gift.'
\z 

(\ref{ex15}) only has the reading that there is a particular question such that answering that question earns Kofi a gift. 

\subsection{Felicity conditions of indefinites}

Ionin (2006) proposes that specific indefinites carry felicity conditions on their use: a specific indefinite can be felicitously used by the speaker only when particular pragmatic conditions have been met. She discusses two felicity conditions: \emph{noteworthiness} and \emph{identifiability} (cf. Abusch and Rooth 1997; Farkas 2002b). Ionin (2013) argues that the indefinite \emph{this} in English carries a condition of noteworthiness and \emph{odin} in Russian carries a condition of identifiability. In this section, following Ionin (2013), I  argue that \emph{bi} carries both a \emph{noteworthiness} and an \emph{identifiability} condition. Ionin (2006) proposes that the use of \emph{this} implies that the speaker knows something noteworthy about the referent of the indefinite. The condition of noteworthiness is not the same thing as speaker knowledge; the speaker can felicitously use a \emph{this}-indefinite even if she does not know the exact identity of the individual under discussion (what counts as knowing the identity of the referent will be discussed in the next section). This is illustrated in the examples below. 
\ea \langinfo{English}{}{Ionin (2006:183)}\\

\ea[\#]{Mary wants to see this new movie; I don’t know which movie it is.} \label{ex18}

\ex[]
{Mary wants to see this new movie; I don’t know which movie it is, but she’s been all excited about seeing it for weeks now.}\label{ex19}
\ex[]
{I want to see this new movie – I can’t remember its name and I have no idea what it’s about, but someone mentioned to me that it’s really interesting.} \label{ex200}

\z
\z

In all the examples above, the speaker states that she does not know the referent of the indefinite, but (\ref{ex19}) and (\ref{ex200}) are felicitous because the referent is noteworthy. (\ref{ex18}) is infelicitous because the condition of noteworthiness is not met. Noteworthiness must be expressed in the sentence.

Abusch and Rooth (1997), propose that the felicitous use of  ``a certain X” requires the speaker to be able to answer the question \emph{``which X is it?"} (see also Ebert et al. (2013) for a similar proposal for \emph{gewiss and bestimmt}). Aloni (2001) argues that in Russian when a  speaker utters \emph{odna kniga ``one book,"} the speaker conveys that she can answer the question ``which book is it?”; the response to this question names an identifying property that singles out a specific book, distinguishing it from all other books. The identifying property does not have to be the name of the book; it may just as easily be some other relevant property that singles out a specific book. More importantly, the identifying property must come from outside of the sentence.


The indefinite determiner \emph{bi} has both a noteworthy and an identifiability felicity condition, but only one of these conditions needs to be satisfied for its felicitous use in a context. The difference between \emph{bi} and the English determiner \emph{this} which only has a noteworthy felicity condition is shown in (\ref{ex20}). In  (\ref{ex20}), the sentence is grammatical even though the noteworthy condition is not satisfied, while in (\ref{ex18}), the lack of noteworthiness makes the sentence infelicitous. (\ref{ex20}) is felicitous in this context because there is an identifiable property `new movie'. As stated above, identifiability does not only have to do with naming (see Aloni and Port (2015) on conceptual covers and methods of identification for epistemic indefinites.) but any description that is able to set the referent of an NP apart from other NPs. Identifiability is context dependent. Context determines what counts as an identifiable property of an NP is in order to assume that the speaker ``knows" it. Context and how it relates to identifiability will be discussed in detail in section 4. In (\ref{ex20}) for instance, \emph{new movie} is an identifiable property that separates the movie Ama wants to watch from other movies.  But there can be countless new movies at any particular time; it appears this identifying property does not qualify as enough information to say that we know the movie in question. 

\ea
\langinfo{Akan}{}{personal knowledge}\\
\ea\label{ex20}
\gll Ama pε sε ɔ- kɔ-hwε sini foforɔ \textbf{bi} a a- ba. Me- n- nim sini koro mpo.\\
Ama want \textsc{comp} \textsc{3sg}- \textsc{MOT}-watch movie new \textsc{ind} \textsc{rel} \textsc{perf}- come \textsc{1sg}- \textsc{neg}- know movie one even \\
\glt `Ama wants to see a certain new movie. I don't even know what movie.'

\ex\label{ex21}
 \gll Ama pε sε ɔ- kɔ-hwε sini foforɔ \textbf{bi} a a- ba, me- n- nim sini koro nanso ɔ- a- ka ho asεm saa ara.\\
Ama want \textsc{comp} \textsc{3sg}- \textsc{MOT}-watch movie new \textsc{ind} \textsc{rel} \textsc{perf}- come, \textsc{1sg}- \textsc{neg}- know movie one but \textsc{3sg}- \textsc{perf}- say self message \textsc{emp} \textsc{emp} \\
\glt `Ama wants to see a certain new movie, I don't even know what movie but she has been talking about it for two weeks.'

\ex\label{ex20a}
\gll Me- pε sε me- kɔ-hwε sini foforɔ \textbf{bi}. Me- n- kae ne din, me- n- nim nea ε- fa ho mpo nanso obi a- ka a- kyerε me sε ε- yε kama.\\
\textsc{1sg}- want \textsc{comp} \textsc{3sg}- \textsc{mot}-watch movie new \textsc{ind} \textsc{1sg}- \textsc{neg}- remember \textsc{3sg}-\textsc{poss} name \textsc{1sg}- \textsc{neg}- know what \textsc{3sg}- take self even but someone \textsc{perf}- say \textsc{cons}- show \textsc{1sg}.\textsc{obj} \textsc{comp} \textsc{3sg}- \textsc{cop} nice\\
\glt `I want to see a certain new movie – I can’t remember its name and I have no idea what it’s about, but someone mentioned to me that it’s really interesting.'
\z\z

We will now turn to the epistemic indefinite analysis of \emph{bi}, keeping in mind the felicity conditions just discussed.

\section{Epistemic Indefinites (EI)}

Aloni and Port (2015) distinguish between two types of indefinites: plain indefinites and epistemic indefinites. Plain indefinites like \emph{somebody}, in addition to their conventional meaning, have an ignorance implicature.

\ea
\ea\label{ex25}\langinfo{English}{}{Aloni \& Port 2015:117}\\
 Somebody arrived late.\\
\ea Conventional meaning: Somebody arrived late.\\
\ex Ignorance implicature: The speaker doesn't know who.
\z \z\z


Epistemic Indefinites, on the other hand, are indefinites in which this ignorance inference is conventionalized, i.e., is part of the meaning of the indefinite. Epistemic indefinites express the knowledge state of the speaker. Examples of epistemic indefinite determiners include German \emph{irgendein} (Haspelmath 1997, Kratzer and Shimoyama 2002, cited in Aloni and Port 2015) and Italian \emph{un qualche}  (Zamparelli 2007, cited in Aloni and Port 2015). The examples below are from Aloni and Port (2015).

\ea\label{ex26}\langinfo{German}{}{Aloni \& Port 2015:119}\\
 \gll\emph{Irgendein} student hat angerufen. \#Rat mal wer?  \\
     Irgend-one student has called guess \textsc{prt} who .\\
\glt `Conventional meaning: Some student called – the speaker doesn’t know who.

\ex \label{ex28}\langinfo{Italian}{}{Aloni \& Port 2015:119}\\
\gll Maria ha sposato \emph{un} \emph{qualche} professore. \#Indovina chi? \\
      Maria has married a qualche professor guess who? .\\
\glt `Conventional meaning: Maria married some professor – the speaker doesn’t know who.
\z 

In addition to expressing an existential proposition, these sentences have the additional claim that the speaker doesn't know who the witness to this proposition is (Aloni and Port (2015). For this reason, the continuation \emph{`guess who?'} results in a contradiction. \emph{`Guess who?'} presupposes that the speaker has some knowledge, which contradicts the ignorance inference of epistemic indefinites, resulting in the oddity.  This assumed ignorance is not necessarily total ignorance of the referent of the NP,  just the contextual relevant property to claim knowledge of the referent. Plain indefinites, on the other hand, allow for this type of continuation.

  \ea\label{ex26}\langinfo{English}{}{Aloni \& Port 2015:117}\\
 Somebody arrived late, guess who?
 \z 
 
 As an epistemic indefinite, therefore, \emph{bi} should behave like \emph{irgendein} and \emph{un qualche}. We expect that it is infelicitous with \emph{`Guess who?'}, when it expresses that the speaker is ignorant about the contextually relevant property to characterize the NP that asserts knowledge.  


\ea \langinfo{Akan}{}{personal knowledge}\\
 \ea\label{ex30}
 \gll Sukuuni \textbf{bi} a- frε wo. \# wo hwε a ε- yε hwan? \\
     student \textsc{ind} \textsc{perf}- call       \textsc{2sg}-\textsc{obj} {} \textsc{2sg}- look  \textsc{rel} \textsc{3sg}- \textsc{cop}. who \\
\glt `Some student has called, guess who?.

\ex \label{ex31}
\gll  Ama a- ware professor \textbf{bi}. \# wo- hwε a ε- yε hwan?\\
Ama \textsc{perf}- marry professor \textsc{ind} {} \textsc{2sg}- look  \textsc{rel}. \textsc{3sg}- \textsc{cop} who\\
\glt `Ama has married some professor, guess who?
\z\z 

I have to point out however that (\ref{ex30}) is felicitous when the speaker is sarcastic, but this context is marked. \emph{Bi}, therefore, appears to have a conventionalized ignorance inference. In addition to \emph{guess who}, epistemic indefinites are also infelicitous with \emph{namely}. It becomes felicitous, however, if the speaker signals that he is reporting the name and does not know anything else about the referent. 
\ea \langinfo{Akan}{}{personal knowledge}\\
\ea\label{ex32}
 \gll Sukuuni \textbf{bi} frε -ε wo. \# Yε- frε no Kwadwo\\
student \textsc{ind} call -\textsc{pst}  \textsc{2sg} {} \textsc{imp}- call  3\textsc{sg}-\textsc{obj} Kwadwo? \\
    
\glt `Some student called you, \# he is called Kwadwo.

\ex \label{ex33}
\gll  Sukuuni \textbf{bi} frε -ε wo. ɔ- se yε- frε no Kwadwo.\\
 student \textsc{ind} call -\textsc{pst}  \textsc{2sg} \textsc{3sg}- say \textsc{imp}- call  \textsc{3sg}-\textsc{obj} Kwadwo\\
\glt `Some student called you, he says he is called Kwadwo.
\z\z In the subsequent sections, I will discuss \emph{bi} in relation to the functions of epistemic indefinite discussed by  Aloni and Port (2015).

\subsection{Functions of epistemic indefinites}

Aloni and Port (2015) discuss four functions of epistemic indefinites. These are (i) \emph{specific unknown function} (SU), when it is used in an unembedded context. (ii) \emph{Epistemic unknown function} (epiU), when it is embedded under an epistemic modal. (iii) \emph{Negative polarity item (NPI) function}, when it gets narrow scope under negation and (iv) \emph{deontic free choice function} (deoFC), when it is embedded under deontic modals. They argue that for an indefinite to qualify for any of these functions, it must (a) be grammatical in the context the function specifies and (b) have the meaning that the function specifies. As we already discussed in the previous section, \emph{bi} cannot be embedded under negation, which means that it does not have an NPI function. Also, though \emph{bi} can be embedded under a deontic modal, it does not have a deoFc function; it only has a specific unknown interpretation under deontic modals. 

\ea\label{ex34}\langinfo{Akan}{}{personal knowledge}\\
 \gll ε- sε  sε Ama ware professor \textbf{bi}.\\
\textsc{3sg}- have \textsc{comp} Ama marry professor \textsc{ind} \\
    
\glt `Ama must marry some professor.' (SU)

\z (\ref{ex34}) only has the interpretation that there is a particular professor that Ama must marry. It does not have the meaning that it is necessarily the case that Ama marries some specific professor, which is the low scope reading of \emph{bi}. 



\subsubsection{Specific Unknown Function (SU)}
Syntactically, the specific unknown function is characterized by an unembedded use of the indefinite, i.e., use in matrix clause and not embedded under negation, modals or attitude verbs. Semantically, they have an obligatory ignorance effect: the speaker does not know the intended referent of the indefinite. Following Aloni and Port (2015), I will use the following continuation to distinguish between the specific and non-specific uses of the indefinite.

\ea\label{ex40}\langinfo{English}{}{Aloni \& Port 2015:118}\\
 John wants to marry a Norwegian\\
 \ea
 She lives in Oslo and is 25 years old.\\
\ex One with blond hair and blue eyes.
\z\z
\emph{Bi} like both \emph{irgendein} and \emph{un qualche} is grammatical in the specific unknown context  and the interpretation specified. 
\ea \langinfo{Akan}{}{personal knowledge}\\
\ea\label{ex42}
 \gll  Ama a- ware professor \textbf{bi}. \# wo- hwε a ε- yε hwan?\\
Ama \textsc{perf}- marry professor \textsc{ind} {} \textsc{2sg}- look  \textsc{cond}. \textsc{3sg}- \textsc{cop} who \\
    
\glt `Ama has married some professor, \#guess who? ' 

\ex 
\gll  Sukuuni \textbf{bi} frε -ε wo. \# wo hwε a ε- yε hwan?\\
student  \textsc{ind} call -\textsc{pst}       \textsc{2sg}-\textsc{obj} {} \textsc{2sg}- look  \textsc{cond} \textsc{3sg}- \textsc{cop} who\\
    
\glt `Some student has called, \#guess who?.' 
\z \z

Like \emph{irgendein} and \emph{un qualche}, \emph{bi} has the weaker modal variation interpretation and not the stronger free choice interpretation. Aloni and Port (ibid) following Alonso-Ovalle and Menendez-Benito (2010) distinguish between two types of modal inference, modal variation, and free choice. 
For the modal variation interpretation, more than one (but not necessarily all) alternatives in the relevant domain qualify as possible options. For the free choice interpretation, all the alternatives in the relevant domain qualify. In (\ref{ex420}), the sentence is still true even if there are professors that the speaker has enough evidence to eliminate from the list of possible professors that Ama could have married.
\ea\label{ex420}\langinfo{Akan}{}{personal knowledge}\\
 \gll  Ama a- ware professor \textbf{bi}. Me- n- nim nipa koro nanso me- yε sure sε ε- n- yε Kofi.\\
Ama \textsc{perf}- marry professor \textsc{ind} \textsc{1sg}- \textsc{neg}- know human person but \textsc{1sg}- do sure \textsc{comp}  \textsc{3sg}- \textsc{neg}- \textsc{cop}. Kofi \\
    
\glt `Ama has married some professor. I don't know who it is. I am sure it is not Kofi.' 
\z 

\subsubsection{Epistemic Unknown Function (epiU)}
An ignorance effect similar to the one with specific unknowns arises when \emph{bi} is embedded under epistemic modals.

\ea\label{ex42}\langinfo{Akan}{}{personal knowledge}\\
 \gll  ε- bε- tumi a- ba sε Ama a- ware professor \textbf{bi}. 
\\
\textsc{3sg}- \textsc{mod}- be.able \textsc{cons}- come \textsc{comp} Ama \textsc{perf}- marry professor \textsc{ind} \\
    
\glt ` `It could be that Ama has married some professor.' 
\z 
This also has the weaker modal variation interpretation. It is compatible with the hide and seek scenario described in Aloni and Port (2015), where not all the alternatives in the relevant domain qualify.
\ea\label{ex42}\langinfo{Akan}{}{personal knowledge}\\
 \gll  ε- bε- tumi a- ba sε Ama a- ware professor \textbf{bi}. ε- te saa a  me- yε sure sε ε- n- yε Kofi. \\
\textsc{3sg} 
 \textsc{mod}- be.able \textsc{cons}- come \textsc{comp} Ama \textsc{perf}- marry professor \textsc{ind} \textsc{3sg}- \textsc{cop} \textsc{dem} \textsc{cond} \textsc{1sg}- do sure \textsc{comp}  \textsc{3sg}- \textsc{neg}- \textsc{cop} Kofi\\
\glt ` It could be that Ama has married some professor. If that is true, I am sure it is not Kofi.' 
\z 

Similar to \emph{irgendein} and \emph{un qualche}, \emph{bi} embedded under propositional attitude verbs have agent oriented ignorance effects.  

 \ea\label{ex42}\langinfo{Akan}{}{personal knowledge}\\
 \gll  Nana gye di sε Ama a- ware professor \textbf{bi}. \\
Nana collect eat \textsc{comp} Ama \textsc{perf}- marry professor \textsc{ind}\\
\glt `Nana believes Ama has married some professor.' \\
Nana believes that Ama married some professor, I don't know who. (SU)\\
Nana believes that Ama married some professor, Nana don't know who. (EpiU)
\z 

\emph{Bi} therefore is more similar to \emph{un qualche} which has no NPI and deoFC functions. \\
 \tabref{tab:owusu:1} below, taken from Aloni and Port (2015), show some cross-linguistic comparison of epistemic indefinites; I have added \emph{bi} to this table.\\

\begin{table}
\caption{Cross-linguistic comparison of epistemic indefinites}
	\begin{tabular}{|c | p{2cm} | p{2cm} | p{2cm}| p{2cm}| }
		\hline
		{}		& SU		& 	epiU & NPI & deoFC \\ \hline
		\emph{irgendein} 			& yes 		& 	yes	& 	yes	& 	yes	\\ \hline
		\emph{alg'un} (SP)        & yes 	 	& 	yes 	& 	yes & 	no		\\ \hline
		 \emph{un qualche} 		& yes 	 	& 	yes 	& 	yes & 	no \\		\hline
		-\emph{si} (Cz)	& 	yes	& 	no &	 no & no			\\ \hline
		\emph{vreun} (Rom) 		& no	 	& 	yes 	& 	yes & 	no \\		\hline	
		\emph{any}(En) 		& no 	 	& 	no 	& 	yes & 	yes \\		\hline	
		\emph{qualunque} (It)		& no	 	& 	no 	& 	no & 	yes \\		\hline
		\emph{bi} (Akan) 		& yes 	 	& 	yes 	& 	no & 	no \\		\hline
	\end{tabular}
	\label{tab:owusu:1}
    \end{table}
    
Akan confirms Aloni and Port's (2015) hypothesis that there should be no language where an epistemic indefinite has a deoFC function but not an NPI function. 

\subsection{Methods of identification and Conceptual Covers}
There are at least two ways in which a context can determine a quantificational domain, domain widening and method of identification (conceptual covers). 
A conceptual cover is a set of individual concepts which exclusively and exhaustively covers the domain of individuals, (Aloni, 2001).

\textbf{Definition 1} [Conceptual covers] Given a set of possible worlds W and a domain of individuals D, a conceptual cover CC based on (W, D) is a set of individual concepts [i.e., functions W → D] such that:\\

$\forall$\emph{w} $\in$\emph{W}: $\forall$d $\in$ D: $\exists$!c $\in$ CC : c(w)=d \\

She explains this with a card scenario which I repeat below.
 In front of you lie two face-down cards, one is the Ace of Hearts, the other is the Ace of Spades. You know that the winning card is the Ace of Hearts, but you don’t know whether it’s the card on the left or the one on the right. Now consider (\ref{ex450}):

\ea\label{ex450}\langinfo{English}{}{Aloni \& Port (2015)}\\
 You know which card is the winning card.
\z Based on the scenario above, sentence (\ref{ex450}) could be true or false in the described scenario. Intuitively, there are two different ways in which the cards can be identified here: by their position (the card on the left, the card on the right) or by their suit (the Ace of Hearts, the Ace of Spades). Our evaluation of (\ref{ex450}) seems to depend on which of these identification methods is adopted. In the semantics of \emph{knowing-wh} constructions proposed in Aloni (2001), the evaluation of (\ref{ex450}) depends on which of these covers is adopted. She adds that this dependency is captured by letting the wh-phrase range over concepts in a conceptual cover instead of plain indefinites. Cover indices \emph{n} are added to their logical form, and context supplies their value. 


\ea\label{ex45}\langinfo{English}{}{Aloni \& Port (2015)}\\
 You know which-\emph{n} card is the winning card.\\
 \textbf{False} if n $\longrightarrow$ {on-the-left, on-the-right}\\
 \textbf{True} if n $\longrightarrow$ {ace-of-spades, ace-of-hearts} \\
 \textbf{Trivial} if n $\longrightarrow$ {the-winning-card, the-losing-card}
\z 
Conceptual covers and methods of identification are essential in understanding especially the specific unknown function of epistemic indefinites. When a speaker uses a specific marker, she signals that she has a particular referent in mind and that she can identify the referent of the indefinite. This appears to conflict with the ignorance inference that we have argued that epistemic indefinites have. The natural way to resolve this conflict is to assume that there are two methods of identification at play, the speaker knows one but not the other. The ignorance is not about all methods of identification for the referent, but for the essential one in that particular context.  

\subsubsection{Methods of Identification}
In this section, I explore the different methods of identification and the context of use that license \emph{bi}. I compare \emph{bi} to the German \emph{irgendein} and the Italian \emph{un qualche}. The method of identification that will be discussed are \emph{naming}, \emph{ostension} and \emph{description}.

\textbf{Description and Naming}

Scenario: You are visiting a foreign university and you want to meet some professor. 
\ea\label{ex45}\langinfo{Akan}{}{personal knowledge}\\
\gll Me- re- hwεhwε professor \textbf{bi}, ɔno na ɔ- yε head of department, me- n- nim ne din.\\
\textsc{1sg}- \textsc{prog}- search professor \textsc{ind}, \textsc{3sg} \textsc{foc}. \textsc{3sg}- \textsc{cop}. head of department, \textsc{1sg}- \textsc{neg}- know \textsc{3sg}-\textsc{poss} name.\\
\glt `I am looking for some professor, he is the head of department but I don't know his name.'\\
	Speaker-can-identify $\rightarrow$ [Description], unknown $\rightarrow$  [Naming]
 \z  In this scenario, the method of identification contextually required for knowledge is naming, but the referent of the epistemic definite can only be identified by description. \\
\textbf{Naming and Ostension}\\
Scenario: At a conference, you have to meet a famous linguist.
\ea\label{ex45}\langinfo{Akan}{}{personal knowledge}\\
\gll ε- wɔ  sε me- hyia professor \textbf{bi}, yε- frε no Nana Aba nanso me- n- nim no.\\
\textsc{3sg}- have \textsc{comp} \textsc{1sg}- meet professor \textsc{ind}, \textsc{3sg} call her Nana Aba, but \textsc{1sg}- \textsc{neg}- know \textsc{3sg}\\
\glt `I have to meet some professor, her name is Nana Aba, but I don't know her.'\\
	Speaker-can-identify $\rightarrow$ [Naming], unknown $\rightarrow$  [Ostension]
 \z In this scenario, the method of identification contextually required for knowledge is ostension, but the referent of the epistemic definite can only be identified by naming. \\

\textbf{Ostension and Naming}

 Scenario: You are watching a football match and a player gets injured, so you tell your friends:

 \ea\label{ex45}\langinfo{Akan}{}{personal knowledge}\\
\gll Hwε player \textbf{bi} a- pira, yε frε no sεn?\\
look player  \textsc{ind} \textsc{perf}- be.injured, \textsc{3pl} call \textsc{3sg}.\textsc{obj} what\\
\glt `Look, some player is injured, what is his name?'\\
	Speaker-can-identify $\rightarrow$ [Ostention], unknown $\rightarrow$  [Naming]

 \z In this scenario, the method of identification contextually required for knowledge is naming, but the referent of the epistemic definite can only be identified by ostension.
 
 Aloni (2001) ranks the method of identification as indicated below:
 
 Ostension $>$ Naming $>$ Description\\
 
Like \emph{bi}, \emph{irgendein} is felicitous in all the scenarios presented, \emph{un qualche} on the other hand is infelicitous in the third scenario when the speaker could identify by ostension, but naming was unknown. \emph{Bi} behaves like Germanic epistemic indefinites.

\section{Analysis}
Following Ionin (2013), I propose the following as the semantics of \emph{bi}.

\textbf{Semantics}: A sentence of the form [bi $\alpha$] $\beta$ expresses a proposition only in those utterance contexts c where the speaker intends to refer to exactly one individual or the max of a group \footnote{\emph{Bi} is compatible with plural nouns.} \emph{y} which is [$\alpha$] in c and the relevant felicity conditions in (\ref{50}) or (\ref{51}) are fulfilled. Then [\emph{bi} $\alpha$] $\beta$ is true at an index \emph{y} if \emph{y} is $\beta$ at \emph{y} and false otherwise. In addition to the felicity conditions, there is a presupposition that states the noteworthy and/or identifying property is all the information the speaker has about [bi $\alpha$]. This presupposition is what differentiates \emph{bi} and the English \emph{some} from English the specific indefinite \emph{a}.   
These sentences are only grammatical when they fulfill either of the conditions below. 
\ea
\ea \label{50} For [bi $\alpha$] $\beta$, the speaker has in mind a \textbf{noteworthy property} $\varphi$ $\in$D$<$s,et$>$ such that $\varphi$ (wc )(y) = 1.

\ex \label{51} The speaker is able to name \textbf{an identifying property} $\varphi$ $\in$D$<$s,et$>$ such that $\varphi$(wc)(y)=1 and $\forall$z[$\alpha$(wc)(z)=1 and z$\neq$ y] → $\varphi$(wc)(z)$\neq$ 1] and $\varphi$ $\neq$ $\alpha$ and $\varphi$ $\neq$ $\beta$.
\ex \textbf{Presupposition}: Speaker is unable to provide any further information about who or what satisfies the existential claim s/he is making.\\ The idea comes from Alonso-Ovalle and Menéndez-Benito (2003) for English \emph{some} and Spanish \emph{alg\'un}. They did not, however, state it as a presupposition.  
\z \z This meaning rules out a kind or generic interpretation for \emph{bi}. (\ref{ex60}) can only mean that there is exactly one dog that Kofi saw, it cannot have the interpretation that Kofi saw something of the kind dog. When a plural is used with the determiner as in (\ref{ex61}), the determiners quantifiers over the sum or the max of the referents.

\ea \langinfo{Akan}{}{personal knowledge}\\
\ea\label{ex60}
\gll Kofi hu -u kraman \textbf{bi}.\\
Kofi see -\textsc{pst} dog bi\\
\glt `Kofi saw a certain dog'\\
	
\ex \label{ex61}
\gll Kofi hu -u n-  kraman \textbf{bi}.\\
     Kofi see -\textsc{pst} PL- dog bi\\
\glt     `Kofi saw some dogs.' 

 \z\z What is considered noteworthy is provided by the sentence, and identifiability supplied by context. 
For an N + bi to be felicitous, the noteworthy or identifying felicity condition must be satisfied, i.e., the speaker needs to be able to name at least one defining property of the referent with which an addresser can construct a referent of their own. And also the presupposition must be satisfied, i.e., whatever the noteworthy or identifying property is, its use is not sufficient to claim ``knowledge" of the referent in the context. What qualifies as sufficient information to ``know" the referent is regulated by the method of identification licensed by the context. In other words, in every context, there are two ways to identify the referent of an NP. One is whatever properties, descriptions or characteristics that are used in the sentence to identify it. This helps an addressee construct a referent of their own. Then, there is the way a particular context requires that the referent of an NP is identified in other to count as ``knowing" the referent. For instance, if an NP is described as a ``red flower'' in a sentence, then the first method of identification is satisfied. But if the contexts require that someone only know what flower it is by knowing the name of the flower, then ``red flower" is inadequate in this context. The use of N+bi signals that the second identifying property, i.e., the one required by context to ``know" the referent is not satisfied. This idea of multiple methods of identification is what conceptual covers by Aloni and Port (2015) appear to capture. ``Suppose \emph{m} is the conceptual cover representing the identification method contextually required for knowledge, then EI signal an obligatory shift to a cover \emph{n} different from \emph{m}. That is, they existentially quantify over a cover which represents a method of identification which is not the one at play in the relevant context" (Aloni and Port 2015:132). 

If we take a look at (\ref{ex620}), the speaker has asserted a noteworthy and identifying property about the flower.  This property does not, however, show that the speaker ``knows" the flower. What does it mean to know a flower? We use names to identify flowers. There are dozens of red flowers and thus knowing the color of a flower does not count as knowing it. 


\ea\label{ex620}\langinfo{Akan}{}{personal knowledge}\\
\gll Me- hu -u nhwiren kɔkɔɔ \textbf{bi}.\\
 1SG- see -\textsc{pst}  flower red \textsc{ind} \\
\glt `I saw a certain red flower.'\\
Speaker-can-identify $\rightarrow$ [Description], unknown $\rightarrow$  [Naming]

 \z  
In this scenario, the method of identification contextually required for knowledge is naming, but the referent of the epistemic definite can only be identified by description. The ignorance component of the indefinite does not conflict with the noteworthy and identifying property requirement.\\

In example (\ref{ex001}), the book is identified as the book lying on the table, as against the one on the bookshelf.  \\
\ea\label{ex001}\langinfo{Akan}{}{personal knowledge}\\
\gll Me- hu -u nwoma \textbf{bi} wɔ pono no so.\\
	     \textsc{1sg}- see -\textsc{pst}  book \textsc{ind} be.located table \textsc{def}  top\\

\glt `There is a book on the table .'\\

 \z 
 
This sentence is (in)felicitous in different contexts.

\textbf{Context 1}:\\
I am sitting by the table and I can see \emph{\emph{War and Peace}} on the table. Someone is looking for \emph{\emph{War and Peace}}.

\ea\label{ex00}\langinfo{Akan}{}{personal knowledge}\\
\gll \# Nhoma \textbf{bi} da pono no so nanso me- n- nim nhoma koro.\\
	     {} book \textsc{ind}  lie table \textsc{def}  top but \textsc{1sg}- \textsc{neg}- know book one\\

\glt `There is some book on the table but I don't one which book it is.'\\

 \z  In this context, the speaker knows the name of the book and can see it. What appears to count as knowing a book is knowing its name. Thus the identification supplied by the sentence and the identification required by the sentence are the same, there is no speaker ignorance. The speaker knows what is necessary and sufficient to identify the book in this context, the ignorance component of \emph{bi} is not fulfilled, and so the sentence is infelicitous. \\

\textbf{Context 2} :\\

I am sitting at the table, and I can see \emph{\emph{War and Peace}} on the table. Someone is looking for some book, but I do not know what book they are looking for.  

\ea\label{ex62}\langinfo{Akan}{}{personal knowledge}\\
\gll Nwoma \textbf{bi} da pono no so.\\
	      book \textsc{ind}  sleep table \textsc{def} top\\

\glt `There is some book book on the table.'

 \z In this context, the speaker knows the name of the book and can see it. What s/he does not know is if the referent of the N satisfies what the addressee is looking for. So here, knowing the name of the book is not sufficient, what counts as ``knowing" in this context is knowing what the speaker is looking for.  The identification supplied by the sentence and the identification required by the sentence is different, speaker ignorance is satisfied, and the sentence is felicitous.
 
\textbf{Context 3}:

I am cooking in the kitchen; someone is looking for \emph{War and Peace}. I remember I saw \emph{War and Peace} on the table.
\ea\label{ex62}\langinfo{Akan}{}{personal knowledge}\\
\gll \# Me- hu -u nwoma \textbf{bi} wɔ pono no so nanso me- n- nim nhoma koro.\\
	     {} \textsc{1sg}- see -\textsc{pst}  book \textsc{ind}  be.located table \textsc{def}  top but \textsc{1sg}- \textsc{neg}- know book one\\

\glt `There is some book on the table but I don't one which book it is.'

 \z In this context, the speaker knows the name of the book and she knows this satisfies the what the addressee is looking for. Knowing the name of the book is sufficient and counts as knowing what the speaker is looking for. The identification supplied by the sentence and the identification required by the sentence is the same. Speaker ignorance is not satisfied, so the sentence is infelicitous.  \\\\
\textbf{Context 4:} \\
I am cooking in the kitchen; someone is looking for \emph{War and Peace}. I remember I saw a book on the table, but I do not remember exactly what book it was.
\ea\label{ex62}\langinfo{Akan}{}{personal knowledge)}\\
\gll Me- hu -u nwoma \textbf{bi} wɔ pono no so nanso me- n- nim nhoma koro.\\
	     \textsc{1sg}- see -\textsc{pst}  book \textsc{ind}  be.located table \textsc{def} top but \textsc{1sg}- \textsc{neg}- know book one\\

\glt `There is some book on the table but I don't one which book it is.'
\z In this context, what counts as knowing is knowing what the addressee is looking for. The speaker does not know the name of the book. Therefore the identification supplied by the sentence (description) and the identification required by the sentence (naming) are different. Speaker ignorance is consequently satisfied, and the sentence is felicitous. 

\section{Conclusion}

I have shown how that \emph{bi} has both a  specific indefinite and an epistemic indefinite and these interpretations do not contradict each other. The identifiability felicity condition that is required for the interpretation of the specific indefinite feeds the epistemic indefinite interpretation. I have argued that when \emph{bi} is used, the speaker signals that she does not have access to all the information that is required to ``know" a referent in a particular context. She has some information about the referent to identify him, but not enough to ``know" him. In so doing, I have highlighted a meaning of the indefinite determiner \emph{bi} that has hitherto not been discussed in the Akan literature. 
 


%\printbibliography[heading=subbibliography,notkeyword=this]
%\printbibliography[localbibliography]
\section*{References}
Abusch, Dorit, and Mats Rooth. ``Epistemic NP modifiers." \emph{Semantics and Linguistic Theory}. Vol. 7. 1997.\\
Alonso-Ovalle, Luis, and Paula Menéndez-Benito. ``Modal indefinites." \emph{Natural Language Semantics}18.1 (2010): 1-31.\\
Alonso-Ovalle, Luis, and Paula Menéndez-Benito. "Some epistemic indefinites." PROCEEDINGS-NELS. Vol. 33. 2003.\\
Aloni, Maria. \emph{Quantification under conceptual covers}. Institute for Logic, Language and Computation, 2001.\\
Aloni,Mari and Port, Angelika. Epistemic indefinites and methods of identification in \emph{Epistemic Indefinites: Exploring Modality Beyond the Verbal Domain.} Eds. Alonso-Ovalle, Luis, and Paula Menéndez-Benito. : Oxford University Press, 2015-04-30. Oxford Scholarship Online. 2015-08-20\\
 Amfo, Nana Aba Appiah. Indefiniteness Marking and Akan Bi. \emph{Journal of Pragmatics}, (2009).\\
Arkoh, Ruby Becky. \emph{Semantics of Akan bi and nʊ.} Diss. UNIVERSITY OF BRITISH COLUMBIA (Vancouver, 2011.\\
Comrie, Bernard. \emph{Language universals and linguistic typology: Syntax and morphology.} University of Chicago press, 1989.\\
Endriss, Cornelia. \emph{Conclusion}. Springer Netherlands, 2009.\\
Farkas, Donka F. ``Varieties of indefinites." \emph{Semantics and Linguistic Theory}. Vol. 12. 2002.\\
Fodor, Janet Dean, and Ivan A. Sag. ``Referential and quantificational indefinites." \emph{Linguistics and philosophy} 5.3 (1982): 355-398.\\
Givón, Talmy. \emph{Syntax: an introduction}. Vol. 1. John Benjamins Publishing, 2001.\\
Gundel, Jeanette K., Nancy Hedberg, and Ron Zacharski. ``Cognitive status and the form of referring expressions in discourse." \emph{language} (1993): 274L-307. \\
 Haspelmath, Martin. \emph{Indefinite pronouns}. Oxford University Press on Demand, 1997.\\
 Hawkins, John A. \emph{Definiteness and indefiniteness: A study in reference and grammaticality prediction.} Vol. 11. Routledge, 2015.
Heim, Irene. "File change semantics and the familiarity theory of definiteness." (1983): 164-189.\\
Hinterwimmwer and  Umbach (2015), Grading and hedging by gewiss in \emph{Epistemic Indefinites: Exploring Modality Beyond the Verbal Domain.} Eds. Alonso-Ovalle, Luis, and Paula Menéndez-Benito. : Oxford University Press, 2015-04-30. Oxford Scholarship Online. 2015-08-20\\
Ionin, Tania. \emph{Pragmatic variation among specificity markers.} Different kinds of specificity across languages. Springer Netherlands, 2013. 75-103.\\
Ionin, Tania. ``This is definitely specific: specificity and definiteness in article systems." \emph{Natural language semantics} 14.2 (2006): 175-234.\\
Payne, Thomas Edward. \emph{Describing morphosyntax: A guide for field linguists.} Cambridge University Press, 1997.\\
Kratzer, Angelika. \emph{Scope or pseudoscope? Are there wide-scope indefinites?.} Springer Netherlands, 1998.
\\Kratzer, Angelika, and Junko Shimoyama. \emph{Indeterminate pronouns: The view from Japanese.} 3rd Tokyo conference on psycholinguistics. 2002.\\
Zamparelli, Roberto. ``On singular existential quantifiers in Italian." \emph{Existence: Semantics and syntax}. Springer, Dordrecht, 2008. 293-328.

\section*{Acknowledgements}
This paper started as a term paper for a Semantic Seminar taught by Veneeta Dayal. I will like to thank the Rutgers Semantics and Syntax Reading group, your comments and insights helped improve this paper. I am also grateful to the participants of  ACAL 48, your questions and comments were beneficial. I have benefited from the comments and insights of two anonymous reviewers. 


\section*{List of abbreviations}
\begin{tabular}{l l} 
\textbf{IND} & Indefinite \\
\textbf{Ref} & Referential \\
\textbf{Fam} & Familiar \\
\textbf{PROG} & Progressive \\
\textbf{FUT} & Fut \\
\textbf{PST} & Past \\
\textbf{CFM} & Clausal Final Marker \\
\textbf{DEF} & Definite  \\
\textbf{MOT} & Verb of motion  \\
\textbf{REL} & Relativizer \\
\textbf{COMP} & Complementizer \\
\textbf{PERF} & Perfect \\
\textbf{NEG} & Negation \\
\textbf{EMP} & Emphatic marker \\
\textbf{prt} & Particle \\
\textbf{COP} & Copular  \\
\textbf{DEM} & Demonstrative \\
\textbf{POSS} & Possessive  \\
\end{tabular}




\end{document}

